\documentclass[10pt,a4paper,twoside]{book}

\input{style}

\usepackage[
	left=2.5cm, % inner
	right=2.5cm, % outer
	top=2.5cm,
	bottom=3cm,
	%showframe,
	]{geometry}

\allowdisplaybreaks[4] % Consente di rompere equazioni su più pagine

\DeclareMathOperator{\Res}{Res}
\DeclareMathOperator{\Log}{Log}
\DeclareMathOperator{\loc}{loc}

%%%%%%%% PARTE ESERCIZI

\usepackage{titlesec}
\usepackage{titletoc}
\usepackage{etoolbox}

\newcommand{\Esercizio}[1]{\subsection{Esercizio~#1}}
\newcommand{\Soluzione}{\subsection{Soluzione}}

\newcommand{\ParteEsercizi}{\section{Esercizi}}
\newcommand{\ParteSoluzioni}{\section{Soluzioni}}

\setcounter{secnumdepth}{2} % mi assicuro che gli esercizi siano numerati
\setcounter{tocdepth}{1} % solo 1 per non avere tutti gli esercizi nel contents

\begin{document}

\frontmatter

\pagestyle{empty}

% COPERTINA

\hypertarget{mytitlepage}{} % set the hypertarget
\bookmark[dest=mytitlepage,level=chapter]{Title Page} % add the bookmark

\vspace*{\fill}

\begin{center}
	{\large \textsc{Appunti di}}\\
	
	\vspace*{0.4cm}
	
	{\Huge \textsc{Analisi Matematica 3}}\\
	
	\vspace*{1cm}
	
	{\large per il corso di Ingegneria Matematica}\\
	\vspace*{0.2cm}
	{\large dell'A.A. 2020/2021}\\

	\vspace*{0.4cm}

	{\large di Teo Bucci}\\
	
	\vspace*{0.4cm}
	
	{\large revisionati per l'A.A. 2022/2023}\\
	\vspace*{0.2cm}
	{\large da Teo Bonfa}\\
	
	\vspace*{1cm}

	Politecnico di Milano
\end{center}
\vspace*{\fill}
\clearpage

% COPYRIGHT PAGE

\hypertarget{mycopyright}{} % set the hypertarget
\bookmark[dest=mycopyright,level=chapter]{Copyright Page} % add the bookmark
%!TEX root = ../main.tex
\vspace*{\stretch{12}}

\textcopyright \ Gli autori, alcuni diritti riservati.

\paragraph{Disclaimer} In questo testo sono raccolti gli appunti tratti dalle lezioni del corso di Analisi Matematica 3, tenuto dal Professor Gianni Arioli per il corso di Ingegneria Matematica, al Politecnico di Milano, nell'Anno Accademico 2022/23.

La proprietà intellettuale rimane ai docenti sopracitati, i quali non hanno revisionato la seguente opera, che si presenta unicamente come un supporto ulteriore alle lezioni, fatta da studenti per studenti, senza pretese di sostituire libri di testo ufficiali o la frequenza delle lezioni.

Quest'opera è rilasciata sotto licenza CC BY-NC-SA 4.0.\\
\url{http://creativecommons.org/licenses/by-nc-sa/4.0/}

In sintesi: potete condividere i contenuti del libro, in tutto o in parte, e apportare le vostre modifiche, a patto di citare la fonte, di condividere le modifiche con la stessa licenza, e di non usare il materiale per scopi commerciali (non è permesso stampare il libro per rivenderlo).

Il codice sorgente \latex è disponibile su \\
\url{https://github.com/teobucci/analisi3}

\vspace*{\stretch{2}}

\textsc{Documento creato il \today}
\IfFileExists{./commit_hash.part}{\\\textsc{Revisione} \texttt{\input{commit_hash.part}}}{}

\vspace*{\stretch{2}}

\textsc{Developed by:}\\
\textsc{Teo Bucci} - \texttt{teobucci8@gmail.com}\\
\textsc{Teo Bonfa} - \texttt{teobonfa2000@gmail.com}


Per segnalare eventuali errori o suggerimenti potete contattare gli autori o fare una Pull Request.

\vspace*{\stretch{5}}
\clearpage

% PREFAZIONE

% \hypertarget{mypreface}{} % set the hypertarget
% \bookmark[dest=mypreface,level=chapter]{Prefazione} % add the bookmark
% \input{firstpages/prefazione}
% \clearpage

% INDICE

\cleardoublepage
\pagestyle{toc}
\hypertarget{mytoc}{} % set the hypertarget
\bookmark[dest=mytoc,level=chapter]{\contentsname} % add the bookmark
\tableofcontents
\cleardoublepage

% MAIN MATTER

\pagestyle{fancy}
\mainmatter

%\part{Teoria}
%!TEX root = ../main.tex

\chapter{Analisi complessa}

Studieremo funzioni complesse $f : \CC\rightarrow \CC$, lavorando con numeri complessi
\begin{equation*}
z = x + iy = (x, y)
\end{equation*}
La differenza tra i numeri complessi e $\RR^{2}$ è che i numeri complessi costituiscono un \textbf{campo}, ovvero rispetta certe proprietà e dove sono definiti determinati concetti. $\RR^{2}$ è uno spazio vettoriale, non un campo. Questa struttura fa una differenza molto grossa in termini di differenziabilità.
\begin{defn}
$f$ è differenziabile in senso complesso se $f(z + h)$ si può scrivere come
\begin{equation*}
f(z + h) = f(z) + f'(z) h + o(|h|)
\end{equation*}
per $h\rightarrow 0$, ricordando che $f'(z) \in \CC$ e $h\in \CC$.
\end{defn}
Quando diciamo che un numero complesso $h\rightarrow 0$ intendiamo che, detto $h = h_{R} + ih_{I}$, si ha $h^{2}_{R} + h^{2}_{I}\rightarrow 0$, entrambe le parti tendono a $0$. Otteniamo poi che
\begin{equation*}
\lim_{h\rightarrow 0}\frac{f(z + h) - f(z)}{h} = f'(z)
\end{equation*}
Il numero a denominatore è un numero complesso, quindi chiamiamo l'argomento del limite \textit{rapporto incrementale}, ma non è un rapporto perché \textbf{i numeri complessi non sono ordinati}. Dal punto di vista formale, i calcoli sono identici, proprio perché $\CC$ è un campo.

\section{Equazioni di Cauchy - Riemann}

Possiamo separare la parte immaginaria e reale anche nella funzione:
\begin{equation*}
f(z) = u(x, y) + iv(x, y)
\end{equation*}
Il teorema sulla derivata direzionale si può estendere ai numeri complessi
\begin{equation*}
\lim_{h\rightarrow 0}\frac{f(x + hv) - f(x)}{h} = v \cdot \nabla f(x) \ \ \ \ | v| = 1, v\in \CC, h\in \RR
\end{equation*}
ovvero
\begin{equation*}
\partial_{v} f(z) = v \cdot f'(z)
\end{equation*}
Possiamo considerare per esempio due direzioni
\begin{align*}
v = (1, 0) = 1 & \ \ \implies \ \ \partial_{x}[u + iv] = u_{x} + iv_{x} = f'(z)\\
v = (0, 1) = i & \ \ \implies \ \ \partial_{y}[u + iv] = u_{y} + iv_{y} = if'(z)
\end{align*}
mettendo insieme i pezzi
\begin{equation*}
u_{y} + iv_{y} = i(u_{x} + iv_{x})
\end{equation*}
Uguagliando le rispettive parti reali e immaginarie otteniamo le \textbf{Equazioni di Cauchy - Riemann (CR)}
\begin{equation*}
\boxed{u_{x} = v_{y}} \ \ \ \ \boxed{u_{y} = - v_{x}}
\end{equation*}
\begin{defn}
[Funzione Olomorfa]
Sia $f: A\subset \CC\rightarrow \CC$ con $A$ \textbf{aperto}. Se $f\ $è derivabile in $A$ allora $f$ si dice \textbf{olomorfa} in $A$.
\end{defn}
È un concetto molto più forte della semplice derivabilità, anche se la definizione la fa sembrare la stessa cosa. Se una funzione è olomorfa, allora è differenziabile infinite volte (molto buono). Inoltre, se una funzione è olomorfa, allora può essere scritta come serie di potenze.

Vale il seguente risultato
\begin{equation*}
\boxed{f\in H(A) \ \ \iff \ \
\begin{cases}
u, v\ \text{differenziabili in} \ A\\
\text{valgono le (CR)}
\end{cases}}
\end{equation*}
Consideriamo ora $u, v\in C^{2}$ e deriviamole. Usando il teorema di Schwarz
\begin{equation*}
\begin{cases}
\partial_{yy} u = - \partial_{xy} v\ \ \ \ \\
\partial_{xx} u = \partial_{xy} v
\end{cases} \implies \ \ \partial_{yy} u + \partial_{xx} u = \Delta u = 0
\end{equation*}
Analogamente per $v$
\begin{equation*}
\Delta v = 0
\end{equation*}
Sono funzioni \textbf{armoniche}.\footnote{Ovvero il loro laplaciano è nullo.}

Data $u$ (o $v$) è possibile risalire a $v$ (o $u$) a meno di una costante, quindi una funzione olomorfa è completamente caratterizzata da $u$ o $v$.

\section{Serie di potenze}

Consideriamo una serie di potenze complessa
\begin{equation*}
f(z) = \sum\limits^{\infty}_{n = 0} a_{n}(z - z_{0})^{n} \ \ \ \ \{a_{n}\} \subset \CC, z\in \CC
\end{equation*}
In primis osserviamo che si può tralasciare $z_{0}$, in quanto si può semplicemente fare una traslazione. Consideriamo quindi una certa serie di potenze centrata in $0$ al variare di $z_{0}$
\begin{equation*}
f(z) = \sum\limits^{\infty}_{n = 0} a_{n} z^{n}_{0}
\end{equation*}
Ci sono delle interessanti proprietà di convergenza
\begin{enumerate}
\item Se converge in $z_{0} \neq 0$ allora converge all'interno del disco $\{| z| < | z_{0}| \}$, totalmente
\item Se diverge in $z_{0} \neq 0$ allora diverge all'esterno del disco $\{| z| > | z_{0}| \}$
\end{enumerate}

Mettendo insieme le due deduciamo che esiste un raggio $R$ di convergenza dentro al quale converge, fuori diverge e sul bordo \textit{nulla si può dire}.\footnote{Può anche convergere in certi punti del bordo e divergere in altri, sempre del bordo.}

\textit{Dimostrazione di}$(1)$\textit{.}

Se converge, allora il termine generale tende a zero
\begin{equation*}
a_{n} z^{n}_{0}\rightarrow 0
\end{equation*}
quindi è limitato
\begin{equation*}
\exists M : \ \ \left| a_{n} z^{n}_{0}\right| \leq M\ \ \forall n
\end{equation*}
Riscriviamo la serie in $z$ e mostriamo che converge totalmente (quindi anche semplicemente) all'interno
\begin{equation*}
\sum\limits^{\infty}_{n = 0}\left| a_{n} z^{n}\right| = \sum\limits^{\infty}_{n = 0}\left| a_{n} z^{n}\right| \cdot \frac{\left| z^{n}_{0}\right|}{\left| z^{n}_{0}\right|} = \sum\limits^{\infty}_{n = 0}\underbrace{\left| a_{n} z^{n}_{0}\right|}_{ \leq M} \cdot \frac{\left| z^{n}\right|}{\left| z^{n}_{0}\right|} \leq M\sum\limits^{\infty}_{n = 0}\left(\frac{| z|}{| z_{0}|}\right)^{n}
\end{equation*}
ma questa è una serie geometrica di ragione $\frac{| z|}{| z_{0}|} < 1$ per ipotesi, quindi converge, allora la serie converge totalmente.

\subsection{Esempi di serie}

\begin{itemize}
\item La serie
\begin{equation*}
\sum\limits^{\infty}_{n = 0} z^{n}
\end{equation*}

converge per $| z| < 1$, ma diverge sul bordo $| z| = 1$
\item La serie
\begin{equation*}
\sum\limits^{\infty}_{n = 0}\frac{z^{n}}{n}
\end{equation*}

diverge se $z = 1$, ma converge se $z = - 1$
\item La serie
\begin{equation*}
\sum\limits^{\infty}_{n = 0}\frac{z^{n}}{n^{2}}
\end{equation*}

diverge con $| z| > 1$, ma converge se $| z| \leq 1$
\end{itemize}
\begin{defn}
[Funzione Analitica]
Una funzione è \textbf{analitica} se si può scrivere localmente in serie di potenze. Se il raggio della serie è infinito allora si dice anche \textbf{intera}.
\end{defn}
\begin{rem}
Osserviamo che si possono estendere al campo complesso tutte le consuete serie di McLaurin
\begin{equation*}
\sin z = \sum\limits^{\infty}_{n = 0}\frac{(- 1)^{n} z^{2n + 1}}{(2n + 1) !} \ \ \ \ \cos z = \sum\limits^{\infty}_{n = 0}\frac{(- 1)^{n} z^{2n}}{(2n) !} \ \ \ \ e^{z} = \sum\limits^{\infty}_{n = 0}\frac{z^{n}}{n!}
\end{equation*}
\end{rem}
\begin{rem}
Ricordiamo anche la forma esponenziale di un numero complesso e la formula di Eulero
\begin{equation*}
e^{i\vartheta} = \cos \vartheta + i\sin \vartheta
\end{equation*}
\end{rem}
\begin{rem}
Consideriamo ora due serie, rispettivamente reale e complessa
\begin{equation*}
\frac{1}{1 + x^{2}} = \sum\limits^{\infty}_{n = 0}\left(- x^{2}\right)^{n} \ \ \ \ R = 1
\end{equation*}
il cui raggio di convergenza $R$ vale $1$
\begin{equation*}
\frac{1}{1 + z^{2}} = \sum\limits^{\infty}_{n = 0}\left(- z^{2}\right)^{n}
\end{equation*}
Notiamo qui che il denominatore si annulla per $z = \pm i$, quindi il disco di convergenza deve per forza avere raggio $1$, abbiamo quindi un'ulteriore giustificazione al perché la sua restrizione ai numeri reale non può convergere oltre $1$. Vedremo a breve che $\pm 1$ si chiameranno poli.
\end{rem}
\begin{thm}
Se una funzione è differenziabile, non è detto che esista automaticamente un'estensione olomorfa,\footnote{Per esempio $f(x) = | x| x$.} ma se esiste è unica.
\end{thm}

\section{Logaritmo complesso}

Notiamo che in $\CC$ si può risolvere $\sin z = 2$ con le formule di Eulero. Una peculiarità da notare è il modo in cui il logaritmo viene gestito, come conseguenza importante del fatto che
\begin{equation*}
e^{z} = e^{z + 2\pi i \cdot k}, \ \ k\in \ZZ
\end{equation*}
Consideriamo quindi
\begin{equation*}
e^{z} = w, \ \ w\neq 0, w\in \CC
\end{equation*}
allora
\begin{equation*}
e^{z} = w\ \ \implies \ \ e^{z + 2\pi i \cdot k} = | w| e^{i\arg w} \ \ \implies \ \ \log e^{z + 2\pi i \cdot k} = \log| w| e^{i\arg w}
\end{equation*}
da cui uguagliando le due parti otteniamo la formulazione del logaritmo complesso
\begin{equation*}
\boxed{z = \log_{\CC} w = \ln| w| + i\arg w + 2\pi i \cdot k, \ \ k\in \ZZ}
\end{equation*}
dove $\log_{\CC}$ indica il logaritmo complesso, mentre $\ln$ è il consueto logaritmo.

Non è una vera e propria funzione, perché ad ogni elemento del dominio $\CC \setminus \{0\}$ non associa uno e un solo valore, ma ne associa infiniti: è detta \textbf{funzione polidroma}.

\subsection{Superfici di Riemann}

Consideriamo ora un certo $w\in \CC$ di modulo $1$, quindi $\ln| w| = 0$. Partiamo da $\vartheta \sim - \pi $ e prendiamo $k = 0$
\begin{equation*}
z = \log_{\CC} w = i\vartheta = - i\pi
\end{equation*}
se aumentiamo $\vartheta $ fino ad arrivare a $\vartheta \sim \pi $ abbiamo
\begin{equation*}
z = \log_{\CC} w = i\vartheta = i\pi
\end{equation*}
I due valori si dovrebbero raccordare con continuità essendo due argomenti corrispondenti allo stesso angolo nel piano di Gauss, ma distano di $2\pi $. Questo salto si visualizza nelle cosiddette \textbf{Superfici di Riemann}\footnote{Il colore rappresenta l'argomento.\\Fonte dell'immagine: \url{https://commons.wikimedia.org/wiki/File:Riemann\_surface\_log.svg}}

\fg{0.6}{superfici-di-Riemann}

\section{Curve in \texorpdfstring{$\CC$}{C}}

Valgono le consuete definizioni date in Analisi $2$ per una curva $r: [a, b]\rightarrow \CC$
\begin{itemize}
\item \textbf{chiusa} se $r(a) = r(b)$
\item \textbf{semplice} se $r$ è iniettiva
\item \textbf{regolare} se $r\in C^{1}([a, b])$
\item il \textbf{sostegno} è $r([a, b])$
\end{itemize}

L'integrale su una curva è dato da
\begin{equation*}
\int_{\gamma} f(z) dz = \int^{b}_{a} f(\gamma (t)) \gamma'(t) dt
\end{equation*}
valgono le proprietà di linearità, cambio verso, e le consuete maggiorazioni.

\section{Teorema dell'Integrale Nullo di Cauchy}

\begin{thm}
[dell'Integrale Nullo di Cauchy] Sia $A\subset \CC$ semplicemente connesso e $f\in H(A)$, allora
\begin{equation*}
\int_{\gamma} f(z) dz = 0, \ \ \forall \gamma \ \text{chiusa}, \gamma \subset A
\end{equation*}
\end{thm}
\begin{proof}
Semplificata al caso in cui $f\in C^{1}$.
\begin{equation*}
\int_{\gamma} f(z) dz = \int_{\gamma}(u + iv)(dx + idy) = \int_{\gamma}(udx - vdy) + i\int_{\gamma}(udy + vdx) = 0
\end{equation*}
in quanto per le condizioni di (CR) quelle sono due forme differenziali esatte.
\end{proof}

\subsection{Estensione alla frontiera}

La tesi vale anche se chiediamo che $f$ sia continua fino al bordo e non che $A$ sia semplicemente connesso.
\begin{equation*}
f\in H(A) \cap C^{0}(\overline{A}) \ \ \implies \ \ \int_{\gamma} f(z) dz = 0, \ \ \gamma = \partial A
\end{equation*}

\subsection{Estensione a regioni a contorno multiplo}

Se ci sono \textit{buchi} nell'insieme $A$ si può semplicemente girarci intorno e tenere conto di tali contributi, mentre i contributi delle linee rette si cancellano
\begin{equation*}
f\in H\left(A\setminus \bigcup\nolimits_{i} A_{i}\right) \cap C^{0}\left(\overline{A\setminus \bigcup\nolimits_{i} A_{i}}\right) \ \ \implies \ \ \int_{\partial A} f(z) dz + \sum\nolimits_{i}\int_{\partial A_{i}} f(z) dz = 0
\end{equation*}

\section{Formula Integrale di Cauchy}

\begin{thm}
Se $f\in H(A) \cap C^{0}(\overline{A})$ e $A$ semplicemente connesso, allora $\forall z\in A$
\begin{equation*}
\boxed{f(z) = \frac{1}{2\pi i}\int_{\gamma}\frac{f(w)}{w - z} dw, \ \ \gamma = \partial A}
\end{equation*}
\end{thm}
\begin{proof}

Poniamo $w = z + re^{i\vartheta}, r > 0, \vartheta \in [0, 2\pi]$, da cui $dw = rie^{i\vartheta} d\vartheta $, calcoliamo l'espressione
\begin{equation*}
\int_{\gamma}\frac{f(w)}{w - z} dw = \int^{2\pi}_{0}\frac{f\left(z + re^{i\vartheta}\right)}{\cancel{re^{i\vartheta}}}\cancel{re^{i\vartheta}} id\vartheta = i\int^{2\pi}_{0}[f(z) + o(r)] d\vartheta
\end{equation*}
ma per continuità possiamo far tendere $r\rightarrow 0$
\begin{gather*}
\int_{\gamma}\frac{f(w)}{w - z} dw\xrightarrow{r\rightarrow 0} if(z)\int^{2\pi}_{0} d\vartheta = 2\pi i \cdot f(z)
\end{gather*}
\end{proof}

\section{Esistenza di primitive}

Non è sufficiente la continuità.
\begin{thm}
Sia $f\in C^{0}(A)$, $A$ aperto. Allora
\begin{equation*}
f\ \text{ammette primitiva} \ \ \iff \ \ udx - vdy \quad \text{e} \quad udy + vdx \quad \text{sono esatte}
\end{equation*}
\end{thm}
\begin{proof}\leavevmode
\begin{itemize}
\item $(\implies)$ consideriamo $\gamma $ chiusa
\begin{equation*}
\int_{\gamma} f(z) dz = \int_{\gamma} F'(z) dz = \int^{b}_{a} F'(\gamma (t)) \gamma'(t) dt = F(\gamma (b)) - F(\gamma (a)) = 0
\end{equation*}
\item $(\impliedby)$ Consideriamo una certa $F$ definita da
\begin{equation*}
F(z) = \int^{z}_{z_{0}} f(w) dw
\end{equation*}

da cui
\begin{equation*}
F(z + h) - F(z) = \int^{z + h}_{z_{0}} f(w) dw - \int^{z}_{z_{0}} f(w) dw = \int^{z + h}_{z} f(w) dw = (\star)
\end{equation*}

parametrizziamo il segmento $w = z + th, t\in [0, 1]$ da cui anche $dw = hdt$
\begin{equation*}
(\star) = \int^{1}_{0} f(z + th) hdt
\end{equation*}

possiamo passare al limite
\begin{gather*}
\lim\limits_{h\rightarrow 0}\frac{F(z + h) - F(z)}{h} = \lim\limits_{h\rightarrow 0}\int^{1}_{0} f(z + th) dt = \int^{1}_{0} f(z) dt = f(z)\int^{1}_{0} dt = f(z)
\end{gather*}

\end{itemize}
\end{proof}

\section{Serie di potenze}

\begin{thm}
[di Weierstrass]
Sia $f\in H(A)$, $A$ aperto. Allora $f$ è analitica, ovvero si può scrivere localmente in serie di potenze
\begin{equation*}
f(z) = \sum^{\infty}_{n = 0} a_{n}(z - z_{0})^{n}, \ \ \ \ a_{n} = \frac{1}{2\pi i}\int_{\gamma}\frac{f(w)}{(w - z_{0})^{n + 1}} dw
\end{equation*}
\end{thm}

\begin{proof}

Per la formula di Cauchy
\begin{equation*}
f(z) = \frac{1}{2\pi i}\int_{\gamma}\frac{f(w)}{w - z} dw
\end{equation*}
Osserviamo che
\begin{equation*}
\frac{1}{w - z} = \frac{1}{w - z_{0} + z_{0} - z} = \frac{1}{w - z_{0}}\frac{1}{1 - \frac{z - z_{0}}{w - z_{0}}} = \frac{1}{w - z_{0}}\sum^{\infty}_{n = 0}\frac{(z - z_{0})^{n}}{(w - z_{0})^{n}}
\end{equation*}
fintanto che ci mettiamo in una regione in cui
\begin{equation*}
\left| \frac{z - z_{0}}{w - z_{0}}\right| < 1\ \ \iff \ \ | z - z_{0}| < | w - z_{0}|
\end{equation*}

\begin{figure}[htpb]
\centering
\tikzset{every picture/.style = {line width = 0.75pt}} %set default line width to 0.75pt

\begin{tikzpicture}[x = 0.75pt, y = 0.75pt, yscale = - 1, xscale = 1]
%uncomment if require: \path (0, 136); %set diagram left start at 0, and has height of 136

%Shape: Ellipse [id: dp7479505317469046]
\draw (159, 71.02) .. controls (159, 40.1) and (222.8, 15.03) .. (301.5, 15.03) .. controls (380.2, 15.03) and (444, 40.1) .. (444, 71.02) .. controls (444, 101.93) and (380.2, 127) .. (301.5, 127) .. controls (222.8, 127) and (159, 101.93) .. (159, 71.02) - - cycle;
%Shape: Circle [id: dp8263143669411064]
\draw (211, 70.5) .. controls (211, 50.89) and (226.89, 35) .. (246.5, 35) .. controls (266.11, 35) and (282, 50.89) .. (282, 70.5) .. controls (282, 90.11) and (266.11, 106) .. (246.5, 106) .. controls (226.89, 106) and (211, 90.11) .. (211, 70.5) - - cycle;
%Shape: Circle [id: dp22698676265510276]
\draw [draw opacity = 0][fill = {rgb, 255: red, 0; green, 0; blue, 0}, fill opacity = 1] (244.5, 70.5) .. controls (244.5, 69.4) and (245.4, 68.5) .. (246.5, 68.5) .. controls (247.6, 68.5) and (248.5, 69.4) .. (248.5, 70.5) .. controls (248.5, 71.6) and (247.6, 72.5) .. (246.5, 72.5) .. controls (245.4, 72.5) and (244.5, 71.6) .. (244.5, 70.5) - - cycle;
%Shape: Circle [id: dp17321185166040398]
\draw [draw opacity = 0][fill = {rgb, 255: red, 0; green, 0; blue, 0}, fill opacity = 1] (233.3, 89.7) .. controls (233.3, 88.6) and (234.2, 87.7) .. (235.3, 87.7) .. controls (236.4, 87.7) and (237.3, 88.6) .. (237.3, 89.7) .. controls (237.3, 90.8) and (236.4, 91.7) .. (235.3, 91.7) .. controls (234.2, 91.7) and (233.3, 90.8) .. (233.3, 89.7) - - cycle;
%Shape: Circle [id: dp9508082191122853]
\draw [draw opacity = 0][fill = {rgb, 255: red, 0; green, 0; blue, 0}, fill opacity = 1] (209, 70.5) .. controls (209, 69.4) and (209.9, 68.5) .. (211, 68.5) .. controls (212.1, 68.5) and (213, 69.4) .. (213, 70.5) .. controls (213, 71.6) and (212.1, 72.5) .. (211, 72.5) .. controls (209.9, 72.5) and (209, 71.6) .. (209, 70.5) - - cycle;

% Text Node
\draw (190.4, 59.6) node [anchor = north west][inner sep = 0.75pt] {$w$};
% Text Node
\draw (240.8, 80.8) node [anchor = north west][inner sep = 0.75pt] {$z$};
% Text Node
\draw (246.8, 50.4) node [anchor = north west][inner sep = 0.75pt] {$z_{0}$};
% Text Node
\draw (388, 5.2) node [anchor = north west][inner sep = 0.75pt] {$A$};
% Text Node
\draw (286, 64) node [anchor = north west][inner sep = 0.75pt] {$\gamma $};

\end{tikzpicture}
\end{figure}
\FloatBarrier

Sostituiamo nella formula di Cauchy
\begin{align*}
f(z) & = \frac{1}{2\pi i}\int_{\gamma} f(w)\sum^{\infty}_{n = 0}\frac{(z - z_{0})^{n}}{(w - z_{0})^{n + 1}} dw\\
 & = \sum^{\infty}_{n = 0}(z - z_{0})^{n}\underbrace{\frac{1}{2\pi i}\int_{\gamma}\frac{f(w)}{(w - z_{0})^{n + 1}} dw}_{a_{n}} = \sum^{\infty}_{n = 0} a_{n}(z - z_{0})^{n}
\end{align*}
Si noti che la regione considerata, come si vede in figura, è quindi una palla centrata in $z_{0}$.

Si ottiene anche la \textbf{formula per derivate}
\begin{gather*}
f^{(n)}(z_{0}) = \frac{n!}{2\pi i}\int_{\gamma}\frac{f(w)}{(w - z_{0})^{n + 1}} dw
\end{gather*}
\end{proof}

\section{Serie di Laurent}

Vi sono dei punti particolari in certe funzioni dove non c'è convergenza della serie, ad esempio in $f(z) = \frac{1}{1 - z}$ il punto $1$ è critico, e la serie geometrica associata non converge in tale punto (e di conseguenza nel disco). Queste "ostruzioni" sono di vario tipo e vengono opportunamente classificate.

Per risolvere il problema si introduce la Serie di Laurent, una nuova espansione su una corona circolare $A$ (quindi \textit{non} semplicemente connessa) centrata nel punto stesso che compromette l'espansione in serie di Taylor. Il teorema inoltre non richiede che la funzione $f$ sia olomorfa su tutto il dominio ma esclusivamente sulla corona circolare $A$.
\begin{thm}
[Serie di Laurent]
Sia $f\in H(A)$, $A = \{z\in \CC : r < | z - z_{0}| < R\}$, $\gamma $ una curva regolare semplice chiusa in $A$ che concateni $z_{0}$. Allora
\begin{equation*}
\boxed{\forall z\in A, \ \ f(z) = \sum\limits^{\infty}_{n = - \infty} a_{n}(z - z_{0})^{n} \ \ \ \ a_{n} = \frac{1}{2\pi i}\int_{\gamma}\frac{f(w)}{(w - z_{0})^{n + 1}} dw}
\end{equation*}
\end{thm}
Si noti che la curva è \textit{qualsiasi} purché concatenata con $z_{0}$; ciò deriva dal fatto che la funzione è olomorfa nel dominio dove si esegue l'integrazione, e pertanto l'integrale non varia per deformazione continua.

Il teorema di Laurent è pertanto una generalizzazione del teorema di Weierstrass, in cui si include anche una somma sui \textbf{termini negativi}. Sono proprio le potenze negative, infatti, a dare informazioni sul comportamento delle singolarità.

\begin{proof}

Tutto si basa sul \textit{fare un giro attorno al punto}

\begin{figure}[htpb]
\centering
\tikzset{every picture/.style = {line width = 0.75pt}} %set default line width to 0.75pt

\begin{tikzpicture}[x = 0.75pt, y = 0.75pt, yscale = - 1, xscale = 1]
%uncomment if require: \path (0, 184); %set diagram left start at 0, and has height of 184

%Shape: Arc [id: dp509954749685253]
\draw [draw opacity = 0] (181.56, 83.63) .. controls (186.19, 44.97) and (219.09, 15) .. (259, 15) .. controls (302.08, 15) and (337, 49.92) .. (337, 93) .. controls (337, 136.08) and (302.08, 171) .. (259, 171) .. controls (219.52, 171) and (186.89, 141.66) .. (181.71, 103.6) - - (259, 93) - - cycle; \draw (181.56, 83.63) .. controls (186.19, 44.97) and (219.09, 15) .. (259, 15) .. controls (302.08, 15) and (337, 49.92) .. (337, 93) .. controls (337, 136.08) and (302.08, 171) .. (259, 171) .. controls (219.52, 171) and (186.89, 141.66) .. (181.71, 103.6);
%Shape: Arc [id: dp39536995790567286]
\draw [draw opacity = 0] (237.74, 83.57) .. controls (241.36, 75.43) and (249.52, 69.75) .. (259, 69.75) .. controls (271.84, 69.75) and (282.25, 80.16) .. (282.25, 93) .. controls (282.25, 105.84) and (271.84, 116.25) .. (259, 116.25) .. controls (249.77, 116.25) and (241.79, 110.87) .. (238.04, 103.07) - - (259, 93) - - cycle; \draw (237.74, 83.57) .. controls (241.36, 75.43) and (249.52, 69.75) .. (259, 69.75) .. controls (271.84, 69.75) and (282.25, 80.16) .. (282.25, 93) .. controls (282.25, 105.84) and (271.84, 116.25) .. (259, 116.25) .. controls (249.77, 116.25) and (241.79, 110.87) .. (238.04, 103.07);
%Straight Lines [id: da39785143879926865]
\draw (181.71, 103.6) - - (239, 103.6);
\draw [shift = {(210.36, 103.6)}, rotate = 0] [fill = {rgb, 255: red, 0; green, 0; blue, 0} ][line width = 0.08] [draw opacity = 0] (12, - 3) - - (0, 0) - - (12, 3) - - cycle;
%Straight Lines [id: da7721203931176552]
\draw (237.74, 83.63) - - (181.56, 83.63);
\draw [shift = {(209.65, 83.63)}, rotate = 180] [fill = {rgb, 255: red, 0; green, 0; blue, 0} ][line width = 0.08] [draw opacity = 0] (12, - 3) - - (0, 0) - - (12, 3) - - cycle;
\draw [draw opacity = 0][fill = {rgb, 255: red, 0; green, 0; blue, 0}, fill opacity = 1] (254.5, 166) - - (264.5, 171) - - (254.5, 176) - - (259.5, 171) - - cycle;
\draw [draw opacity = 0][fill = {rgb, 255: red, 0; green, 0; blue, 0}, fill opacity = 1] (262, 20) - - (252, 15) - - (262, 10) - - (257, 15) - - cycle;
%Shape: Circle [id: dp12847517277803466]
\draw [draw opacity = 0][fill = {rgb, 255: red, 0; green, 0; blue, 0}, fill opacity = 1] (257, 93) .. controls (257, 91.9) and (257.9, 91) .. (259, 91) .. controls (260.1, 91) and (261, 91.9) .. (261, 93) .. controls (261, 94.1) and (260.1, 95) .. (259, 95) .. controls (257.9, 95) and (257, 94.1) .. (257, 93) - - cycle;

% Text Node
\draw (253, 71.9) node [anchor = north west][inner sep = 0.75pt] {$z_{0}$};
% Text Node
\draw (266.5, 114.9) node [anchor = north west][inner sep = 0.75pt] {$r$};
% Text Node
\draw (304, 157.9) node [anchor = north west][inner sep = 0.75pt] {$R$};
% Text Node
\draw (323.5, 25.4) node [anchor = north west][inner sep = 0.75pt] {$\gamma_{1}$};
% Text Node
\draw (281, 62.9) node [anchor = north west][inner sep = 0.75pt] {$\gamma_{2}$};
% Text Node
\draw (372, 81.4) node [anchor = north west][inner sep = 0.75pt] {$\gamma = \gamma_{1} \cup \gamma_{2}$};

\end{tikzpicture}

\end{figure}
\FloatBarrier

L'insieme \textit{pacman} è effettivamente semplicemente connesso, e lì la funzione è olomorfa, allora
\begin{equation*}
\int_{\gamma} f(z) dz = 0 = \int_{\gamma_{1}} f(z) dz - \int_{\gamma_{2}} f(z) dz
\end{equation*}
i termini paralleli si cancellano. Consideriamo un $z$ nella corona $A$
\begin{equation*}
f(z) = \frac{1}{2\pi i}\int_{\gamma}\frac{f(w)}{w - z} dw = \underbrace{\frac{1}{2\pi i}\int_{\gamma_{1}}\frac{f(w)}{w - z} dw}_{A}\underbrace{- \frac{1}{2\pi i}\int_{\gamma_{2}}\frac{f(w)}{w - z} dw}_{B}
\end{equation*}
Analizziamo i termini.

Termine $A$
\begin{gather*}
\frac{1}{w - z} = \frac{1}{w - z_{0} - (z - z_{0})} = \frac{1}{w - z_{0}}\frac{1}{1 - \frac{z - z_{0}}{w - z_{0}}} = \sum\limits^{\infty}_{n = 0}\frac{(z - z_{0})^{n}}{(w - z_{0})^{n + 1}}\\
\implies \ \ A = \frac{1}{2\pi i}\int_{\gamma_{1}} f(w)\sum\limits^{\infty}_{n = 0}\frac{(z - z_{0})^{n}}{(w - z_{0})^{n + 1}} dw = \sum\limits^{\infty}_{n = 0} a_{n}(z - z_{0})^{n}
\end{gather*}
Termine $B$, quello \textit{sospetto} che non c'era nel caso di Weierstrass
\begin{align*}
\frac{- 1}{w - z} & = \frac{- 1}{w - z_{0} - (z - z_{0})} = \frac{1}{z - z_{0}}\frac{1}{1 - \frac{w - z_{0}}{z - z_{0}}} = \sum\limits^{\infty}_{n = 0}\frac{(w - z_{0})^{n}}{(z - z_{0})^{n + 1}}\\
 & = \sum\limits^{\infty}_{n = 1}\frac{(w - z_{0})^{n - 1}}{(z - z_{0})^{n}} = \sum\limits^{- \infty}_{n = - 1}\frac{(z - z_{0})^{n}}{(w - z_{0})^{n + 1}}
\end{align*}
Sostituendo tale quantità in $B$ si ottiene la tesi.
\end{proof}
\textit{Esempio.}
\begin{equation*}
f(z) = \frac{\sin z}{z} = \frac{1}{\cancel{z}}\sum\limits^{\infty}_{n = 0}\frac{(- 1)^{n} z^{2n\cancel{+ 1}}}{(2n + 1) !} = \sum\limits^{\infty}_{n = 0}\frac{(- 1)^{n} z^{2n}}{(2n + 1) !}
\end{equation*}
Abbiamo potuto eliminare la singolarità!

\subsection{Classificazione delle singolarità}

\begin{defn}
[Punto singolare]
Un punto $z_{0}$ si dice \textbf{singolare} (o critico) per una funzione $f(z)$ se non può essere incluso in nessun disco di convergenza della serie di potenze.
\end{defn}
Classifichiamo i punti singolari in base ai coefficienti $a_{n} \neq 0, n < 0$
\begin{itemize}
\item se non ce n'è \textit{nessuno:} \textbf{singolarità eliminabile} (esiste un prolungamento continuo)
\item se sono \textit{finiti:} \textbf{polo}
\begin{itemize}
\item definiamo inoltre l'\textbf{ordine} di un polo come l'indice $n > 0$ dei coefficienti $a_{- n} \neq 0$ tale che $a_{m} = 0, \forall m < - n$. Ovvero stiamo prendendo il più piccolo coefficiente non nullo.
\end{itemize}
\item se sono \textit{infiniti:} \textbf{singolarità essenziale}
\end{itemize}

\textit{Esempio.}
\begin{equation*}
z_{0} = 1, \ \ f(z) = \frac{1}{z - 1} = \sum\limits^{\infty}_{n = - \infty} a_{n}(z - 1)^{n} \ \ \ \ a_{n} =
\begin{cases}
0, & n\neq - 1\\
1, & n = - 1
\end{cases}
\end{equation*}
in questo caso $z_{0}$ è un polo.

\textit{Esempio.}
\begin{equation*}
z_{0} = 0, \ \ f(z) = \sin\left(\frac{1}{z}\right) = \sum\limits^{\infty}_{n = 0}\frac{\left(\frac{1}{z}\right)^{2n + 1}(- 1)^{n}}{(2n + 1) !} = \sum\limits^{- \infty}_{n = 0}\frac{z^{2n - 1}(- 1)^{n}}{(1 - 2n) !}
\end{equation*}
in questo caso $z_{0}$ è una singolarità essenziale.
\begin{defn}
[Punto singolare isolato]
Un punto singolare $z_{0}$ per $f(z)$ si dice \textbf{isolato} esiste un intorno che non contiene punti critici per $f$.
\end{defn}
In $\CC$ non ha senso parlare di $\pm \infty $, essendoci infinite direzioni, tuttavia è utile considerare tutti i punti infinitamente distanti \textit{come un unico punto all'infinito}, più nello specifico vale la seguente definizione.
\begin{defn}
[Punto singolare all'infinito] Il punto all'infinito $z_{\infty}$ è singolare per $f(z)$ se $z_{0} = 0$ è singolare per $f\left(\frac{1}{z}\right)$.
\end{defn}
\textit{Esempio.}

La funzione $f(z) = e^{z}$ ha singolarità isolata all'infinito, mentre $g(z) = \frac{1}{\sin z}$ ha punto all'infinito non isolato.

\begin{thm}
[Caratterizzazione dei poli]
Sia $z_{0} \neq \infty $ un punto singolare isolato per $f$. Allora le seguenti affermazioni sono equivalenti:
\begin{enumerate}
\item $f$ ha un polo di ordine $n$ in $z_{0}$;
\item $g(z) = (z - z_{0})^{n} f(z)$ ha una singolarità eliminabile in $z_{0}$ e inoltre $\lim_{z\rightarrow z_{0}} g(z)\neq 0$;
\item è rispettata la seguente relazione tra ordini di grandezza:

\begin{equation*}
|f(z)|\asymp \frac{1}{| z - z_{0}|^{n}}
\end{equation*}
\item $\varphi (z) = \frac{1}{f(z)}$ ha in $z_{0}$ uno zero di ordine $n$.
\end{enumerate}
\end{thm}

\begin{thm}
[Caratterizzazione delle singolarità] Valgono le seguenti affermazioni, con $z_{0}$ punto singolare isolato per $f$ :
\begin{enumerate}
\item $\lim_{z\rightarrow z_{0}} f(z) = \infty \iff z_{0}$ è un polo.
\item $\lim_{z\rightarrow z_{0}} f(z)$ esiste finito $\iff z_{0}$ è una singolarità eliminabile.
\item $\lim_{z\rightarrow z_{0}} f(z)$ non esiste $\iff z_{0}$ è una singolarità essenziale.
\end{enumerate}
\end{thm}
\begin{thm}
[Caratterizzazione delle singolarità essenziali] Sia $z_{0}$ una singolarità essenziale isolata per $f(z)$. Allora la funzione $\frac{1}{f(z)}$ ha in $z_{0}$ una singolarità essenziale oppure un punto di accumulazione di poli.
\end{thm}
In generale si può sempre studiare la serie di Laurent centrata nel punto, ma ciò può risultare complicato e lungo da fare, specialmente se le singolarità sono molte.

\section{Teorema dei residui}

Il coefficiente $a_{- 1}$ della serie di Laurent centrata in $z_{0}$ è molto importante e prende il nome di \textbf{residuo integrale}
\begin{equation*}
a_{n} = \frac{1}{2\pi i}\int_{\gamma}\frac{f(w)}{(w - z_{0})^{n + 1}} dw\ \ \implies \ \ \Res (f, z_{0}) = a_{- 1} = \frac{1}{2\pi i}\int_{\gamma} f(w) dw
\end{equation*}
Il seguente teorema semplifica tantissimi calcoli di integrali, anche reali.
\begin{thm}
[dei Residui]
Sia $A\subset \CC$ aperto, $f\in H(A\setminus \{z_{1}, \dotsc, z_{n}\} \cap C^{0}(\overline{A} \setminus \{z_{1}, \dotsc, z_{n}\})$. Allora
\begin{equation*}
\boxed{\int_{\gamma} f(z) dz = 2\pi i \cdot \sum\limits^{n}_{j = 1}\Res (f, z_{j})}
\end{equation*}
\end{thm}
\begin{defn}
Se consideriamo una $\gamma $ chiusa che contiene tutte le singolarità di $f(z)$ al finito, e $f\in H(| z| > R)$, se consideriamo lo sviluppo di $f$ in un intorno dell'infinito, allora definiamo il \textbf{residuo all'infinito}
\begin{equation*}
\Res (f, \infty) : = - a_{- 1} = \frac{1}{2\pi i}\int_{- \gamma} f(w) dw
\end{equation*}
\end{defn}
\begin{thm}
Se abbiamo un numero finito di singolarità, la somma di tutte, sia al finito che all'infinito, vale zero.
\begin{equation*}
\sum\limits^{n}_{j = 1}\Res (f, z_{j}) + \Res (f, \infty) = 0
\end{equation*}
\end{thm}

\subsection{Come calcolare i residui}

Se $z_{0} \neq \infty $
\begin{itemize}
\item se è eliminabile allora $\Res (f, z_{0}) = 0$
\item se è polo di ordine $n$ allora $\Res (f, z_{0}) = \frac{1}{(n - 1)!}\lim\limits_{z\rightarrow z_{0}}\frac{d^{n - 1}}{dz^{n - 1}}(z - z_0)^n f(z)$
\item se è polo di ordine $1$ allora $\Res \left(\frac{f}{g}, z_{0}\right) = \lim\limits_{z\rightarrow z_{0}}\frac{f(z)}{g'(z)}$
\item vale De l'Hôpital $\lim\limits_{z\rightarrow z_{0}}\frac{f(z)}{g(z)} = \lim\limits_{z\rightarrow z_{0}}\frac{f'(z)}{g'(z)}$
\end{itemize}

Se $z_{0} = \infty $
\begin{itemize}
\item $\Res (f, \infty) = \Res \left(- \frac{1}{z^{2}} f\left(\frac{1}{z}\right), 0\right)$
\item se è zero di ordine $1$ allora $\Res (f, \infty) = - \lim\limits_{z\rightarrow \infty} zf(z)$
\item se è zero di ordine $ \geq 2$ allora $\Res (f, \infty) = 0$
\end{itemize}

\section{Lemmi di Jordan}

Consideriamo un generico arco di circonferenza:
\begin{equation*}
C_{R}(\vartheta_{1}, \vartheta_{2}) = \left\{z\in \CC : z = Re^{i\vartheta}, \vartheta \in [\vartheta_{1}, \vartheta_{2}]\right\}
\end{equation*}
\begin{thm}
Sia $f(z)$ continua su $C_{R}$. Allora:
\begin{equation*}
\left| \int_{C_{R}} f(z)dz\right| \leq R(\vartheta_{2} - \vartheta_{1})\max_{|z\in C_{R}}| f(z)|
\end{equation*}
dove $R(\vartheta_{2} - \vartheta_{1})$ non è altro che la lunghezza dell'arco.
\end{thm}
\begin{thm}
[Lemma di Jordan al cerchio grande]
Sia $K > 0$ e sia $f(z)$ continua per $|z| > K$. Se $\exists \alpha > 1, c > 0$ tali che $|f(z)| \leq \frac{c}{|z|^{\alpha}}, \ \forall z: |z| > K, $ allora:
\begin{equation*}
\lim_{R\rightarrow \infty}\int_{C_{R}(\vartheta_{1}, \vartheta_{2})} f(z)dz = 0\ \ \forall \vartheta_{1}, \vartheta_{2}
\end{equation*}
\end{thm}
\begin{proof}
\begin{equation*}
\left| \int f(z) dz\right| \leq R(\vartheta_{2} - \vartheta_{1})\frac{c}{|z|^{\alpha}} = R(\vartheta_{2} - \vartheta_{1})\frac{c}{R^{\alpha}}\xrightarrow[\alpha > 1]{R\rightarrow \infty} 0
\end{equation*}
\end{proof}
\begin{thm}
[Lemma di Jordan al cerchio piccolo]
Con analoghe ipotesi ma con $\alpha < 1$
\begin{equation*}
\lim_{R\rightarrow 0}\int_{C_{R}(\vartheta_{1}, \vartheta_{2})} f(z)dz = 0\ \ \forall \vartheta_{1}, \vartheta_{2}
\end{equation*}
\end{thm}
Tuttavia $f$ può andare a zero anche più lentamente per ottenere dei risultati interessanti
\begin{thm}
[Lemmi di Jordan]
Sia $f(z)$ continua per $|z| > K.$ Siano inoltre $\vartheta_{1}, \vartheta_{2} \in [0, 2\pi]$ e $a > 0, $tali che:
\begin{equation*}
\lim_{R\rightarrow \infty}\sup_{z\in C_{R}}| f(z)| = 0
\end{equation*}
Allora:
\begin{enumerate}
\item se $\vartheta_{1} = 0$ e $\vartheta_{2} = \pi $

\begin{equation*}
\lim_{R\rightarrow \infty}\int_{C_{R} (0, \pi)} e^{iaz} f(z)dz = 0
\end{equation*}
\item se $\vartheta_{1} = \pi $ e $\vartheta_{2} = 2\pi $,

\begin{equation*}
\lim_{R\rightarrow \infty}\int_{C_{R} (\pi, 2\pi)} e^{- iaz} f(z)dz = 0
\end{equation*}
\item se $\vartheta_{1} = - \frac{\pi}{2}$ e $\vartheta_{2} = \frac{\pi}{2}$,

\begin{equation*}
\lim_{R\rightarrow \infty}\int_{C_{R}\left(- \frac{\pi}{2}, \frac{\pi}{2}\right)} e^{- az} f(z)dz = 0
\end{equation*}
\item se $\vartheta_{1} = \frac{\pi}{2}$ e $\vartheta_{2} = \frac{3}{2} \pi $,

\begin{equation*}
\lim_{R\rightarrow \infty}\int_{C_{R}\left(\frac{\pi}{2}, \frac{3}{2} \pi \right)} e^{az} f(z)dz = 0
\end{equation*}
\end{enumerate}
\end{thm}
\begin{proof}[Dimostrazione di $(1)$.]

Sia $z = Re^{i\vartheta}$, $\vartheta \in [0, \pi]$, sostituiamo
\begin{equation*}
\int_{C_{R} (0, \pi)} e^{iaz} f(z)dz = \int^{\pi}_{0} e^{iaRe^{i\vartheta}} f\left(Re^{i\vartheta}\right) iRe^{i\vartheta} d\vartheta
\end{equation*}
Poniamo $M = \sup_{z\in C_{R}}| f(z)| $, per ipotesi $\lim_{R\rightarrow \infty} M = 0$. Notiamo che
\begin{equation*}
e^{iaRe^{i\vartheta}} = e^{iaR[\cos \vartheta + i\sin \vartheta]} = \underbrace{e^{iaR\cos \vartheta}}_{\substack{\text{modulo}\\\text{unitario}}} e^{- aR\sin \vartheta}
\end{equation*}
Maggioriamo
\begin{equation*}
\left| \int_{C_{R} (0, \pi)} e^{iaz} f(z)dz\right| \leq M\int^{\pi}_{0}\left| e^{iaRe^{i\vartheta}} iRe^{i\vartheta}\right| d\vartheta = 2MR\int^{\pi /2}_{0} e^{- aR\sin \vartheta} d\vartheta = (\star)
\end{equation*}
a questo punto possiamo minorare il seno con la retta che passa per $0$ e $(\pi /2, 1)$
\begin{equation*}
\sin \vartheta \geq \frac{2\vartheta}{\pi}
\end{equation*}
Di conseguenza
\begin{equation*}
(\star) \leq 2MR\int^{\pi /2}_{0} e^{- aR\frac{2\vartheta}{\pi}} d\vartheta = \cancel{2} M\cancel{R}\left[\left(- \frac{\pi}{\cancel{2} a\cancel{R}}\right) e^{- aR\frac{2\vartheta}{\pi}}\right]^{\pi /2}_{0} = - \frac{\pi M}{a}\underbrace{\left[e^{- aR} - 1\right]}_{\in [ - 1, 0]} \leq \frac{\pi M}{a}
\end{equation*}
Per ipotesi $M\rightarrow 0$ e $a > 0$, da cui la tesi.
\end{proof}

\chapter{Spazi di Banach e Spazi di Hilbert}

\begin{defn}
[Norma]
Sia $X$ uno spazio vettoriale. Si dice \textbf{norma} un'applicazione $ \Vert \cdot \Vert : X\rightarrow [0, + \infty)$ che verifica, $\forall \lambda \in \RR$, $\forall x, y\in X$
\begin{itemize}
\item $ \Vert x \Vert = 0\iff x = 0$
\item $ \Vert \lambda x \Vert = | \lambda | \Vert x \Vert $
\item $ \Vert x + y \Vert \leq \Vert x \Vert + \Vert y \Vert $
\end{itemize}
\end{defn}

\textit{Esempio.}

Dato $1 \leq p < + \infty $, l'applicazione $ \Vert \cdot \Vert : \RR^{n}\rightarrow [0, + \infty)$ definita da
\begin{equation*}
\Vert x \Vert_{p} = \left(| x_{1}|^{p} + \dotsc + | x_{n}|^{p}\right)^{1/p}
\end{equation*}
è la norma $p$.

\textit{Esempio.}

L'applicazione $ \Vert \cdot \Vert : \RR^{n}\rightarrow [0, + \infty)$ definita da
\begin{equation*}
\Vert x \Vert_{\infty} = \max\{| x_{1}|, \dotsc, | x_{n}| \}
\end{equation*}
è la norma infinito.
\begin{defn}
[Spazio normato]
Uno spazio vettoriale dotato di norma si dice \textbf{spazio normato}.
\end{defn}
\begin{defn}
[Metrica]
La funzione $d: X\times X\rightarrow [0, + \infty)$ definita da $(u, v) = \Vert u - v \Vert $ è detta \textbf{metrica} indotta dalla norma, mentre la quantità $ \Vert u - v \Vert $ è detta \textbf{distanza}.
\end{defn}
\begin{defn}
[Spazio metrico]
Uno spazio vettoriale dotato di metrica si dice \textbf{spazio metrico}, quindi uno spazio normato è anche metrico.
\end{defn}
\begin{defn}
[Successione di Cauchy]
Sia $V$ uno spazio normato. Una successione $x_{n}$ è di Cauchy se $\forall \varepsilon > 0$, $\exists N$ tale che $\forall n, m \geq N$ si ha $ \Vert x_{n} - x_{m} \Vert < \varepsilon $. Ogni successione convergente è di Cauchy.
\end{defn}
\begin{defn}
[Spazio completo]
Uno spazio metrico si dice \textbf{completo} se ogni successione di Cauchy è convergente.
\end{defn}
\begin{defn}
[Spazio di Banach]
Uno spazio normato si dice di Banach se è completo.
\end{defn}
\begin{defn}
[Prodotto scalare]
Il prodotto scalare è un'applicazione $( \cdot, \cdot) : X\times X\rightarrow \CC$ tale che
\begin{itemize}
\item $(x, x) = 0\iff x = 0$
\item $(x, x) \geq 0, \forall x\in X$
\item $(x, y) = \overline{(y, x)}$
\item Vale la \textit{sesquilinearità}
\begin{itemize}
\item $(c_{1} x_{1} + c_{2} x_{2}, y) = c_{1}(x_{1}, y) + c_{2}(x_{2}, y)$
\item $\overline{(y, c_{1} x_{1} + c_{2} x_{2})} = c_{1}\overline{(y, x_{1})} + c_{2}\overline{(y, x_{2})}$
\item $(y, c_{1} x_{1} + c_{2} x_{2}) = \overline{c_{1}}(y, x_{1}) + \overline{c_{2}}(y, x_{2})$
\end{itemize}
\end{itemize}
\end{defn}
\begin{defn}
[Spazio di Hilbert]
Uno spazio vettoriale \textit{dotato di prodotto scalare} (che induce quindi una norma $ \Vert x \Vert = \sqrt{(x, x)}$ e una distanza) \textit{e completo} rispetto a quella norma si dice \textbf{spazio di Hilbert}.
\end{defn}
\begin{defn}
Definiamo lo spazio delle successioni convergenti
\begin{equation*}
l^{2} = \left\{\{a_{n}\} \subset \CC : \sum\limits^{\infty}_{n = 0}| a_{n}|^{2} < + \infty \right\}
\end{equation*}
\end{defn}
\begin{defn}
[Spazio convesso]
Dato uno spazio di Hilbert $H$, $K\subset H$, $K$ è \textbf{convesso} se $\forall x, y\in K, \forall \lambda \in [0, 1]$ allora $\lambda x + (1 - \lambda) y\in K$.
\end{defn}
\begin{thm}
[di proiezione]
Sia $H$ di Hilbert e $C\subset H$ convesso e chiuso. Sia $\overline{x} \notin C$, allora $\exists P\in C$ tale che
\begin{equation*}
\Vert \overline{x} - P \Vert \leq \Vert \overline{x} - x \Vert \ \ \forall x\in C
\end{equation*}
e
\begin{equation*}
(\overline{x} - P, x - p) \leq 0\ \ \forall x\in C
\end{equation*}
\end{thm}

\begin{figure}[htpb]
\centering
\tikzset{every picture/.style = {line width = 0.75pt}} %set default line width to 0.75pt

\begin{tikzpicture}[x = 0.75pt, y = 0.75pt, yscale = - 1, xscale = 1]
%uncomment if require: \path (0, 106); %set diagram left start at 0, and has height of 106

%Shape: Ellipse [id: dp5649356596549697]
\draw (267, 60.66) .. controls (267, 40.58) and (300.58, 24.31) .. (342, 24.31) .. controls (383.42, 24.31) and (417, 40.58) .. (417, 60.66) .. controls (417, 80.73) and (383.42, 97) .. (342, 97) .. controls (300.58, 97) and (267, 80.73) .. (267, 60.66) - - cycle;
%Straight Lines [id: da02100518467366297]
\draw (283, 37.31) - - (189, 37.31);
\draw [shift = {(189, 37.31)}, rotate = 180] [color = {rgb, 255: red, 0; green, 0; blue, 0} ][fill = {rgb, 255: red, 0; green, 0; blue, 0} ][line width = 0.75] (0, 0) circle [x radius = 3.35, y radius = 3.35];
\draw [shift = {(283, 37.31)}, rotate = 180] [color = {rgb, 255: red, 0; green, 0; blue, 0} ][fill = {rgb, 255: red, 0; green, 0; blue, 0} ][line width = 0.75] (0, 0) circle [x radius = 3.35, y radius = 3.35];
%Straight Lines [id: da7214240046750127]
\draw (363, 80.31) - - (283, 37.31);
\draw [shift = {(283, 37.31)}, rotate = 208.26] [color = {rgb, 255: red, 0; green, 0; blue, 0} ][fill = {rgb, 255: red, 0; green, 0; blue, 0} ][line width = 0.75] (0, 0) circle [x radius = 3.35, y radius = 3.35];
\draw [shift = {(363, 80.31)}, rotate = 208.26] [color = {rgb, 255: red, 0; green, 0; blue, 0} ][fill = {rgb, 255: red, 0; green, 0; blue, 0} ][line width = 0.75] (0, 0) circle [x radius = 3.35, y radius = 3.35];
%Shape: Arc [id: dp82817523637768]
\draw [draw opacity = 0] (309.33, 51.7) .. controls (304.23, 61.01) and (294.35, 67.31) .. (283, 67.31) .. controls (266.58, 67.31) and (253.25, 54.13) .. (253, 37.77) - - (283, 37.31) - - cycle; \draw (309.33, 51.7) .. controls (304.23, 61.01) and (294.35, 67.31) .. (283, 67.31) .. controls (266.58, 67.31) and (253.25, 54.13) .. (253, 37.77);

% Text Node
\draw (173, 14.4) node [anchor = north west][inner sep = 0.75pt] {$\overline{x}$};
% Text Node
\draw (276, 10.4) node [anchor = north west][inner sep = 0.75pt] {$P$};
% Text Node
\draw (365, 55.4) node [anchor = north west][inner sep = 0.75pt] {$x$};
% Text Node
\draw (196, 67.4) node [anchor = north west][inner sep = 0.75pt] {$ > \pi /2$};

\end{tikzpicture}
\end{figure}
\FloatBarrier

Chiameremo $P_{C}\overline{x}$ la proiezione di $\overline{x}$ su $C$.

Consideriamo $M\subset H$ sottospazio chiuso (essenziale per includere anche il caso in cui $H$ ha dimensione infinita). Sia $\overline{x} \in M$, consideriamo il caso bidimensionale con $M$ che ha dimensione $1$, si può chiaramente scomporre
\begin{equation*}
\textcolor[rgb]{0.29, 0.56, 0.89}{\overline{x}} = \overline{x} - P_{M}\overline{x} + P_{M}\overline{x} = \textcolor[rgb]{0.49, 0.83, 0.13}{Q}\textcolor[rgb]{0.49, 0.83, 0.13}{_{M}}\textcolor[rgb]{0.49, 0.83, 0.13}{\overline{x}} + \textcolor[rgb]{0.82, 0.01, 0.11}{P}\textcolor[rgb]{0.82, 0.01, 0.11}{_{M}}\textcolor[rgb]{0.82, 0.01, 0.11}{\overline{x}}
\end{equation*}

\begin{figure}[htpb]
\centering
\tikzset{every picture/.style = {line width = 0.75pt}} %set default line width to 0.75pt

\begin{tikzpicture}[x = 0.75pt, y = 0.75pt, yscale = - 1, xscale = 1]
%uncomment if require: \path (0, 184); %set diagram left start at 0, and has height of 184

%Shape: Axis 2D [id: dp7134109691601231]
\draw (198, 127.91) - - (415, 127.91)(244, 30) - - (244, 153.91) (408, 122.91) - - (415, 127.91) - - (408, 132.91) (239, 37) - - (244, 30) - - (249, 37);
%Shape: Axis 2D [id: dp2976528104012621]
\draw (201.57, 145.68) - - (401.72, 61.83)(206.17, 37.6) - - (254.05, 151.89) (393.33, 59.93) - - (401.72, 61.83) - - (397.2, 69.15) (204.26, 45.99) - - (206.17, 37.6) - - (213.49, 42.13);
%Straight Lines [id: da6603280695196603]
\draw [color = {rgb, 255: red, 208; green, 2; blue, 27}, draw opacity = 1][line width = 2.25] (244, 127.91) - - (334.38, 90.66);
\draw [shift = {(339, 88.75)}, rotate = 517.6] [fill = {rgb, 255: red, 208; green, 2; blue, 27}, fill opacity = 1][line width = 0.08] [draw opacity = 0] (16.07, - 7.72) - - (0, 0) - - (16.07, 7.72) - - (10.67, 0) - - cycle;
%Straight Lines [id: da12421560052711267]
\draw [color = {rgb, 255: red, 74; green, 144; blue, 226}, draw opacity = 1][line width = 2.25] (244, 127.91) - - (314.81, 42.38);
\draw [shift = {(318, 38.53)}, rotate = 489.62] [fill = {rgb, 255: red, 74; green, 144; blue, 226}, fill opacity = 1][line width = 0.08] [draw opacity = 0] (16.07, - 7.72) - - (0, 0) - - (16.07, 7.72) - - (10.67, 0) - - cycle;
%Straight Lines [id: da9584665613249306]
\draw [color = {rgb, 255: red, 126; green, 211; blue, 33}, draw opacity = 1][line width = 2.25] (339, 88.75) - - (319.93, 43.15);
\draw [shift = {(318, 38.53)}, rotate = 427.31] [fill = {rgb, 255: red, 126; green, 211; blue, 33}, fill opacity = 1][line width = 0.08] [draw opacity = 0] (16.07, - 7.72) - - (0, 0) - - (16.07, 7.72) - - (10.67, 0) - - cycle;

% Text Node
\draw (408.5, 48.4) node [anchor = north west][inner sep = 0.75pt] {$M$};
% Text Node
\draw (256, 151.9) node [anchor = north west][inner sep = 0.75pt] {$M^{\perp}$};
% Text Node
\draw (271.5, 55.4) node [anchor = north west][inner sep = 0.75pt] {$\overline{x}$};
% Text Node
\draw (335, 44.9) node [anchor = north west][inner sep = 0.75pt] {$Q_{M}\overline{x}$};
% Text Node
\draw (323.5, 103.9) node [anchor = north west][inner sep = 0.75pt] {$P_{M}\overline{x}$};

\end{tikzpicture}
\end{figure}
\FloatBarrier

\begin{defn}
[Spazio ortogonale]
Sia $H$ di Hilbert, sia $M\subset H$ un sottospazio chiuso. Chiamiamo spazio ortogonale a $M$ lo spazio
\begin{equation*}
M^{\perp} = \{x\in H: (x, y) = 0, \forall y\in M\}
\end{equation*}
\end{defn}
Uno spazio di Hilbert è la più naturale estensione di uno spazio euclideo a dimensione infinita.
\begin{thm}
[di proiezione]
Sia $H$ di Hilbert, sia $M\subset H$ un sottospazio \textbf{chiuso proprio}. Allora
\begin{enumerate}
\item $\forall x\in H, \exists !$ decomposizione in $\RR^{n}$ generica
\begin{equation*}
x = P_{x} + Q_{x}
\end{equation*}
tale che $P_{x} \in M$, $Q_{x} \in M^{\perp}$.
\item detti $P = P_{M}$ e $Q = P_{M^{\perp}}$ le proiezioni di un certo punto rispettivamente su $M$ e $M^{\perp}$, allora
\begin{gather*}
\Vert P_{x} - x \Vert \leq \Vert y - x \Vert \ \ \forall y\in M\\
\Vert Q_{x} - x \Vert \leq \Vert y - x \Vert \ \ \forall y\in M^{\perp}
\end{gather*}
\item le applicazioni $P: H\rightarrow M$ e $Q: H\rightarrow M^{\perp}$ sono lineari
\item vale Pitagora generalizzato
\begin{equation*}
\Vert x \Vert^{2} = \Vert P \Vert^{2} + \Vert Q \Vert^{2}
\end{equation*}
\end{enumerate}
\end{thm}

Il teorema ha come corollario del quarto punto che proiettando due punti non li possiamo allontanare più di quanto non lo erano
\begin{equation*}
\Vert P_{x} - P_{y} \Vert \leq \Vert x - y \Vert \ \ \forall x, y
\end{equation*}
\begin{rem}
I casi banali sono
\begin{gather*}
x\in M\implies P = x, Q = 0\\
x\in M^{\perp} \implies P = 0, Q = x
\end{gather*}
\end{rem}
\begin{thm}
[Disuguaglianza di Schwarz]
\begin{equation*}
| (x, y)| \leq \Vert x \Vert \Vert y \Vert \ \ \forall x, y\in H
\end{equation*}
\end{thm}
\begin{thm}
[Identità del parallelogramma] La somma dei quadrati sulle diagonali è uguale alla somma dei quadrati su tutti i lati
\begin{equation*}
\Vert x + y \Vert^{2} + \Vert x - y \Vert^{2} = 2\left(\Vert x \Vert^{2} + \Vert y \Vert^{2}\right)
\end{equation*}
\end{thm}
\begin{defn}
[Spazio separabile]
Uno spazio normato $X$ si dice separabile se $\exists Y\subset X$, $Y$ numerabile e tale che $\overline{Y} = X$.
\end{defn}
In uno spazio separabile si ha che $\forall x\in X, \exists \{y_{n}\} \subset Y$ tale che $y_{n}\rightarrow x$, cioè nel senso della norma $ \Vert y_{n} - x \Vert \rightarrow 0$. Si dice che $Y$ è denso in $X$, ovvero possiamo sempre trovare una successione approssimante.
\begin{defn}
[Base hilbertiana]
Sia $H$ di Hilbert a dimensione infinita. Un insieme di vettori $\{e_{k}\} \subset H$ tali che $(e_{j}, e_{k}) = \delta_{jk}$ si dice \textbf{base hilbertiana} se $\forall x\in H, \exists \{\lambda_{n}\} \subset \CC$ tale che $x = \sum^{\infty}_{n = 1} \lambda_{n} e_{n}$.
\end{defn}
\begin{rem}
Non essendo finita, non è una base nel senso dell'algebra lineare, ma è la cosa più vicina in dimensione infinita.
\end{rem}
\begin{rem}
È l'equivalente di una base ortonormale.
\end{rem}
\begin{rem}
Se $n = \mathrm{dim}(V) < \infty $, $\{e_{1}, \dotsc, e_{n}\}$, $(e_{j}, e_{k}) = \delta_{jk}$, allora $\forall v\in V\implies v = \sum^{n}_{k = 1} \lambda_{k} e_{k}$.
\end{rem}
\begin{rem}
Dire
\begin{equation*}
x = \sum^{\infty}_{n = 1} \lambda_{n} e_{n} \ \ \text{è equivalente a dire} \ \ \lim\limits_{n\rightarrow + \infty}\left \Vert x - \sum^{n}_{k = 1} \lambda_{k} e_{k}\right \Vert = 0
\end{equation*}
usando la norma indotta dal prodotto scalare.
\end{rem}
\begin{thm}
Ogni spazio di Hilbert separabile ammette base Hilbertiana.
\end{thm}
\begin{rem}
Il concetto di separabilità è un concetto abbastanza debole con poche richieste, soddisfatte nella stragrande maggioranza dei casi.
\end{rem}
\begin{thm}
Sia $H$ separabile e sia $\{e_{n}\}$ una base hilbertiana. Allora $\forall x\in H$, $x$ si scrive come \textbf{serie di Fourier generalizzata}
\begin{equation*}
x = \sum^{\infty}_{n = 1}(x, e_{n}) e_{n}
\end{equation*}
dove pertanto definiamo i \textbf{coefficienti di Fourier}
\begin{equation*}
\lambda_{n} = (x, e_{n})
\end{equation*}
inoltre vale l'\textbf{identità di Parseval}
\begin{equation*}
\Vert x \Vert^{2} = \sum\limits^{\infty}_{n = 1}| (x, e_{n})|^{2}
\end{equation*}
\end{thm}
\begin{proof}[Semi - dimostrazione.]

Ammettendo base hilbertiana, esistono $\lambda_{n}$ tali che
\begin{equation*}
x = \sum^{\infty}_{n = 1} \lambda_{n} e_{n}
\end{equation*}
moltiplichiamo \textit{a destra} scalarmente per un generico $e_{k}$
\begin{equation*}
(x, e_{k}) = \left(\sum^{\infty}_{n = 1} \lambda_{n} e_{n}, e_{k}\right) = \sum^{\infty}_{n = 1} \lambda_{n}(e_{n}, e_{k})\overset{\delta_{nk}}{=} \lambda_{k}
\end{equation*}
\begin{rem}
È una semidimostrazione in quanto la linearità non vale per la serie che va a $\infty $, bisognerebbe passare al limite ma ci accontentiamo.
\end{rem}
Mostriamo ora Parseval
\begin{align*}
\Vert x \Vert^{2} & = (x, x) = \left(\sum^{\infty}_{n = 1}(x, e_{n}) e_{n}, \sum^{\infty}_{k = 1}(x, e_{k}) e_{k}\right)\\
 & = \sum\limits^{\infty}_{n, k = 1}(x, e_{n})\overline{(x, e_{k})}(e_{n}, e_{k}) = \sum\limits^{\infty}_{n = 1}| (x, e_{n})|^{2}
\end{align*}
\end{proof}
\begin{thm}
Sia $H$ di Hilbert separabile. Allora $H$ è isomorfo a $l^{2}$. In particolare $\exists L: H\rightarrow l^{2}$, lineare, biiettiva ed isometria.
\end{thm}
\begin{rem}
Quindi tutti gli spazi di Hilbert sono isomorfi tra di loro.
\end{rem}
\begin{proof}

Essendo separabile esiste una BH\footnote{Base Hilbertiana.} $\{e_{n}\}$.
\begin{equation*}
\forall x\in H\ \ \exists \lambda_{n} = (x, e_{n}) \ \ x = \sum^{\infty}_{n = 1} \lambda_{n} e_{n}
\end{equation*}
La successione dei coefficienti è contenuta in $l^{2}$
\begin{equation*}
\lambda = \{\lambda_{n}\} \subset l^{2}
\end{equation*}
grazie all'identità di Parseval
\begin{equation*}
\Vert x \Vert^{2} = \sum\limits^{\infty}_{n = 1}| \lambda_{n}|^{2}
\end{equation*}
Questa successione è anche unica, quindi la mappa è ben definita.
\begin{itemize}
\item \textit{Linearità.}
\begin{equation*}
\lambda_{n} = (\alpha x + \beta y, e_{n}) = \alpha (x, e_{n}) + \beta (y, e_{n}) = \alpha \mu_{n} + \beta \nu_{n}
\end{equation*}
\item \textit{Isometria.}

Per Parseval
\begin{equation*}
\Vert x \Vert^{2}_{H} = \sum^{\infty}_{n = 1}| \lambda_{n}|^{2} = \Vert \lambda \Vert^{2}_{l^{2}}
\end{equation*}
\item \textit{Iniettività.}

Isometria significa che $ \Vert L(x) \Vert = \Vert x \Vert $, grazie a questo possiamo dire che\footnote{Un'applicazione \textit{lineare} tra spazi di dimensione infinita \textit{può non essere continua}, ma se è isometria lo è di sicuro.}
\begin{equation*}
L(x) = L(y) \ \ \implies \ \ \Vert L(x) - L(y) \Vert = 0\ \ \implies \ \ \Vert x - y \Vert = 0\ \ \implies \ \ x = y
\end{equation*}
\item \textit{Suriettività.}

Data $\lambda \subset l^{2} \mapsto x\in H, \lambda = L(x)$ dobbiamo costruire $x$ avendo $\lambda $, lo facciamo chiedendoci se converge
\begin{equation*}
x = \sum^{\infty}_{n = 1} \lambda_{n} e_{n}
\end{equation*}

Usiamo la \textbf{completezza} dello spazio
\begin{equation*}
\lambda \subset l^{2} \ \ \implies \ \ \sum\limits^{\infty}_{n = 1}| \lambda_{n}|^{2} < + \infty \ \ \implies \ \ \mu_{k} = \sum\limits^{k}_{n = 1}| \lambda_{n}|^{2} \ \ \text{è di Cauchy}
\end{equation*}

da cui possiamo considerare
\begin{equation*}
x_{k} = \sum^{k}_{n = 1} \lambda_{n} e_{n}
\end{equation*}

e dimostrare che

\begin{gather*}
\Vert x_{k} - x_{l} \Vert = \left \Vert \sum\limits^{k}_{n = l + 1} \lambda_{n} e_{n}\right \Vert = \sum\limits^{k}_{n = l + 1}| \lambda_{n}|^{2} = \mu_{k} - \mu_{l}\\
\implies \ \ x_{k} = \sum^{k}_{n = 1} \lambda_{n} e_{n} \ \ \text{è di Cauchy} \ \ \implies \ \ \text{converge per completezza.}
\end{gather*}

\end{itemize}
\end{proof}

\section{Integrale di Lebesgue}

Consideriamo un \textbf{plurirettangolo}
\begin{equation*}
P\subset \RR^{n}, \ \ P = [a_{1}, b_{1}] \times \dotsc \times [a_{n}, b_{n}]
\end{equation*}
la sua \textbf{misura} è data da
\begin{equation*}
| P| = (b_{1} - a_{1}) \dotsc (b_{n} - a_{n})
\end{equation*}
Consideriamo dei generici insiemi
\begin{itemize}
\item $A$ aperto
\begin{equation*}
| A| = \sup \left\{| P|, P\in \text{pluri - intervalli}, P\subset A\right\}
\end{equation*}
\item $C$ chiuso
\begin{equation*}
| C| = \inf\left\{| P|, P\in \text{pluri - intervalli}, P\supset C\right\}
\end{equation*}
\end{itemize}
\begin{rem}
Gli insiemi aperti non sono chiusi per intersezioni numerabili.
\end{rem}
\begin{rem}
Gli insiemi chiusi non sono chiusi per unioni numerabili.
\end{rem}
Dato un insieme limitato qualunque, si dice
\begin{itemize}
\item \textbf{misura esterna} di $E$, $| E|^{*} = \inf\left\{| A|, A\ \text{aperto}, A\supset E\right\}$
\item \textbf{misura interna} di $E$, $| E|_{*} = \sup \left\{| C|, C\ \text{chiuso}, C\subset E\right\}$
\end{itemize}
\begin{defn}
Un insieme $E$ si dice \textbf{misurabile secondo Lebesgue} se $| E|^{*} = | E|_{*}$.
\end{defn}
Si ha quasi sempre questa condizione, tranne in casi patologici.

\textbf{Proprietà.}

Siano $E_{1}, \dotsc, E_{n}$ un'infinità numerabile di insiemi misurabili.
\begin{itemize}
\item allora $\bigcup^{\infty}_{n = 1} E_{n}$ è misurabile $\left| \bigcup^{\infty}_{n = 1} E_{n}\right| \leq \sum\nolimits^{\infty}_{n = 1}| E_{n}| $, l'uguaglianza solo se sono disgiunti
\item allora $\bigcap^{\infty}_{n = 1} E_{n}$ è misurabile.
\end{itemize}
\begin{defn}
Un insieme \textbf{illimitato} è misurabile se $E\cap B_{r}$ è misurabile. Si definisce $| E| : = \lim\limits_{r\rightarrow + \infty}| E\cap B_{r}| $.
\end{defn}
\begin{defn}
Una funzione si dice\textbf{a scala} se la sua immagine è finita.
\end{defn}
\begin{defn}
Una funzione $f: A\subset \RR^{n}\rightarrow \RR$ è \textbf{misurabile secondo Lebesgue} se l'insieme
\begin{equation*}
\{x\in A: f(x) < c\}
\end{equation*}
è misurabile $\forall c\in \RR$.
\end{defn}
\begin{defn}
Si dice integrale di Lebesgue di $f: A\subset \RR^{n}\rightarrow [0, + \infty)$ e $A$ misurabile
\begin{equation*}
\int_{A} f(x) dx = \sup \int_{A} g(x) dx
\end{equation*}
dove $g(x)$ è una funzione a scala tale che $g(x) \leq f(x)$ e inoltre vale
\begin{equation*}
\int_{A} g(x) dx = \sum\limits^{N}_{i = 1} g_{i}| A_{i}|, \ \ g(x) = g_{i} \ \forall x\in A_{i}
\end{equation*}
\end{defn}

\begin{figure}[htpb]
\centering
\tikzset{every picture/.style = {line width = 0.75pt}} %set default line width to 0.75pt

\begin{tikzpicture}[x = 0.75pt, y = 0.75pt, yscale = - 1, xscale = 1]
%uncomment if require: \path (0, 171); %set diagram left start at 0, and has height of 171

%Shape: Axis 2D [id: dp6527686546165439]
\draw (138, 140.65) - - (460, 140.65)(170.2, 5) - - (170.2, 155.72) (453, 135.65) - - (460, 140.65) - - (453, 145.65) (165.2, 12) - - (170.2, 5) - - (175.2, 12);
%Curve Lines [id: da4027262347286087]
\draw [color = {rgb, 255: red, 208; green, 2; blue, 27}, draw opacity = 1][line width = 2.25] (210, 55) .. controls (299.8, - 13.4) and (275.8, 107) .. (377, 55.72);
%Straight Lines [id: da24315154470517308]
\draw [color = {rgb, 255: red, 126; green, 211; blue, 33}, draw opacity = 1][line width = 2.25] (210, 55) - - (235, 55);
%Straight Lines [id: da22825901363329582]
\draw [color = {rgb, 255: red, 126; green, 211; blue, 33}, draw opacity = 1][line width = 2.25] (235, 40) - - (274.6, 40);
%Straight Lines [id: da5730415093182173]
\draw [color = {rgb, 255: red, 126; green, 211; blue, 33}, draw opacity = 1][line width = 2.25] (274.6, 60) - - (301, 60);
%Straight Lines [id: da5297578793317113]
\draw [color = {rgb, 255: red, 126; green, 211; blue, 33}, draw opacity = 1][line width = 2.25] (301, 70) - - (327.4, 70);
%Straight Lines [id: da9350109074443522]
\draw [color = {rgb, 255: red, 126; green, 211; blue, 33}, draw opacity = 1][line width = 2.25] (327.4, 70) - - (353.8, 70);
%Straight Lines [id: da5780729033254488]
\draw [color = {rgb, 255: red, 126; green, 211; blue, 33}, draw opacity = 1][line width = 2.25] (353.8, 66) - - (377.4, 66);
%Straight Lines [id: da7836728935339827]
\draw (210, 140.6) - - (210, 55);
%Straight Lines [id: da4255882760356502]
\draw (235, 140.6) - - (235, 40);
%Straight Lines [id: da9443315667203329]
\draw (275, 140.6) - - (275, 39.8);
%Straight Lines [id: da4615497493154914]
\draw (301, 140.6) - - (301, 60);
%Straight Lines [id: da2871328359876997]
\draw (354, 140.6) - - (354, 66);
%Straight Lines [id: da718006435968708]
\draw (377, 140.6) - - (377, 55.72);
%Straight Lines [id: da4407818981447249]
\draw [color = {rgb, 255: red, 74; green, 144; blue, 226}, draw opacity = 1][line width = 2.25] (235, 140.6) - - (274.6, 140.6);

% Text Node
\draw (384, 34.4) node [anchor = north west][inner sep = 0.75pt] [color = {rgb, 255: red, 208; green, 2; blue, 27}, opacity = 1] {$f(x)$};
% Text Node
\draw (384, 59.2) node [anchor = north west][inner sep = 0.75pt] [color = {rgb, 255: red, 126; green, 211; blue, 33}, opacity = 1] {$g(x)$};
% Text Node
\draw (242.6, 144) node [anchor = north west][inner sep = 0.75pt] [color = {rgb, 255: red, 74; green, 144; blue, 226}, opacity = 1] {$A_{i}$};

\end{tikzpicture}
\end{figure}
\FloatBarrier

\begin{rem}
È una definizione più pulita di quella di Riemann. Il dominio e la funzione possono essere illimitati (e naturalmente l'integrale può valere $\infty $).
\end{rem}
Per definirlo per funzioni anche negative procediamo come segue.

Sia $f: A\subset \RR^{n}\rightarrow \CC$
\begin{equation*}
f(x) = R^{+}(x) - R^{-}(x) + I^{+}(x) - I^{-}(x)
\end{equation*}
\begin{itemize}
\item $R^{+}(x) = \max(\Re (f(x)), 0), \forall x$
\item $R^{-}(x) = \max(- \Re (f(x)), 0), \forall x$
\item $I^{+}(x) = \max(\Im (f(x)), 0), \forall x$
\item $I^{-}(x) = \max(- \Im (f(x)), 0), \forall x$
\end{itemize}

Allora definiamo
\begin{equation*}
\int_{A} f(x) dx = \int_{A} R^{+}(x) dx - \int_{A} R^{-}(x) dx + i\left[\int_{A} I^{+}(x) dx - \int_{A} I^{-}(x) dx\right]
\end{equation*}
\begin{defn}
Una funzione $f: A\subset \RR^{n}\rightarrow \CC$ si dice \textbf{Lebesgue - integrabile}
\begin{equation*}
\boxed{f\in L^{1} \ \ \iff \ \ \int_{A}| f(x)| dx < + \infty}
\end{equation*}
\end{defn}
Per esempio la funzione
\begin{equation*}
f(x) =
\begin{cases}
1, & x\in \QQ\\
0, & x\notin \QQ
\end{cases}
\end{equation*}
non è integrabile secondo Riemann, ma il suo integrale di Lebesgue esiste e vale $0$.
\begin{rem}
Se $f(x) = g(x), \forall x\in A$ tranne al più un insieme di misura nulla, allora
\begin{equation*}
\int_{A} f(x) dx = \int_{A} g(x) dx\ \ \text{quasi ovunque}
\end{equation*}
\end{rem}
\begin{thm}
[della Convergenza Monotona]
Siano $f_{n} : A\rightarrow [0, + \infty)$. Sia $\lim\limits_{n\rightarrow + \infty} f_{n} = f(x)$ quasi ovunque, $f_{n}(x) \leq f_{n + 1}(x), \forall x$ quasi ovunque e $\forall n$. Allora
\begin{equation*}
\boxed{\lim\limits_{n\rightarrow + \infty}\int_{A} f_{n}(x) dx = \int_{A}\lim\limits_{n\rightarrow + \infty} f_{n}(x) dx = \int_{A} f(x) dx}
\end{equation*}
\end{thm}
\begin{thm}
[della Convergenza Dominata]
Siano $f_{n} : A\rightarrow \CC$. Sia $\lim\limits_{n\rightarrow + \infty} f_{n} = f(x)$ quasi ovunque, $\exists g\in L^{1}(A)$ tale che $| f_{n}(x)| \leq g(x)$ quasi ovunque e $\forall n$. Allora
\begin{equation*}
\boxed{\lim\limits_{n\rightarrow + \infty}\int_{A} f_{n}(x) dx = \int_{A}\lim\limits_{n\rightarrow + \infty} f_{n}(x) dx = \int_{A} f(x) dx}
\end{equation*}
\end{thm}
\begin{thm}
[di Fubini]
Sia $f: \RR^{n + m}\rightarrow \CC, f\in L^{1}\left(\RR^{n + m}\right)$. Sia $f = f(x, y)$ con $x\in \RR^{n}$ e $y\in \RR^{m}$. Allora
\begin{equation*}
\boxed{\int_{\RR^{n + m}} f(x, y) dxdy = \int_{\RR^{n}}\left[\int_{\RR^{m}} f(x, y) dy\right] dx}
\end{equation*}
\end{thm}

\section{Spazi $L^{p}$}

\begin{defn}
[Spazio $L^{p}$] Sia $A\subset \RR^{n}$ misurabile. Sia $1 \leq p < \infty $. Definiamo lo spazio
\begin{equation*}
L^{p}(A) = \left\{f: A\rightarrow \CC \ \text{misurabili, tali che} \ \int_{A}| f(x)|^{p} dx < \infty \right\}
\end{equation*}
con norma
\begin{equation*}
\Vert f \Vert_{L^{p}(A)} = \left(\int_{A}| f(x)|^{p} dx\right)^{1/p}
\end{equation*}
\end{defn}
\begin{defn}
Se $p = \infty $ si definisce
\begin{equation*}
L^{\infty}(A) = \left\{f: A\rightarrow \CC \ \text{misurabili, tali che} \ \exists c \geq 0: | f(x)| \leq c\ \text{q.o. in} \ A\right\}
\end{equation*}
con norma
\begin{equation*}
\Vert f \Vert_{L^{\infty}(A)} = \inf\left\{c \geq 0: | f(x)| \leq c\ \text{q.o. in} \ A\right\}
\end{equation*}
\end{defn}
\begin{defn}
[Estremo superiore essenziale]
Consideriamo lo spazio $L^{\infty}(A)$, definiamo estremo superiore essenziale
\begin{align*}
\esssup f(x) & = \inf\{c\in \RR : | f(x) > c| = 0\}\\
 & = \inf\left\{c\in \RR : | f(x)| < c\ \text{q.o. in} \ A\right\}
\end{align*}
\end{defn}
\begin{rem}
$ \Vert f \Vert_{L^{\infty}} = \esssup| f(x)| $.
\end{rem}
\begin{thm}
Sia $p\in [1, + \infty)$, $a, b\in L^{p}$, allora anche $(a + b) \in L^{p}$.
\end{thm}
\begin{proof}

Sia $f(x) = | x|^{p}, x \geq 0, p > 1$ è una funzione convessa, ovvero
\begin{equation*}
f(\lambda x + (1 - \lambda) y) \leq \lambda f(x) + (1 - \lambda) f(y)
\end{equation*}
poniamo $\lambda = \frac{1}{2}$ e aggiungiamo il modulo
\begin{equation*}
\left| \frac{1}{2} x + \frac{1}{2} y\right|^{p} \leq \frac{1}{2}| x|^{p} + \frac{1}{2}| y|^{p}
\end{equation*}
integriamo
\begin{equation*}
\int \left| \frac{1}{2} x + \frac{1}{2} y\right|^{p} dx \leq \frac{1}{2}\int | x|^{p} dx + \frac{1}{2}\int | y|^{p} dx
\end{equation*}
questo prova la tesi, se al posto di $x$ e $y$ immaginiamo che ci siano funzioni in $L^{p}$.
\end{proof}
\begin{defn}
[Esponenti coniugati]
Due numeri $p, q \geq 1$ si dicono coniugati se
\begin{equation*}
\frac{1}{p} + \frac{1}{q} = 1
\end{equation*}
Poniamo inoltre
\begin{equation*}
\frac{1}{\infty} = 0\ \ \ \ \frac{1}{0} = \infty
\end{equation*}
quindi se $p = 1\implies q = + \infty $, osserviamo anche che se $p\in (1, + \infty) \implies q\in (1, + \infty)$.
\end{defn}
\begin{thm}
[Disuguaglianza di Young]
Siano $p, q$ coniugati entrambi maggiori di $1$. Allora $\forall a, b \geq 0$
\begin{equation*}
\frac{a^{p}}{p} + \frac{b^{q}}{q} \geq ab
\end{equation*}
\end{thm}
\begin{proof}

La funzione logaritmo è concava, poniamo $\lambda = \frac{1}{p}$ e di conseguenza $1 - \lambda = \frac{1}{q}$
\begin{gather*}
\log\left(\frac{a^{p}}{p} + \frac{b^{q}}{q}\right) \geq \frac{\log a^{p}}{p} + \frac{\log b^{q}}{q} = \log(ab)
\end{gather*}
\end{proof}
\begin{thm}
[Disuguaglianza di Hölder]
Siano $p, q$ coniugati e siano $f\in L^{p}, g\in L^{q}$. Allora
\begin{equation*}
f(x) g(x) \in L^{1} \ \ \text{e} \ \ \int | f(x) g(x)| dx \leq \Vert f \Vert_{L^{p}} \Vert g \Vert_{L^{q}}
\end{equation*}
\end{thm}
\begin{proof}\leavevmode
\begin{itemize}
\item Se $p = 1, q = + \infty $ allora $g\in L^{\infty}$ cioè $\esssup| g(x)| < + \infty $
\begin{align*}
\implies & | g(x)| \leq \Vert g \Vert_{L^{\infty}} \ \ \text{q.o. in} \ x\\
\implies & \int | f(x) g(x)| dx \leq \Vert g \Vert_{L^{\infty}}\int | f(x)| dx = \Vert g \Vert_{L^{\infty}} \Vert f \Vert_{L^{1}}
\end{align*}
\item Se $p, q > 1$ allora
\begin{equation*}
| f(x) g(x)| = \left| \lambda^{\frac{p - 1}{p}} f(x)\right| \left| \lambda^{\frac{1 - p}{p}} g(x)\right|
\end{equation*}

Notiamo che
\begin{equation*}
\frac{1}{p} + \frac{1}{q} = 1\ \ \implies \ \ \frac{1}{q} = 1 - \frac{1}{p} = \frac{p - 1}{p}
\end{equation*}

voglio utilizzare la disuguaglianza di Young
\begin{align*}
\left| \lambda^{\frac{p - 1}{p}} f(x)\right| \left| \lambda^{\frac{1 - p}{p}} g(x)\right| & \leq \frac{\lambda^{\frac{p - 1}{p} \cdot p}| f(x)|^{p}}{p} + \frac{\lambda^{- \frac{1}{q} \cdot q}| g(x)|^{q}}{q}\\
 & = \frac{\lambda^{p - 1}| f(x)|^{p}}{p} + \frac{| g(x)|^{q}}{\lambda q}
\end{align*}

leggiamo tutta la maggiorazione
\begin{equation}
\int | f(x) g(x)| dx \leq \frac{\lambda^{p - 1}}{p}\underbrace{\int | f(x)|^{p} dx}_{\Vert f \Vert^{p}_{L^{p}}} + \frac{1}{\lambda q}\underbrace{\int | g(x)|^{q} dx}_{\Vert g \Vert^{q}_{L^{q}}} \ \ \implies \ \ fg\in L^{1} \tag{*}
\end{equation}

Per dimostrare il risultato della disuguaglianza cerchiamo il minimo in funzione di $\lambda > 0$ derivando
\begin{align*}
\frac{p - 1}{p} \lambda^{p - 2} \Vert f \Vert^{p}_{L^{p}} - \frac{1}{\lambda^{2} q} \Vert g \Vert^{q}_{L^{q}} & = 0\\
\implies \ \ \frac{p - 1}{p} \lambda^{p} \Vert f \Vert^{p}_{L^{p}} & = \frac{1}{q} \Vert g \Vert^{q}_{L^{q}}\\
\lambda^{p} & = \cancel{\frac{1}{q}} \Vert g \Vert^{q}_{L^{q}} \cdot \cancel{\frac{p}{p - 1}} \cdot \frac{1}{\Vert f \Vert^{p}_{L^{p}}}\\
\lambda & = \left[\frac{\Vert g \Vert^{q}_{L^{q}}}{\Vert f \Vert^{p}_{L^{p}}}\right]^{1/p}
\end{align*}

sostituendolo nella $(\star)$ otteniamo la tesi.

\end{itemize}
\end{proof}
\begin{thm}
[Disuguaglianza di Minkowski]
Sia $p \geq 1, f\in L^{p}, g\in L^{p}$. Allora
\begin{equation*}
(f + g) \in L^{p} \ \ \text{e} \ \ \Vert f + g \Vert_{L^{p}} \leq \Vert f \Vert_{L^{p}} + \Vert g \Vert_{L^{p}}
\end{equation*}
\end{thm}
\begin{proof}
\begin{align*}
\Vert f + g \Vert^{p}_{L^{p}} & = \int | f(x) + g(x)|^{p} dx\\
 & = \int | f(x) + g(x)|^{p - 1}| f(x) + g(x)| dx\\
 & \leq \int \underbrace{| f(x) + g(x)|^{p - 1}}_{A}\underbrace{| f(x)|}_{B} dx + \int \underbrace{| f(x) + g(x)|^{p - 1}}_{C}\underbrace{| g(x)|}_{D} dx
\end{align*}
Consideriamo $h(x) = A$
\begin{equation*}
[h(x)]^{q} = [h(x)]^{\frac{p}{p - 1}} = | f(x) + g(x)|^{p}
\end{equation*}
Sappiamo già, per dimostrazione precedente, che $(f + g) \in L^{p}$, allora $h\in L^{q}$. Usiamo la disuguaglianza di Hölder su $A\in L^{q}, B\in L^{p}$ e $C\in L^{q}, D\in L^{p}$. Otteniamo
\begin{align*}
\Vert f + g \Vert^{p}_{L^{p}} & \leq \Vert f \Vert_{L^{p}}\left \Vert \ | f + g|^{p - 1} \ \right \Vert_{L^{q}} + \Vert g \Vert_{L^{p}}\left \Vert \ | f + g|^{p - 1} \ \right \Vert_{L^{q}}
\end{align*}
osserviamo il termine che si presenta due volte
\begin{align*}
\left \Vert \ | f + g|^{p - 1} \ \right \Vert_{L^{q}} & = \left(\int | f(x) + g(x)|^{(p - 1) q} dx\right)^{1/q}\\
 & = \left(\int | f(x) + g(x)|^{p} dx\right)^{1/q}\\
 & = \left(\Vert f + g \Vert^{p}_{L^{p}}\right)^{1/q}\\
 & = \Vert f + g \Vert^{p/q}_{L^{p}}\\
 & = \Vert f + g \Vert^{p - 1}_{L^{p}}
\end{align*}
allora
\begin{gather*}
\Vert f + g \Vert^{p}_{L^{p}} \leq \Vert f \Vert_{L^{p}} \Vert f + g \Vert^{p - 1}_{L^{p}} + \Vert g \Vert_{L^{p}} \Vert f + g \Vert^{p - 1}_{L^{p}}\\
\implies \ \ \Vert f + g \Vert_{L^{p}} \leq \Vert f \Vert_{L^{p}} + \Vert g \Vert_{L^{p}}
\end{gather*}
\end{proof}
\begin{thm}
Per ogni $A\subset \RR^{n}$ misurabile, per ogni $p\in [1, + \infty)$, lo spazio $L^{p}(A)$ è di Banach, ed è separabile tranne nel caso $p = \infty $.
\end{thm}
\begin{thm}
Lo spazio $L^{2}(A)$ è uno spazio di Hilbert rispetto al prodotto scalare
\begin{equation*}
(f, g)_{2} = \int_{A} f(x)\overline{g(x)} dx
\end{equation*}
\end{thm}
\begin{rem}
L'unico esponente coniugato con se stesso è $2$. Lo spazio $L^{2}$ è il giusto \textit{equilibrio}.
\end{rem}

\section{Serie di Fourier}

Consideriamo come spazio di lavoro $L^{2}([ - \pi, \pi])$.
\begin{thm}
Consideriamo
\begin{equation*}
B = \left\{\frac{1}{\sqrt{2}}, \cos x, \sin x, \cos(2x), \sin(2x), \dotsc \right\}
\end{equation*}
Consideriamo $f, g\in L^{2}([ - \pi, \pi])$ e definiamo il prodotto scalare
\begin{equation*}
\boxed{(f, g) = \frac{1}{\pi}\int^{\pi}_{- \pi} f(x) g(x) dx} \ \ \text{cioè} \ \ \Vert f \Vert^{2}_{L^{2}} = \frac{1}{\pi}\int^{\pi}_{- \pi} f^{2}(x) dx
\end{equation*}
allora $B$ è una base di Hilbert.
\end{thm}
Si vede facilmente che i suoi vettori sono normali
\begin{gather*}
\left \Vert \frac{1}{\sqrt{2}}\right \Vert = \frac{1}{\pi}\int^{\pi}_{- \pi}\left(\frac{1}{\sqrt{2}}\right)^{2} dx = 1\\
\Vert \sin(kx) \Vert = \frac{1}{\pi}\int^{\pi}_{- \pi}(\sin(kx))^{2} dx = 1
\end{gather*}
e ortogonali a due a due
\begin{equation*}
\frac{1}{\pi}\int^{\pi}_{- \pi}\frac{1}{\sqrt{2}}\sin(kx) dx = 0\ \ \ \ \frac{1}{\pi}\int^{\pi}_{- \pi}\cos(kx)\cos(jx) dx = 0, \ \forall j\neq k
\end{equation*}
Vale inoltre, dalla teoria degli spazi di Hilbert, la seguente relazione, che ci permette di costruire ogni elemento dello spazio di Hilbert tramite la base
\begin{equation*}
\forall f\in L^{2}([ - \pi, \pi]), \ \ f(x)\textcolor[rgb]{0.82, 0.01, 0.11}{=}\frac{a_{0}}{\sqrt{2}} + \sum\limits^{\infty}_{n = 1}[a_{n}\cos(nx) + b_{n}\sin(nx)]
\end{equation*}
Cerchiamo l'espressione per calcolare i coefficienti, e soprattutto cerchiamo di capire in che senso vale tale uguaglianza in rosso.

Sappiamo che i coefficienti si calcolano come prodotto scalare della cosa da costruire (la funzione in $L^{2}$ nel nostro caso) per il rispettivo elemento della base
\begin{gather*}
a_{n} = (f, e_{n}) = \frac{1}{\pi}\int^{\pi}_{- \pi} f(x)\cos(nx) dx\\
b_{n} = (f, e_{n}) = \frac{1}{\pi}\int^{\pi}_{- \pi} f(x)\sin(nx) dx\\
a_{0} = \frac{1}{\pi}\int^{\pi}_{- \pi} f(x) \cdot \textcolor[rgb]{0.29, 0.56, 0.89}{\frac{1}{\sqrt{2}}} dx
\end{gather*}
Possiamo "togliere" dalla formula il termine blu e accorparlo nella formula completa, in modo da non avere radici, otteniamo la formula per le serie di Fourier\footnote{L'importanza della serie di Fourier è cruciale nelle equazioni differenziali. Infatti, nell'equazione del calore, che deriva direttamente dal postulato di Fourier, la soluzione si trova facilmente se è composta da soli seni. In questo modo si può costruire facilmente la soluzione usando una serie di Fourier, che sarà ancora soluzione per il principio di sovrapposizione delle equazioni differenziali. Fourier, quando sviluppò questa teoria, stava proprio cercando di risolvere questo problema, e grazie ciò, ottenne la sua \textit{immortalità}.}
\begin{equation*}
\boxed{\forall f\in L^{2}([ - \pi, \pi]), \ \ f(x) = \frac{a_{0}}{2} + \sum\limits^{\infty}_{n = 1}[a_{n}\cos(nx) + b_{n}\sin(nx)]}
\end{equation*}
\begin{rem}
[Convergenza $L^{2}$] L'uguaglianza vale in senso $L^{2}$, ovvero che se indichiamo con $S_{N}(x)$ le somme parziali
\begin{equation*}
S_{N}(x) = \frac{a_{0}}{2} + \sum\limits^{N}_{n = 1}[a_{n}\cos(nx) + b_{n}\sin(nx)]
\end{equation*}
vale il limite\footnote{come di consueto, dire che qualcosa si avvicina a un'altra cosa, è equivalente a dire che \textit{una certa norma} della differenza tra le due tende a zero.}
\begin{equation*}
\lim\limits_{N\rightarrow + \infty}\int^{\pi}_{- \pi}[f(x) - S_{N}(x)]^{2} dx = 0
\end{equation*}
\end{rem}
\begin{rem}
La funzione che scegliamo di approssimare deve essere periodica di periodo $2\pi $, si può chiaramente generalizzare il risultato a periodi generici $T$
\begin{equation*}
\forall f\in L^{2}\left(\left[ - \frac{T}{2}, \frac{T}{2}\right]\right), \ \ f(x) = \frac{a_{0}}{2} + \sum\limits^{\infty}_{n = 1}\left[a_{n}\cos\left(\frac{2n\pi}{T} x\right) + b_{n}\sin\left(\frac{2n\pi}{T} x\right)\right]
\end{equation*}
con prodotto scalare
\begin{equation*}
(f, g) = \frac{1}{\frac{T}{2}}\int^{T/2}_{- T/2} f(x) g(x) dx
\end{equation*}
quindi per esempio
\begin{equation*}
a_{n} = \frac{2}{T}\int^{T/2}_{- T/2} f(x)\cos\left(\frac{2n\pi}{T} x\right) dx
\end{equation*}
\end{rem}
\begin{rem}
[Convergenza uniforme]
Dal momento che
\begin{equation*}
| a_{n}\cos(nx) + b_{n}\sin(nx)| \leq | a_{n}| + | b_{n}|
\end{equation*}
Per il test di Weierstrass vale la seguente relazione che ci permette di stabilire se la serie converge uniformemente
\begin{equation*}
\sum\limits^{\infty}_{n = 1}(| a_{n}| + | b_{n}|) < \infty \ \ \implies \ \ f(x) \ \text{converge uniformemente}
\end{equation*}
\end{rem}
\textit{Esempio.}

Sia $f(x) = | x| $ nell'intervallo $[ - \pi, \pi]$ estesa con periodicità a tutto $\RR$, calcoliamo lo sviluppo di Fourier.
\begin{align*}
a_{n} & = \frac{1}{\pi}\int^{\pi}_{- \pi}| x| \cos(nx) dx = \frac{2}{\pi}\int^{\pi}_{0} x\cos(nx) dx\\
 & = \frac{2}{\pi}\left\{\cancel{x \cdot \left. \frac{\sin(nx)}{n}\right|^{\pi}_{0}} - \int^{\pi}_{0}\frac{\sin(nx)}{n} dx\right\}\\
 & = \frac{2}{\pi}\left\{\left. \frac{\cos(nx)}{n^{2}}\right|^{\pi}_{0}\right\} = \frac{2}{\pi n^{2}}\left[(- 1)^{n} - 1\right]\\
 & \\
b_{n} & = \frac{1}{\pi}\int^{\pi}_{- \pi}| x| \sin(nx) dx = 0\\
 & \\
a_{0} & = \frac{2}{\pi}\int^{\pi}_{0} xdx = \pi
\end{align*}
allora si scrive in serie di Fourier
\begin{equation*}
| x| = \frac{\pi}{2} + \sum\limits^{\infty}_{n = 1}\frac{2}{\pi n^{2}}\left[(- 1)^{n} - 1\right]\cos(nx)
\end{equation*}
\begin{defn}
Una funzione $f: [ - \pi, \pi]\rightarrow \RR$ si dice \textbf{regolare a tratti} se è derivabile e ha un numero finito di punti di discontinuità $\{x_{n}\}_{n = 1\dotsc N}$ e in tali punti esistono finiti i limiti
\begin{equation*}
\lim\limits_{x\rightarrow x^{+}_{n}} f(x) \ \ \ \ \ \ \lim\limits_{x\rightarrow x^{-}_{n}} f(x)
\end{equation*}
Escludiamo quindi asintoti, punti di cuspide, ma i limiti possono eventualmente essere diversi.
\end{defn}
\begin{thm}
[Convergenza puntuale]
Una funzione $f: [ - \pi, \pi]\rightarrow \RR$ regolare a tratti. Allora la serie di Fourier converge $\forall x\in [ - \pi, \pi]$ tale che $f(x)$ è continua. Nei punti in cui non è continua invece converge al valore medio tra i due limiti.
\end{thm}
\begin{thm}
[Derivazione]
Per la derivazione possiamo dire che
\begin{equation*}
f'(x) = \sum\limits^{\infty}_{n = 1}[ - na_{n}\sin(nx) + nb_{n}\cos(nx)]
\end{equation*}
osserviamo che
\begin{equation*}
\sup_{x\in [ - \pi, \pi]}[ - na_{n}\sin(nx) + nb_{n}\cos(nx)] \leq n(| a_{n}| + | b_{n}|)
\end{equation*}
per cui per il test di Weierstrass se
\begin{equation*}
\sum\limits^{\infty}_{n = 1} n(| a_{n}| + | b_{n}|) < \infty \ \ \implies \ \ f'(x) \ \text{converge uniformemente}
\end{equation*}
e naturalmente anche $f(x)$. La condizione \textit{sufficiente} affinché la serie di Fourier sia derivabile $k$ volte è che
\begin{equation*}
\sum\limits^{\infty}_{n = 1} n^{k}(| a_{n}| + | b_{n}|) < \infty
\end{equation*}
\end{thm}
\begin{rem}
Lo usiamo normalmente solo come controllo per sapere se abbiamo calcolato bene la serie di Fourier, di solito sappiamo già se $f$ è derivabile o meno.
\end{rem}
Esiste una base alternativa, che troviamo se scriviamo le relazioni complesse
\begin{equation*}
\cos(nx) = \frac{e^{inx} + e^{- inx}}{2} \ \ \ \ \ \ \ \ \sin(nx) = \frac{e^{inx} - e^{- inx}}{2i}
\end{equation*}
allora la serie di Fourier si scrive nella sua \textbf{forma esponenziale}
\begin{equation*}
f(x) = \frac{a_{0}}{2} + \sum\limits^{\infty}_{n = 1}\left[a_{n}\frac{e^{inx} + e^{- inx}}{2} + b_{n}\frac{e^{inx} - e^{- inx}}{2i}\right] \ \ \implies \ \ \boxed{f(x) = \sum\limits^{\infty}_{n = - \infty} c_{n} e^{inx}}
\end{equation*}
\begin{rem}
In generale questi coefficienti sono complessi $c_{n} = \overline{c_{- n}}$.
\end{rem}
\begin{rem}
La base $\left\{e^{inx}\right\}_{n\in \ZZ}$ è una base ortonormale
\begin{equation*}
\int^{\pi}_{- \pi} e^{inx} e^{imx} dx = \int^{\pi}_{- \pi} e^{i(n - m) x} dx =
\begin{cases}
2\pi, & n = m\\
\left. \frac{e^{i(n - m) x}}{i(n - m)}\right|^{\pi}_{- \pi} = 0, & n\neq m
\end{cases}
\end{equation*}
normalizziamo, quindi il prodotto scalare sarà
\begin{equation*}
(f, g) = \frac{1}{2\pi}\int^{\pi}_{- \pi} f(x)\overline{g(x)} dx
\end{equation*}
e i coefficienti saranno
\begin{equation*}
c_{n} = \frac{1}{2\pi}\int^{\pi}_{- \pi} f(x) e^{- inx} dx
\end{equation*}
Il $ - $ è dovuto al fatto che $g$ è coniugato nel prodotto scalare.
\end{rem}

\chapter{Distribuzioni}

\section{Spazi di Banach e spazi duali}

Per arrivare a definirle e vedere a cosa servono dobbiamo fare alcuni passaggi preliminari. Definiamo il concetto di spazio duale. Parliamo di spazio duale non necessariamente, ma specialmente nel contesto degli spazi di Banach.

Supponiamo di avere due spazi di Banach $X, Y$ in modo assolutamente generale, poi ci concentreremo su uno specifico.
\begin{defn}
[Linearità] Possiamo considerare tutte le funzioni lineari da $X$ in $Y$
\begin{equation*}
L: X\rightarrow Y\ \ \ \ L(\alpha x_{1} + \beta x_{2}) = \alpha L(x_{1}) + \beta L(x_{2})
\end{equation*}
\end{defn}
Teniamo presente che quando parliamo di spazi di Banach spesso intendiamo spazi a dimensione infinita, in questo caso certe idee che abbiamo molto naturali e intuitive delle funzioni lineari non valgono necessariamente. In particolare la continuità \textbf{non} è necessariamente garantita in caso di funzioni lineari. In uno spazio di Banach possiamo avere funzioni lineari continue e funzioni lineari non continue.
\begin{defn}
[Continuità] Una funzione è continua se
\begin{equation*}
x_{n}\rightarrow x\ \ \implies \ \ L(x_{n})\rightarrow L(x)
\end{equation*}
\end{defn}
Ricordiamo un'altra definizione non scontata negli spazi a dimensione infinita
\begin{defn}
[Limitatezza]
Una funzione lineare è limitata se
\begin{equation*}
\exists c\ \ \Vert L(x) \Vert \leq c \Vert x \Vert \ \ \forall x\in X
\end{equation*}
\end{defn}
Vale un teorema che collega queste due definizioni.
\begin{thm}
Siano $X, Y$ due spazi di Banach qualsiasi. Una funzione lineare $L: X\rightarrow Y$ è limitata se e solo se è continua.
\end{thm}
Quello che ci interessa è un tipo particolare di funzioni tra spazi di Banach. Il secondo spazio sarà $\RR$, che è esso stesso uno spazio di Banach, è uno spazio lineare dotato di norma (il valore assoluto) e continuo rispetto a questa norma.
\begin{equation*}
L: X\rightarrow \RR \ \ \ \ X\ \text{di Banach}
\end{equation*}
\begin{defn}
Dato uno spazio $X$ di Banach, si dice \textbf{spazio duale} l'insieme di tutte le \underline{funzioni lineari continue} da $X$ in $\RR$. Queste funzioni si chiamano anche \textbf{funzionali}.
\end{defn}
Supponiamo per esempio che $X$ sia $\RR^{n}$, possiamo considerare tutte le funzioni lineari (e quindi continue, essendo $\RR^{n}$ a dimensione finita)
\begin{equation*}
L: \RR^{n}\rightarrow \RR
\end{equation*}
Come possiamo descriverle tutte? Le identifichiamo con il prodotto scalare con un altro vettore. Dato $y\in \RR^{n}$ possiamo definire $L(x) : = (y, x)$. Per esempio il duale di $\RR^{n}$ è $\RR^{n}$ stesso.

La categoria più importante di spazi di Banach che abbiamo visto fin'ora è $L^{p}$, cos'è il suo duale? Abbiamo un modo molto semplice.
\begin{thm}
Se $1 < p < + \infty $, allora il duale di $L^{p}$ è $L^{q}$ dove $q$ è l'esponente coniugato di $p$: $\frac{1}{p} + \frac{1}{q} = 1$.
\end{thm}
Sia $f\in L^{p}$ e $L(f) \in \RR$, possiamo dimostrare che $\forall L$ funzionale lineare su $L^{p}$ esiste una funzione $h\in L^{q}$ tale che $L(f) = \int hf$. Non stiamo specificando in quale dominio stiamo definendo questi spazi. Grazie alla disuguaglianza di Hölder otteniamo questo risultato. Non indaghiamo ulteriormente.

Torniamo alla definizione di duale. Una possibile notazione dello spazio duale di $X$ è $X'$. Questo spazio duale è lui stesso uno spazio di Banach. È ovviamente uno spazio vettoriale, ma dobbiamo far vedere che c'è una \textit{norma}. Un elemento del duale è un funzionale $L: X\rightarrow \RR$, questo vuol dire che dato un elemento $x\in X$ allora $L(x) \in \RR$
\begin{equation*}
\Vert L \Vert = \sup_{\Vert x \Vert = 1}| L(x)|
\end{equation*}
È vero che questo è effettivamente un numero reale? Abbiamo detto che il duale è fatto da funzionali lineari continui, ovvero limitati, quindi sicuramente questo estremo superiore è limitato. La definizione di poco fa di funzione limitata era che $\exists c$ tale che $| L(x)| \leq c \Vert x \Vert $, quindi sicuramente questo estremo superiore è minore di $c$. Si tratta di far vedere che è effettivamente una norma, abbiamo allora questo teorema.
\begin{thm}
Sia $X$ spazio di Banach. Allora $X'$ è uno spazio di Banach lui stesso rispetto alla norma
\begin{equation*}
\Vert L \Vert = \sup_{\Vert x \Vert = 1}| L(x)|
\end{equation*}
\end{thm}
\begin{defn}
[Supporto di una funzione]
Supponiamo ora di avere una funzione $f: A\subset \RR^{n}\rightarrow \RR$. Si dice supporto di $f$ l'insieme
\begin{equation*}
\supp f = \overline{\{x\in A, f(x) \neq 0\}}
\end{equation*}
È la chiusura dell'insieme dei punti dove la funzione non si annulla.
\end{defn}
Diamo la definizione di un nuovo spazio vettoriale, se abbiamo $A\subset \RR^{n}$ sappiamo cos'è $L^{p}(A)$.
\begin{defn}
Si dice $L^{p}_{\loc}$ l'insieme delle funzioni $f: A\rightarrow \RR$ tali che $\forall K\subset A$ compatto
\begin{equation*}
\exists \ \ \int_{K}| f|^{p} dx\ \ \text{finito} .
\end{equation*}
\end{defn}
È molto più grande dello spazio $L^{p}$.

\section{Funzioni test e distribuzioni}

Adesso dobbiamo definire un nuovo spazio di funzioni, che non è uno spazio di Banach.
\begin{defn}
[Spazio delle funzioni test]
Sia $A\subset \RR^{n}$ aperto. Chiamiamo funzioni $C^{\infty}$ a supporto compatto, o funzioni test, e lo indichiamo con $\Dc(A)$ l'insieme delle funzioni $f: A\rightarrow \RR$, derivabili infinite volte $f\in C^{\infty}$ e tali che il supporto $\supp f$ è compatto.
\end{defn}
Una funzione in $\Dc(A)$ viene detta \textbf{funzione test}. Queste funzioni sono un insieme estremamente piccolo. Non è banale neanche fare degli esempi in $\Dc(\RR)$. Per esempio
\begin{equation*}
f(x) =
\begin{cases}
e^{\frac{1}{x^{2} - 1}} & | x| < 1\\
0 & | x| \geq 1
\end{cases}
\end{equation*}
Somiglia a una gaussiana, ma fuori da $[ - 1, 1]$ si annulla.

\fg{0.7}{esempioFunzioneD}

Non è banale che sia $C^{\infty}$ nei punti $ - 1, 1$, ma si può dimostrare.

Nessuna funzione elementare appartiene a questo insieme.

Questo insieme non è uno spazio di Banach, se definiamo una norma, non può essere uno spazio di Banach rispetto a quella norma. Nonostante ciò vogliamo definire un concetto di convergenza.

Daremo le prossime definizioni in dimensione $1$ per alleggerire le notazioni.
\begin{defn}
[Convergenza in $\Dc$]Sia $A\subset \RR$ aperto e consideriamo l'insieme $\Dc(A)$, consideriamo una successione di funzioni $\{f_{n}\} \subset \Dc(A)$. Diciamo che $f_{n}\rightarrow f$ in $\Dc(A)$ se
\begin{enumerate}
\item $\exists K\subset A$ compatto tale che il supporto $\supp f_{n} \subset K$, ciascuna funzione è a supporto compatto, ma vogliamo che esista un compatto che contenga i supporti di tutte le funzioni.
\item $\forall k\in \NN$ la derivata di ordine $k$ della successione di funzioni converge alla derivata di ordine $k$ della funzione uniformemente $f^{(k)}_{n}\rightarrow f^{(k)}$.
\end{enumerate}
\end{defn}
Si può dimostrare che non possiamo definire questa convergenza in norma o distanza, motivo per cui l'abbiamo definita noi così. Ora possiamo finalmente arrivare alla definizione di distribuzione.
\begin{defn}
[Distribuzione]
Sia $A\subset \RR$ aperto. Chiamiamo \textbf{distribuzione} in $A$ un funzionale (lineare e continuo) su $\Dc(A)$. Quindi lo spazio delle distribuzione è il suo duale $\Dc'(A)$.
\end{defn}
Stiamo cercando di generalizzare il concetto di funzione.
\begin{thm}
A ogni funzione $u\in L^{1}_{\loc}(A)$ corrisponde una distribuzione in $\Dc'(A)$.

Indichiamo con $(u, f)$ il prodotto di dualità, cioè la distribuzione (funzionale lineare e continuo) associata ad $u$ che agisce sulla funzione test $f$
\begin{equation*}
\boxed{(u, f) = \int_{A} ufdx} \ \ \ \ u\in L^{1}_{\loc}(A) \ \ f\in \Dc(A)
\end{equation*}
\end{thm}
\begin{proof}

Dobbiamo verificare che l'integrale sia ben definito, lineare e continuo (o limitato).
\begin{itemize}
\item \textit{ben definito.} Il supporto di $f$ è un certo insieme $K\subset A$ compatto, quindi si annulla al di fuori di $K$, allora
\begin{equation*}
\int_{A} ufdx = \int_{K} ufdx
\end{equation*}ma $f$ è anche limitata (essendo continua a supporto compatto) e $u\in L^{1}_{\loc}$:
\begin{equation*}
\left| \int_{K} ufdx\right| \leq \sup | f| \left| \int_{K} udx\right| < + \infty
\end{equation*}
\item \textit{lineare}, infatti
\begin{equation*}
(u, c_{1} f_{1} + c_{2} f_{2}) = \int_{A} u(c_{1} f_{1} + c_{2} f_{2}) dx
\end{equation*}per la linearità dell'integrale
\begin{equation*}
\int_{A} u(c_{1} f_{1} + c_{2} f_{2}) dx = c_{1}\int_{A} uf_{1} dx + c_{2}\int_{A} uf_{2} dx = c_{1}(u, f_{1}) + c_{2}(u, f_{2})
\end{equation*}
\item \textit{continuo, o limitato} (sono equivalenti). Per ipotesi $f_{n}\rightarrow f$ in $\Dc(A)$ e $u\in L^{1}_{\loc}$. Allora la convergenza in $\Dc(A)$ garantisce che il supporto di tutte le $f_{n}$ sia contenuto in un compatto $K\subset A$, e inoltre $f_{n}\rightarrow f$ (e le derivate) uniformemente, cioè $ \Vert f_{n} - f \Vert_{\infty}\rightarrow 0$. Per definizione di $L^{1}_{\loc}$ la funzione $u$ ristretta al compatto $K$ è $L^{1}(K)$. Possiamo allora usare la disuguaglianza di Hölder con $p = 1, q = \infty $ per far vedere che
\begin{equation*}
\left| \int_{K} u(f_{n} - f) dx\right| \leq \Vert u \Vert_{L^{1}(K)} \Vert f_{n} - f \Vert_{\infty}\rightarrow 0
\end{equation*}allora abbiamo fatto vedere che
\begin{equation*}
\int_{A} u(f_{n} - f) dx\rightarrow 0\ \ \ \ \iff \ \ \ \ \int_{A} uf_{n} dx\rightarrow \int_{A} ufdx
\end{equation*}quindi qualsiasi funzione $L^{1}_{\loc}$ è una distribuzione su quell'insieme.
\end{itemize}
\end{proof}

Si noti che lo spazio $L^{1}_{\loc}$ è molto più grande di $L^{1}$, ad esempio banalmente qualunque polinomio non identicamente nullo non è integrabile su tutto $\RR$, ma è integrabile su qualunque compatto.

Tutte le distribuzioni si possono vedere come associate a una funzione $L^{1}_{\loc}$? No, vediamo un controesempio con la \textbf{delta di Dirac}, che definiamo come distribuzione su $\RR$. Ad ogni funzione $f\in \Dc(\RR)$ e $y\in \RR$ vogliamo associare una distribuzione $\delta_{y} \in \Dc'(\RR)$
\begin{equation*}
\boxed{(\delta_{y}, f) = f(y) \in \RR}
\end{equation*}
\begin{proof}\leavevmode
\begin{itemize}
\item ha valori in $\RR$, $\delta_{y} : \Dc(\RR)\rightarrow \RR$
\item è lineare $(\delta_{y}, c_{1} f_{1} + c_{2} f_{2}) = (c_{1} f_{1} + c_{2} f_{2})(y) = c_{1} f_{1}(y) + c_{2} f_{2}(y)$ che è proprio uguale a $c_{1}(\delta_{y}, f_{1}) + c_{2}(\delta_{y}, f_{2})$
\item è continua. Sia $f_{n}\rightarrow f$, cioè il supporto di tutte le $f_{n}$ è contenuto in $K$ compatto e $f_{n}\rightarrow f$ uniformemente, e quindi anche puntualmente:
\begin{itemize}
\item se $y\notin K$ allora $(\delta_{y}, f_{n}) = 0$ e $(\delta_{y}, f) = 0$
\item se $y\in K$ allora $(\delta_{y}, f_{n}) = f_{n}(y)\rightarrow f(y) = (\delta_{y}, f)$
\end{itemize}

\end{itemize}
\end{proof}
Spesso si scrive
\begin{equation*}
(\delta_{y}, f) = \int_{\RR} \delta_{y}(x) f(x) dx = f(y)
\end{equation*}
La delta di Dirac nasce per estrarre in modo semplice il valore di una funzione in un punto, e da qui si è poi sviluppata la teoria delle distribuzioni. Ad ogni funzione $L^{1}_{\loc}$ si può quindi associare una distribuzione, ma non viceversa, le distribuzioni sono di più.

Un altro modo di interpretare la delta di Dirac è il limite di una gaussiana
\begin{equation*}
f(x) = \frac{1}{\sqrt{2\pi}} e^{- \frac{1}{2} x^{2}} \ \ \ \ \Vert f \Vert_{L^{1}} = 1
\end{equation*}
definiamo
\begin{equation*}
f_{n}(x) = nf(nx)
\end{equation*}
\fg{0.7}{deltaGaussianaLimite}

dove
\begin{equation*}
\Vert f_{n} \Vert_{L^{1}} = \int_{\RR}| f_{n}(x)| dx = \int_{\RR} nf(nx) dx
\end{equation*}
chiamo $y = nx$ allora $dy = ndx$
\begin{equation*}
\int_{\RR} nf(nx) dx = \int_{\RR} f(y) dy = 1
\end{equation*}
Notiamo che
\begin{equation*}
f_{n}(x) = \frac{n}{\sqrt{2\pi}} e^{- \frac{1}{2}(nx)^{2}}\xrightarrow{n\rightarrow \infty} 0\ \ \forall x\neq 0
\end{equation*}
Essendo $L^{1}$, quindi anche $L^{1}_{\loc}$, le possiamo vedere come distribuzioni. Possiamo dimostrare che $\forall \varphi \in \Dc(A)$, $(f_{n}, \varphi)\rightarrow (\delta_{0}, \varphi)$.

Più uno spazio è piccolo e più il suo duale è grande, e viceversa. Lo spazio $\Dc(A)$ è molto piccolo, quindi il duale (lo spazio delle distribuzioni) sarà molto grande, ma pur essendo molto grande, avrà delle proprietà molto forti (derivabilità infinita). \textit{Stiamo cercando capre e cavoli}.

\begin{defn}
[Convergenza di una distribuzione]
Cosa vuol dire avere una successione di distribuzioni $\{\Lambda_{n}\}$ che converge a $\Lambda $ \underline{nel senso delle distribuzioni}? Come prima, se $\forall \varphi \in \Dc(A) \implies (\Lambda_{n}, \varphi)\rightarrow (\Lambda, \varphi)$. La definizione è simmetrica rispetto all'altra.
\end{defn}
Veniamo al supporto. Una distribuzione non ha una variabile, non si annulla come una funzione, non è una funzione.
\begin{defn}
[Supporto di una distribuzione]
Sia $\Lambda \subset \Dc'(A)$. Consideriamo $B\subset A$ e $\varphi \in \Dc(B)$. Supponiamo che $(\Lambda, \varphi) = 0\ \ \forall \varphi $. Questo è il modo per dire che "si annulla", diciamo
\begin{equation*}
\boxed{
\supp \Lambda =
\left(\bigcup_{\substack{B\subset A\\B\text{aperti}}} \{B: (\Lambda, \varphi) = 0, \ \forall \varphi \in \Dc(B)\}\right)^{C}
}
\end{equation*}
\end{defn}

NB. Quando diciamo che a $u\in L^{1}_{\loc}(A)$ corrisponde una distribuzione, diciamo che esiste una $\Lambda_{u}$ distribuzione definita come $(\Lambda_{u}, \varphi) = \int_{A} u\varphi $, ma talvolta scriveremo direttamente $u$ con abuso di notazione.
\begin{defn}
[Derivata di una distribuzione]
Sia $\Lambda \in \Dc'(A)$ una distribuzione. Definiamo la sua distribuzione derivata $\Lambda'\in \Dc'(A)$. Per ogni $\varphi \in \Dc(A)$ dobbiamo definire quanto vale $(\Lambda', \varphi)$
\begin{equation*}
\boxed{(\Lambda', \varphi) = - (\Lambda, \varphi')}
\end{equation*}
\end{defn}
Questo somiglia molto all'integrazione per parti (in realtà è molto più che una somiglianza). Controlliamo che sia una distribuzione
\begin{itemize}
\item definita $\Lambda': \Dc(A)\rightarrow \RR$, sì per come è costruita, $\varphi'$ è comunque definito essendo $C^{\infty}$ a supporto compatto
\item lineare:
\begin{equation*}
(\Lambda', a_{1} \varphi_{1} + a_{2} \varphi_{2}) = a_{1}(\Lambda', \varphi_{1}) + a_{2}(\Lambda', \varphi_{2})
\end{equation*}
\item continuo:
\begin{equation*}
\varphi_{k}\rightarrow \varphi \ \ \text{in} \ \Dc(A) \ \ \implies \ \ (\Lambda', \varphi_{k})\rightarrow (\Lambda', \varphi)
\end{equation*}

infatti, dato che $\varphi'_{k}\rightarrow \varphi'$
\begin{gather*}
(\Lambda', \varphi_{k}) = - (\Lambda, \varphi'_{k})\rightarrow - (\Lambda, \varphi') = (\Lambda', \varphi)
\end{gather*}
\end{itemize}

Andiamo ora a capire che ha le stesse proprietà di una derivata. Consideriamo una funzione derivabile $u\in C^{1}(A) \subset L^{1}_{\loc}(A)$, ma allora è associata ad una distribuzione, che continueremo a chiamare $u$
\begin{equation*}
(u, \varphi) = \int_{A} u(x) \varphi (x) dx
\end{equation*}
Sappiamo che esiste $u'\in C^{0}(A) \subset L^{1}_{\loc}$, associata a sua a volta ad una distribuzione
\begin{gather*}
\underbrace{(u', \varphi) = - (u, \varphi')}_{\text{definizione di prima}}\\
(u', \varphi) = \int_{A} u'(x) \varphi (x) dx = \cancel{u(b) \varphi (b)} - \cancel{u(a) \varphi (a)} - \int^{b}_{a} u(x) \varphi'(x) dx\ \ \ \ A = [a, b]
\end{gather*}
i termini si cancellano perché $\varphi $ è a supporto compatto, che è contenuto in $A$, quindi $\varphi $ si annulla agli estremi $\varphi (a) = 0 = \varphi (b)$.

\textit{Esempio.}

Consideriamo la funzione
\fg{0.6}{rampa}
\begin{equation*}
u(x) =
\begin{cases}
x & x \geq 0\\
0 & x < 0
\end{cases}
\end{equation*}
Non è derivabile nell'origine, ma è $L^{1}_{\loc}$, quindi derivabile nel senso delle distribuzioni. Sia $\varphi \in \Dc(\RR)$
\begin{equation*}
(u', \varphi) = - (u, \varphi') = - \int_{\RR} u(x) \varphi'(x) dx = - \int^{M}_{0} x\varphi'(x) dx
\end{equation*}
$M$ perché $\varphi $ è a supporto compatto, quindi ci sarà un $M$; applico l'integrazione per parti
\begin{align*}
- \int^{M}_{0} x\varphi'(x) dx & = - \cancel{[x\varphi (x)]^{M}_{0}} + \int^{M}_{0} \varphi (x) dx = \int^{M}_{0} \varphi (x) dx\\
 & = \int^{+ \infty}_{0} \varphi (x) dx = \int_{\RR} H(x) \varphi (x) dx = (H, \varphi)
\end{align*}
dove $H(x)$ è la \textbf{funzione di Heaviside}, che abbiamo interpretato come distribuzione
\begin{equation*}
H(x) =
\begin{cases}
1 & x \geq 0\\
0 & x < 0
\end{cases}
\end{equation*}
NB. Tutto quello che vediamo riguardante gli spazi $L^{p}$ riguarda gli integrali, quindi che valga $0$ o $1$ quando $x = 0$ è del tutto irrilevante.

Allora la derivata di $u$, cioè $u'$, nel senso delle distribuzioni, è $H$.

Sembra che non abbiamo fatto granché se non togliere di mezzo un punto angoloso: spingiamoci oltre e deriviamo la funzione di Heaviside, che non è nemmeno continua.
\begin{align*}
(H', \varphi) & = - (H, \varphi') \ \ \forall \varphi \in \Dc(\RR)\\
 & = - \int_{\RR} H(x) \varphi'(x) dx = - \int^{+ \infty}_{0} \varphi'(x) dx\\
 & = \varphi (0) - \lim_{x\rightarrow + \infty} \varphi (x) = \varphi (0) = (\delta_{0}, \varphi)
\end{align*}
La derivata di $H$ nel senso delle distribuzioni è la distribuzione delta di Dirac (centrata nel punto di discontinuità). Può sembrare strano derivare la delta, ma si può fare.
\begin{equation*}
(\delta'_{a}, \varphi) = - (\delta_{a}, \varphi') = - \varphi'(a)
\end{equation*}
estendiamo a un ordine $n$
\begin{equation*}
\left(\delta^{(n)}_{a}, \varphi \right) = (- 1)^{n} \varphi^{(n)}(a)
\end{equation*}

\section{Funzioni a decrescita rapida e distribuzioni temperate}

Lo spazio delle $\varphi $ è lo spazio delle funzioni test, delle funzioni $C^{\infty}$ a supporto compatto, uno spazio molto stringente. Cerchiamo di allentarlo un po' introducendo un nuovo spazio chiamato \textbf{spazio delle funzioni a decrescita rapida} (o \textbf{spazio di Schwartz}.\footnote{Si noti che si tratta di un matematico differente da quello del teorema sulle derivate seconde, Karl Hermann Amandus Schwarz (1843 - 1921). Qui si parla di Laurent Schwartz (1915 - 2002), medaglia Fields nel 1950.})
\begin{defn}
[Spazio delle funzioni a decrescita rapida]
Diciamo che $f\in \Sc(\RR)$ se $\boxed{f\in C^{\infty}}$ e $\boxed{x^{k} f^{(n)}(x)}$ è \underline{limitata} $\forall k, n\in \NN$.
\end{defn}
Le definiamo su tutto $\RR$. Non chiediamo che si annullino fuori da un compatto, ma che si annullino più rapidamente di qualsiasi polinomio insieme a tutte le derivate per $x\rightarrow \pm \infty $. Una gaussiana per esempio tende a $0$ più velocemente di qualsiasi polinomio con le sue derivate, ma non si annulla mai. Questo spazio $\Sc(\RR)$ è tale che
\begin{equation*}
\Dc(\RR) \subset \Sc(\RR)
\end{equation*}
Come si definisce la convergenza in questo insieme?
\begin{defn}
[Convergenza in $\Sc$]Sia una successione $\{f_{n}\} \subset \Sc(\RR)$. Diciamo che $f_{n}\rightarrow f$ in $\Sc(\RR)$ se $x^{k} f^{(m)}_{n}(x)\xrightarrow{n\rightarrow + \infty} x^{k} f^{(m)}(x)$ uniformemente $\forall k, m\in \NN$.
\end{defn}
Anche questo non è uno spazio di Banach e non c'è norma, motivo per cui \textit{definiamo} così la successione.
\begin{defn}
[Distribuzione temperata]
Un funzionale lineare $\Lambda : \Sc(\RR)\rightarrow \RR$ continuo si dice distribuzione temperata.
\end{defn}
Una distribuzione temperata è anche una distribuzione normale, in altre parole, lo spazio delle distribuzioni temperate $\Sc'(\RR)$ è contenuto in $\Dc'(\RR)$
\begin{equation*}
\Sc'(\RR) \subset \Dc'(\RR)
\end{equation*}
In generale se abbiamo due spazi qualsiasi $A\subset B$, i loro duali sono legati da $B'\subset A'$.
\begin{figure}[htpb]
\centering

\tikzset{every picture/.style = {line width = 0.75pt}} %set default line width to 0.75pt

\begin{tikzpicture}[x = 0.75pt, y = 0.75pt, yscale = - 1, xscale = 1]
%uncomment if require: \path (0, 181); %set diagram left start at 0, and has height of 181

%Shape: Ellipse [id: dp7503563455876929]
\draw (80, 108.68) .. controls (80, 74.82) and (122.53, 47.36) .. (175, 47.36) .. controls (227.47, 47.36) and (270, 74.82) .. (270, 108.68) .. controls (270, 142.55) and (227.47, 170) .. (175, 170) .. controls (122.53, 170) and (80, 142.55) .. (80, 108.68) - - cycle;
%Shape: Ellipse [id: dp5108599722199094]
\draw (101.55, 108.68) .. controls (101.55, 94.32) and (116.87, 82.68) .. (135.77, 82.68) .. controls (154.68, 82.68) and (170, 94.32) .. (170, 108.68) .. controls (170, 123.04) and (154.68, 134.68) .. (135.77, 134.68) .. controls (116.87, 134.68) and (101.55, 123.04) .. (101.55, 108.68) - - cycle;
%Shape: Ellipse [id: dp5851131130928555]
\draw (330, 108.68) .. controls (330, 74.82) and (372.53, 47.36) .. (425, 47.36) .. controls (477.47, 47.36) and (520, 74.82) .. (520, 108.68) .. controls (520, 142.55) and (477.47, 170) .. (425, 170) .. controls (372.53, 170) and (330, 142.55) .. (330, 108.68) - - cycle;
%Shape: Ellipse [id: dp4435469561254657]
\draw (351.55, 108.68) .. controls (351.55, 94.32) and (366.87, 82.68) .. (385.77, 82.68) .. controls (404.68, 82.68) and (420, 94.32) .. (420, 108.68) .. controls (420, 123.04) and (404.68, 134.68) .. (385.77, 134.68) .. controls (366.87, 134.68) and (351.55, 123.04) .. (351.55, 108.68) - - cycle;

% Text Node
\draw (205, 101.08) node [anchor = north west][inner sep = 0.75pt] {$\Sc(\RR)$};
% Text Node
\draw (114.77, 101.08) node [anchor = north west][inner sep = 0.75pt] {$\Dc(\RR)$};
% Text Node
\draw (455, 101.08) node [anchor = north west][inner sep = 0.75pt] {$\Dc'(\RR)$};
% Text Node
\draw (364.77, 101.08) node [anchor = north west][inner sep = 0.75pt] {$\Sc'(\RR)$};
% Text Node
\draw (151, 11) node [anchor = north west][inner sep = 0.75pt] [align = left] {primale};
% Text Node
\draw (411, 11) node [anchor = north west][inner sep = 0.75pt] [align = left] {duale};

\end{tikzpicture}
\end{figure}
\FloatBarrier
Questa riduzione dello spazio, è una riduzione stretta, cioè perdiamo alcuni elementi, vediamo un esempio.
\begin{equation*}
u = e^{x^{2}} \in L^{1}_{\loc} \ \ \implies \ \ \text{c'è una distribuzione associata}
\end{equation*}
tuttavia se consideriamo come funzione test una funzione di $\Sc(\RR)$ $\varphi = e^{- x^{2}}$
\begin{equation*}
(u, \varphi) = \int_{\RR} u\varphi dx = \int_{\RR} dx = + \infty \ \ \implies \ \ \text{non è temperata}
\end{equation*}
\begin{defn}
[Convergenza di distribuzioni temperate]
Sia $u_{n}$ una successione in $\{u_{n}\} \subset \Sc'(\RR)$. Diciamo che $u_{n}\rightarrow u\in \Sc'(\RR)$ se per ogni funzione test a decrescita rapida $\varphi \in \Sc(\RR)$, si ha $(u_{n}, \varphi)\rightarrow (u, \varphi)$.
\end{defn}
La definizione di derivata è la stessa.

Ora chiediamoci, se consideriamo una funzione $u\in L^{1}_{\loc}$, possiamo dire che vi è associata una distribuzione temperata? Abbiamo questo risultato.
\begin{thm}
Se $\exists n\in \NN$ tale che $(1 + | x|)^{- n} u(x) \in L^{1}(\RR)$ allora esiste una distribuzione temperata associata $u\in \Sc'(\RR)$.
\end{thm}
\begin{thm}
Se $u\in L^{p}(\RR)$ per un qualsiasi $p\in [1, + \infty]$ allora esiste una distribuzione temperata associata $u\in \Sc'(\RR)$.
\end{thm}

\chapter{Trasformate integrali}

\section{Introduzione}

È la versione \textit{continua} della serie di Fourier. Vedremo tutto sulla retta reale, ma si può applicare anche in $\RR^{n}$. Consideriamo $\Dc(\RR) \subset L^{p}(\RR) \ \ \forall p\in [1, + \infty]$, questa inclusione è densa $\forall p < + \infty $ (escluso).
\begin{defn}
[Inclusione densa]
Dire che l'inclusione è densa significa dire che $\forall p < + \infty, \forall f\in L^{p}(\RR)$ esiste $\varphi_{n} \in \Dc(\RR)$ tale che $\varphi_{n}\rightarrow f$ in senso $L^{p}$, cioè $ \Vert \varphi_{n} - f \Vert_{L^{p}}\rightarrow 0$.
\end{defn}
Posso \textit{approssimare} ogni funzione in $L^{p}$ con $p < + \infty $ (funzioni anche molto irregolari e discontinue) con funzioni di $\Dc(\RR)$ (funzioni molto regolari e con cui è più semplice fare i conti), nel senso della norma ovviamente.

Introduciamo lo spazio
\begin{equation*}
\boxed{C^{0}_{0}(\RR) = \left\{f\in C^{0}(\RR), \lim_{x\rightarrow \pm \infty} f(x) = 0\right\}} \subset L^{\infty}(\RR)
\end{equation*}
Le funzioni in $C^{0}_{0}(\RR)$ sono automaticamente limitate, quindi contenute nello spazio $L^{\infty}(\RR)$ delle funzioni \textit{essenzialmente} limitate. Questa inclusione \textbf{non} è densa.

Sia $f\in L^{1}(\RR)$ definiamo la funzione traslata
\begin{equation*}
f_{y}(x) = f(x - y)
\end{equation*}
\begin{thm}
Sia $f\in L^{1}(\RR)$, allora
\begin{equation*}
\lim\limits_{y\rightarrow 0} \Vert f_{y} - f \Vert_{L^{1}} = 0
\end{equation*}
\end{thm}
\begin{proof}

È basata sulla densità a cui abbiamo accennato. Sia $\varepsilon > 0$, $\exists \varphi \in \Dc(\RR)$ tale che $ \Vert f - \varphi \Vert < \varepsilon $. Se trasliamo una funzione, essa non cambia di norma $\implies \ \Vert f_{y} - \varphi_{y} \Vert < \varepsilon $. Chiamiamo $K$ un certo insieme che contiene sia il supporto di $\varphi $ che della sua traslata $\varphi_{y}$, per un certo $y$ sufficientemente piccolo. Allora\footnote{Teorema di Heine - Cantor. Una funzione continua su un compatto è uniformemente continua.} la funzione $\varphi $ è uniformemente continua, cioè
\begin{gather*}
\forall \varepsilon > 0\ \ \exists \delta \ \ | \varphi (x) - \varphi (y)| < \varepsilon \ \ \text{purché} \ | x - y| < \delta \\
\Updownarrow \\
\forall \varepsilon > 0\ \ \exists \delta \ \ | \varphi (x - y) - \varphi (x)| < \varepsilon \ \ \text{purché} \ | y| < \delta
\end{gather*}
il delta che otteniamo \textit{non} dipende dal punto.

Dall'uniforme continuità deduciamo
\begin{equation*}
\Vert \varphi_{y} - \varphi \Vert = \int_{K}| \varphi (x - y) - \varphi (x)| dx < \varepsilon | K|
\end{equation*}
Veniamo ora a
\begin{align*}
\Vert f_{y} - f \Vert & = \Vert (f_{y} - \varphi_{y}) + (\varphi_{y} - \varphi) + (\varphi - f) \Vert \\
 & \leq \Vert f_{y} - \varphi_{y} \Vert + \Vert \varphi_{y} - \varphi \Vert + \Vert \varphi - f \Vert \\
 & < \varepsilon + \varepsilon | K| + \varepsilon \\
 & = \varepsilon (2 + | K|)
\end{align*}
dove $(2 + | K|)$ è costante e $\varepsilon $ è arbitrario, cioè $ \Vert f_{y} - f \Vert $ è tanto piccola quanto vogliamo.
\end{proof}

Prima di procedere riassumiamo brevemente le inclusioni fra gli spazi che abbiamo visto

\begin{center}

\begin{tabular}{lc}
$\Dc(\RR) \subset L^{p}(\RR)$ & densa $\forall p < + \infty $ \\
$\Dc(\RR) \subset L^{p}(\RR)$ & densa $\forall p < + \infty $ \\
$\Dc(\RR) \subset \Sc(\RR)$ & standard \\
$\Sc'(\RR) \subset \Dc'(\RR)$ & standard \\
$C^{0}_{0}(\RR) \subset L^{\infty}(\RR)$ & standard \\

\end{tabular}
\end{center}

\section{Trasformata di Fourier}

I numeri complessi entrano necessariamente nella trasformata di Fourier, quindi considereremo sempre funzioni $f: \RR\rightarrow \CC$.

Consideriamo $f\in L^{1}(\RR)$, definiamo \textbf{trasformata di Fourier} della funzione $f$
\begin{equation*}
\boxed{\hat{f}(z) = \int_{\RR} e^{- izx} f(x) dx} \ \ \ \ f\in L^{1}(\RR)
\end{equation*}
Ci ricordiamo che i coefficienti di Fourier erano
\begin{equation*}
c_{n} = \frac{1}{\pi}\int^{\pi}_{- \pi} e^{- inx} f(x) dx
\end{equation*}
È corretta la definizione?
\begin{equation*}
| f(x)| = \left| e^{- ixz} f(x)\right|
\end{equation*}
ma $f$, e quindi anche $| f| $, è integrabile secondo Lebesgue, quindi anche il secondo membro è integrabile e convergente. Ha senso dare questa definizione! Vediamo di capire le proprietà, riassunte nel teorema fondamentale.
\begin{thm}
[di Riemann - Lebesgue] Sia $f\in L^{1}(\RR)$, allora
\begin{equation*}
\boxed{\hat{f} \in C^{0}_{0}(\RR)} \ \ \ \ \boxed{\Vert \hat{f} \Vert_{\infty} \leq \Vert f \Vert_{L^{1}}}
\end{equation*}
\end{thm}
La trasformata ha proprietà molto diverse e molto più regolari della $f$, che è banalmente integrabile, sta in uno spazio completamente diverso.

\begin{proof}

\textit{Continuità.}

Sia $z_{n}\rightarrow z\in \RR$
\begin{equation*}
\lim\limits_{n\rightarrow + \infty}\hat{f}(z_{n}) = \lim\limits_{n\rightarrow + \infty}\int_{\RR} e^{- iz_{n} x} f(x) dx
\end{equation*}
Notiamo che abbiamo la convergenza puntuale, quasi $\forall x$. Se usassimo l'integrale di Riemann non avremmo questo risultato, con Lebesgue, sì, usando la convergenza dominata.
\begin{equation*}
g_{n}\rightarrow g\ \text{q.o.}, \ \exists h\in L^{1}, \ | g_{n}| \leq h\ \forall n\ \ \implies \ \ \lim\limits_{n\rightarrow + \infty}\int g_{n} = \int \lim\limits_{n\rightarrow + \infty} g_{n}
\end{equation*}
qui possiamo maggiorare
\begin{equation*}
\underbrace{\left| e^{- iz_{n} x} f(x)\right|}_{| g_{n}|} \leq \underbrace{| f(x)|}_{h} \in L^{1}
\end{equation*}
quindi
\begin{equation*}
\lim\limits_{n\rightarrow + \infty}\int_{\RR} e^{- iz_{n} x} f(x) dx = \int_{\RR}\lim\limits_{n\rightarrow + \infty} e^{- iz_{n} x} f(x) dx = \int_{\RR} e^{- izx} f(x) dx = \hat{f}(z)
\end{equation*}
\textit{Disuguaglianza delle norme.}

Chiaramente poi possiamo scrivere
\begin{equation*}
\Vert \hat{f} \Vert_{\infty} = | \hat{f}(z)| = \left| \int_{\RR} e^{- ixz} f(x) dx\right| \leq \int_{\RR}\left| e^{- ixz} f(x)\right| dx \leq \int_{\RR}| f(x)| dx = \Vert f \Vert_{L^{1}}
\end{equation*}
\textit{Ultima parte.}

Ci resta da far vedere che
\begin{equation*}
\lim\limits_{z\rightarrow \pm \infty}\hat{f}(z) = 0
\end{equation*}
Ricordiamo che $ - 1 = e^{i\pi}$, scriviamo
\begin{equation*}
\hat{f}(z) = - e^{i\pi}\int_{\RR} e^{- ixz} f(x) dx = - \int_{\RR} e^{- i(xz - \pi)} f(x) dx = - \int_{\RR} e^{- iz\left(x - \frac{\pi}{z}\right)} f(x) dx
\end{equation*}
cambiamo la variabile e ricordiamo che l'integrale è in $dx$
\begin{equation*}
x' = x - \frac{\pi}{z}
\end{equation*}
allora
\begin{equation*}
- \int_{\RR} e^{- iz\left(x - \frac{\pi}{z}\right)} f(x) dx = - \int_{\RR} e^{- izx'} f\left(x' + \frac{\pi}{z}\right) dx' = - \int_{\RR} e^{- izx} f\left(x + \frac{\pi}{z}\right) dx
\end{equation*}
Scriviamo ora il doppio di $\hat{f}$
\begin{align*}
2\hat{f}(z) & = \int_{\RR} e^{- ixz} f(x) dx - \int_{\RR} e^{- izx} f\left(x + \frac{\pi}{z}\right) dx\\
 & = \int_{\RR}\left[e^{- ixz} f(x) - e^{- izx} f\left(x + \frac{\pi}{z}\right)\right] dx\\
 & = \int_{\RR} e^{- ixz}\left[f(x) - f\left(x + \frac{\pi}{z}\right)\right] dx
\end{align*}
passiamo al modulo
\begin{align*}
2| \hat{f}(z)| & \leq \int_{\RR}\left| e^{- ixz}\left[f(x) - f\left(x + \frac{\pi}{z}\right)\right]\right| dx\\
 & = \int_{\RR}\left| f(x) - f\left(x + \frac{\pi}{z}\right)\right| dx\\
 & = \Vert f - f_{- \pi /z} \Vert_{L^{1}}
\end{align*}
Quando $z\rightarrow \pm \infty $ la traslazione $ - \pi /z\rightarrow 0$ e quindi, per quanto detto prima sulla traslazione di funzioni $L^{1}$ abbiamo $2| \hat{f}(z)| \rightarrow 0$.

\end{proof}

Questo vuol dire che se chiamiamo
\begin{equation*}
\Fc : L^{1}(\RR)\rightarrow C^{0}_{0}(\RR)
\end{equation*}
abbiamo un operatore trasformata di Fourier che è, ovviamente, lineare, e, grazie alla disuguaglianza $ \Vert \hat{f} \Vert_{\infty} \leq \Vert f \Vert_{L^{1}}$ anche limitata (e quindi equivalentemente anche continua, per quanto precedentemente). Cominciamo a verificare alcune proprietà.

Notazione: indichiamo con $\Fc(f(x), z) = \hat{f}(z)$ la trasformata di $f(x)$ scritta in funzione di $z$.
\begin{itemize}
\item Traslazione di $y\in \RR$.
\begin{align*}
\Fc(f(x - y), z) & = \int e^{- ixz} f(x - y) dx & x' = x - y\\
 & = \int e^{- ix'z - iyz} f(x') dx' & \\
 & = e^{- iyz}\int e^{- ixz} f(x) dx & \\
 & = e^{- iyz}\hat{f}(z) & \\
 & = e^{- iyz}\Fc(f(x), z) &
\end{align*}
\item Il contrario

\begin{align*}
\Fc\left(e^{iyx} f(x), z\right) & = \int e^{- ixz} e^{iyx} f(x) dx\\
 & = \int e^{- ix(z - y)} f(x) dx\\
 & = \hat{f}(z - y)\\
 & = \Fc(f(x), z - y)
\end{align*}
\item $\Fc\left(\overline{f(x)}, z\right) = \overline{\hat{f}(- z)}$
\item $f$ pari $\implies $ $\hat{f}$ pari
\item $f$ dispari $\implies $ $\hat{f}$ dispari
\item $f$ pari e reale $\implies $ $\hat{f}$ pari e reale
\item $f$ dispari e reale $\implies $ $\hat{f}$ dispari e immaginaria
\end{itemize}

Cerchiamo di capire se una trasformata di Fourier è derivabile, quindi anche $C^{1}$.
\begin{thm}
[Derivata della trasformata]
Sia $f\in L^{1}$, $g(x) = xf(x)$, $g\in L^{1}$. Allora $\hat{f} \in C^{1}$ e
\begin{equation*}
\hat{f}'(z) = - i\hat{g}(z) \ \ \text{cioè} \ \ \boxed{\frac{d^{n}}{dz^{n}}\Fc(f(x), z) = \Fc\left((- ix)^{n} \cdot f(x), z\right)}
\end{equation*}
\end{thm}
Chiedere che anche $xf(x)$ stia in $L^{1}$ significa che la $f$ deve in qualche senso tendere a $0$ abbastanza rapidamente, vogliamo che l'integrale su $\RR$ converga.

\begin{proof}

Calcoliamo il limite del rapporto incrementale
\begin{align*}
\lim\limits_{h\rightarrow 0}\frac{\hat{f}(z + h) - \hat{f}(z)}{h} & = \lim\limits_{h\rightarrow 0}\frac{1}{h}\left(\int e^{- ix(z + h)} f(x) dx - \int e^{- ixz} f(x) dx\right)\\
 & = \lim\limits_{h\rightarrow 0}\int e^{- ixz}\frac{e^{- ixh} - 1}{h} f(x) dx\\
 & = \int e^{- ixz}(- ix) f(x) dx\\
 & = - i\Fc(xf(x), z)\\
 & = - i\hat{g}(z)
\end{align*}
dove possiamo passare al limite sotto il segno di integrale perché in generale
\begin{equation*}
\left| e^{i\lambda} - 1\right| \leq | \lambda | \ \ \lambda \in \RR \ \ \ \ \ \ \ \ \lambda = - xh\ \ \implies \ \ \text{l'integrale converge per conv. dom.}
\end{equation*}
\end{proof}
\begin{thm}
[Trasformata della derivata]
Sia $f\in L^{1} \cap C^{1}$, $f'\in L^{1}$.
\begin{equation*}
\boxed{\Fc\left(\frac{d^{n}}{dx^{n}} f(x), z\right) = (iz)^{n} \cdot \Fc(f(x), z)}
\end{equation*}
\end{thm}
Tutto questo è vero purché le cose scritte abbiano senso, da cui le ipotesi.

\section{Trasformazione inversa}

\begin{thm}
Sia $f\in L^{1}$ e tale che $\hat{f} \in L^{1}$. Allora $f\in C^{0}_{0}$ e
\begin{equation*}
\boxed{f(x) = \frac{1}{2\pi}\int e^{ixz}\hat{f}(z) dz}
\end{equation*}
\end{thm}
Perché le richieste sugli spazi di appartenenza? Supponiamo che questa sia l'antitrasformata, ha senso? È corretta? C'è in generale questo problema
\begin{equation*}
\begin{array}{c c c}
f & \rightarrow & \hat{f}\\
\in L^{1} & & \in C^{0}_{0}
\end{array} \ \ \ \ \ \ \ \ \ \ \ \
\begin{array}{c c c}
\hat{f} & \rightarrow & f\\
\in L^{1} & & \in C^{0}_{0}
\end{array}
\end{equation*}
La trasformata di Fourier non è simmetrica, lavora su spazi diversi. Se vogliamo che sia simmetrica nel senso che mandi uno spazio nello stesso spazio e ci permetta di tornare indietro possiamo farlo prendendo uno spazio un po' strano: $L^{1} \cap C^{0}_{0}$, vale infatti (non dimostriamo)
\begin{equation*}
\Fc : L^{1} \cap C^{0}_{0}\rightarrow L^{1} \cap C^{0}_{0}
\end{equation*}
\textbf{Problema!} Lo spazio $L^{1} \cap C^{0}_{0}$ non è uno spazio di Banach, non è completo né rispetto alla norma $L^{1}$ né rispetto alla norma $L^{\infty}$. Cercheremo ora di estendere la definizione allo spazio $L^{2}$, che oltre a essere di Banach è anche di Hilbert
\begin{equation*}
\Fc : L^{2}\rightarrow L^{2}
\end{equation*}

\section{Convoluzione}

\begin{thm}
Siano $u, v\in L^{1}(\RR)$. Allora la funzione
\begin{equation*}
\boxed{(u*v)(x) = \int_{\RR} u(x - t) v(t) dt}
\end{equation*}
è definita per \underline{quasi ogni} $x\in \RR$, $(u*v) \in L^{1}$ e inoltre, stando in $L^{1}$, avrà una norma $ \Vert u*v \Vert_{L^{1}} \leq \Vert u \Vert_{L^{1}} \Vert v \Vert_{L^{1}}$.
\end{thm}
Discende direttamente dal teorema di Fubini.
\begin{itemize}
\item È simmetrica

\begin{align*}
(u*v)(x) & = \int_{\RR} u(x - t) v(t) dt & t' = x - t\\
 & = \int_{\RR} u(t') v(x - t') dt' & \\
 & = (v*u)(x) &
\end{align*}
\item È associativa
\begin{align*}
(u*v) *w & = u*(v*w)
\end{align*}
\item Supponiamo di avere

\begin{gather*}
u(x) =
\begin{cases}
n & x\in \left[0, \frac{1}{n}\right]\\
0 & \text{altrove}
\end{cases} \ \ \ \ \Vert u \Vert = 1\\
(u*v)(x) = \int_{\RR} u(x - t) v(t) dt = n\int^{x}_{x - \frac{1}{n}} v(t) dt
\end{gather*}

è la media della funzione $v$ calcolata in un intorno di $x$. L'effetto della convoluzione è in qualche modo di smussare il valore della funzione $v$, in ogni punto sostituiamo la media in un suo intorno.
\end{itemize}

Qual è il legame con la trasformata di Fourier? Siano $u, v\in L^{1}$, calcoliamo la trasformata di $u*v$, dato che sta in $L^{1}$
\begin{align*}
\Fc(u*v, z) & = \int e^{- ixz}\left[\int u(x - t) v(t) dt\right] dx\\
 & = \int e^{- iz(x - t)} e^{- izt} u(x - t) v(t) dtdx\\
 & = \int e^{- izt} v(t)\left[\int e^{- iz(x - t)} u(x - t) dx\right] dt
\end{align*}
con un cambio di variabile $x' = x - t$
\begin{align*}
\int e^{- izt} v(t)\left[\int e^{- iz(x - t)} u(x - t) dx\right] dt & = \int e^{- izt} v(t)\left[\int e^{- izx'} u(x') dx'\right] dt\\
 & = \hat{u}(z)\hat{v}(z)
\end{align*}
Vale anche il contrario
\begin{equation*}
\Fc^{- 1}(\hat{u} *\hat{v}, x) = u(x) v(x)
\end{equation*}

\section{Estensione della trasformata di Fourier}

Introduzione
\begin{itemize}
\item Ci serviremo della proprietà che $\Dc(\RR) \subset \Sc(\RR) \subset L^{p}(\RR)$ è un'inclusione densa per ogni $p < + \infty $
\item Lavoriamo in $\boxed{\Sc(\RR)}$ per estendere la trasformata perché la trasformata di una funzione in $\Sc(\RR)$ è ancora una funzione in $\Sc(\RR)$, mentre la trasformata di una funzione in $\Dc(\RR)$ è ancora $C^{\infty}$, ma non più necessariamente a supporto compatto.
\end{itemize}

\subsection{Allo spazio di funzioni a descrescenza rapida}

\begin{thm}
La trasformata agisce su
\begin{equation*}
\Fc : \Sc(\RR)\rightarrow \Sc(\RR)
\end{equation*}
\end{thm}
\begin{proof}

Lo spazio di partenza ha senso essendo densamente incluso in $L^{p}(\RR), p = 1$.

Sia $u\in S(\RR)$, consideriamo $v(x) = x^{N} u(x) \in \Sc(\RR) \subset L^{1}(\RR)$.
\begin{itemize}
\item Trasformata della derivata di $v$ $M$ volte
\begin{equation*}
\Fc\left(v^{(M)}, z\right) = (iz)^{M}\hat{v} \in L^{\infty}(\RR)
\end{equation*}

dato che $v$ sta in $L^{1}$, $\hat{v}$ sta in $C^{0}_{0}$, quindi la quantità $(iz)^{M}\hat{v}$ è limitata
\item Derivata della trasformata di $u$ $N$ volte
\begin{equation*}
\hat{u}^{(N)} = (- i)^{N}\hat{v}
\end{equation*}

moltiplichiamola per $z^{M}$, il termine che otteniamo a destra abbiamo appena fatto vedere che, a meno di coefficienti $i$ opportunamente elevati, è limitato
\begin{equation*}
z^{M}\hat{u}^{(N)} = z^{M}(- i)^{N}\hat{v} \ \ \boxed{\in L^{\infty}(\RR)} \ \ \forall N, M \geq 0
\end{equation*}ma questa è proprio la definizione di funzione a decrescenza rapida, quindi $\hat{u} \in \Sc(\RR)$ e abbiamo dimostrato la tesi.
\end{itemize}
\end{proof}

\subsection{Allo spazio $L^{2}$}

Vogliamo mostrare che
\begin{equation*}
\Fc : L^{2}(\RR)\rightarrow L^{2}(\RR)
\end{equation*}
Non possiamo definirla in $L^{2}$ come $\int e^{- ixz} u(x) dx$ perché se $u\in L^{2}$ non abbiamo nessuna garanzia che la funzione sia integrabile. Partiamo definendo la trasformata in $L^{1}(\RR) \cap L^{2}(\RR)$.
\begin{thm}
[di Plancherel]
Sia $u\in L^{1}(\RR) \cap L^{2}(\RR)$, allora
\begin{equation*}
\boxed{\Vert \hat{u} \Vert_{L^{2}} = \sqrt{2\pi} \Vert u \Vert_{L^{2}}} \ \ \text{quindi} \ \ \boxed{\hat{u} \in L^{2}(\RR)}
\end{equation*}
\end{thm}
\begin{proof}\leavevmode
\begin{itemize}
\item \textit{Dimostrazione nel caso di}$u\in \Sc(\RR)$

Dato che $u\in \Sc(\RR)$, allora $u\in L^{2}$. Ci serve partire da $u\in \Sc(\RR)$ per poter usare correttamente il risultato di inversione, in accordo a quanto appena detto sull'estensione a $\Fc : \Sc(\RR)\rightarrow \Sc(\RR)$.

\begin{align*}
\Vert u \Vert^{2}_{L^{2}} & = \int | u(x)|^{2} dx = \int \overline{u(x)} u(x) dx & \text{essendo} \ u\in L^{2}(\RR)\\
u(x) & = \frac{1}{2\pi}\int e^{ixz}\hat{u}(z) dz & \text{essendo} \ u\in \Sc(\RR)
\end{align*}

Sostituiamo

\begin{align*}
\Vert u \Vert^{2}_{L^{2}} & = \frac{1}{2\pi}\int \overline{u(x)}\int e^{ixz}\hat{u}(z) dzdx & \\
 & = \frac{1}{2\pi}\int \hat{u}(z)\int \overline{e^{- ixz} u(x)} dxdz & \text{(Fubini)}\\
 & = \frac{1}{2\pi}\int \hat{u}(z)\overline{\hat{u}(z)} dz & \\
 & = \frac{1}{2\pi} \Vert \hat{u} \Vert^{2}_{L^{2}} \ \ \implies \ \ \Vert \hat{u} \Vert_{L^{2}} = \sqrt{2\pi} \Vert u \Vert_{L^{2}} &
\end{align*}

Stando $u\in L^{2}$, ciò che sta a destra dell'uguale è un numero finito, quindi anche ciò che è a sinistra, quindi $\hat{u}$ appartiene ad $L^{2}$.
\item \textit{Estensione a}$u\in L^{2}(\RR)$

Per \textbf{densità} dell'inclusione $\Sc(\RR) \subset L^{2}(\RR)$,
\begin{itemize}
\item posso costruire una successione approssimante
\begin{equation*}
L^{2}(\RR) \supset \Sc(\RR) \ni u_{n}\xrightarrow{L^{2}} u\in L^{2}(\RR)
\end{equation*}
\item posso fare le trasformate $\widehat{u_{n}}$ dato che $u_{n} \in \Sc(\RR)$
\item per l'uguaglianza delle norme appena trovata, valida appunto per $u_{n} \in \Sc(\RR)$
\begin{equation*}
\Vert u_{n} - u_{m} \Vert_{L^{2}} = \Vert \hat{u}_{n} - \hat{u}_{m} \Vert_{L^{2}}\frac{1}{\sqrt{2\pi}}
\end{equation*}

Grazie alla completezza dello spazio $L^{2}$, se $u_{n}$ converge, è di Cauchy, allora lo è anche la sua trasformata, allora anche lei converge. Definisco $\hat{u}$ il limite della successione $\widehat{u_{n}}\xrightarrow{L^{2}}\hat{u}$, abbiamo trovato la trasformata $\Fc : L^{2}\rightarrow L^{2}$.
\end{itemize}
\item Con questo ragionamento potremmo anche dedurre che

\begin{equation*}
(u, v) = \frac{1}{2\pi}(\hat{u}, \hat{v}) \ \ \ \ \text{(si mantiene l'ortogonalità)}
\end{equation*}
\end{itemize}
\end{proof}

\section{Trasformata di Fourier di una distribuzione temperata}

Ci manca solo la trasformata di Fourier di una distribuzione temperata, consideriamo quindi $S'(\RR)$. Vogliamo mostrare che
\begin{equation*}
\Fc : S'(\RR)\rightarrow S'(\RR)
\end{equation*}
Ricordiamo che
\begin{equation*}
(u', \varphi) = - (u, \varphi')
\end{equation*}
Supponiamo che $u$ sia una distribuzione temperata $u\in \Sc'(\RR)$, vorremmo che $\hat{u} \in \Sc'(\RR)$. Definiamo senza nessuno sforzo
\begin{equation*}
\boxed{(\hat{u}, \varphi) = (u, \hat{\varphi})} \ \ \ \ \varphi \in \Sc(\RR)
\end{equation*}
dato che se $\varphi \in \Sc(\RR)$ allora $\hat{\varphi} \in \Sc(\RR)$.
\begin{thm}
La trasformata di Fourier $\Fc : \Sc'(\RR)\rightarrow \Sc'(\RR)$ è un'applicazione

lineare biunivoca e continua.
\end{thm}
\begin{proof}\leavevmode
\begin{itemize}
\item È effettivamente un funzionale \textbf{lineare} dalla definizione
\item \textbf{Continuo} grazie alla (non dimostrata) continuità della trasformata di Fourier delle funzioni in $\Sc(\RR)$.
\begin{equation*}
(\hat{u}, \varphi_{n}) = \left(u, \widehat{\varphi_{n}}\right)\rightarrow (u, \hat{\varphi}) = (\hat{u}, \varphi)
\end{equation*}
\item \textbf{Suriettiva.}
\begin{equation*}
\begin{array}{c c c}
\Fc : \Sc'(\RR) & \rightarrow & \Sc'(\RR)\\
u & & \hat{u} = v
\end{array}
\end{equation*}

Data una distribuzione temperata $v\in \Sc'(\RR)$ tale che $v = \hat{u}$ dobbiamo trovare $u\in \Sc'(\RR)$. Diciamo che $u = \Fc^{- 1}(v)$, e infatti
\begin{equation*}
(\hat{u}, \varphi) = (u, \hat{\varphi}) = \left(\Fc^{- 1}(v), \hat{\varphi}\right) = \left(v, \Fc^{- 1}(\hat{\varphi})\right) = (v, \varphi) \ \ \forall \varphi
\end{equation*}
\item \textbf{Iniettiva}.

Dobbiamo far vedere che se abbiamo $u_{1}, u_{2} \in \Sc'(\RR)$ e $\hat{u}_{1} = \hat{u}_{2}$ allora $u_{1} = u_{2}$. Prendiamo una nuova funzione test $\psi $ tale che $\hat{\psi} = \varphi $.
\begin{equation*}
(u_{1}, \varphi) = (u_{1}, \hat{\psi}) = (\hat{u}_{1}, \psi) = (\hat{u}_{2}, \psi) = (u_{2}, \hat{\psi}) = (u_{2}, \varphi)
\end{equation*}
\end{itemize}
\end{proof}
\textit{Esempio.}

Calcoliamo la trasformata di Fourier di una costante, cioè una funzione $L^{1}_{\loc}$ a cui quindi è associata una distribuzione temperata\footnote{Si noti che in testi differenti si potrebbe trovare un risultato diverso in base a come si è definita la trasformata di Fourier. Non tutti, infatti, sono d'accordo sull'uso della costante $2\pi $ tra la trasformazione diretta e quella inversa.}
\begin{equation*}
\Fc^{- 1}\{\delta_{0}(\xi)\} = \frac{1}{2\pi}\int_{\RR} e^{i\xi x} \delta_{0}(\xi) d\xi = \frac{1}{2\pi} \langle \delta_{0}(\xi), e^{i\xi x} \rangle = \frac{1}{2\pi}
\end{equation*}
allora
\begin{equation*}
\Fc\left\{\frac{1}{2\pi}\right\} = \delta_{0}(\xi) \ \ \implies \ \ \boxed{\Fc\{1\} = 2\pi \delta_{0}(\xi)}
\end{equation*}

\section{Trasformata di Laplace}

È parente stretta della trasformata di Fourier, solo che Fourier si riferisce a tutto $\RR^{n}$, Laplace su funzioni definite su una semiretta. Questa differenza si riflette sul loro impiego per le equazioni differenziali.

Cosa deve avere una funzione $u: \RR\rightarrow \RR$ affinché sia Laplace - trasformabile o $\Lc$ - trasformabile?
\begin{itemize}
\item $\supp(u) \subset [0, + \infty)$

A questo proposito ricordiamo la funzione di Heaviside, che sarà utile, in quanto non trasformeremo mai $\sin x$, piuttosto trasformeremo $H(x)\sin x$, anche se talvolta sorvoleremo su questa cosa dandola per scontata
\begin{equation*}
H(x) =
\begin{cases}
1, & \text{se} \ x \geq 0\\
0, & \text{se} \ x < 0
\end{cases}
\end{equation*}
\item $u\in L^{1}_{\loc}$
\item $\exists \lambda \in \RR$ tale che $e^{- \lambda x} u(x) \in L^{1}(\RR)$

Chiamiamo $\lambda (u)$ l'estremo inferiore di questi $\lambda $
\end{itemize}
\begin{defn}
[Trasformata di Laplace]
Definiamo
\begin{equation*}
\boxed{\Lc(u(x), s) = \int^{+ \infty}_{0} e^{- sx} u(x) dx} \ \ \ \ s\in \CC
\end{equation*}
dove
\begin{equation*}
e^{- sx} = e^{- s_{R} x} e^{- is_{I} x}
\end{equation*}
Questo integrale ha senso purché $e^{- \lambda x} u(x) \in L^{1}(\RR)$, cioè $\Re (s) > \lambda (u)$, che è quindi il dominio della trasformata, mentre Fourier aveva senso sempre.
\end{defn}
\textbf{Proprietà}
\begin{itemize}
\item Linearità $\boxed{\Lc(\alpha u(x) + \beta v(x), s) = \alpha \Lc(u(x), s) + \beta \Lc(v(x), s)}$

Attenzione, dobbiamo intersecare i due domini
\begin{equation*}
\boxed{\Re (s) > \max(\lambda (u), \lambda (v))}
\end{equation*}
\item $\boxed{\lim\limits_{\Re (s)\rightarrow + \infty}\Lc(u(x), s) = 0}$

\begin{proof}
\begin{equation*}
\lim\limits_{\Re (s)\rightarrow + \infty}\Lc(u(x), s) = \lim\limits_{\Re (s)\rightarrow + \infty}\int^{+ \infty}_{0} e^{- s_{R} x} e^{- is_{I} x} u(x) dx
\end{equation*}

possiamo portare il limite dentro per convergenza dominata, purché $\Re (s) > \lambda (u)$ possiamo trovare una funzione che la domini
\begin{equation*}
\lim\limits_{\Re (s)\rightarrow + \infty}\int^{+ \infty}_{0} e^{- s_{R} x} e^{- is_{I} x} u(x) dx = \int^{+ \infty}_{0}\lim\limits_{\Re (s)\rightarrow + \infty} e^{- s_{R} x} e^{- is_{I} x} u(x) dx = 0
\end{equation*}
\end{proof}
\item $\boxed{\Lc(u(x - x_{0}), s) = e^{- sx_{0}}\Lc(u(x), s)}$

Attenzione che essendo $\supp(u) \in [0, + \infty)$
\begin{itemize}
\item se $\boxed{x_{0} > 0}$ il supporto è ancora quello
\item se $x_{0} < 0$ il supporto potrebbe includere numeri negativi, quindi scartiamo questo caso
\end{itemize}

\begin{proof}
\begin{align*}
\Lc(u(x - x_{0}), s) & = \int^{+ \infty}_{0} e^{- sx} u(x - x_{0}) dx & x' = x - x_{0}\\
 & = \int^{+ \infty}_{- x_{0}} e^{- s(x' + x_{0})} u(x') dx' & \\
 & = \int^{+ \infty}_{0} e^{- s(x' + x_{0})} u(x') dx' & \\
 & = e^{- sx_{0}}\Lc(u(x), s) &
\end{align*}
\end{proof}
\item $\boxed{\Lc\left(e^{s_{0} x} u(x), s\right) = \Lc(u(x), s - s_{0})}$ $\boxed{\Re (s - s_{0}) > \lambda (u)}$

\begin{proof}
\begin{equation*}
\Lc\left(e^{s_{0} x} u(x), s\right) = \int^{+ \infty}_{0} e^{- x(s - s_{0})} u(x) dx = \Lc(u(x), s - s_{0})
\end{equation*}
\end{proof}
\item $\boxed{\Lc(u(cx), s) = \frac{1}{c}\Lc\left(u(x), \frac{s}{c}\right)}$ $\boxed{c\in \RR, c > 0}$

\begin{proof}
\begin{align*}
\Lc(u(cx), s) & = \int^{+ \infty}_{0} e^{- sx} u(cx) dx & x' = cx\ \ \implies \ \ dx' = cdx\\
 & = \int^{+ \infty}_{0} e^{- s\frac{x'}{c}} u(x')\frac{1}{c} dx' & \\
 & = \frac{1}{c}\Lc\left(u(x), \frac{s}{c}\right) &
\end{align*}
\end{proof}
\end{itemize}

\subsection{Trasformate di funzioni notevoli}

\begin{itemize}
\item Heaviside
\begin{align*}
\Lc(H(x), s) & = \int^{+ \infty}_{0} e^{- sx} H(x) dx = \int^{+ \infty}_{0} e^{- sx} dx\\
 & = \left[ - \frac{e^{- sx}}{s}\right]^{+ \infty}_{0} = \frac{1}{s} - \lim_{r\rightarrow + \infty}\frac{e^{- sr}}{s}\\
 & = \frac{1}{s} \ \ \ \ \text{purché} \ \Re (s) > 0
\end{align*}

il che ci dice anche che $\lambda (H) = 0$, confermato da
\begin{equation*}
e^{- \lambda x} H(x) \in L^{1}(\RR) \ \ \ \ \lambda > 0
\end{equation*}

Perché ci serve?
\begin{equation*}
\Lc\left(e^{s_{0} x} H(x), s\right) = \Lc(H(x), s - s_{0}) = \frac{1}{s - s_{0}}
\end{equation*}
\item Seno e coseno
\begin{align*}
\Lc(\cos(\omega x) H(x), s) & = \Lc\left(\frac{e^{i\omega x} + e^{- i\omega x}}{2}, s\right)\\
 & = \frac{1}{2}\left(\frac{1}{s - i\omega x} + \frac{1}{s + i\omega x}\right) = \frac{1}{2}\frac{2s}{s^{2} + \omega^{2}} = \frac{s}{s^{2} + \omega^{2}}\\
\Lc(\sin(\omega x) H(x), s) & = \frac{\omega}{s^{2} + \omega^{2}}
\end{align*}
\item Delta di Dirac \textit{trattandola come funzione}.
\begin{align*}
\Lc(\delta_{0}, s) & = \int^{+ \infty}_{0} e^{- sx} \delta_{0}(x) dx = 1
\end{align*}

per capirlo meglio possiamo vedere la Delta come
\begin{gather*}
\delta_{0} = \lim\limits_{\varepsilon \rightarrow 0} u_{\varepsilon}(x) \ \ \ \ u_{\varepsilon}(x) =
\begin{cases}
\frac{1}{\varepsilon}, & x\in [0, \varepsilon]\\
0, & \text{altrimenti}
\end{cases}\\
\Lc(u_{\varepsilon}(x), s) = \frac{1}{\varepsilon}\int^{\varepsilon}_{0} e^{- sx} dx = e^{- sx_{\varepsilon}}\rightarrow 1\ \ x_{\varepsilon} \in (0, \varepsilon)
\end{gather*}
\end{itemize}
\begin{thm}
[Derivata della trasformata]
Sia $u$ una funzione $\Lc$ - trasformabile.
\begin{equation*}
\boxed{\frac{d^{n}}{ds^{n}}\Lc(u(x), s) = (- 1)^{n} \Lc\left(x^{n} u(x), s\right)} \ \ \ \ \forall s\ \text{tale che} \ \Re (s) > \lambda (u)
\end{equation*}
\end{thm}
\begin{proof}
\begin{align*}
\lim\limits_{h\rightarrow 0}\frac{\Lc(u(x), s + h) - \Lc(u(x), s)}{h} & = \lim\limits_{h\rightarrow 0}\int^{+ \infty}_{0} e^{- sx}\frac{e^{- hx} - 1}{h} u(x) dx\\
 & = \int^{+ \infty}_{0}\lim\limits_{h\rightarrow 0} e^{- sx}\textcolor[rgb]{0.82, 0.01, 0.11}{\frac{e^{- hx} - 1}{- hx}}(- x) u(x) dx\\
 & = \int^{+ \infty}_{0} e^{- sx}(- x) u(x) dx\\
 & = - \Lc(xu(x), s)
\end{align*}
Possiamo passare al limite sotto il segno di integrale? Sì perché la quantità rossa è limitata, il resto è $L^{1}$.

È derivabile in un aperto in senso complesso, quindi la funzione trasformata di Laplace è olomorfa, quindi è anche $C^{\infty}$.
\begin{thm}
[Trasformata della derivata]
Sia $u$ una funzione $\Lc$ - trasformabile e $u\in C^{1}([0, + \infty))$. Chiamiamo $v(x)$ la derivata
\begin{equation*}
v(x) =
\begin{cases}
u'(x) & x \geq 0\\
0 & x < 0
\end{cases}
\end{equation*}
dove $u'(x)$ è estesa per continuità nell'origine.

Sia quindi anche $v$ $\Lc$ - trasformabile. Allora
\begin{equation*}
\boxed{\Lc(v(x), s) = s\Lc(u(x), s) - u(0)}
\end{equation*}
attenzione che in realtà $u(0)$ è il limite per $x\rightarrow 0^{+}$, dato che non è di solito continua nell'origine.

Inoltre
\begin{equation*}
\lim_{\Re (s)\rightarrow + \infty} s\Lc(u(x), s) = u(0)
\end{equation*}
Se $v\in L^{1}$, allora $\lambda (u) \leq 0$ e
\begin{equation*}
\lim\limits_{s\rightarrow 0, \ \Re (s) > 0} s\Lc(u(x), s) = \lim_{x\rightarrow + \infty} u(x)
\end{equation*}
\end{thm}
\textit{Dimostrazione della prima parte.}
\begin{align*}
\Lc(u'(x), s) & = \int^{+ \infty}_{0} e^{- sx} u'(x) dx & \text{(per parti)}\\
 & = \left[e^{- sx} u(x)\right]^{+ \infty}_{0} - (- s)\int^{+ \infty}_{0} e^{- sx} u(x) dx & \\
 & = \lim_{x\rightarrow + \infty} e^{- sx} u(x) - u(0) + s\Lc(u(x), s) &
\end{align*}
Notiamo che
\begin{equation*}
e^{- \lambda x} u(x) \in L^{1} \ \ \notimplies \ \ e^{- \lambda x} u(x)\rightarrow 0\ \text{per} \ x\rightarrow + \infty
\end{equation*}
per esempio questa funzione non tende a $0$ pur essendo integrabile

\begin{figure}[htpb]
\centering

\tikzset{every picture/.style = {line width = 0.75pt}} %set default line width to 0.75pt

\begin{tikzpicture}[x = 0.75pt, y = 0.75pt, yscale = - 1, xscale = 1]
%uncomment if require: \path (0, 159); %set diagram left start at 0, and has height of 159

%Shape: Axis 2D [id: dp9522434839194833]
\draw (171, 111) - - (441, 111)(171, 11) - - (171, 111) - - cycle (434, 106) - - (441, 111) - - (434, 116) (166, 18) - - (171, 11) - - (176, 18);
%Shape: Rectangle [id: dp6209341938001463]
\draw (191, 41) - - (251, 41) - - (251, 111) - - (191, 111) - - cycle;
%Shape: Rectangle [id: dp5551932751144555]
\draw (281, 41) - - (311, 41) - - (311, 111) - - (281, 111) - - cycle;
%Shape: Rectangle [id: dp45095610787740714]
\draw (341, 41) - - (351, 41) - - (351, 111) - - (341, 111) - - cycle;
%Shape: Rectangle [id: dp9386110377703627]
\draw (381, 41) - - (386, 41) - - (386, 111) - - (381, 111) - - cycle;

% Text Node
\draw (152, 33.4) node [anchor = north west][inner sep = 0.75pt] {$1$};
% Text Node
\draw (291, 122.4) node [anchor = north west][inner sep = 0.75pt] [font = \scriptsize] {$\frac{1}{4}$};
% Text Node
\draw (216, 122.4) node [anchor = north west][inner sep = 0.75pt] {$1$};
% Text Node
\draw (157, 112.4) node [anchor = north west][inner sep = 0.75pt] {$0$};
% Text Node
\draw (340, 122.4) node [anchor = north west][inner sep = 0.75pt] [font = \scriptsize] {$\frac{1}{9}$};
% Text Node
\draw (376, 122.4) node [anchor = north west][inner sep = 0.75pt] [font = \scriptsize] {$\frac{1}{16}$};

\end{tikzpicture}
\end{figure}
\FloatBarrier

\begin{equation*}
\int^{+ \infty}_{0} f(x) dx = \sum^{\infty}_{n = 1}\frac{1}{n^{2}} < + \infty \ \ \ \ \text{ma} \ f(x) \nrightarrow 0
\end{equation*}
Analizziamo bene però la nostra funzione, derivandola
\begin{equation*}
\frac{d}{dx}\left(e^{- \lambda x} u(x)\right) = \underbrace{- \lambda e^{- \lambda x} u(x)}_{\in L^{1}} + \underbrace{e^{- \lambda x} u'(x)}_{\in L^{1}} \in L^{1}
\end{equation*}
le due parti sono integrabili per ipotesi, grazie al fatto che anche $u'$ deve essere $\Lc$ - trasformabile. Per il teorema del calcolo
\begin{equation*}
e^{- \lambda x} u(x) = u(0) + \int^{x}_{0}\frac{d}{dt}\left(e^{- \lambda t} u(t)\right) dt
\end{equation*}
il limite a destra esiste finito, allora esiste finito il limite di $e^{- \lambda x} u(x)$, ma se ammette limite deve necessariamente essere
\begin{gather*}
\lim_{x\rightarrow + \infty} e^{- sx} u(x) = 0
\end{gather*}
\end{proof}

\subsection{Trasformazione inversa}

Se abbiamo $f(s) = \Lc(u(x), s)$ come troviamo $u(x) = \Lc^{- 1}(f(s), s)$?

Scriviamo $s = a + ib$.
\begin{equation*}
f(a + ib) = \int^{+ \infty}_{0} e^{- ax} e^{- ibx} u(x) dx = \int_{\RR} e^{- ax} e^{- ibx} u(x) dx = \Fc\left(e^{- ax} u(x), b\right)
\end{equation*}
Ma allora possiamo fare l'antitrasformata di Fourier!
\begin{equation*}
e^{- ax} u(x) = \frac{1}{2\pi}\int_{\RR} f(a + ib) e^{ibx} db
\end{equation*}
allora
\begin{equation*}
\boxed{u(x) = \frac{1}{2\pi}\int_{\RR} e^{(a + ib) x} f(a + ib) db} \ \ \ \ \forall a > \lambda (u)
\end{equation*}
Questa è già una formula valida, ma possiamo chiamare, per $x$ fissato
\begin{equation*}
\Gamma_{x} = \{z\in \CC, z = x + iy, y\in \RR\}
\end{equation*}

\begin{figure}[htpb]
\centering
\tikzset{every picture/.style = {line width = 0.75pt}} %set default line width to 0.75pt

\begin{tikzpicture}[x = 0.75pt, y = 0.75pt, yscale = - 1, xscale = 1]
%uncomment if require: \path (0, 113); %set diagram left start at 0, and has height of 113

%Shape: Axis 2D [id: dp6665420354493967]
\draw (251, 63.66) - - (351, 63.66)(260.5, 7) - - (260.5, 107) (344, 58.66) - - (351, 63.66) - - (344, 68.66) (255.5, 14) - - (260.5, 7) - - (265.5, 14);
%Straight Lines [id: da4009061220418899]
\draw [color = {rgb, 255: red, 208; green, 2; blue, 27}, draw opacity = 1] (288, 18.66) - - (288, 103);

% Text Node
\draw (290, 64.23) node [anchor = north west][inner sep = 0.75pt] [color = {rgb, 255: red, 208; green, 2; blue, 27}, opacity = 1] {$x$};

\end{tikzpicture}
\end{figure}
\FloatBarrier

E possiamo allora scrivere l'antitrasformata come
\begin{equation*}
\boxed{u(x) = \frac{1}{2\pi}\int_{\Gamma_{x}} e^{zx} f(z) dz}
\end{equation*}

\section{Relazione con la convoluzione}

Ricordiamo la convoluzione di due funzioni $f, g\in L^{1}$
\begin{equation*}
(f*g)(x) = \int_{\RR} f(x - t) g(t) dt
\end{equation*}
Con Fourier abbiamo, utile in entrambi i versi,
\begin{equation*}
\Fc(f*g, z) = \hat{f}(z)\hat{g}(z)
\end{equation*}
Per Laplace abbiamo le seguenti richieste sulle funzioni affinché siano $\Lc$ - trasformabili
\begin{itemize}
\item $f, g\in L^{1}_{\loc}$
\item $\supp f\subset [0, + \infty), \ \supp g\subset [0, + \infty)$
\item $\exists \lambda_{1}, \lambda_{2} \ \implies \ e^{- \lambda_{1} x} f(x) \in L^{1}(\RR), \ e^{- \lambda_{2} x} g(x) \in L^{1}(\RR)$
\end{itemize}
\begin{thm}
La convoluzione di due funzioni trasformabili secondo Laplace, è trasformabile secondo Laplace.
\end{thm}
\begin{proof}
\begin{equation*}
(f*g)(x) = \int_{\RR} f(x - t) g(t) dt = \int^{x}_{0} f(x - t) g(t) dt
\end{equation*}
$f(x - t) = 0$ se $x - t < 0$, quindi se $t > x$. Mentre $g(t) = 0$ se $t < 0$. Allora esiste l'integrale.
\begin{align*}
e^{- \lambda x}(f*g)(x) & = e^{- \lambda x}\int^{x}_{0} f(x - t) g(t) dt\\
 & = \int^{x}_{0} e^{- \lambda (x - t)} f(x - t) e^{- \lambda t} g(t) dt\\
 & = \left(e^{- \lambda x} f(x)\right) *\left(e^{- \lambda x} g(x)\right) \ \ \ \ \lambda \geq \max(\lambda (f), \lambda (g)) \ \implies \ \in L^{1}
\end{align*}
Soddisfa i requisiti, calcoliamone l'espressione
\begin{align*}
\Lc((f*g), s) & = \int_{\RR} e^{- sx}\left(\int_{\RR} f(x - t) g(t) dt\right) dx\\
 & = \iint_{\RR^{2}} e^{- s(x - t)} e^{- st} f(x - t) g(t) dtdx\\
 & = \int_{\RR} e^{- s(x - t)} f(x - t)\int_{\RR} e^{- st} g(t) dtdx = (\star)
\end{align*}
cambiamo la variabile $x' = x - t$
\begin{gather*}
(\star) = \left[\int_{\RR} e^{- sx'} f(x') dx'\right]\left[\int_{\RR} e^{- st} g(t) dt\right] = \Lc(f) \cdot \Lc(g)
\end{gather*}
\end{proof}

\chapter{Conclusioni}

\section{EDP e Fourier}

Consideriamo l'equazione del calore
\begin{equation*}
u_{t} - u_{xx} = 0\ \ \ \ u(t, x) \ \ \ \ u_{t} = \partial_{t} u\ \ \ \ u_{xx} = \partial_{x} \partial_{x} u\ \ \ \ x\in \RR, t \geq 0
\end{equation*}
Supponiamo di sapere il dato iniziale
\begin{equation*}
\begin{cases}
u_{t} - u_{xx} = 0\\
u(0, x) = g(x)
\end{cases}
\end{equation*}
Indico con
\begin{equation*}
\hat{u}(t, z) = \int_{\RR} e^{izx} u(t, x) dx
\end{equation*}
trattando $t$ come un parametro
\begin{align*}
\frac{\partial}{\partial t}\hat{u}(t, z) & = \frac{\partial}{\partial t}\int_{\RR} e^{izx} u(t, x) dx\overset{!!!}{=}\int_{\RR} e^{izx}\frac{\partial}{\partial t} u(t, x) dx = \Fc(u_{t}(t, x), z)\\
 & \\
\Fc(u_{xx}(t, x), z) & = \int_{\RR} e^{- ixz} u_{xx}(t, x) dx\ \ \forall t \geq 0\\
 & \overset{\text{ipp}}{=}\int_{\RR}(iz)^{2} e^{- ixz} u(t, x) dx\\
 & = - z^{2}\Fc(u(t, x), z)
\end{align*}
Allora il problema diventa
\begin{equation*}
\begin{cases}
\partial_{t}\hat{u}(t, z) + z^{2}\hat{u}(t, z) = 0\\
\hat{u}(0, z) = \hat{g}(z)
\end{cases}
\end{equation*}
per ogni $z$, è un'equazione differenziale ordinaria
\begin{equation*}
\hat{u}(t, z) = \hat{g}(z) e^{- z^{2} t}
\end{equation*}
antitrasformiamo
\begin{equation*}
u(t, x) = \Fc^{- 1}\left(\hat{g}(z) e^{- z^{2} t}, x\right)
\end{equation*}
torna utile il prodotto di convoluzione
\begin{gather*}
u(t, x) = g(x) *\Fc^{- 1}\left(e^{- z^{2} t}, x\right) = \frac{1}{\sqrt{4\pi t}}\int_{\RR} e^{- \frac{(x - y)^{2}}{2t}} g(y) dy\\
\Fc^{- 1}\left(e^{- z^{2} t}, x\right) = \frac{1}{\sqrt{4\pi t}} e^{- \frac{x^{2}}{2t}}
\end{gather*}
ma non è definita in $t = 0$, tuttavia possiamo considerare il limite nel senso delle distribuzioni (non dimostrato, molto extra)
\begin{equation*}
\lim_{t\rightarrow 0^{+}} u(t, x) = g(x) \ \ \ \ u(t, x) : [0, + \infty) \times \RR \ \ \ \ g: \RR\rightarrow \RR
\end{equation*}

\section{EDP e Laplace}

Con Laplace risolviamo cose del tipo
\begin{equation*}
\begin{cases}
a_{n} y^{(n)}(t) + a_{n - 1} y^{(n - 1)}(t) + \cdots + a_{0} y(t) = f(t)\\
y^{(n - 1)}(0) = \textcolor[rgb]{0.82, 0.01, 0.11}{y}\textcolor[rgb]{0.82, 0.01, 0.11}{_{n - 1}}\\
\vdots \\
y(0) = \textcolor[rgb]{0.29, 0.56, 0.89}{y}\textcolor[rgb]{0.29, 0.56, 0.89}{_{0}}
\end{cases}
\end{equation*}
Trasformiamo
\begin{align*}
\Lc(y'(t), s) & = s\Lc(y(t), s) - y(0)\\
\Lc(y''(t), s) & = s\Lc(y'(t), s) - y'(0)\\
 & = s[s\Lc(y(t), s) - y(0)] - y'(0)\\
 & = s^{2}\Lc(y(t), s) - sy(0) - y'(0)\\
\Lc\left(y^{(n)}(t), s\right) & = s^{n}\Lc(y(t), s) - s^{n - 1}\textcolor[rgb]{0.29, 0.56, 0.89}{y}\textcolor[rgb]{0.29, 0.56, 0.89}{(}\textcolor[rgb]{0.29, 0.56, 0.89}{0}\textcolor[rgb]{0.29, 0.56, 0.89}{)} - s^{n - 1} y'(0) - \cdots - \textcolor[rgb]{0.82, 0.01, 0.11}{y}\textcolor[rgb]{0.82, 0.01, 0.11}{^{(n - 1)}}\textcolor[rgb]{0.82, 0.01, 0.11}{(}\textcolor[rgb]{0.82, 0.01, 0.11}{0}\textcolor[rgb]{0.82, 0.01, 0.11}{)}
\end{align*}
Esempio del secondo ordine
\begin{equation*}
\begin{cases}
a_{2} y''(y) + a_{1} y'(t) + a_{0} y(t) = f(t)\\
y(0) = y_{0}\\
y'(0) = y_{1}
\end{cases}
\end{equation*}
Trasformiamo
\begin{align*}
a_{2}\left(s^{2}\Lc(y(t), s) - sy_{0} - y_{1}\right) + a_{1}(s\Lc(y(t), s) - y_{0}) + a_{0}\Lc(y(t), s) & = \Lc(f(t), s)\\
\left(a_{2} s^{2} + a_{1} s + a_{0}\right)\Lc(y(t), s)\underbrace{- a_{2}(sy_{0} + y_{1}) - a_{1} y_{0}}_{\text{polinomio in} \ s} & = \Lc(f(t), s)
\end{align*}
quindi
\begin{equation*}
	\Lc(y(t), s) = \frac{\Lc(f(t), s) + a_{2}(sy_{0} + y_{1}) + a_{1} y_{0}}{a_{2} s^{2} + a_{1} s + a_{0}}
\end{equation*}
ci permette di introdurre in maniera naturale anche i dati iniziali.
\begin{gather*}
y(t) = f(t) *\Lc^{- 1}\left(\frac{1}{a_{2} s^{2} + a_{1} s + a_{0}}, t\right) + \Lc^{- 1}\left(\frac{a_{2}(sy_{0} + y_{1}) + a_{1} y_{0}}{a_{2} s^{2} + a_{1} s + a_{0}}, t\right)\\
\\
\frac{1}{s - s_{0}} + \frac{1}{s - s_{1}} + \cdots \ \ \ \ \Lc\left(e^{\alpha t} H(t), s\right) = \frac{1}{s - \alpha}
\end{gather*}

\section{Trasformata vs Serie di Fourier}

Sono definite da
\begin{equation*}
\hat{f}(z) = \int_{\RR} e^{- ixz} f(x) dx\ \ \ \ \ \ \ \ f(x) = \sum_{n\in \ZZ} e^{inx} c_{n}
\end{equation*}
Agiscono sugli spazi
\begin{equation*}
\Fc : L^{2}(A)\rightarrow L^{2}(A) \ \ \ \ \ \ \ \ L^{2}[ - \pi, \pi]\rightarrow l^{2}
\end{equation*}
Valgono due identità importanti
\begin{equation*}
\Vert f \Vert_{L^{2}} = \frac{1}{\sqrt{2\pi}} \Vert \hat{f} \Vert_{L^{2}} \ \ \text{(Plancherel)} \ \ \ \ \ \ \ \ \Vert f \Vert_{L^{2}} = \sqrt{\pi} \Vert c \Vert_{l^{2}} \ \ \text{(Parseval)}
\end{equation*}
dove
\begin{equation*}
\{c_{n}\} \in l^{2}, \ \text{ovvero} \ \ \sum\nolimits_{n\in \ZZ}| c_{n}|^{2} < + \infty
\end{equation*}
Infine se per la serie prendiamo come dominio $L^{2}([ - L, L])$ e facciamo tendere $L\rightarrow \infty $, la serie
\begin{equation*}
f(x) = \sum_{n\in \ZZ} e^{i\frac{2\pi}{L} nx} c_{n}
\end{equation*}
viene fatta su una griglia sempre più fine e inizia ad assomigliare all'integrale!

 \cleardoublepage

%\part{Esercizi}
%%!TEX root = ../main.tex

% TEMPLATE

% \chapter{Spazi campionari}
%
% \ParteEsercizi
%
% \Esercizio{(Nome di un esercizio speciale)}
% \Esercizio{}
% \Esercizio{}
%
% \ParteSoluzioni
%
% \Soluzione
% \Soluzione
% \Soluzione
%
%
% \chapter{Indipendenza}
%
% \ParteEsercizi
%
% \Esercizio{}
% \begin{enumerate}
% \item ine
% \item oh
% \item $\!\!\!\!{^*}$ jhk b
% \end{enumerate}
% \Esercizio{$*$}
% \Esercizio{}
%
% \ParteSoluzioni
%
% \Soluzione
% \Soluzione
% \Soluzione









\chapter{Esercitazione 1 - Boella}

\ParteEsercizi

\Esercizio{}

Risolvere l'equazione
\begin{equation*}
\sin z = 2
\end{equation*}

\Esercizio{}

Scrivere la serie di potenze di
\begin{equation*}
f(z) = \frac{1}{z^{2} - 3z + 2}
\end{equation*}

\Esercizio{}

Scrivere la serie di potenze di
\begin{equation*}
f(z) = \frac{e^{z}}{1 + 2z}
\end{equation*}

\Esercizio{}

La funzione
\begin{equation*}
u(x, y) = y^{3} - 3x^{2} y
\end{equation*}
è armonica? Qual è la sua armonica coniugata?

\Esercizio{}

Calcolare $\int\nolimits_{\gamma_{1}} f(z)dz$ con
\begin{equation*}
f(z) = \overline{z} \ \ \ \ \gamma_{1} = \left(t, it^{2}\right) \ \ \ \ t\in [0, 1]
\end{equation*}

\Esercizio{}

Calcolare $\int\nolimits_{\gamma_{1}} f(z)dz$ con
\begin{equation*}
f(z) = \overline{z} \ \ \ \ \gamma_{2} = (t, it)\ \ \ \
\end{equation*}

\Esercizio{}

Calcolare $\int_{\gamma_{1}} g(z)dz$ e $\int_{\gamma_{2}} g(z)dz$ con
\begin{equation*}
g(z) = z^{2} \ \ \ \ \gamma_{1} = \left(t, it^{2}\right) \ \ \ \ \gamma_{2} = (t, it)\ \ \ \ t\in [0, 1]
\end{equation*}

\ParteSoluzioni

\Soluzione

\begin{thm}
[Formule di Eulero] Il seno e il coseno complessi possono essere espressi come
\begin{equation*}
\sin z = \frac{e^{iz} - e^{- iz}}{2i} \ \ \cos z = \frac{e^{iz} + e^{- iz}}{2}
\end{equation*}
\end{thm}
Si ha
\begin{align*}
\sin z = \frac{e^{iz} - e^{- iz}}{2i} & = 2\\
e^{iz} - 4i - e^{- iz} & = 0\\
e^{2iz} - 4ie^{iz} - 1 & = 0
\end{align*}
ponendo $w = e^{iz}$ abbiamo
\begin{equation*}
w^{2} - 4iw - 1 = 0\ \ \implies \ \ w = 2i\pm \sqrt{3} i = (2\pm \sqrt{3})i
\end{equation*}
Eguagliamo quindi
\begin{equation*}
(2\pm \sqrt{3})i = e^{iz}
\end{equation*}
Ricordiamo che $z = x + iy$ e da ciò segue che
\begin{equation*}
e^{iz} = e^{ix - y} = e^{- y} e^{ix} \ \ \ \ \ \ \ \ (2\pm \sqrt{3})i = (2\pm \sqrt{3})e^{i\pi /2}
\end{equation*}
Uguagliando il modulo abbiamo
\begin{equation*}
e^{- y} = (2\pm \sqrt{3})\ \ \implies \ \ y = -\ln (2\pm \sqrt{3})
\end{equation*}
Uguagliando l'argomento abbiamo
\begin{equation*}
e^{ix} = e^{i\pi /2} \ \ \implies \ \ x = \frac{\pi}{2} + 2k\pi
\end{equation*}
L'equazione ha quindi \textit{infinite} soluzioni, date da infinite coppie di punti disposti su due rette parallele.

\Soluzione

\begin{defn}
La serie centra nel punto $z_{0}$ è data da
\begin{equation*}
\sum\limits^{\infty}_{n = 0} a_{n}\left(z - z_{0}\right)^{n}
\end{equation*}
\end{defn}
\begin{rem}
Il raggio di convergenza è $R$, ed è centrato in $z_{0}$. In tutti i punti interni la serie converge, nei punti esterni non converge, sulla circonferenza nulla si può dire in generale.
\end{rem}
\begin{thm}
[Criterio del rapporto]
\begin{equation*}
\frac{1}{R} = \lim\limits_{n\rightarrow + \infty}\frac{\left| a_{n + 1}\right|}{\left| a_{n}\right|}
\end{equation*}
Se il limite vale $\infty /0$, il raggio è $0/\infty $. Se il limite non esiste, il criterio fallisce.
\end{thm}
\begin{thm}
[Criterio della radice]
\begin{equation*}
\frac{1}{R} = \limsup\limits_{n\rightarrow \infty}\sqrt[n]{\left| a_{n}\right|}
\end{equation*}
Si noti che il limite superiore esiste sempre (a differenza del limite).
\end{thm}
\begin{defn}
Serie prodotto secondo Cauchy
\begin{equation*}
A = \sum\limits^{\infty}_{n = 0} a_{n}, \ \ B = \sum\limits^{\infty}_{n = 0} b_{n} \ \ \implies \ \ C = \sum\limits^{\infty}_{n = 0} a_{n}\sum\limits^{\infty}_{n = 0} b_{n} = \sum\limits^{\infty}_{n = 0} c_{n}
\end{equation*}
con
\begin{equation*}
c_{n} = \sum\limits^{n}_{k = 0} a_{k} b_{n - k}
\end{equation*}
\end{defn}
\begin{thm}
La serie $C$ converge assolutamente ad $A \cdot B$, per il teorema di Mertens, se almeno una fra $A$ e $B$ converge assolutamente.

Se $A$ e $B$ hanno $R = +\infty, $ si può ovviamente applicare il teorema di Mertens, e si può scrivere
\begin{equation*}
\sum\limits^{\infty}_{n = 0} a_{n}\sum\limits^{\infty}_{n = 0} b_{n} = \sum\limits^{\infty}_{n, m = 0} a_{n} b_{m}
\end{equation*}
\end{thm}
\begin{rem}
[Serie notevoli]
\begin{align*}
e^{z} & = \sum^{\infty}_{n = 0}\frac{z^{n}}{n!} & R = +\infty \\
\cosh z & = \sum^{\infty}_{n = 0}\frac{z^{2n}}{(2n)!} & R = +\infty \\
\sinh z & = \sum^{\infty}_{n = 0}\frac{z^{2n + 1}}{(2n + 1)!} & R = +\infty \\
\cos z & = \sum^{\infty}_{n = 0}\frac{(- 1)^{n} z^{2n}}{(2n)!} & R = +\infty \\
\sin z & = \sum^{\infty}_{n = 0}\frac{(- 1)^{n} z^{2n + 1}}{(2n + 1)!} & R = +\infty \\
\frac{1}{1 - z} & = \sum^{\infty}_{n = 0} z^{n} & R = 1
\end{align*}
\end{rem}
Separiamo in due termini la frazione
\begin{equation*}
f(z) = \frac{1}{z^{2} - 3z + 2} = \frac{1}{(z - 1)(z - 2)} = \frac{1}{z - 2} - \frac{1}{z - 1} = \frac{1}{1 - z} - \frac{1}{2 - z}
\end{equation*}
Raccogliamo nel secondo termine
\begin{equation*}
\frac{1}{2 - z} = \frac{\frac{1}{2}}{1 - \frac{z}{2}} = \frac{1}{2}\frac{1}{1 - \frac{z}{2}}
\end{equation*}
valgono poi
\begin{equation*}
\frac{1}{1 - z} = \sum^{\infty}_{n = 0} z^{n} \ \ \ \ \frac{1}{2 - z} = \frac{1}{2}\sum^{\infty}_{n = 0}\left(\frac{z}{2}\right)^{n} = \sum^{\infty}_{n = 0}\frac{1}{2^{n + 1}} z^{n}
\end{equation*}
pertanto
\begin{equation*}
f(z) = \sum^{\infty}_{n = 0}\left(1 - \frac{1}{2^{n + 1}}\right) z^{n}
\end{equation*}
Il raggio di convergenza è l'intersezione tra $|z| < 1$ e $|z| < 2$, ovvero $|z| < 1$.

\Soluzione

\begin{equation*}
f(z) = \frac{e^{z}}{1 + 2z} = e^{z} \cdot \frac{1}{1 - (- 2z)} = (\star)
\end{equation*}
Sviluppiamo in serie e scriviamo la serie prodotto secondo Cauchy
\begin{equation*}
(\star) = \sum^{\infty}_{n = 0}\frac{z^{n}}{n!} \cdot \sum^{\infty}_{m = 0} (- 2z)^{m} = \sum^{\infty}_{n = 0}\sum^{n}_{k = 0}\frac{z^{k}}{k!} (- 2)^{n - k} z^{n - k} = \sum^{\infty}_{n = 0}\left(\sum^{n}_{k = 0}\frac{(- 2)^{n - k}}{k!}\right) z^{n}
\end{equation*}
In alternativa, sfruttando Mertens
\begin{equation*}
f(z) = \frac{e^{z}}{1 + 2z} = \sum^{\infty}_{n = 0}\frac{z^{n}}{n!} \cdot \sum^{\infty}_{n = 0} (- 2z)^{n} = \sum^{\infty}_{n, m = 0}\frac{z^{n}}{n!} (- 2)^{m} z^{m}
\end{equation*}

\Soluzione

\begin{defn}
Una funzione si dice \textbf{armonica} se la somma delle sue derivate seconda è pari a zero.
\end{defn}
\begin{defn}
Una funzione si dice \textbf{olomorfa} se è derivabile in senso complesso.
\end{defn}
\begin{thm}
Sia $f(z)$ olomorfa: $f(z) = f(x, y) = u(x, y) + iv(x, y)$. Allora $u$ e $v$ sono armoniche, e l'una è l'armonica coniugata dell'altra, secondo le condizioni di Cauchy - Riemann
\begin{equation*}
u_{x} = v_{y} \ \ \ \ u_{y} = - v_{x}
\end{equation*}
\end{thm}
Deriviamo la funzione
\begin{gather*}
u_{x} = - 6xy = v_{y}\\
u_{y} = 3y^{2} - 3x^{2} = - v_{x}
\end{gather*}
pertanto è armonica.

Integrando la $v_{x}$ abbiamo
\begin{equation*}
\int v_{x} dx = \int \left(3x^{2} - 3y^{2}\right) dx = x^{3} - 3xy^{2} + h(y)
\end{equation*}
integrando la $v_{y}$ abbiamo
\begin{equation*}
\int v_{y} dy = \int - 6xydy = -3xy^{2} + k(x)
\end{equation*}
Da ciò segue che $k(x) = x^{3}$ e $h(y) = \text{costante}$. Pertanto
\begin{equation*}
v(x, y) = x^{3} - 3xy^{2}
\end{equation*}
e
\begin{align*}
f(z) & = f(x, y)\\
 & = u(x, y) + iv(x, y)\\
 & = \left(y^{3} - 3x^{2} y\right) + i\left(x^{3} - 3xy^{2}\right)\\
 & = iz^{3}
\end{align*}

\Soluzione

È noto che
\begin{equation*}
f(z) = \overline{z} = x - iy\ \ \ \ dz = d\left(t + it^{2}\right) = dt + 2itdt = \left(1 + 2it\right) dt
\end{equation*}
Pertanto
\begin{align*}
\int_{\gamma_{1}} f(z)dz & = \int^{1}_{0}\left(t - it^{2}\right) (1 + 2it)dt\\
 & = \int^{1}_{0}\left(t + it^{2} + 2t^{3}\right) dt\\
 & = \left[\frac{t^{2}}{2} + i\frac{t^{3}}{3} + \frac{t^{4}}{2}\right]^{1}_{0}\\
 & = 1 + \frac{i}{3}
\end{align*}

\Soluzione

È noto che
\begin{gather*}
f(z) = \overline{z} = x - iy\\
dz = dx + idy\\
dx = dt\\
dy = dt
\end{gather*}
Pertanto
\begin{equation*}
\int_{\gamma_{2}} f(z)dz = \int^{1}_{0} (t - it)(1 + i)dt = \int^{1}_{0} 2tdt = 1
\end{equation*}
Notiamo che l'integrale dipende dal percorso scelto, essendo $\overline{z}$ una funzione \textit{non} olomorfa.

\Soluzione

Osserviamo che
\begin{gather*}
z^{2} = x^{2} - y^{2} + 2ixy\\
dx + idy = dt + 2itdt
\end{gather*}
Pertanto
\begin{align*}
\int_{\gamma_{1}} g(z)dz & = \int^{1}_{0}\left(t^{2} - t^{4} + 2it^{3}\right) (1 + 2it)dt\\
 & = \left[\frac{t^{3}}{3} + it^{4} - 5t^{5} - \frac{i}{3} t^{6}\right]^{1}_{0}\\
 & = - \frac{2}{3} + \frac{2}{3} i
\end{align*}
\textit{Esercizio:} svolgere l'integrale $\gamma_{2}$, spiegando perché ora il risultato sia uguale.
\chapter{Esercitazione 1 - Potrich}

\ParteEsercizi

\Esercizio{}

Determinare in quali punti del piano complesso la funzione
\begin{equation*}
f(z) = z^{2}\left| z\right|^{2} e^{i\overline{z}^{2}}
\end{equation*}
è derivabile in senso complesso.

\Esercizio{}

Mostrare che
\begin{equation*}
u\left(x, y\right) = 2x - 2xy
\end{equation*}
è armonica e trovare la sua armonica coniugata.

\Esercizio{}

Sia $\gamma (t) = t + it^{2}, t\in \left[ - 1, 0\right]$. Calcolare
\begin{equation*}
\int\nolimits_{\gamma}\left[\overline{z} + \left(z + 1\right)^{7}\right] dz
\end{equation*}

\Esercizio{}

Calcolare
\begin{equation*}
\int\nolimits_{\gamma}\left(\overline{z}\right)^{2} dz
\end{equation*}
dove $\gamma $ è la circonferenza centrata in $z = 1$ e raggio unitario percorsa in senso antiorario.

\Esercizio{}

Risolvere
\begin{equation*}
\sin z = 3
\end{equation*}

\Esercizio{}

Calcolare la somma della serie
\begin{equation*}
\sum\limits^{\infty}_{n = 1}\left(3n + 7\right) z^{n}
\end{equation*}
all'interno del suo disco di convergenza.

\ParteSoluzioni

\Soluzione

\begin{defn}
Siano $\Omega \subseteq \CC $, $f:\Omega \rightarrow \CC $ è derivabile, in senso complesso, in $z_{0}$ se $\exists $ finito
\begin{equation*}
\lim\limits_{h\rightarrow 0}\frac{f\left(z_{0} + h\right) - f\left(z_{0}\right)}{h}, \ \ \ \ h\in \CC
\end{equation*}
\end{defn}
\begin{defn}
Sia $\Omega \in \CC $ aperto. Se $f$ è derivabile in ogni punto di $\Omega $, allora $f$ è \textbf{olomorfa} in $\Omega $, $f\in \Hc \left(\Omega \right)$.
\end{defn}
\begin{rem}
Se $\Omega \equiv \CC $ allora $f$ è \textbf{intera}.
\end{rem}
\begin{thm}
[di Cauchy - Riemann] Sia $\Omega \subseteq \CC $ aperto, $f:\Omega \rightarrow \CC $ definita come
\begin{equation*}
f(z) = u\left(x, y\right) + iv\left(x, y\right) = \Re (z) + i\Im (z)
\end{equation*}
è olomorfa in $\Omega $ se e solo se $u, v:\Omega \rightarrow \RR $ sono differenziabili in senso classico in $\Omega $ e valgono le \textbf{condizioni di Cauchy - Riemann}
\begin{equation}
\begin{cases}
u_{x}\left(x, y\right) = v_{y}\left(x, y\right)\\
u_{y}\left(x, y\right) = - v_{x}\left(x, y\right)
\end{cases}
\end{equation}
\end{thm}
\begin{rem}
Un altro modo per scrivere le (1) è
\begin{equation*}
\frac{\partial f}{\partial \overline{z}} = 0
\end{equation*}
Infatti
\begin{equation*}
\begin{array}{l}
x = \Re (z) = \frac{z + \overline{z}}{2}\\
y = \Im (z) = - i\frac{z - \overline{z}}{2}
\end{array}
\end{equation*}
allora
\begin{equation*}
f\left(x, y\right) = f\left(\frac{z + \overline{z}}{2}, - i\frac{z - \overline{z}}{2}\right)
\end{equation*}
Per la regola della catena
\begin{equation*}
\frac{\partial f}{\partial \overline{z}} = \frac{1}{2}\left(\frac{\partial f}{\partial x} + i\frac{\partial f}{\partial y}\right) = 0
\end{equation*}
che vale in quanto
\begin{equation*}
\begin{cases}
f_{x} = f' = u_{x} + iv_{x} = v_{y} - iu_{y}\\
f_{y} = if' = iu_{x} - v_{x} = iv_{y} + u_{y}
\end{cases}
\end{equation*}
\end{rem}
Poniamo $z = x + iy$
\begin{align*}
f(z) & = \left(x + iy\right)^{2}\left(x^{2} + y^{2}\right) e^{i\left(x - iy\right)^{2}}\\
 & = \left(x + iy\right)^{2}\left(x^{2} + y^{2}\right) e^{i\left(x^{2} - y^{2} - 2ixy\right)}\\
 & = \left(x^{2} - y^{2} + 2ixy\right)\left(x^{2} + y^{2}\right) e^{2xy + i\left(x^{2} - y^{2}\right)}\\
 & = \left(x^{2} - y^{2} + 2ixy\right)\left(x^{2} + y^{2}\right) e^{2xy}\left[\cos\left(x^{2} - y^{2}\right) + i\sin\left(x^{2} - y^{2}\right)\right]\\
 & = \left(x^{2} + y^{2}\right) e^{2xy}\left[\left(x^{2} - y^{2}\right)\cos\left(x^{2} - y^{2}\right) - 2xy\sin\left(x^{2} - y^{2}\right)\right]\\
 & + i\left(x^{2} + y^{2}\right) e^{2xy}\left[ 2xy\cos\left(x^{2} - y^{2}\right) + \left(x^{2} - y^{2}\right)\sin\left(x^{2} - y^{2}\right)\right]
\end{align*}
$u, v\in C^{1}\left(\RR^{2}\right)$.

Osserviamo che $f(z) = z^{2} \cdot z\overline{z} \cdot e^{i\overline{z}^{2}} = z^{3}\overline{z} e^{i\overline{z}^{2}}$
\begin{align*}
\frac{\partial f}{\partial \overline{z}} & = z^{3} e^{i\overline{z}^{2}} + z^{3}\overline{z} 2i\overline{z} e^{i\overline{z}^{2}}\\
 & = z^{3} e^{i\overline{z}^{2}}\left(1 + 2i\overline{z}^{2}\right) = 0
\end{align*}
Notiamo intanto che $z_{1} = 0$ è soluzione.
\begin{equation*}
1 + 2i\overline{z}^{2} = 0\ \ \implies \ \ \overline{z}^{2} = - \frac{1}{2i} = \frac{i}{2}
\end{equation*}
Scrivendo $\overline{z} = x - iy$ abbiamo
\begin{gather*}
\begin{aligned}
\left(x - iy\right)^{2} & = \frac{i}{2}\\
x^{2} - y^{2} - 2ixy & = \frac{i}{2}
\end{aligned}\\
\Downarrow \\
\begin{cases}
x^{2} - y^{2} = 0\\
- 2xy = \frac{1}{2}
\end{cases} \ \ \implies \ \
\begin{cases}
x^{2} - \frac{1}{16x^{2}} = 0\\
y = -\frac{1}{4x}
\end{cases} \ \ \implies \ \
\begin{cases}
16x^{4} - 1 = 0\\
y = -\frac{1}{4x}
\end{cases}
\end{gather*}
Le cui soluzioni sono
\begin{gather*}
x_{2} = \frac{1}{2} \ \ \rightarrow \ \ y_{2} = - \frac{1}{2} \ \ \rightarrow \ \ z_{2} = \frac{1}{2} - \frac{i}{2}\\
x_{3} = - \frac{1}{2} \ \ \rightarrow \ \ y_{3} = \frac{1}{2} \ \ \rightarrow \ \ z_{3} = - \frac{1}{2} + \frac{i}{2}
\end{gather*}

\Soluzione

\begin{defn}
Le condizioni che verificano l'armonicità di $u$ sono
\begin{equation}
u\in C^{2}\left(\RR^{2}\right), \ \Delta u = u_{xx} + u_{yy} = 0
\end{equation}
\end{defn}
La prima è certamente vera, controlliamo che il laplaciano di $u$ sia nullo
\begin{equation*}
\begin{array}{r}
u_{x} = 2 - 2y\ \ \rightarrow \ \ u_{xx} = 0\\
u_{y} = - 2x\ \ \rightarrow \ \ u_{yy} = 0
\end{array} \ \ \implies \ \ \Delta u = 0
\end{equation*}
deduciamo che $u$ è armonica. Trovare la funzione armonica coniugata di $u$ equivale a trovare una funzione armonica $v$ tale che $u, v$ soddisfano le condizioni di Cauchy - Riemann.
\begin{equation*}
\begin{cases}
u_{x} = v_{y}\\
u_{y} = - v_{x}
\end{cases} \ \ \implies \ \
\begin{cases}
2 - 2y = v_{y}\\
- 2x = -v_{x}
\end{cases} \ \ \implies \ \ v\left(x, y\right) = x^{2} - y^{2} + 2y + c, \ \ c\in \RR
\end{equation*}

\Soluzione

\begin{defn}
Sia $\Omega \subseteq \CC $ aperto, $f:\Omega \rightarrow \CC $ continua in $\Omega $, $\gamma :\left[ a, b\right]\rightarrow \Omega $ una curva $C^{1}$ a tratti, allora
\begin{equation*}
\int\nolimits_{\gamma} f(z) dz = \int\nolimits^{b}_{a} f\left(\gamma (t)\right) \gamma'(t) dt
\end{equation*}
dove
\begin{gather*}
\gamma (t) = x(t) + iy(t)\\
\gamma'(t) = x'(t) + iy'(t)
\end{gather*}
\end{defn}
\begin{rem}
L'integrale curvilineo dipende dall'orientazione di $\gamma $
\begin{equation*}
\int\nolimits_{- \gamma} f(z) dz = -\int\nolimits_{\gamma} f(z) dz
\end{equation*}
\end{rem}
\begin{thm}
Sia $\Omega \subseteq \CC $ aperto, $f:\Omega \rightarrow \CC $ continua. TFAE (the followings are equivalent):

1) $\int_{\gamma} f(z) dz$ dipende solo dagli estremi di $\gamma $

2) $\forall \gamma $ chiusa $C^{1}$ a tratti si ha $\int_{\gamma} f(z) dz = 0$

3) $\exists F\in \Hc \left(\Omega \right) \cap C^{1}\left(\Omega \right)$ tale che $F' = f$ in $\Omega $
\end{thm}
\textbf{\underline{1 modo}}
\begin{align*}
\int\nolimits_{\gamma}\left[\overline{z} + \left(z + 1\right)^{7}\right] dz & = \int\nolimits^{0}_{- 1}\underbrace{\left[\left(t - it^{2}\right) + \left(t + it^{2} + 1\right)^{7}\right]}_{f\left(\gamma (t)\right)}\underbrace{\left(1 + 2it\right)}_{\gamma'(t)} dt\\
 & = \int\nolimits^{0}_{- 1}\left[\left(t + 2it^{2} - it^{2} + 2t^{3}\right) + \left(t + it^{2} + 1\right)^{7}\left(1 + 2it\right)\right] dt\\
 & = \int\nolimits^{0}_{- 1}\left[\left(t + it^{2} + 2t^{3}\right) + \left(t + it^{2} + 1\right)^{7}\left(1 + 2it\right)\right] dt\\
 & = \left[\frac{t^{2}}{2} + i\frac{t^{3}}{3} + \frac{t^{4}}{2} + \frac{\left(t + it^{2} + 1\right)^{8}}{8}\right]^{0}_{- 1}\\
 & = \cancel{\frac{1}{8}} - \left\{\frac{1}{2} - \frac{i}{3} + \frac{1}{2} + \cancel{\frac{1}{8}}\right\} = - 1 + \frac{i}{3}
\end{align*}
\begin{rem}
$\overline{z}$ non è olomorfa
\begin{align*}
\lim\limits_{h\rightarrow 0}\frac{f\left(z_{0} + h\right) - f\left(z_{0}\right)}{h} & = \lim\limits_{h\rightarrow 0}\frac{\overline{z_{0} + h} - \overline{z_{0}}}{h}\\
 & = \lim\limits_{h\rightarrow 0}\frac{\cancel{\overline{z_{0}}} + \overline{h} - \cancel{\overline{z_{0}}}}{h}
\end{align*}
non esiste finito. Intuiamo che anche $\left| z\right|^{2} = z\overline{z}$ avrà problemi. In particolare si dimostra che $\left| z\right|^{2}$ è derivabile in senso complesso solo nell'origine. Il suo analogo in $\RR^{2}$ sarebbe $f\left(x, y\right) = x^{2} + y^{2}$, che però è derivabile ovunque.
\end{rem}
\textbf{\underline{2 modo}}

Consideriamo le due parti di $f(z)$
\begin{equation*}
f(z) = \overline{z} + \left(z + 1\right)^{7} = g(z) + h(z)
\end{equation*}
dove
\begin{equation*}
\begin{aligned}
g(z) = \overline{z} & \notin \Hc \left(\Omega \right)\\
h(z) = \left(z + 1\right)^{7} & \in \Hc \left(\Omega \right)
\end{aligned}
\end{equation*}
La funzione $H(z) = \frac{\left(z + 1\right)^{8}}{8}$ è tale che $H'(z) = h(z)$, quindi per il teorema visto l'integrale dipende solo dagli estremi
\begin{equation*}
\int\nolimits_{\gamma} f(z) dz = \int\nolimits_{\gamma}\overline{z} dz + \underbrace{H\left(\gamma (0)\right) - H\left(\gamma (1)\right)}_{\int\nolimits_{\gamma}\left(z + 1\right)^{7} dz} = \dotsc
\end{equation*}

\Soluzione

Scriviamo una possibile parametrizzazione della curva
\begin{equation*}
\gamma (t) = 1 + e^{it}, \ \ t\in \left[ 0, 2\pi \right]
\end{equation*}
e calcoliamo
\begin{align*}
\int\nolimits_{\gamma}\left(\overline{z}\right)^{2} dz & = \int\nolimits^{2\pi}_{0}\left(\overline{1 + e^{it}}\right)^{2} ie^{it} dt\\
 & = \int\nolimits^{2\pi}_{0}\left(1 + e^{- it}\right)^{2} ie^{it} dt\\
 & = \int\nolimits^{2\pi}_{0}\left(1 + 2e^{- it} + e^{- 2it}\right) ie^{it} dt\\
 & = i\int\nolimits^{2\pi}_{0}\left(e^{it} + 2 + e^{- it}\right) dt\\
 & = \left[ e^{it} + 2it - e^{it}\right]^{2\pi}_{0} = 4\pi i
\end{align*}
\textit{Domanda:} $f(z) = \left(\overline{z}\right)^{2}$ ammette primitiva in un qualsiasi dominio $D$ contenente $\gamma $? No, se per assurdo la ammettesse, allora ogni integrale lungo qualsiasi curva chiusa sarebbe nullo.

\Soluzione

Ragioniamo come segue
\begin{align*}
\frac{e^{iz} - e^{- iz}}{2i} & = 3\\
e^{iz} - e^{- iz} & = 6i\\
e^{iz} - \frac{1}{e^{iz}} - 6i & = 0\\
e^{2iz} - 6ie^{iz} - 1 & = 0\\
e^{iz} & = 3i\pm 2\sqrt{2} i = \left(3\pm 2\sqrt{2}\right) i
\end{align*}
Risolviamo per $z$
\begin{align*}
iz_{k} & = \log\left[\left(3\pm 2\sqrt{2}\right) i\right]\\
 & = \ln\left| \left(3\pm 2\sqrt{2}\right) i\right| + i\arg\left[\left(3\pm 2\sqrt{2}\right) i\right]\\
 & = \ln\left| 3\pm 2\sqrt{2}\right| + i\left(\frac{\pi}{2} + 2k\pi \right), \ \ \ \ k\in \ZZ
\end{align*}
Allora
\begin{equation*}
z_{k} = - i\ln\left| 3\pm 2\sqrt{2}\right| + \left(\frac{\pi}{2} + 2k\pi \right), \ \ \ \ k\in \ZZ
\end{equation*}
Che può essere facilmente rappresentato nel piano di Gauss come un insieme di punti.

\Soluzione

Utilizziamo il criterio del rapporto per determinare $R$
\begin{equation*}
\frac{\left| a_{n + 1}\right|}{\left| a_{n}\right|} = \frac{3\left(n + 1\right) + 7}{3n + 7}\xrightarrow{n\rightarrow \infty} 1 = \frac{1}{R}
\end{equation*}
Il raggio di convergenza è $R = 1$.

La serie converge assolutamente $\forall z\in \CC :\left| z\right| < 1$ e non converge per $\left| z\right| > 1$. Se $\left| z\right| = 1$ la serie non converge perché il suo termine generale non tende a $0$. Per ogni $z\in \CC :\left| z\right| < 1$ si ha
\begin{align*}
\sum\limits^{\infty}_{n = 1}\left(3n + 7\right) z^{n} & = 3\sum\limits^{\infty}_{n = 1} nz^{n} + 7\sum\limits^{\infty}_{n = 1} z^{n}\\
 & = 3z\sum\limits^{\infty}_{n = 1} nz^{n - 1} + 7\sum\limits^{\infty}_{n = 1} z^{n}\\
 & = 3z\sum\limits^{\infty}_{n = 1}\frac{d}{dz} z^{n} + 7\sum\limits^{\infty}_{n = 1} z^{n}\\
 & = 3z\frac{d}{dz}\sum\limits^{\infty}_{n = 1} z^{n} + 7\sum\limits^{\infty}_{n = 1} z^{n}\\
 & = 3z\frac{d}{dz}\left(\frac{1}{1 - z} - 1\right) + 7\left(\frac{1}{1 - z} - 1\right)\\
 & = \frac{3z}{\left(1 - z\right)^{2}} + \frac{7}{1 - z} - 7
\end{align*}
\chapter{Esercitazione 2 - Boella}

\ParteEsercizi

\Esercizio{}

Scrivere la serie di potenze di
\begin{equation*}
f(z) = \frac{1}{1 - z} - \frac{1}{2 - z}
\end{equation*}

\Esercizio{}

Scrivere la serie di potenze di
\begin{equation*}
f(z) = z^{3}\sin\frac{1}{z}
\end{equation*}

\Esercizio{}

Classificare le singolarità di
\begin{equation*}
f(z) = \frac{z^{2}}{\sin z}
\end{equation*}

\Esercizio{}

Classificare le singolarità di
\begin{equation*}
f(z) = \frac{z^{2}}{\cos^{2}\left(\frac{1}{z}\right)}
\end{equation*}

\Esercizio{}

Classificare le singolarità di
\begin{equation*}
f(z) = \frac{e^{1/z}}{1 + z^{2}}
\end{equation*}

\Esercizio{(Integrale di Fresnel)}

Calcolare
\begin{equation*}
\int^{+ \infty}_{0}\cos x^{2} dx
\end{equation*}

\ParteSoluzioni

\Soluzione

Osserviamo che ha singolarità in $1$ e $2$, che sono due poli di ordine $1$.
\begin{align*}
f(z) & = \frac{1}{1 - z} - \frac{1}{2 - z}\\
 & = \frac{1}{1 - z} - \frac{1}{2}\frac{1}{1 - \frac{z}{2}}\\
 & = \sum\limits^{\infty}_{n = 0} z^{n} - \frac{1}{2}\sum\limits^{\infty}_{n = 0}\frac{1}{2^{n}} z^{n}\\
 & = \sum\limits^{\infty}_{n = 0}\left(1 - \frac{1}{2^{n + 1}}\right) z^{n}
\end{align*}
Manipoliamola in modo diverso
\begin{align*}
f(z) & = - \frac{1}{z} \cdot \frac{1}{1 - \frac{1}{z}} - \frac{1}{2}\frac{1}{1 - \frac{z}{2}}\\
 & = - \frac{1}{z}\sum\limits^{\infty}_{n = 0} z^{- n} - \frac{1}{2}\sum\limits^{\infty}_{n = 0}\frac{z^{n}}{2^{n}}\\
 & = - \sum\limits^{- 1}_{n = -\infty} z^{n} - \sum\limits^{\infty}_{n = 0}\frac{z^{n}}{2^{n + 1}}
\end{align*}
Le due serie convergono per
\begin{equation*}
1 < \left| z\right| < 2
\end{equation*}

\Soluzione

\begin{align*}
f(z) & = z^{3}\sin\frac{1}{z}\\
 & = z^{3}\sum\limits^{\infty}_{n = 0}\frac{\left(- 1\right)^{n}}{\left(2n + 1\right) !}\left(\frac{1}{z}\right)^{2n + 1}\\
 & = \sum\limits^{\infty}_{n = 0}\frac{\left(- 1\right)^{n}}{\left(2n + 1\right) !} z^{2 - 2n}\\
 & = z^{2} - \frac{1}{3!} + \frac{1}{5!z^{2}}
\end{align*}
Ha singolarità in $z = 0$ di tipo essenziale (ci sono infinite potenze negative nello sviluppo). Il raggio di convergenza è $R = +\infty $. Il residuo nell'origine è $0$ (è il coefficiente $a_{- 1}$ dello sviluppo). All'infinito la funzione si comporta come $z^{2}$. Il punto all'infinito è un polo di ordine $2$. Lo sviluppo vale sia all'infinito che nell'origine.

\Soluzione

I punti singolari sono
\begin{equation*}
z = k\pi, \ \ k\in \ZZ
\end{equation*}
In particolare
\begin{itemize}
\item $z = 0$ è una singolarità eliminabile, perché annulla sia il denominatore che il numeratore, quindi ha residuo nullo.\footnote{L'implicazione \textit{singolarità eliminabile} allora \textit{residuo nullo} vale solo per i \textbf{punti finiti}, non per il punto all'infinito.}
\item $z = k\pi $ sono poli del I ordine, $k\in \ZZ, k\neq 0$, il cui residuo vale
\begin{equation*}
\Res \left(f, k\pi \right) = \frac{g\left(k\pi \right)}{h'\left(k\pi \right)} = \frac{\left(k\pi \right)^{2}}{\left(\sin k\pi \right)'} = \frac{\left(k\pi \right)^{2}}{\cos k\pi} = \frac{\left(k\pi \right)^{2}}{\left(- 1\right)^{k}}
\end{equation*}
\item $z_{\infty}$ è una singolarità non isolata perché non esiste un raggio che contiene tutti i punti singolari. Andando all'infinito avremo infiniti punti singolari.
\end{itemize}

\Soluzione

I punti di singolarità sono dati da
\begin{equation*}
\frac{1}{z_{k}} = \frac{\pi}{2} + k\pi \ \ \ \ k\in \ZZ \ \ \ \ \implies \ \ \ \ z_{k} = \frac{1}{\pi \left(\frac{1}{2} + k\right)}
\end{equation*}
Sono poli del II ordine perché il coseno qui si annulla con ordine $2$. In $z = 0$ è una singolarità non isolata dato che gli $z_{k}$ si addensano intorno a $0$. Il punto $z_{\infty}$ è un polo di ordine II perché per $z\rightarrow z_{\infty}, f(z) \sim z^{2}$.

\Soluzione

Il denominatore si annulla in $z = \pm i$, con un polo di ordine $1$. Ricordando
\begin{equation*}
e^{1/z} = \sum\limits^{\infty}_{n = 0}\frac{1}{n!}\left(\frac{1}{z}\right)^{n} = \sum\limits^{\infty}_{n = 0}\frac{1}{n!} z^{- n}
\end{equation*}
deduciamo che nell'origine si ha una singolarità essenziale. Si ha poi
\begin{align*}
f(z) & = \sum\limits^{\infty}_{n = 0}\frac{1}{n!} z^{- n}\sum\limits^{\infty}_{m = 0}\left(- 1\right)^{m} z^{2m} = \sum\limits^{\infty}_{n, m = 0}\frac{\left(- 1\right)^{m}}{n!} z^{2m - n}
\end{align*}
che converge per $\left| z\right| < 1$. Il residuo vale
\begin{equation*}
a_{- 1} = \sum\limits^{\infty}_{m = 0, n = 2m + 1}\frac{\left(- 1\right)^{m}}{n!} = \sum\limits^{\infty}_{m = 0}\frac{\left(- 1\right)^{m}}{\left(2m + 1\right) !} = \sin 1
\end{equation*}

\Soluzione

Consideriamo $f(z) = e^{iz^{2}}$ e $\gamma = \gamma_{1} + \gamma_{2} + \gamma_{3}$


\begin{figure}[htpb]
	\centering
\tikzset{every picture/.style = {line width = 0.75pt}} %set default line width to 0.75pt

\begin{tikzpicture}[x = 0.75pt, y = 0.75pt, yscale = -1, xscale = 1]
%uncomment if require: \path (0, 176); %set diagram left start at 0, and has height of 176

%Shape: Axis 2D [id:dp035634818552425784]
\draw (200, 140.05) - - (400, 140.05)(220.5, 33.92) - - (220.5, 156) (393, 135.05) - - (400, 140.05) - - (393, 145.05) (215.5, 40.92) - - (220.5, 33.92) - - (225.5, 40.92) ;
%Straight Lines [id:da7560460514537715]
\draw [color = {rgb, 255:red, 208; green, 2; blue, 27}, draw opacity = 1 ][line width = 1.5] (220.5, 140.8) - - (283.5, 77.8) ;
\draw [shift = {(252, 109.3)}, rotate = 315] [fill = {rgb, 255:red, 208; green, 2; blue, 27}, fill opacity = 1 ][line width = 0.08] [draw opacity = 0] (13.4, - 6.43) - - (0, 0) - - (13.4, 6.44) - - (8.9, 0) - - cycle ;
%Curve Lines [id:da9635953000928117]
\draw [color = {rgb, 255:red, 208; green, 2; blue, 27}, draw opacity = 1 ][line width = 1.5] (283.5, 77.8) .. controls (298.5, 95.8) and (306.5, 108.8) .. (310, 140.11) ;
\draw [shift = {(302.31, 106.84)}, rotate = 65.41] [fill = {rgb, 255:red, 208; green, 2; blue, 27}, fill opacity = 1 ][line width = 0.08] [draw opacity = 0] (13.4, - 6.43) - - (0, 0) - - (13.4, 6.44) - - (8.9, 0) - - cycle ;
%Straight Lines [id:da9987973908776102]
\draw [color = {rgb, 255:red, 208; green, 2; blue, 27}, draw opacity = 1 ][line width = 1.5] (220.5, 140.8) - - (310, 140.11) ;
\draw [shift = {(265.25, 140.45)}, rotate = 539.56] [fill = {rgb, 255:red, 208; green, 2; blue, 27}, fill opacity = 1 ][line width = 0.08] [draw opacity = 0] (13.4, - 6.43) - - (0, 0) - - (13.4, 6.44) - - (8.9, 0) - - cycle ;

% Text Node
\draw (235, 81.4) node [anchor = north west][inner sep = 0.75pt] {$\gamma_{3}$};
% Text Node
\draw (312, 91.4) node [anchor = north west][inner sep = 0.75pt] {$\gamma_{2}$};
% Text Node
\draw (275.25, 142.85) node [anchor = north west][inner sep = 0.75pt] {$\gamma_{1}$};
% Text Node
\draw (267, 57.4) node [anchor = north west][inner sep = 0.75pt] {$R$};


\end{tikzpicture}
\end{figure}
\FloatBarrier

La funzione $f(z)$ è olomorfa, allora
\begin{equation*}
{\displaystyle \int\nolimits_{\gamma} f(z) dz = 0 = \int\nolimits_{\gamma_{1}} f(z) dz + \int\nolimits_{\gamma_{2}} f(z) dz + \int\nolimits_{\gamma_{3}} f}(z) dz
\end{equation*}
\begin{itemize}
\item La curva $\gamma_{3}$ ha parametrizzazione $z = re^{i\vartheta}$ con $r$ variabile e $\vartheta $ fissato. Il valore di $\vartheta $ si determina imponendo di potersi ricondurre all'integrale di Gauss, ovvero
\begin{equation*}
iz^{2} = i\left(re^{i\vartheta}\right)^{2} = - r^{2} \ \ \implies \ \ \vartheta = \frac{\pi}{4}
\end{equation*}

Deduciamo il nuovo differenziale
\begin{equation*}
z = re^{i\vartheta} \ \ \implies \ \ dz = e^{i\vartheta} dr = e^{i\frac{\pi}{4}} dr
\end{equation*}

Allora il terzo integrale diventa
\begin{equation*}
\int\nolimits_{\gamma_{3}} e^{iz^{2}} dz = \int\nolimits^{0}_{R} e^{- r^{2}} e^{i\frac{\pi}{4}} dr
\end{equation*}

Facciamo ora tendere $R\rightarrow + \infty $
\begin{equation*}
\lim\limits_{R\rightarrow + \infty}\int\nolimits^{0}_{R} e^{- r^{2}} e^{i\frac{\pi}{4}} dr = e^{i\frac{\pi}{4}}\lim\limits_{R\rightarrow + \infty}\int\nolimits^{0}_{R} e^{- r^{2}} dr = -e^{i\frac{\pi}{4}}\int\nolimits^{+ \infty}_{0} e^{- r^{2}} dr = -e^{i\frac{\pi}{4}}\frac{\sqrt{\pi}}{2}
\end{equation*}
\item La curva $\gamma_{2}$ ha parametrizzazione $z = Re^{i\vartheta}$, con $R$ fissato e $\vartheta $ variabile in $\left[ 0, \frac{\pi}{4}\right]$. Deduciamo il nuovo differenziale
\begin{equation*}
z = Re^{i\vartheta} \ \ \implies \ \ dz = Rie^{i\vartheta} d\vartheta
\end{equation*}Il secondo integrale diventa

\begin{align*}
\int\nolimits_{\gamma_{2}} e^{iz^{2}} dz & = \int\nolimits^{\pi /4}_{0} e^{iR^{2} e^{i2\vartheta}} Rie^{i\vartheta} d\vartheta \\
 & = \int\nolimits^{\pi /4}_{0} iRe^{iR^{2} e^{i2\vartheta} + i\vartheta} d\vartheta \\
 & = \int\nolimits^{\pi /4}_{0} iRe^{iR^{2}\left[\cos\left(2\vartheta \right) + i\sin\left(2\vartheta \right)\right] + i\vartheta} d\vartheta \\
 & = \int\nolimits^{\pi /4}_{0} iRe^{iR^{2}\cos\left(2\vartheta \right) - R^{2}\sin\left(2\vartheta \right) + i\vartheta} d\vartheta \\
 & = \int\nolimits^{\pi /4}_{0} iRe^{iR^{2}\cos\left(2\vartheta \right)} e^{- R^{2}\sin\left(2\vartheta \right)} e^{i\vartheta} d\vartheta
\end{align*}

Poi
\begin{align*}
\left| \int\nolimits_{\gamma_{2}} e^{iz^{2}} dz\right| & = \left| \int\nolimits^{\pi /4}_{0} iRe^{iR^{2}\cos\left(2\vartheta \right)} e^{- R^{2}\sin\left(2\vartheta \right)} e^{i\vartheta} d\vartheta \right| \\
 & \leq \int\nolimits^{\pi /4}_{0}\left| i\right| \left| R\right| \left| e^{iR^{2}\cos\left(2\vartheta \right)}\right| \left| e^{- R^{2}\sin\left(2\vartheta \right)}\right| \left| e^{i\vartheta}\right| d\vartheta = R\int\nolimits^{\pi /4}_{0} e^{- R^{2}\sin\left(2\vartheta \right)} d\vartheta
\end{align*}

Ricordando ora la seguente disuguaglianza
\begin{equation*}
\sin\left(2\vartheta \right) > \frac{4\vartheta}{\pi}
\end{equation*}

otteniamo
\begin{equation*}
R\int\nolimits^{\pi /4}_{0} e^{- R^{2}\sin\left(2\vartheta \right)} d\vartheta \leq R\int\nolimits^{\pi /4}_{0} e^{- R^{2}\frac{4\vartheta}{\pi}} d\vartheta = \left[ - \frac{\pi}{4R^{2}} e^{- R^{2}\frac{4\vartheta}{\pi}}\right]^{\pi /4}_{0} = \frac{\pi}{4R^{2}}\left(1 - e^{- R^{2}}\right)
\end{equation*}

la quale
\begin{equation*}
\frac{\pi}{4R^{2}}\left(1 - e^{- R^{2}}\right)\xrightarrow{R\rightarrow + \infty} 0
\end{equation*}
\item La curva $\gamma_{1}$ ha parametrizzazione $z = x = R$, tuttavia non serve fare calcoli in quanto sappiamo che la somma dei tre integrali fa $0$ e ne abbiamo già calcolati due, pertanto
\begin{equation*}
\int\nolimits_{\gamma_{1}} e^{iz^{2}} dz = -\int\nolimits_{\gamma_{3}} e^{iz^{2}} dz = e^{i\frac{\pi}{4}}\frac{\sqrt{\pi}}{2}
\end{equation*}
\end{itemize}

Notiamo a questo punto che l'integrale
\begin{equation*}
\int\nolimits_{\gamma_{1}} e^{iz^{2}} dz
\end{equation*}
non è altro che un integrale complesso, fatto però sulla curva $\gamma_{1}$, la quale coincide con la retta reale positiva. Di conseguenza
\begin{equation*}
\int\nolimits_{\gamma_{1}} e^{iz^{2}} dz = \int\nolimits^{+ \infty}_{0} e^{ix^{2}} dx
\end{equation*}
che sarà quindi una quantità complessa, ma l'integrale richiesto è proprio la parte reale di questa quantità
\begin{equation*}
\int\nolimits^{+ \infty}_{0}\cos\left(x^{2}\right) dx = \Re \left(\int\nolimits^{+ \infty}_{0} e^{ix^{2}} dz\right) = \Re \left(e^{i\frac{\pi}{4}}\frac{\sqrt{\pi}}{2}\right) = \cos\left(\frac{\pi}{4}\right)\frac{\sqrt{\pi}}{2} = \sqrt{\frac{\pi}{8}}
\end{equation*}
\chapter{Esercitazione 2 - Potrich}

\ParteEsercizi

\Esercizio{(Integrali di Fresnel)}

Calcolare
\begin{equation*}
\int\nolimits^{+ \infty}_{0}\cos t^{2} dt\ \ \ \ \ \ \ \ \int\nolimits^{+ \infty}_{0}\sin t^{2} dt
\end{equation*}

\Esercizio{}

Determinare l'anello di convergenza della seguente serie di Laurent
\begin{equation*}
f(z) = \sum\limits^{\infty}_{n = -\infty}\frac{z^{2n}}{9^{\left| n\right|}}
\end{equation*}

\Esercizio{}

Determinare lo sviluppo di Laurent di
\begin{equation*}
f(z) = \frac{1}{2iz - z^{2}}
\end{equation*}
in
\begin{equation*}
A = \left\{z\in \CC :0 < \left| z\right| < 2\right\} \ \ \ \ \ \ \ \ B = \left\{z\in \CC :\left| z - 2i\right| > 2\right\}
\end{equation*}

\Esercizio{}

Determinare le singolarità e i residui.
\begin{equation*}
F(z) = \frac{1}{z - z^{3}} = \frac{1}{z\left(1 - z^{2}\right)} = \frac{1}{z\left(1 + z\right)\left(1 - z\right)}
\end{equation*}

\Esercizio{}

Determinare le singolarità e i residui.
\begin{equation*}
f(z) = \frac{z^{5}}{\left(1 + iz\right)^{2}}
\end{equation*}

\Esercizio{}

Determinare le singolarità e i residui.
\begin{equation*}
f(z) = \frac{e^{\frac{1}{z - 1}}}{z^{2}\left(z^{2} + 4\right)}
\end{equation*}

\Esercizio{}

Sia
\begin{equation*}
f(z) = \frac{e^{\frac{1}{z - 1}}}{z - 2}
\end{equation*}
\begin{enumerate}
\item Classificare le singolarità di $f$
\item Determinare la serie di Laurent centrata in $z = 1$, precisandone l'insieme di convergenza
\item Calcolare i residui di $f$
\item Calcolare $\int_{\gamma} f(z) dz$ dove $\gamma :\left| z\right| = 4$ percorsa in senso antiorario
\end{enumerate}

\ParteSoluzioni

\Soluzione

Si tratta di due integrali che convergono in senso improprio. Dimostriamolo:
\begin{equation*}
\int\nolimits^{+ \infty}_{0}\cos t^{2} dt = \left\{
\begin{array}{c}
t^{2} = x\\
t = \sqrt{x}\\
dt = \frac{1}{2\sqrt{x}} dx
\end{array}\right\} = \frac{1}{2}\int\nolimits^{+ \infty}_{0}\frac{\cos x}{\sqrt{x}} dx
\end{equation*}
Tale integrale converge in un intorno destro di $0$ in senso improprio. Studiamo ora la convergenza in un intorno di $ + \infty $. Considerando le seguenti funzioni
\begin{equation*}
\begin{array}{r l}
f(x) = \frac{1}{\sqrt{x}} & \rightarrow f'(x) = - \frac{1}{2x^{3/2}}\\
g(x) = \sin x & \rightarrow g'(x) = \cos x
\end{array}
\end{equation*}
possiamo integrare per parti
\begin{equation*}
\int\nolimits^{a}_{1}\frac{\cos x}{\sqrt{x}} dx = \left[\frac{\sin x}{\sqrt{x}}\right]^{a}_{1} + \int\nolimits^{a}_{1}\frac{\sin x}{2x^{3/2}} dx = \frac{\sin a}{\sqrt{a}} - \sin 1 + \int\nolimits^{a}_{1}\frac{\sin x}{2x^{3/2}} dx
\end{equation*}
che per $a\rightarrow + \infty $ tende a
\begin{equation*}
- \sin 1 + \int\nolimits^{+ \infty}_{1}\frac{\sin x}{2x^{3/2}} dx
\end{equation*}
dobbiamo quindi studiare la convergenza di tale integrale.
\begin{rem}
Convergenza di integrali.
\begin{gather*}
\int\nolimits^{+ \infty}_{1}\frac{1}{x^{\alpha}} dx\ \ \text{converge} \ \ \iff \ \ \alpha > 1\\
\int\nolimits^{1}_{0}\frac{1}{x^{\alpha}} dx\ \ \text{converge} \ \ \iff \ \ \alpha < 1
\end{gather*}
\end{rem}
Pertanto
\begin{equation*}
\left| \int\nolimits^{+ \infty}_{1}\frac{\sin x}{2x^{3/2}} dx\right| \leq \int\nolimits^{+ \infty}_{1}\left| \frac{\sin x}{2x^{3/2}}\right| dx \leq \int\nolimits^{+ \infty}_{1}\frac{1}{2x^{3/2}} dx
\end{equation*}
in cui $\alpha = 3/2 > 1$, pertanto converge.

Calcoliamo
\begin{equation*}
J = \int\nolimits^{+ \infty}_{0} e^{it^{2}} dt = \int\nolimits^{+ \infty}_{0}\cos t^{2} dt + i\int\nolimits^{+ \infty}_{0}\sin t^{2} dt
\end{equation*}
Poniamo
\begin{equation*}
f(z) = e^{iz^{2}} \in \Hc \left(\CC \right)
\end{equation*}
Consideriamo una curva $\gamma = \gamma_{1} \cup \gamma_{2} \cup \gamma_{3}$ fatta come


\begin{figure}[htpb]
	\centering
\tikzset{every picture/.style = {line width = 0.75pt}} %set default line width to 0.75pt

\begin{tikzpicture}[x = 0.75pt, y = 0.75pt, yscale = -1, xscale = 1]
%uncomment if require: \path (0, 176); %set diagram left start at 0, and has height of 176

%Shape: Axis 2D [id:dp17170898643653087]
\draw (220, 126.6) - - (420, 126.6)(240.5, 20.47) - - (240.5, 142.55) (413, 121.6) - - (420, 126.6) - - (413, 131.6) (235.5, 27.47) - - (240.5, 20.47) - - (245.5, 27.47) ;
%Straight Lines [id:da9725409241349892]
\draw [color = {rgb, 255:red, 208; green, 2; blue, 27}, draw opacity = 1 ][line width = 1.5] (240.5, 127.34) - - (303.5, 64.34) ;
\draw [shift = {(272, 95.84)}, rotate = 315] [fill = {rgb, 255:red, 208; green, 2; blue, 27}, fill opacity = 1 ][line width = 0.08] [draw opacity = 0] (13.4, - 6.43) - - (0, 0) - - (13.4, 6.44) - - (8.9, 0) - - cycle ;
%Curve Lines [id:da9846937875734199]
\draw [color = {rgb, 255:red, 208; green, 2; blue, 27}, draw opacity = 1 ][line width = 1.5] (303.5, 64.34) .. controls (318.5, 82.34) and (326.5, 95.34) .. (330, 126.66) ;
\draw [shift = {(322.31, 93.38)}, rotate = 65.41] [fill = {rgb, 255:red, 208; green, 2; blue, 27}, fill opacity = 1 ][line width = 0.08] [draw opacity = 0] (13.4, - 6.43) - - (0, 0) - - (13.4, 6.44) - - (8.9, 0) - - cycle ;
%Straight Lines [id:da17260088434526732]
\draw [color = {rgb, 255:red, 208; green, 2; blue, 27}, draw opacity = 1 ][line width = 1.5] (240.5, 127.34) - - (330, 126.66) ;
\draw [shift = {(285.25, 127)}, rotate = 539.56] [fill = {rgb, 255:red, 208; green, 2; blue, 27}, fill opacity = 1 ][line width = 0.08] [draw opacity = 0] (13.4, - 6.43) - - (0, 0) - - (13.4, 6.44) - - (8.9, 0) - - cycle ;

% Text Node
\draw (255, 67.95) node [anchor = north west][inner sep = 0.75pt] {$\gamma_{3}$};
% Text Node
\draw (332, 77.95) node [anchor = north west][inner sep = 0.75pt] {$\gamma_{2}$};
% Text Node
\draw (295.25, 129.4) node [anchor = north west][inner sep = 0.75pt] {$\gamma_{1}$};
% Text Node
\draw (287, 43.95) node [anchor = north west][inner sep = 0.75pt] {$R$};


\end{tikzpicture}
\end{figure}
\FloatBarrier

le cui parti sono parametrizzate come segue
\begin{itemize}
\item $\gamma_{1}(t) = t$ con $t\in \left[ 0, R\right]$
\item $\gamma_{2}(t) = Re^{it}$ con $t\in \left[ 0, \frac{\pi}{4}\right]$
\item $ - \gamma_{3}(t) = te^{i\frac{\pi}{4}}$ con $t\in \left[ 0, R\right]$
\end{itemize}

Analizziamo cosa succede all'integrale di $f(z)$ sui vari percorsi
\begin{itemize}
\item su $\gamma_{1}$
\begin{equation*}
\int\nolimits_{\gamma_{1}} f(z) dz = \int\nolimits^{R}_{0} e^{it^{2}} dt\xrightarrow{R\rightarrow + \infty}\int\nolimits^{+ \infty}_{0} e^{it^{2}} dt = J
\end{equation*}
\item su $\gamma_{3}$
\begin{align*}
\int\nolimits_{\gamma_{3}} f(z) dz & = - \int\nolimits^{R}_{0} f\left(\gamma (t)\right) \gamma'(t) dt = -\int\nolimits^{R}_{0} e^{it^{2} e^{i\frac{\pi}{2}}} e^{i\frac{\pi}{4}} dt\\
 & = - \int\nolimits^{R}_{0} e^{it^{2} i} e^{i\frac{\pi}{4}} dt = -e^{i\frac{\pi}{4}}\int\nolimits^{R}_{0} e^{- t^{2}} dt\xrightarrow{R\rightarrow + \infty} - e^{i\frac{\pi}{4}}\int\nolimits^{+ \infty}_{0} e^{- t^{2}} dt
\end{align*}

\begin{rem}
Ricordiamo l'integrale di Gauss
\begin{equation*}
\int\nolimits_{\RR} e^{- x^{2}} dx = \sqrt{\pi}
\end{equation*}
\end{rem}

Otteniamo che
\begin{equation*}
\int\nolimits_{\gamma_{3}} f(z) dz\xrightarrow{R\rightarrow + \infty} - e^{i\frac{\pi}{4}}\frac{\sqrt{\pi}}{2}
\end{equation*}
\item su $\gamma_{2}$ dobbiamo cercare una maggiorazione in modo da far vedere che tende a $0$. Ricordiamo anche che $\left| e^{z}\right| = e^{\Re (z)}$
\begin{align*}
0 & \leq \left| \int\nolimits_{\gamma_{2}} f(z) dz\right| = \left| \int\nolimits^{\pi /4}_{0} f\left(\gamma (t)\right) \gamma'(t) dt\right| = \left| \int\nolimits^{\pi /4}_{0} e^{iR^{2} e^{2it}} Rie^{it} dt\right| \\
 & \leq R\int\nolimits^{\pi /4}_{0}\left| e^{iR^{2} e^{2it}}\right| dt = R\int\nolimits^{\pi /4}_{0}\left| e^{iR^{2} e^{2it}}\right| dt = R\int\nolimits^{\pi /4}_{0} e^{\Re \left[ iR^{2}\left(\cos\left(2t\right) + i\sin\left(2t\right)\right)\right]} dt\\
 & = R\int\nolimits^{\pi /4}_{0} e^{- R^{2}\sin\left(2t\right)} dt = \left\{
 \begin{array}{c}
2t = x\\
dt = \frac{dx}{2}
\end{array}\right\} = \frac{R}{2}\int\nolimits^{\pi /2}_{0} e^{- R^{2}\sin x} dx
\end{align*}

minoriamo ora il seno



\begin{figure}[htpb]
	\centering
\tikzset{every picture/.style = {line width = 0.75pt}} %set default line width to 0.75pt

\begin{tikzpicture}[x = 0.75pt, y = 0.75pt, yscale = -1, xscale = 1]
%uncomment if require: \path (0, 140); %set diagram left start at 0, and has height of 140

%Shape: Axis 2D [id:dp521862829598075]
\draw (200, 100.03) - - (361.5, 100.03)(210.67, 4) - - (210.67, 115.03) (354.5, 95.03) - - (361.5, 100.03) - - (354.5, 105.03) (205.67, 11) - - (210.67, 4) - - (215.67, 11) ;
%Curve Lines [id:da7711478785582004]
\draw [line width = 1.5] (210.5, 100.03) .. controls (238.5, 70.03) and (261.5, 41.03) .. (316.5, 41.03) ;
%Straight Lines [id:da5280090308997825]
\draw [dash pattern = {on 0.84pt off 2.51pt}] (210.5, 41.03) - - (316.5, 41.03) ;
%Straight Lines [id:da665885556586004]
\draw [dash pattern = {on 0.84pt off 2.51pt}] (316.5, 41.03) - - (316.5, 100.03) ;
%Straight Lines [id:da2630936419717216]
\draw [line width = 1.5] (316.5, 41.03) - - (210.5, 100.03) ;

% Text Node
\draw (193, 28.4) node [anchor = north west][inner sep = 0.75pt] {$1$};
% Text Node
\draw (311.5, 106.43) node [anchor = north west][inner sep = 0.75pt] {$\frac{\pi}{2}$};
% Text Node
\draw (195, 104.4) node [anchor = north west][inner sep = 0.75pt] {$0$};
% Text Node
\draw (325, 54.4) node [anchor = north west][inner sep = 0.75pt] [font = \small] {$y = \frac{2}{\pi} x$};
% Text Node
\draw (247, 11.4) node [anchor = north west][inner sep = 0.75pt] [font = \small] {$y = \sin x$};


\end{tikzpicture}
\end{figure}
\FloatBarrier


otteniamo
\begin{align*}
\frac{R}{2}\int\nolimits^{\pi /2}_{0} e^{- R^{2}\sin x} dx & \leq \frac{R}{2}\int\nolimits^{\pi /2}_{0} e^{- \frac{2}{\pi} R^{2} x} dx\\
 & = \frac{R}{2}\left[ - \frac{\pi}{2R^{2}} e^{- \frac{2R^{2}}{\pi} x}\right]^{\pi /2}_{0} = \frac{\pi}{4R}\left(1 - e^{- R^{2}}\right)\xrightarrow{R\rightarrow + \infty} 0
\end{align*}
\end{itemize}

Per il teorema dell'integrale nullo di Cauchy
\begin{equation*}
0 = \oint\nolimits_{\gamma} f(z) dz = \int\nolimits_{\gamma_{1}} f(z) dz + \int\nolimits_{\gamma_{2}} f(z) dz + \int\nolimits_{\gamma_{3}} f(z) dz
\end{equation*}
sostituiamo i vari elementi, sempre per $R\rightarrow + \infty $
\begin{equation*}
0 = J + 0 - e^{i\frac{\pi}{4}}\frac{\sqrt{\pi}}{2}
\end{equation*}
allora
\begin{equation*}
J = \int\nolimits^{+ \infty}_{0} e^{it^{2}} dt = e^{i\frac{\pi}{4}}\frac{\sqrt{\pi}}{2} = \frac{\sqrt{2\pi}}{4}\left(1 + i\right)
\end{equation*}
ma quindi
\begin{align*}
\int\nolimits^{+ \infty}_{0}\cos t^{2} dt & = \Re (J) = \frac{\sqrt{2\pi}}{4}\\
\int\nolimits^{+ \infty}_{0}\sin t^{2} dt & = \Im (J) = \frac{\sqrt{2\pi}}{4}
\end{align*}

\Soluzione

\begin{align*}
f(z) = & \sum\limits^{\infty}_{n = -\infty}\frac{z^{2n}}{9^{\left| n\right|}} = \sum_{n < 0}\frac{z^{2n}}{9^{\left| n\right|}} + \sum_{n \geq 0}\frac{z^{2n}}{9^{n}}\\
 & = \sum_{n > 0}\frac{z^{- 2n}}{9^{n}} + \sum_{n \geq 0}\frac{z^{2n}}{9^{n}}\\
 & = \sum_{n > 0}\frac{1}{9^{n} z^{2n}} + \sum_{n \geq 0}\frac{z^{2n}}{9^{n}}
\end{align*}
Affinché la serie di Laurent converga devono convergere entrambe le serie. La prima converge se
\begin{equation*}
\lim\limits_{n\rightarrow \infty}\left| \frac{a_{n + 1}}{a_{n}}\right| = \lim\limits_{n\rightarrow \infty}\frac{9^{n}\left| z\right|^{2n}}{9^{n + 1}\left| z\right|^{2\left(n + 1\right)}} = \frac{1}{9\left| z\right|^{2}} < 1
\end{equation*}
ossia $\left| z\right|^{2} > \frac{1}{9}$, cioè $\left| z\right| > \frac{1}{3}$.

La seconda serie converge se
\begin{equation*}
\lim\limits_{n\rightarrow \infty}\left| \frac{b_{n + 1}}{b_{n}}\right| = \lim\limits_{n\rightarrow \infty}\frac{\left| z\right|^{2\left(n + 1\right)}}{9^{n + 1}}\frac{9^{n}}{\left| z\right|^{2n}} = \frac{1}{9}\left| z\right|^{2} < 1
\end{equation*}
ossia $\left| z\right|^{2} < 9$, cioè $\left| z\right| < 3$.

Allora l'anello di convergenza è $A = \left\{z\in \CC :\frac{1}{3} < \left| z\right| < 3\right\}$.

\Soluzione

Possiamo vedere la $f$ come
\begin{equation*}
f(z) = \frac{1}{2iz - z^{2}} = \frac{1}{z\left(2i - z\right)} = \frac{1}{z} \cdot \frac{1}{2i} \cdot \frac{1}{1 - \frac{z}{2i}}
\end{equation*}
In $A$
\begin{align*}
f(z) & = \frac{1}{z} \cdot \frac{1}{2i} \cdot \frac{1}{1 - \frac{z}{2i}} = \frac{1}{z} \cdot \frac{1}{2i} \cdot \sum\limits^{\infty}_{n = 0}\frac{z^{n}}{\left(2i\right)^{n}}\\
 & = \sum\limits^{\infty}_{n = 0}\frac{z^{n - 1}}{\left(2i\right)^{n + 1}} = \sum\limits^{\infty}_{n = -1}\frac{z^{n}}{\left(2i\right)^{n + 2}}
\end{align*}
In $B$
\begin{align*}
f(z) & = \frac{1}{z\left(2i - z\right)} = \frac{A}{z} + \frac{B}{2i - z} = \frac{A2i - Az + Bz}{z\left(2i - z\right)} = \frac{A2i + z\left(B - A\right)}{z\left(2i - z\right)}
\end{align*}
allora $A = B = -\frac{i}{2}$
\begin{align*}
f(z) & = - \frac{i}{2}\left\{\frac{1}{z} + \frac{1}{2i - z}\right\}\\
 & = - \frac{i}{2}\left\{\frac{1}{\left(z - 2i\right) + 2i} + \frac{1}{2i - z}\right\}\\
 & = - \frac{i}{2}\left\{\frac{1}{z - 2i} \cdot \frac{1}{1 + \frac{2i}{z - 2i}} + \frac{1}{2i - z}\right\}\\
 & = - \frac{i}{2}\left\{\frac{1}{z - 2i} \cdot \frac{1}{1 - \frac{- 2i}{z - 2i}} - \frac{1}{z - 2i}\right\}
\end{align*}
Essendo in $B$
\begin{equation*}
\left| \frac{- 2i}{z - 2i}\right| < 1\ \ \iff \ \ 2 < \left| z - 2i\right|
\end{equation*}
possiamo scrivere la serie geometrica
\begin{align*}
f(z) & = - \frac{i}{2}\left\{\frac{1}{z - 2i} \cdot \sum\limits^{\infty}_{n = 0}\left(\frac{- 2i}{z - 2i}\right)^{n} - \frac{1}{z - 2i}\right\}\\
 & = - \frac{i}{2}\left\{\frac{1}{z - 2i}\left[\sum\limits^{\infty}_{n = 0}\left(\frac{- 2i}{z - 2i}\right)^{n} - 1\right]\right\}\\
 & = - \frac{i}{2}\left\{\frac{1}{z - 2i}\sum\limits^{\infty}_{n = 1}\left(\frac{- 2i}{z - 2i}\right)^{n}\right\}\\
 & = - \frac{i}{2}\left\{\sum\limits^{\infty}_{n = 1}\frac{\left(- 2i\right)^{n}}{\left(z - 2i\right)^{n + 1}}\right\}
\end{align*}

\Soluzione

\begin{thm}
Sia $f\in \Hc \left(D\left(z_{0}, R\right) \setminus \left\{z_{0}\right\}\right)$, cioè $z_{0}$ è una \textbf{singolarità isolata} per $f$. Allora localmente
\begin{equation*}
f(z) = \sum\limits^{+ \infty}_{n = -\infty} a_{n}\left(z - z_{0}\right)^{n}
\end{equation*}
\end{thm}
\begin{thm}
Sia $M = \left\{n\in \ZZ :n < 0, a_{n} \neq 0\right\}$ l'insieme degli indici della parte singolare dello sviluppo di Laurent. $z_{0}$ è

\begin{itemize}
\item \textbf{singolarità eliminabile} se

\begin{equation*}
\exists \lim\limits_{z\rightarrow z_{0}} f(z) \in \CC \ \ \iff \ \ M = \emptyset \ \ \iff \ \ f\ \text{è limitata su} \ \Uc \left(z_{0}\right)
\end{equation*}
\item \textbf{polo} se

\begin{equation*}
\exists \lim\limits_{z\rightarrow z_{0}} f(z) = \infty \ \ \iff \ \ 0 < \left| M\right| < + \infty
\end{equation*}

dove $ - \min M = k$ è l'\textbf{ordine} del polo.
\item \textbf{singolarità essenziale} se

\begin{gather*}
\nexists \lim\limits_{z\rightarrow z_{0}} f(z) \ \ \iff \ \ \left| M\right| = \infty \\
\iff \ \ \forall \varepsilon \in \left(0, R\right) \ \text{l'insieme immagine} \ f\left(\left\{0 < \left| z - z_{0}\right| < \varepsilon \right\}\right) \ \text{è denso in} \ \CC
\end{gather*}
\end{itemize}
\end{thm}
\begin{defn}
Si dice residuo di $f$ in $z_{0}$
\begin{equation*}
\Res \left(f, z_{0}\right) \coloneqq a_{- 1}
\end{equation*}
\end{defn}
\begin{thm}
Sia $f\in \Hc \left(\CC \setminus \overline{D\left(z_{0}, R\right)}\right)$ ed $M = \left\{n\in \ZZ :n > 0, a_{n} \neq 0\right\}$. Il punto $\infty $ è

\begin{itemize}
\item \textbf{singolarità eliminabile} se e solo se
\begin{equation*}
\exists \lim\limits_{z\rightarrow \infty} f(z) \in \CC \ \ \iff \ \ M = \emptyset
\end{equation*}
\item \textbf{polo} se e solo se
\begin{equation*}
\exists \lim\limits_{z\rightarrow \infty} f(z) = \infty \ \ \iff \ \ 0 < \left| M\right| < \infty
\end{equation*}
\item \textbf{singolarità essenziale} se e solo se
\begin{equation*}
\nexists \lim\limits_{z\rightarrow + \infty} f(z) \ \ \iff \ \ \left| M\right| = \infty
\end{equation*}
\end{itemize}
\end{thm}
\begin{defn}
Si dice residuo di $f$ in $\infty $
\begin{equation*}
\Res \left(f, \infty \right) \coloneqq - a_{- 1}
\end{equation*}
\end{defn}
\begin{thm}
[di De l'Hôpital] Siano $f$ e $g$ funzioni analitiche in $B_{R}\left(z_{0}\right)$ tali che $f\left(z_{0}\right) = g\left(z_{0}\right) = 0$ e $g'\left(z_{0}\right) \neq 0$ (quindi solo la forma $\frac{0}{0}$ al finito). Allora
\begin{equation*}
\lim\limits_{z\rightarrow z_{0}}\frac{f(z)}{g(z)} = \frac{f'\left(z_{0}\right)}{g'\left(z_{0}\right)}
\end{equation*}
\end{thm}
\begin{thm}
Le seguenti affermazioni sono equivalenti
\begin{itemize}
\item $f$ ha in $z_{0}$ un polo di ordine $m$
\item $g(z) = \left(z - z_{0}\right)^{m} f(z)$ ha in $z_{0}$ una singolarità eliminabile e $\lim\limits_{z\rightarrow z_{0}} g(z) \neq 0$
\item $f(z) \approx \left| z - z_{0}\right|^{m}$ per $z\rightarrow z_{0}$
\item $\frac{1}{f(z)}$ ha in $z_{0}$ uno zero di ordine $m$
\end{itemize}
\end{thm}
\begin{thm}
Siano $f(z)$ e $g(z)$ tali che $f\left(z_{0}\right) = 0 = g\left(z_{0}\right)$ e siano $n$ ed $m$ rispettivamente, gli ordini di $z_{0}$ come zero di $f$ e $g$. La funzione $\frac{f(z)}{g(z)}$ ha in $z_{0}$:

\begin{itemize}
\item una singolarità eliminabile che è uno zero d'ordine $n - m$ se e solo se $n > m$
\item una singolarità eliminabile che non è uno zero se e solo se $n = m$
\item un polo d'ordine $m - n$ se e solo se $m > n$.
\end{itemize}
\end{thm}
\begin{thm}
Sia $f\in \Hc \left(\CC \setminus \left\{z_{1}, \dotsc, z_{n}\right\}\right)$ allora
\begin{equation*}
\sum\limits^{n}_{k = 1}\Res \left(f, z_{k}\right) + \Res \left(f, \infty \right) = 0
\end{equation*}
\end{thm}
\begin{thm}
Sia $f\in \Hc \left(\CC \setminus \left\{z_{1}, \dotsc, z_{n}\right\}\right)$. Se $R > \max_{1 \leq k \leq n}\left| z_{k}\right| $ allora
\begin{equation*}
\int\nolimits_{\partial D\left(0, R\right)} f(z) dz = 2\pi i\left[\sum\limits^{n}_{k = 1}\Res \left(f, z_{k}\right)\right]
\end{equation*}
\end{thm}
\begin{rem}
Come si calcolano i residui?
Nei punti all'infinito si definisce
\begin{equation*}
\Res \left(f, \infty \right) = \Res \left(- \frac{1}{z^{2}} f\left(\frac{1}{z}\right), 0\right)
\end{equation*}
Nei punti $z_{0} \neq \infty $
\begin{enumerate}
\item se è una singolarità eliminabile:
\begin{equation*}
\Res \left(f, z_{0}\right) = 0
\end{equation*}

(attenzione: solo per i punti finiti)
\item se è un polo, c'è la \textbf{formula dei poli}
\begin{enumerate}
\item semplice (di ordine $p = 1$)
\begin{equation*}
\Res \left(f, z_{0}\right) = \lim\limits_{z\rightarrow z_{0}} f(z)\left(z - z_{0}\right)
\end{equation*}
\item di ordine $p \geq 2$
\begin{equation*}
\Res \left(f, z_{0}\right) = \frac{1}{\left(p - 1\right) !}\lim\limits_{z\rightarrow z_{0}}\left\{\frac{d^{p - 1}}{dz^{p - 1}}\left[ f(z)\left(z - z_{0}\right)^{p}\right]\right\}
\end{equation*}
\end{enumerate}
\item se è una singolarità essenziale bisogna per forza fare lo sviluppo di Laurent (che è valido in tutti i casi in realtà, ma è decisamente più laborioso la maggior parte delle volte)
\end{enumerate}
\end{rem}
\begin{thm}
Sia $F = \frac{f}{g}$. Se $f\in \Hc \left(\Uc \left(z_{0}\right)\right)$ e $g$ ha in $z_{0}$ uno zero semplice ($g\left(z_{0}\right) = 0$ e $g'\left(z_{0}\right) \neq 0$), allora
\begin{equation*}
\Res \left(\frac{f}{g}, z_{0}\right) = \frac{f\left(z_{0}\right)}{g'\left(z_{0}\right)}
\end{equation*}

Ricordiamo che se $f$ non si annulla in $z_{0}$ allora $F$ ha in $z_{0}$ un polo del I ordine. Se $f$ si annulla in $z_{0}$ allora $F$ ha in $z_{0}$ una singolarità eliminabile.
\end{thm}
Le singolarità isolate sono $0, 1, - 1, \infty $.

$z = 0, z = 1, z = -1$ annullano il denominatore con molteplicità $1$ e non annullano il numeratore, quindi sono degli $1$ - poli. Per il teorema appena enunciato:
\begin{align*}
\Res \left(F, \phantom{-}0\right) & = \left\{g(z) = z, \ f(z) = \frac{1}{1 - z^{2}}\right\} = \frac{1}{1} = 1\\
\Res \left(F, \phantom{-}1\right) & = \left\{g(z) = 1 - z, \ f(z) = \frac{1}{z\left(1 + z\right)}\right\} = \frac{\frac{1}{2}}{- 1} = - \frac{1}{2}\\
\Res \left(F, - 1\right) 			 & = \left\{g(z) = 1 + z, \ f(z) = \frac{1}{z\left(1 - z\right)}\right\} = - \frac{1}{2}
\end{align*}
Mentre
\begin{equation*}
F(z) = \frac{1}{z\left(1 + z\right)\left(1 - z\right)} \sim \frac{1}{z^{3}} \ \ \ \ z\rightarrow + \infty
\end{equation*}
quindi $\infty $ è una singolarità eliminabile.
\begin{equation*}
\Res \left(F, \infty \right) = - \sum\limits^{3}_{k = 1}\Res \left(F, z_{k}\right) = - \left(1 - \frac{1}{2} - \frac{1}{2}\right) = 0
\end{equation*}
\begin{rem}
Se $\infty $ è singolarità eliminabile, ciò non implica per forza $\Res \left(f, \infty \right) = 0$. Infatti $f(z) = \frac{1}{z}\rightarrow 0$ per $z\rightarrow + \infty $ quindi $\infty $ è singolarità eliminabile, ma $\Res \left(f, \infty \right) = - a_{- 1} = - 1\neq 0$.
\end{rem}
\begin{rem}
Nel nostro caso è vero perché vale il seguente teorema.
\end{rem}
\begin{thm}
Se $f$ ha uno zero di ordine almeno $2$ in $\infty $ allora $\Res \left(f, \infty \right) = 0$, perché
\begin{equation*}
f(z) = \sum\limits^{- 2}_{n = -\infty} a_{n} z^{n} \ \ \implies \ \ \Res \left(f, \infty \right) = - a_{- 1} = 0
\end{equation*}
\end{thm}

\Soluzione

Le singolarità isolate di $f$ sono $i, \infty $.

$z = i$ annulla il denominatore con molteplicità $2$ e non annulla il numeratore, quindi è un polo. Usando la formula dei poli
\begin{equation*}
\Res \left(f, i\right) = \frac{d}{dz} f(z)\left(z - i\right)^{2} = \left. - 5z^{4}\right|_{z = i} = - 5
\end{equation*}
$f(z) \sim - z^{3}$ per $z\rightarrow + \infty $ quindi $\infty $ è un $3$ - polo.
\begin{equation*}
\Res \left(f, \infty \right) = - \Res \left(f, i\right) = 5
\end{equation*}

\Soluzione

Le singolarità isolate di $f$ sono $1, 0, \pm 2i, \infty $.

$z = \pm 2i$ sono zeri di molteplicità $1$ per il denominatore di $f$ e non annullano il denominatore, quindi sono $1$ - poli. Usando la formula dei poli
\begin{gather*}
\Res \left(f, 2i\right) = \lim\limits_{z\rightarrow 2i}\left(z - 2i\right) f(z) = \dotsc = \frac{e^{- \frac{1 + 2i}{5}}}{16} i\\
\Res \left(f, - 2i\right) = \lim\limits_{z\rightarrow - 2i}\left(z + 2i\right) f(z) = \dotsc = - \frac{e^{\frac{2i - 1}{5}}}{16} i
\end{gather*}
$z = 0$ è uno zero di molteplicità $2$ per il denominatore di $f$ e non annulla il denominatore, quindi è un $2$ - polo. Usando la formula dei poli
\begin{equation*}
\Res \left(f, 0\right) = \lim\limits_{z\rightarrow 0}\frac{d}{dz} z^{2} f(z) = \dotsc = - \frac{1}{4e}
\end{equation*}
$z = 1$ è una singolarità essenziale perché $\nexists \lim\limits_{z\rightarrow 1} f(z)$.

Abbiamo che $f(z) \sim \frac{e}{z^{4}}$ per $z\rightarrow + \infty $, allora per il teorema $\Res \left(f, \infty \right) = 0$.

Infine
\begin{equation*}
\Res \left(f, 1\right) = - \sum \Res \left(f, \dotsc \right) = \frac{1}{4e} - \frac{\sin\left(\frac{2}{5}\right)}{8e^{\frac{1}{5}}}
\end{equation*}

\Soluzione

\begin{enumerate}
\item Le singolarità di $f$ sono $1, 2, \infty $.

$z = 1$ è singolarità essenziale perché $\nexists \lim\limits_{z\rightarrow 1} f(z)$

$z = 2$ è $1$ - polo perché annulla il denominatore con molteplicità $1$ e non annulla il numeratore.

$z = \infty $ è singolarità eliminabile perché $\exists \lim\limits_{z\rightarrow \infty} f(z) \in \CC $.
\item Ricordiamo che

\begin{rem}
\begin{equation*}
e^{z} = \sum\limits^{\infty}_{n = 0}\frac{z^{n}}{n!}, \ \ z\in \CC \ \ \ \ \ \ \ \ \frac{1}{1 - z} = \sum\limits^{\infty}_{n = 0} z^{n}, \ \ \left| z\right| < 1
\end{equation*}
\end{rem}

Qui
\begin{equation*}
e^{\frac{1}{z - 1}} = \sum\limits^{\infty}_{h = 0}\frac{1}{h!\left(z - 1\right)^{h}}
\end{equation*}mentre
\begin{equation*}
\frac{1}{z - 2} = - \frac{1}{2 - z} = - \frac{1}{1 - \left(z - 1\right)} = - \sum\limits^{\infty}_{k = 0}\left(z - 1\right)^{k}
\end{equation*}La prima serie converge in $\left| z - 1\right| > 0$ e la seconda in $\left| z - 1\right| < 1$. La serie di Laurent converge quindi in $0 < \left| z - 1\right| < 1$.
\begin{rem}
Per scrivere la serie di Laurent sfruttiamo il prodotto alla Cauchy.
\begin{equation*}
\left(\sum\limits^{\infty}_{n = 0} a_{n}\right)\left(\sum\limits^{\infty}_{n = 0} b_{n}\right) = \sum\limits^{\infty}_{n = 0}\left(\sum\limits^{n}_{k = 0} a_{k} b_{n - k}\right)
\end{equation*}
\end{rem}Quindi
\begin{equation*}
f(z) = - \sum\limits^{\infty}_{k = 0}\sum\limits^{k}_{h = 0}\frac{\left(z - 1\right)^{k - 2h}}{h!}
\end{equation*}
\item $\Res \left(f, 2\right) = e$ per la formula dei poli

$\Res \left(f, 1\right) = a_{- 1}$ lo trovo ponendo $k - 2h = -1$, quindi $a_{- 1} = - \sum\limits^{\infty}_{h = 1}\frac{1}{h!} = 1 - e$

$\Res \left(f, \infty \right) = - 1$
\item Applico il teorema dei residui
\begin{equation*}
\int\nolimits_{\gamma} f(z) dz = 2\pi i\left[\Res \left(f, 1\right) + \Res \left(f, 2\right)\right] = 2\pi i
\end{equation*}
\end{enumerate}
\chapter{Esercitazione 3 - Boella}

\ParteEsercizi

\Esercizio{}

Calcolare il seguente integrale
\begin{equation*}
\int_{\RR}\frac{1}{x^{4} + 4a^{4}} dx
\end{equation*}

\Esercizio{}

Calcolare il seguente integrale
\begin{equation*}
\int_{\RR}\frac{e^{- i\lambda x}}{x^{6} + 64} dx\ \ \ \ \forall \lambda \in \RR
\end{equation*}

\Esercizio{}

Calcolare il seguente integrale
\begin{equation*}
\int^{\pi}_{0}\frac{1}{a + b\cos \vartheta} d\vartheta \ \ \ \ 0 < b < a
\end{equation*}

\Esercizio{}

Calcolare il seguente integrale, per i valori di $\alpha $ in cui converge
\begin{equation*}
\int_{\RR}\frac{e^{\alpha x}}{1 + e^{x}} dx\ \ \ \ \alpha \in \RR
\end{equation*}

\Esercizio{}

Calcolare il seguente integrale
\begin{equation*}
\int^{+ \infty}_{0}\frac{\cos\left(\alpha x\right)}{x^{2} + \beta^{2}} dx\ \ \ \ \alpha \neq 0, \beta \neq 0
\end{equation*}

\ParteSoluzioni

\Soluzione

Per cominciare pensiamo il nostro integrale come
\begin{equation*}
\int_{\RR}\frac{1}{x^{4} + 4a^{4}} dx = \lim\limits_{R\rightarrow + \infty}\int^{R}_{- R}\frac{1}{x^{4} + 4a^{4}} dx
\end{equation*}
Consideriamo un percorso $\gamma_{R}$ fatto dall'unione della curva da $ - R$ a $R$ e la semicirconferenza di raggio $R$ che chiameremo $\sigma_{R}$


\begin{figure}[htpb]
	\centering
\tikzset{every picture/.style = {line width = 0.75pt}} %set default line width to 0.75pt

\begin{tikzpicture}[x = 0.75pt, y = 0.75pt, yscale = -1, xscale = 1]
%uncomment if require: \path (0, 179); %set diagram left start at 0, and has height of 179

%Shape: Axis 2D [id:dp39099083924259226]
\draw (180, 120.75) - - (420, 120.75)(300, 10) - - (300, 150) (413, 115.75) - - (420, 120.75) - - (413, 125.75) (295, 17) - - (300, 10) - - (305, 17) ;
%Shape: Arc [id:dp3884788829218313]
\draw [draw opacity = 0][line width = 1.5] (241, 121.83) .. controls (241, 121.59) and (241, 121.34) .. (241, 121.09) .. controls (241, 88.23) and (267.64, 61.59) .. (300.5, 61.59) .. controls (333.36, 61.59) and (360, 88.23) .. (360, 121.09) .. controls (360, 121.25) and (360, 121.41) .. (360, 121.57) - - (300.5, 121.09) - - cycle ; \draw [color = {rgb, 255:red, 208; green, 2; blue, 27}, draw opacity = 1 ][line width = 1.5] (241, 121.83) .. controls (241, 121.59) and (241, 121.34) .. (241, 121.09) .. controls (241, 88.23) and (267.64, 61.59) .. (300.5, 61.59) .. controls (333.36, 61.59) and (360, 88.23) .. (360, 121.09) .. controls (360, 121.25) and (360, 121.41) .. (360, 121.57) ;
%Shape: Circle [id:dp8434618645208203]
\draw [fill = {rgb, 255:red, 0; green, 0; blue, 0}, fill opacity = 1 ] (278, 100) .. controls (278, 98.9) and (278.9, 98) .. (280, 98) .. controls (281.1, 98) and (282, 98.9) .. (282, 100) .. controls (282, 101.1) and (281.1, 102) .. (280, 102) .. controls (278.9, 102) and (278, 101.1) .. (278, 100) - - cycle ;
%Shape: Circle [id:dp7789399295573649]
\draw [fill = {rgb, 255:red, 0; green, 0; blue, 0}, fill opacity = 1 ] (278, 140) .. controls (278, 138.9) and (278.9, 138) .. (280, 138) .. controls (281.1, 138) and (282, 138.9) .. (282, 140) .. controls (282, 141.1) and (281.1, 142) .. (280, 142) .. controls (278.9, 142) and (278, 141.1) .. (278, 140) - - cycle ;
%Shape: Circle [id:dp730262806413829]
\draw [fill = {rgb, 255:red, 0; green, 0; blue, 0}, fill opacity = 1 ] (318, 100) .. controls (318, 98.9) and (318.9, 98) .. (320, 98) .. controls (321.1, 98) and (322, 98.9) .. (322, 100) .. controls (322, 101.1) and (321.1, 102) .. (320, 102) .. controls (318.9, 102) and (318, 101.1) .. (318, 100) - - cycle ;
%Shape: Circle [id:dp7565318380097881]
\draw [fill = {rgb, 255:red, 0; green, 0; blue, 0}, fill opacity = 1 ] (318, 140) .. controls (318, 138.9) and (318.9, 138) .. (320, 138) .. controls (321.1, 138) and (322, 138.9) .. (322, 140) .. controls (322, 141.1) and (321.1, 142) .. (320, 142) .. controls (318.9, 142) and (318, 141.1) .. (318, 140) - - cycle ;
\draw [draw opacity = 0][fill = {rgb, 255:red, 208; green, 2; blue, 27}, fill opacity = 1 ] (310, 114) - - (324, 121) - - (310, 128) - - (317, 121) - - cycle ;
%Straight Lines [id:da6746169366745891]
\draw [color = {rgb, 255:red, 208; green, 2; blue, 27}, draw opacity = 1 ][line width = 1.5] (241, 121.83) - - (360, 121.57) ;
\draw [draw opacity = 0][fill = {rgb, 255:red, 208; green, 2; blue, 27}, fill opacity = 1 ] (336.51, 81.83) - - (328.43, 68.42) - - (344, 70) - - (334.34, 72.17) - - cycle ;

% Text Node
\draw (411, 62.4) node [anchor = north west][inner sep = 0.75pt] {$\gamma_{R}$};
% Text Node
\draw (353, 72.4) node [anchor = north west][inner sep = 0.75pt] {$\sigma_{R}$};
% Text Node
\draw (351, 122.4) node [anchor = north west][inner sep = 0.75pt] {$R$};
% Text Node
\draw (231, 122.4) node [anchor = north west][inner sep = 0.75pt] {$ - R$};


\end{tikzpicture}
\end{figure}
\FloatBarrier

Cerchiamo le singolarità della funzione in campo complesso
\begin{equation*}
z^{4} + 4a^{4} = 0\ \ \iff \ \ z = \pm a\pm ia
\end{equation*}
Calcoliamo ora l'integrale
\begin{equation*}
\int_{\gamma_{R}} f(z) dz = \int^{R}_{- R} f(x) dx + \int_{\sigma_{R}} f(z) dz
\end{equation*}
che vale, per il teorema dei residui
\begin{equation*}
\int_{\gamma_{R}} f(z) dz = 2\pi i\left[\Res \left(f, - a + ia\right) + \Res \left(f, a + ia\right)\right]
\end{equation*}
Calcoliamo ora questi residui
\begin{gather*}
\Res \left(f, - a + ia\right) = \frac{g\left(z_{0}\right)}{h'\left(z_{0}\right)} = \left. \frac{1}{4z^{3}}\right|_{z = -a + ia} = \left. \frac{z}{4z^{4}}\right|_{z = -a + ia} = \frac{a + ia}{- 16a^{4}}\\
\Res \left(f, a + ia\right) = \frac{g\left(z_{0}\right)}{h'\left(z_{0}\right)} = \left. \frac{1}{4z^{3}}\right|_{z = a + ia} = \left. \frac{z}{4z^{4}}\right|_{z = a + ia} = \frac{- a + ia}{- 16a^{4}}
\end{gather*}
Allora
\begin{equation*}
\int_{\gamma_{R}} f(z) dz = 2\pi i\left[\frac{a + ia}{- 16a^{4}} + \frac{- a + ia}{- 16a^{4}}\right] = \frac{\pi}{4a^{3}}
\end{equation*}
Ci resta da far vedere che
\begin{equation*}
\int_{\sigma_{R}} f(z) dz\rightarrow 0
\end{equation*}
Non possiamo usare il Lemma di Jordan, perché al numeratore non c'è un esponenziale immaginario. Dobbiamo sporcarci le mani. In generale sarà un numero complesso, quindi conviene prenderne il modulo e maggiorarlo.
\begin{rem}
\begin{equation*}
\left| \int_{\gamma} f(z) dz\right| \leq M \cdot \text{lunghezza}\left(\gamma \right)
\end{equation*}
dove $M$ è il massimo valore che $f$ può assumere lungo $\gamma $.
\end{rem}
Per la disuguaglianza triangolare
\begin{equation*}
\left| a\pm b\right| \geq \left| a\right| - \left| b\right|
\end{equation*}
nel nostro caso
\begin{equation*}
\left| \frac{1}{z^{4} + 4a^{4}}\right| = \frac{1}{\left| z^{4} + 4a^{4}\right|} \leq \frac{1}{\left| z^{4}\right| - \left| 4a^{4}\right|} = \frac{1}{R^{4} - \left| 4a^{4}\right|}
\end{equation*}
e quindi utilizzando il Nota Bene
\begin{equation*}
\left| \int_{\sigma_{R}} f(z) dz\right| \leq \frac{1}{R^{4} - \left| 4a^{4}\right|} \cdot \pi R\xrightarrow{R\rightarrow \infty} 0
\end{equation*}
Pertanto il nostro integrale iniziale vale
\begin{equation*}
\int_{\RR}\frac{1}{x^{4} + 4a^{4}} dx = \frac{\pi}{4a^{3}}
\end{equation*}

\Soluzione

\begin{thm}
Lemmi di Jordan. Considerando
\begin{equation*}
g(z) e^{i\alpha z}
\end{equation*}
\begin{itemize}
\item $\alpha > 0$ l'integrale sulla semicirconferenza di sopra tende a zero
\item $\alpha < 0$ l'integrale sulla semicirconferenza di sotto tende a zero (quello sopra fa un casino dell'altro mondo)
\end{itemize}
\end{thm}
Analizziamo il caso $\lambda > 0$. Applichiamo il Lemma di Jordan, il quale ci dice che l'integrale lungo la semicirconferenza inferiore tende a $0$, quindi possiamo direttamente scrivere


\begin{figure}[htpb]
	\centering
\tikzset{every picture/.style = {line width = 0.75pt}} %set default line width to 0.75pt

\begin{tikzpicture}[x = 0.75pt, y = 0.75pt, yscale = -1, xscale = 1]
%uncomment if require: \path (0, 160); %set diagram left start at 0, and has height of 160

%Shape: Axis 2D [id:dp508857242987794]
\draw (170, 80) - - (410, 80)(290.4, 9.41) - - (290.4, 149.41) (403, 75) - - (410, 80) - - (403, 85) (285.4, 16.41) - - (290.4, 9.41) - - (295.4, 16.41) ;
%Shape: Arc [id:dp2429361463821933]
\draw [draw opacity = 0][line width = 1.5] (350, 80.14) .. controls (350, 80.26) and (350, 80.38) .. (350, 80.5) .. controls (350, 113.36) and (323.36, 140) .. (290.5, 140) .. controls (257.64, 140) and (231, 113.36) .. (231, 80.5) .. controls (231, 80.4) and (231, 80.31) .. (231, 80.21) - - (290.5, 80.5) - - cycle ; \draw [color = {rgb, 255:red, 208; green, 2; blue, 27}, draw opacity = 1 ][line width = 1.5] (350, 80.14) .. controls (350, 80.26) and (350, 80.38) .. (350, 80.5) .. controls (350, 113.36) and (323.36, 140) .. (290.5, 140) .. controls (257.64, 140) and (231, 113.36) .. (231, 80.5) .. controls (231, 80.4) and (231, 80.31) .. (231, 80.21) ;
%Shape: Circle [id:dp8781397774455211]
\draw [fill = {rgb, 255:red, 0; green, 0; blue, 0}, fill opacity = 1 ] (288.67, 30) .. controls (288.67, 28.9) and (289.56, 28) .. (290.67, 28) .. controls (291.77, 28) and (292.67, 28.9) .. (292.67, 30) .. controls (292.67, 31.1) and (291.77, 32) .. (290.67, 32) .. controls (289.56, 32) and (288.67, 31.1) .. (288.67, 30) - - cycle ;
%Shape: Circle [id:dp3148978419514976]
\draw [fill = {rgb, 255:red, 0; green, 0; blue, 0}, fill opacity = 1 ] (331.39, 54.67) .. controls (331.39, 53.56) and (332.29, 52.67) .. (333.39, 52.67) .. controls (334.5, 52.67) and (335.39, 53.56) .. (335.39, 54.67) .. controls (335.39, 55.77) and (334.5, 56.67) .. (333.39, 56.67) .. controls (332.29, 56.67) and (331.39, 55.77) .. (331.39, 54.67) - - cycle ;
%Shape: Circle [id:dp8279624449312546]
\draw [fill = {rgb, 255:red, 0; green, 0; blue, 0}, fill opacity = 1 ] (245.94, 104) .. controls (245.94, 102.9) and (246.84, 102) .. (247.94, 102) .. controls (249.05, 102) and (249.94, 102.9) .. (249.94, 104) .. controls (249.94, 105.1) and (249.05, 106) .. (247.94, 106) .. controls (246.84, 106) and (245.94, 105.1) .. (245.94, 104) - - cycle ;
%Shape: Circle [id:dp5338705832900976]
\draw [fill = {rgb, 255:red, 0; green, 0; blue, 0}, fill opacity = 1 ] (331.39, 104) .. controls (331.39, 102.9) and (332.29, 102) .. (333.39, 102) .. controls (334.5, 102) and (335.39, 102.9) .. (335.39, 104) .. controls (335.39, 105.1) and (334.5, 106) .. (333.39, 106) .. controls (332.29, 106) and (331.39, 105.1) .. (331.39, 104) - - cycle ;
\draw [draw opacity = 0][fill = {rgb, 255:red, 208; green, 2; blue, 27}, fill opacity = 1 ] (314, 87.41) - - (300, 80.41) - - (314, 73.41) - - (307, 80.41) - - cycle ;
%Straight Lines [id:da4843584145229285]
\draw [color = {rgb, 255:red, 208; green, 2; blue, 27}, draw opacity = 1 ][line width = 1.5] (231, 80.21) - - (349.99, 79.95) ;
\draw [draw opacity = 0][fill = {rgb, 255:red, 208; green, 2; blue, 27}, fill opacity = 1 ] (265.21, 126.99) - - (276.31, 138.03) - - (260.82, 140.28) - - (269.66, 135.83) - - cycle ;
%Shape: Regular Polygon [id:dp2801234129072723]
\draw [dash pattern = {on 0.84pt off 2.51pt}] (290.67, 128.67) - - (247.94, 104) - - (247.94, 54.67) - - (290.67, 30) - - (333.39, 54.67) - - (333.39, 104) - - cycle ;
%Shape: Circle [id:dp21637549294959646]
\draw [fill = {rgb, 255:red, 0; green, 0; blue, 0}, fill opacity = 1 ] (288.67, 128.67) .. controls (288.67, 127.56) and (289.56, 126.67) .. (290.67, 126.67) .. controls (291.77, 126.67) and (292.67, 127.56) .. (292.67, 128.67) .. controls (292.67, 129.77) and (291.77, 130.67) .. (290.67, 130.67) .. controls (289.56, 130.67) and (288.67, 129.77) .. (288.67, 128.67) - - cycle ;
%Shape: Circle [id:dp24691945495063838]
\draw [fill = {rgb, 255:red, 0; green, 0; blue, 0}, fill opacity = 1 ] (245.94, 54.67) .. controls (245.94, 53.56) and (246.84, 52.67) .. (247.94, 52.67) .. controls (249.05, 52.67) and (249.94, 53.56) .. (249.94, 54.67) .. controls (249.94, 55.77) and (249.05, 56.67) .. (247.94, 56.67) .. controls (246.84, 56.67) and (245.94, 55.77) .. (245.94, 54.67) - - cycle ;

% Text Node
\draw (268, 54.4) node [anchor = north west][inner sep = 0.75pt] {$\gamma_{R}$};
% Text Node
\draw (351, 58.4) node [anchor = north west][inner sep = 0.75pt] {$R$};
% Text Node
\draw (205, 57.4) node [anchor = north west][inner sep = 0.75pt] {$ - R$};


\end{tikzpicture}
\end{figure}
\FloatBarrier

\begin{equation*}
\int_{\RR}\frac{e^{- i\lambda x}}{x^{6} + 64} dx = -\lim\limits_{R\rightarrow \infty}\int_{\gamma_{R}}\frac{e^{- i\lambda z}}{z^{6} + 64} dz = -2\pi i\left[\sum \Res \left(f, z_{k}\right)\right]
\end{equation*}
Gli zeri al denominatore di $f$ sono nelle $6$ radici seste di $ - 64$ (i vertici di un esagono)
\begin{equation*}
\pm \sqrt{3} \pm i\ \ \ \ \pm 2i
\end{equation*}
Quelli che stanno dentro la parte di piano interessata sono
\begin{equation*}
\pm \sqrt{3} - i\ \ \ \ - 2i
\end{equation*}
Per calcolare il residuo in tali punti utilizziamo la formula di calcolo vista anche nel precedente esercizio
\begin{equation*}
\Res \left(f, z_{k}\right) = \left. \frac{e^{- i\lambda z}}{6z^{5}}\right|_{z = z_{k}} = \left. \frac{ze^{- i\lambda z}}{6z^{6}}\right|_{z = z_{k}}
\end{equation*}
Pertanto
\begin{itemize}
\item $\Res \left(f, - \sqrt{3} - i\right) = \frac{\left(- \sqrt{3} - i\right) e^{- i\lambda \left(- \sqrt{3} - i\right)}}{6 \cdot \left(- 64\right)}$
\item $\Res \left(f, \sqrt{3} - i\right) = \frac{\left(\sqrt{3} - i\right) e^{- i\lambda \left(\sqrt{3} - i\right)}}{6 \cdot \left(- 64\right)}$
\item $\Res \left(f, - 2i\right) = \frac{\left(- 2i\right) e^{- i\lambda \left(- 2i\right)}}{6 \cdot \left(- 64\right)}$
\end{itemize}

Allora
\begin{align*}
\int_{\RR}\frac{e^{- i\lambda x}}{x^{6} + 64} dx & = - 2\pi i\left[\frac{\left(- \sqrt{3} - i\right) e^{- i\lambda \left(- \sqrt{3} - i\right)}}{6 \cdot \left(- 64\right)} + \frac{\left(\sqrt{3} - i\right) e^{- i\lambda \left(\sqrt{3} - i\right)}}{6 \cdot \left(- 64\right)} + \frac{\left(- 2i\right) e^{- i\lambda \left(- 2i\right)}}{6 \cdot \left(- 64\right)}\right]\\
 & = \frac{2\pi i}{6 \cdot 64}\left[\left(- \sqrt{3} - i\right) e^{i\lambda \sqrt{3} - \lambda} + \left(\sqrt{3} - i\right) e^{- i\lambda \sqrt{3} - \lambda} + \left(- 2i\right) e^{- 2\lambda}\right]\\
 & = \frac{2\pi i}{6 \cdot 64}\left[ - \sqrt{3} e^{i\lambda \sqrt{3} - \lambda} - ie^{i\lambda \sqrt{3} - \lambda} + \sqrt{3} e^{- i\lambda \sqrt{3} - \lambda} - ie^{- i\lambda \sqrt{3} - \lambda} - 2ie^{- 2\lambda}\right]\\
 & = \frac{2\pi i}{6 \cdot 64}\left[ - \frac{\sqrt{3}}{e^{\lambda}} e^{i\lambda \sqrt{3}} - \frac{i}{e^{\lambda}} e^{i\lambda \sqrt{3}} + \frac{\sqrt{3}}{e^{\lambda}} e^{- i\lambda \sqrt{3}} - \frac{i}{e^{\lambda}} e^{- i\lambda \sqrt{3}} - 2ie^{- 2\lambda}\right]\\
 & = \frac{2\pi i}{6 \cdot 64 \cdot e^{\lambda}}\left[ - \sqrt{3} e^{i\lambda \sqrt{3}} - ie^{i\lambda \sqrt{3}} + \sqrt{3} e^{- i\lambda \sqrt{3}} - ie^{- i\lambda \sqrt{3}} - 2ie^{- \lambda}\right]\\
 & = \frac{2\pi i}{6 \cdot 64 \cdot e^{\lambda}}\left[ - \sqrt{3}\left(e^{i\lambda \sqrt{3}} - e^{- i\lambda \sqrt{3}}\right) - i\left(e^{i\lambda \sqrt{3}} + e^{- i\lambda \sqrt{3}}\right) - 2ie^{- \lambda}\right]\\
 & = \frac{2\pi i}{6 \cdot 64 \cdot e^{\lambda}}\left[ - 2i\sqrt{3}\frac{\left(e^{i\lambda \sqrt{3}} - e^{- i\lambda \sqrt{3}}\right)}{2i} - 2i\frac{\left(e^{i\lambda \sqrt{3}} + e^{- i\lambda \sqrt{3}}\right)}{2} - 2ie^{- \lambda}\right]\\
 & = \frac{2\pi i}{6 \cdot 64 \cdot e^{\lambda}}\left[ - 2i\sqrt{3}\sin\left(\lambda \sqrt{3}\right) - 2i\cos\left(\lambda \sqrt{3}\right) - 2ie^{- \lambda}\right]\\
 & = \frac{4\pi}{6 \cdot 64 \cdot e^{\lambda}}\left[\sqrt{3}\sin\left(\lambda \sqrt{3}\right) + \cos\left(\lambda \sqrt{3}\right) + e^{- \lambda}\right]\\
 & = \frac{\pi}{96} e^{- \lambda}\left[\sqrt{3}\sin\left(\lambda \sqrt{3}\right) + \cos\left(\lambda \sqrt{3}\right) + e^{- \lambda}\right]
\end{align*}
Abbiamo appena calcolato la nostra prima trasformata di Fourier. Fare per casa i casi $\lambda = 0, \lambda < 0$.

\Soluzione

Questa è un'altra tipologia di integrali, in cui $R$ è una funzione razionale, con seni e coseni, e le loro potenze
\begin{equation*}
\int^{2\pi}_{0} R\left(\cos \vartheta, \sin \vartheta \right) d\vartheta
\end{equation*}
li facciamo per \textit{sostituzione}
\begin{equation*}
\boxed{e^{i\vartheta} = z} \ \ \implies \ \ \boxed{\cos \vartheta = \frac{z + \frac{1}{z}}{2}} \ \boxed{\sin \vartheta = \frac{z - \frac{1}{z}}{2i}}
\end{equation*}
mentre per il differenziale
\begin{equation*}
ie^{i\vartheta} d\vartheta = dz\ \ \implies \ \ \boxed{d\vartheta = \frac{1}{iz} dz}
\end{equation*}
Veniamo ora al nostro integrale
\begin{equation*}
\int^{\pi}_{0}\frac{1}{a + b\cos \vartheta} d\vartheta = \left\{\text{funzione pari}\right\} = \frac{1}{2}\int^{2\pi}_{0}\frac{1}{a + b\cos \vartheta} d\vartheta
\end{equation*}
Riscriviamo il coseno in modo più comodo
\begin{equation*}
\cos \vartheta = \frac{z + \frac{1}{z}}{2} = \frac{z^{2} + 1}{2z}
\end{equation*}
allora
\begin{equation*}
\frac{1}{2}\int^{2\pi}_{0}\frac{1}{a + b\cos \vartheta} d\vartheta = \frac{1}{2}\int_{\left| z\right| = 1}\frac{1}{a + b\frac{z^{2} + 1}{2z}} \cdot \frac{1}{iz} dz = \frac{1}{i}\int_{\left| z\right| = 1}\frac{1}{bz^{2} + 2az + b} dz
\end{equation*}
calcoliamo le radici del denominatore
\begin{equation*}
z = \frac{- a\pm \sqrt{a^{2} - b^{2}}}{b}
\end{equation*}
la radice col $ + $ è interna alla circonferenza, quella con il meno sta a sinistra di $ - 1$ invece, di conseguenza prendiamo
\begin{equation*}
\overline{z} = \frac{- a + \sqrt{a^{2} - b^{2}}}{b}
\end{equation*}
allora
\begin{equation*}
\frac{1}{i}\int_{\left| z\right| = 1}\frac{1}{bz^{2} + 2az + b} dz = \frac{1}{i} \cdot 2\pi i \cdot \Res \left(f, \overline{z}\right) = 2\pi \cdot \Res \left(f, \overline{z}\right)
\end{equation*}
il residuo vale
\begin{equation*}
\Res \left(f, \overline{z}\right) = \left. \frac{1}{2bz + 2a}\right|_{z = \overline{z}} = \frac{1}{2\sqrt{a^{2} - b^{2}}}
\end{equation*}
sostituiamo nel nostro integrale
\begin{equation*}
2\pi \cdot \Res \left(f, \overline{z}\right) = \cancel{2} \pi \cdot \frac{1}{\cancel{2}\sqrt{a^{2} - b^{2}}} = \frac{\pi}{\sqrt{a^{2} - b^{2}}}
\end{equation*}
abbiamo calcolato un integrale facendo una derivata!

\Soluzione

Per il criterio del confronto asintotico, per $x\rightarrow + \infty $
\begin{equation*}
\frac{e^{\alpha x}}{1 + e^{x}} \sim \frac{e^{\alpha x}}{e^{x}} = e^{\left(\alpha - 1\right) x}
\end{equation*}
converge per $\left(\alpha - 1\right) < 0$, cioè $\alpha < 1$. Mentre per $x\rightarrow - \infty $
\begin{equation*}
\frac{e^{\alpha x}}{1 + e^{x}} \sim \frac{e^{\alpha x}}{1} = e^{\alpha x}
\end{equation*}
converge per $\alpha > 0$. Quindi complessivamente converge per $0 < \alpha < 1$.

Calcoliamo
\begin{equation*}
\int_{\gamma_{R}} f(z) dz\ \ \ \ f(z) = \frac{e^{\alpha z}}{1 + e^{z}}
\end{equation*}
Non possiamo usare il lemma di Jordan e ci ricordiamo che $e^{z}$ è periodica
\begin{equation*}
e^{z} = e^{z} \cdot 1 = e^{z} \cdot e^{2\pi i} = e^{z + 2\pi i}
\end{equation*}
la nostra funzione ha quindi singolarità in
\begin{equation*}
e^{z} = - 1\ \ \implies \ \ z = i\left(\pi + 2k\pi \right)
\end{equation*}
infinite lungo l'asse immaginaria. Non ha senso prendere una semicirconferenza perché dovremmo inglobare nuovi residui man mano che aumentiamo $R$. Invece prendiamo un rettangolo.


\begin{figure}[htpb]
	\centering
\tikzset{every picture/.style = {line width = 0.75pt}} %set default line width to 0.75pt

\begin{tikzpicture}[x = 0.75pt, y = 0.75pt, yscale = -1, xscale = 1]
%uncomment if require: \path (0, 160); %set diagram left start at 0, and has height of 160

%Shape: Axis 2D [id:dp6952694616964656]
\draw (150, 110.75) - - (450, 110.75)(300, 30) - - (300, 150) (443, 105.75) - - (450, 110.75) - - (443, 115.75) (295, 37) - - (300, 30) - - (305, 37) ;
%Shape: Circle [id:dp010000296023957134]
\draw [fill = {rgb, 255:red, 0; green, 0; blue, 0}, fill opacity = 1 ] (297, 90.5) .. controls (297, 89.12) and (298.12, 88) .. (299.5, 88) .. controls (300.88, 88) and (302, 89.12) .. (302, 90.5) .. controls (302, 91.88) and (300.88, 93) .. (299.5, 93) .. controls (298.12, 93) and (297, 91.88) .. (297, 90.5) - - cycle ;
%Straight Lines [id:da3282163126700619]
\draw [color = {rgb, 255:red, 208; green, 2; blue, 27}, draw opacity = 1 ][line width = 1.5] (200, 70) - - (200, 110) ;
\draw [shift = {(200, 90)}, rotate = 270] [fill = {rgb, 255:red, 208; green, 2; blue, 27}, fill opacity = 1 ][line width = 0.08] [draw opacity = 0] (13.4, - 6.43) - - (0, 0) - - (13.4, 6.44) - - (8.9, 0) - - cycle ;
%Straight Lines [id:da8194840593110315]
\draw [color = {rgb, 255:red, 208; green, 2; blue, 27}, draw opacity = 1 ][line width = 1.5] (400, 110) - - (400, 70) ;
\draw [shift = {(400, 90)}, rotate = 450] [fill = {rgb, 255:red, 208; green, 2; blue, 27}, fill opacity = 1 ][line width = 0.08] [draw opacity = 0] (13.4, - 6.43) - - (0, 0) - - (13.4, 6.44) - - (8.9, 0) - - cycle ;
%Straight Lines [id:da32015283387712334]
\draw [color = {rgb, 255:red, 208; green, 2; blue, 27}, draw opacity = 1 ][line width = 1.5] (400, 70) - - (200, 70) ;
\draw [shift = {(300, 70)}, rotate = 360] [fill = {rgb, 255:red, 208; green, 2; blue, 27}, fill opacity = 1 ][line width = 0.08] [draw opacity = 0] (13.4, - 6.43) - - (0, 0) - - (13.4, 6.44) - - (8.9, 0) - - cycle ;
%Straight Lines [id:da18099188002221434]
\draw [color = {rgb, 255:red, 208; green, 2; blue, 27}, draw opacity = 1 ][line width = 1.5] (200, 110) - - (400, 110) ;
\draw [shift = {(300, 110)}, rotate = 180] [fill = {rgb, 255:red, 208; green, 2; blue, 27}, fill opacity = 1 ][line width = 0.08] [draw opacity = 0] (13.4, - 6.43) - - (0, 0) - - (13.4, 6.44) - - (8.9, 0) - - cycle ;

% Text Node
\draw (391, 120.4) node [anchor = north west][inner sep = 0.75pt] {$R$};
% Text Node
\draw (181, 120.4) node [anchor = north west][inner sep = 0.75pt] {$ - R$};
% Text Node
\draw (373, 49.4) node [anchor = north west][inner sep = 0.75pt] {$R + 2\pi i$};
% Text Node
\draw (159, 50.4) node [anchor = north west][inner sep = 0.75pt] {$ - R + 2\pi i$};
% Text Node
\draw (270, 50.4) node [anchor = north west][inner sep = 0.75pt] {$2\pi i$};
% Text Node
\draw (321, 112.4) node [anchor = north west][inner sep = 0.75pt] {$\gamma_{1}$};
% Text Node
\draw (411, 80.4) node [anchor = north west][inner sep = 0.75pt] {$\gamma_{2}$};
% Text Node
\draw (321, 45.4) node [anchor = north west][inner sep = 0.75pt] {$\gamma_{3}$};
% Text Node
\draw (171, 80.4) node [anchor = north west][inner sep = 0.75pt] {$\gamma_{4}$};


\end{tikzpicture}
\end{figure}
\FloatBarrier

\begin{equation*}
\int_{\gamma_{R}} f(z) dz = 2\pi i\left[\Res \left(f, \pi i\right)\right] = 2\pi i\left[\frac{e^{\alpha z}}{e^{z}}\right]_{z = \pi i} = - 2\pi ie^{i\alpha \pi}
\end{equation*}
Vediamo quanto valgono i vari integrali. Quello su $\gamma_{2}$ (e analogamente su $\gamma_{4}$)
\begin{align*}
\left| \int_{\gamma_{2}} f(z) dz\right| & = \left\{\gamma_{2} :x = R, y\in \left[ 0, 2\pi \right], dz = idy\right\} = \\
 & = \left| \int^{2\pi}_{0}\frac{e^{\alpha R + i\alpha y}}{1 + e^{R + iy}} idy\right| \leq \int^{2\pi}_{0}\frac{e^{\alpha R}}{e^{R} - 1} dy = 2\pi \frac{e^{\alpha R}}{e^{R} - 1}\xrightarrow{R\rightarrow + \infty} 0
\end{align*}
Quello su $\gamma_{3}$
\begin{align*}
\int_{\gamma_{3}} f(z) dz & = \left\{z = x + 2\pi i\right\} = \\
 & = - \int^{R}_{- R}\frac{e^{\alpha x} e^{i2\pi \alpha}}{1 + e^{x + 2\pi i}} dx = -e^{i2\pi \alpha}\int^{R}_{- R} f(x) dx
\end{align*}
quindi
\begin{align*}
\int_{\gamma_{R}} f(z) dz & \rightarrow - 2\pi ie^{i\alpha \pi}\\
\int^{R}_{- R} f(x) dx - e^{i2\pi \alpha}\int^{R}_{- R} f(x) dx + \underbrace{\int_{\gamma_{2}} f(z) dz + \int_{\gamma_{4}} f(z) dz}_{\rightarrow 0} & \rightarrow - 2\pi ie^{i\alpha \pi}\\
\left(1 - e^{i2\pi \alpha}\right)\int^{R}_{- R} f(x) dx & \rightarrow - 2\pi ie^{i\alpha \pi}
\end{align*}
in conclusione
\begin{equation*}
\int_{\RR} f(x) dx = \frac{2\pi ie^{i\alpha \pi}}{e^{i2\pi \alpha} - 1} = 2\pi i\frac{e^{i\alpha \pi}}{e^{2i\alpha \pi} - 1} \cdot \frac{e^{- i\alpha \pi}}{e^{- i\alpha \pi}} = 2\pi i\frac{1}{e^{i\alpha \pi} - e^{- i\alpha \pi}} = \frac{\pi}{\sin\left(\alpha \pi \right)}
\end{equation*}
\chapter{Esercitazione 3 - Potrich}

\ParteEsercizi

\Esercizio{}

Calcolare, al variare di $R > 0$
\begin{equation*}
I_{R} = \int_{\gamma_{R}}\frac{\left(z + 1\right) e^{\pi z}}{z^{2}\left(z^{2} + 1\right)} dz
\end{equation*}
dove
\begin{equation*}
\gamma_{R} :\left| z - 1 - i\right| = R
\end{equation*}
percorsa positivamente.

\Esercizio{}

Vediamo alcuni integrali reali con metodi di analisi complessa.
\begin{equation*}
I = \boxed{\int^{2\pi}_{0} R\left(\cos t, \sin t\right) dt}
\end{equation*}
dove $R$ si intende una funzione razionale.
\begin{equation*}
I = \left\{
\begin{array}{c c}
z = e^{it} & \cos t = \frac{e^{it} + e^{- it}}{2}\\
dz = ie^{it} dt & \sin t = \frac{e^{it} - e^{- it}}{2i}
\end{array}\right\} = \int_{\left| z\right| = 1} R\left(\frac{z + z^{- 1}}{2}, \frac{z - z^{- 1}}{2i}\right)\frac{1}{iz} dz
\end{equation*}
Calcolare
\begin{equation*}
\int^{2\pi}_{0}\frac{1}{3 + \sin t} dt
\end{equation*}

\Esercizio{}

Vediamo una seconda tipologia di integrali
\begin{equation*}
\boxed{\int^{+ \infty}_{- \infty} R(x) dx} \ \ \ \text{oppure} \ \ \ \ \boxed{\int^{+ \infty}_{- \infty} R(x)\left\{
\begin{array}{c}
e^{ix}\\
\cos x\\
\sin x
\end{array}\right\} dx}
\end{equation*}
Per risolverli basta porre $f(z) = R(z)$ oppure $f(z) = R(z) e^{iz}$ e si integra sui semicerchi, infine si fa tendere $R\rightarrow + \infty $.

Se ci fossero poli sull'asse reale devo fare dei morsi e applicare il seguente Lemma. Si fa quindi tendere $\varepsilon \rightarrow 0^{+}$ e $R\rightarrow + \infty $.
\begin{thm}
[Lemma del cerchio piccolo] Sia $f\in \Hc \left(D\left(z_{0}, r\right) \setminus \left\{z_{0}\right\}\right)$, sia $z_{0}$ un polo semplice per $f$, sia $\varphi_{\varepsilon}(t) = z_{0} + \varepsilon e^{it}$ con $t\in \left[ \alpha, \beta \right]$. Allora
\begin{equation*}
\lim_{\varepsilon \rightarrow 0^{+}}\int_{\varphi_{\varepsilon}} f(z) dz = \left(\beta - \alpha \right) i \cdot \Res \left(f, z_{0}\right)
\end{equation*}
\end{thm}
\begin{figure}[htpb]
	\centering
\tikzset{every picture/.style = {line width = 0.75pt}} %set default line width to 0.75pt

\begin{tikzpicture}[x = 0.75pt, y = 0.75pt, yscale = -1, xscale = 1]
%uncomment if require: \path (0, 169); %set diagram left start at 0, and has height of 169

%Shape: Axis 2D [id:dp21276168492453773]
\draw (110, 120.09) - - (490, 120.09)(300.5, 10) - - (300.5, 160) (483, 115.09) - - (490, 120.09) - - (483, 125.09) (295.5, 17) - - (300.5, 10) - - (305.5, 17) ;
%Shape: Arc [id:dp9066154861810967]
\draw [draw opacity = 0][line width = 1.5] (231, 120.09) .. controls (231, 120.09) and (231, 120.09) .. (231, 120.09) .. controls (231, 81.71) and (262.12, 50.59) .. (300.5, 50.59) .. controls (338.88, 50.59) and (370, 81.71) .. (370, 120.09) - - (300.5, 120.09) - - cycle ; \draw [color = {rgb, 255:red, 208; green, 2; blue, 27}, draw opacity = 1 ][line width = 1.5] (231, 120.09) .. controls (231, 120.09) and (231, 120.09) .. (231, 120.09) .. controls (231, 81.71) and (262.12, 50.59) .. (300.5, 50.59) .. controls (338.88, 50.59) and (370, 81.71) .. (370, 120.09) ;
%Shape: Arc [id:dp2361543100002761]
\draw [draw opacity = 0][line width = 1.5] (250.48, 119.98) .. controls (250.48, 119.98) and (250.48, 119.98) .. (250.48, 119.98) .. controls (250.48, 108.93) and (259.44, 99.96) .. (270.5, 99.96) .. controls (281.56, 99.96) and (290.52, 108.93) .. (290.52, 119.98) - - (270.5, 119.98) - - cycle ; \draw [color = {rgb, 255:red, 208; green, 2; blue, 27}, draw opacity = 1 ][line width = 1.5] (250.48, 119.98) .. controls (250.48, 119.98) and (250.48, 119.98) .. (250.48, 119.98) .. controls (250.48, 108.93) and (259.44, 99.96) .. (270.5, 99.96) .. controls (281.56, 99.96) and (290.52, 108.93) .. (290.52, 119.98) ;
%Shape: Circle [id:dp06145462056718265]
\draw [fill = {rgb, 255:red, 0; green, 0; blue, 0}, fill opacity = 1 ] (269, 119.98) .. controls (269, 119.15) and (269.67, 118.48) .. (270.5, 118.48) .. controls (271.33, 118.48) and (272, 119.15) .. (272, 119.98) .. controls (272, 120.81) and (271.33, 121.48) .. (270.5, 121.48) .. controls (269.67, 121.48) and (269, 120.81) .. (269, 119.98) - - cycle ;
\draw [draw opacity = 0][fill = {rgb, 255:red, 208; green, 2; blue, 27}, fill opacity = 1 ] (342.64, 72.63) - - (332.87, 59) - - (349.63, 59.36) - - (339.5, 62.5) - - cycle ;
%Straight Lines [id:da8537685439706526]
\draw [color = {rgb, 255:red, 208; green, 2; blue, 27}, draw opacity = 1 ][line width = 1.5] (230, 120) - - (250.48, 119.98) ;
%Straight Lines [id:da3820014294051135]
\draw [color = {rgb, 255:red, 208; green, 2; blue, 27}, draw opacity = 1 ][line width = 1.5] (290.52, 119.98) - - (370, 120) ;
\draw [shift = {(330.26, 119.99)}, rotate = 180.01] [fill = {rgb, 255:red, 208; green, 2; blue, 27}, fill opacity = 1 ][line width = 0.08] [draw opacity = 0] (13.4, - 6.43) - - (0, 0) - - (13.4, 6.44) - - (8.9, 0) - - cycle ;
\draw [draw opacity = 0][fill = {rgb, 255:red, 208; green, 2; blue, 27}, fill opacity = 1 ] (264, 95) - - (274, 100) - - (264, 105) - - (269, 100) - - cycle ;

% Text Node
\draw (197, 98.4) node [anchor = north west][inner sep = 0.75pt] {$ - R$};
% Text Node
\draw (374, 99.4) node [anchor = north west][inner sep = 0.75pt] {$R$};
% Text Node
\draw (227, 127.4) node [anchor = north west][inner sep = 0.75pt] [font = \scriptsize] {$z_{0} - \varepsilon $};
% Text Node
\draw (267, 127.4) node [anchor = north west][inner sep = 0.75pt] [font = \scriptsize] {$z_{0} + \varepsilon $};
% Text Node
\draw (264, 105.4) node [anchor = north west][inner sep = 0.75pt] [font = \scriptsize] {$z_{0}$};


\end{tikzpicture}
\end{figure}
\FloatBarrier
Calcolare
\begin{equation*}
I_{\alpha, k} = \int^{+ \infty}_{- \infty}\frac{x^{k}}{\alpha^{2} + x^{2k}} dx\ \ \ \ \alpha > 0, k\in \NN
\end{equation*}

\Esercizio{}

Calcolare
\begin{equation*}
I = \int^{+ \infty}_{0}\frac{\sin x}{x} dx
\end{equation*}

\Esercizio{}

Calcolare
\begin{equation*}
\int^{+ \infty}_{- \infty}\frac{\sin\left(\pi x\right)}{\left(x - 3\right)\left(x^{2} - 2x + 2\right)} dx
\end{equation*}

\Esercizio{}

Vediamo una terza tipologia di integrali
\begin{equation*}
\boxed{\int^{+ \infty}_{- \infty} R\left(e^{x}\right)\left\{
\begin{array}{c}
1\\
e^{ix}
\end{array}\right\} dx}
\end{equation*}
Si risolvono integrando sui rettangoli, facendo degli opportuni morsi se e dove necessario.

Calcolare
\begin{equation*}
\int^{+ \infty}_{- \infty}\frac{\cos x}{\cosh x} dx
\end{equation*}

\ParteSoluzioni

\Soluzione

$f\in \Hc (\CC \setminus \{0, \pm i\})$ dove $\pm i$ sono poli semplici, mentre $0$ è un polo del II ordine.


\begin{figure}[htpb]
	\centering
\tikzset{every picture/.style = {line width = 0.75pt}} %set default line width to 0.75pt

\begin{tikzpicture}[x = 0.75pt, y = 0.75pt, yscale = -1, xscale = 1]
%uncomment if require: \path (0, 205); %set diagram left start at 0, and has height of 205

%Shape: Axis 2D [id:dp9998288753339848]
\draw (180, 140.25) - - (410, 140.25)(250, 27.25) - - (250, 190) (403, 135.25) - - (410, 140.25) - - (403, 145.25) (245, 34.25) - - (250, 27.25) - - (255, 34.25) ;
%Straight Lines [id:da12445530205724542]
\draw [dash pattern = {on 0.84pt off 2.51pt}] (291, 101) - - (250.5, 99.5) ;
%Shape: Ellipse [id:dp7674970843501638]
\draw [fill = {rgb, 255:red, 0; green, 0; blue, 0}, fill opacity = 1 ] (248, 140.75) .. controls (248, 139.37) and (249.12, 138.25) .. (250.5, 138.25) .. controls (251.88, 138.25) and (253, 139.37) .. (253, 140.75) .. controls (253, 142.13) and (251.88, 143.25) .. (250.5, 143.25) .. controls (249.12, 143.25) and (248, 142.13) .. (248, 140.75) - - cycle ;
%Shape: Ellipse [id:dp5063104104492224]
\draw [fill = {rgb, 255:red, 0; green, 0; blue, 0}, fill opacity = 1 ] (248, 180.75) .. controls (248, 179.37) and (249.12, 178.25) .. (250.5, 178.25) .. controls (251.88, 178.25) and (253, 179.37) .. (253, 180.75) .. controls (253, 182.13) and (251.88, 183.25) .. (250.5, 183.25) .. controls (249.12, 183.25) and (248, 182.13) .. (248, 180.75) - - cycle ;
%Shape: Ellipse [id:dp3475276859167522]
\draw [fill = {rgb, 255:red, 0; green, 0; blue, 0}, fill opacity = 1 ] (248, 99.5) .. controls (248, 98.12) and (249.12, 97) .. (250.5, 97) .. controls (251.88, 97) and (253, 98.12) .. (253, 99.5) .. controls (253, 100.88) and (251.88, 102) .. (250.5, 102) .. controls (249.12, 102) and (248, 100.88) .. (248, 99.5) - - cycle ;
%Shape: Circle [id:dp9945561623234371]
\draw (250, 100) .. controls (250, 77.91) and (267.91, 60) .. (290, 60) .. controls (312.09, 60) and (330, 77.91) .. (330, 100) .. controls (330, 122.09) and (312.09, 140) .. (290, 140) .. controls (267.91, 140) and (250, 122.09) .. (250, 100) - - cycle ;
%Shape: Circle [id:dp6603014030673247]
\draw (233, 100) .. controls (233, 68.52) and (258.52, 43) .. (290, 43) .. controls (321.48, 43) and (347, 68.52) .. (347, 100) .. controls (347, 131.48) and (321.48, 157) .. (290, 157) .. controls (258.52, 157) and (233, 131.48) .. (233, 100) - - cycle ;
%Shape: Circle [id:dp8201517506911262]
\draw (200, 100) .. controls (200, 50.29) and (240.29, 10) .. (290, 10) .. controls (339.71, 10) and (380, 50.29) .. (380, 100) .. controls (380, 149.71) and (339.71, 190) .. (290, 190) .. controls (240.29, 190) and (200, 149.71) .. (200, 100) - - cycle ;
%Shape: Ellipse [id:dp147014916814163]
\draw [fill = {rgb, 255:red, 0; green, 0; blue, 0}, fill opacity = 1 ] (289, 101) .. controls (289, 100.45) and (289.45, 100) .. (290, 100) .. controls (290.55, 100) and (291, 100.45) .. (291, 101) .. controls (291, 101.55) and (290.55, 102) .. (290, 102) .. controls (289.45, 102) and (289, 101.55) .. (289, 101) - - cycle ;
%Straight Lines [id:da32148847298984107]
\draw [dash pattern = {on 0.84pt off 2.51pt}] (250, 140.25) - - (290, 101) ;
%Straight Lines [id:da4835195406384667]
\draw [dash pattern = {on 0.84pt off 2.51pt}] (250.5, 180.75) - - (290, 101) ;

% Text Node
\draw (292.5, 100.9) node [anchor = north west][inner sep = 0.75pt] [font = \scriptsize] {$1 + i$};
% Text Node
\draw (232, 182.4) node [anchor = north west][inner sep = 0.75pt] [font = \scriptsize] {$ - i$};
% Text Node
\draw (238.5, 142.4) node [anchor = north west][inner sep = 0.75pt] [font = \scriptsize] {$0$};
% Text Node
\draw (239, 93.4) node [anchor = north west][inner sep = 0.75pt] [font = \scriptsize] {$i$};


\end{tikzpicture}
\end{figure}
\FloatBarrier

Calcoliamo i residui
\begin{gather*}
\Res \left(f, 0\right) = \lim_{z\rightarrow 0}\frac{d}{dz}\left(\left(z - 0\right)^{2}\frac{\left(z + 1\right) e^{\pi z}}{z^{2}\left(z^{2} + 1\right)}\right) = \lim_{z\rightarrow 0}\frac{d}{dz}\frac{\left(z + 1\right) e^{\pi z}}{z^{2} + 1} = \dotsc = 1 + \pi \\
\Res \left(f, i\right) = \lim_{z\rightarrow i}\left(z - i\right) f(z) = \lim_{z\rightarrow i}\frac{\left(z + 1\right) e^{\pi z}}{z^{2}\left(z + i\right)} = \dotsc = \frac{1 + i}{2i}\\
\Res \left(f, - i\right) = \lim_{z\rightarrow - i}\left(z + i\right) f(z) = \dotsc = - \frac{1 - i}{2i}
\end{gather*}
Le distanze dei punti singolari di $f$ dal centro della curva $\gamma_{R}$ sono
\begin{gather*}
d\left(i, i + 1\right) = \left| i - 1 - i\right| = 1\\
d\left(0, i + 1\right) = \left| 1 + i\right| = \sqrt{2}\\
d\left(- i, i + 1\right) = \left| - i - 1 - i\right| = \left| - 2i - 1\right| = \sqrt{5}
\end{gather*}
Allora
\begin{itemize}
\item se $0 < R < 1$ per il teorema dei residui
\begin{equation*}
I_{R} = 0
\end{equation*}
\item se $1 < R < \sqrt{2}$ $f$ è olomorfa all'interno di $\gamma_{R}$ tranne che in $z = i$
\begin{equation*}
I_{R} = 2\pi i \cdot \Res \left(f, i\right) = \left(1 + i\right) \pi
\end{equation*}
\item se $\sqrt{2} < R < \sqrt{5}$ $f$ è olomorfa all'interno di $\gamma_{R}$ tranne che in $z = i$ e $z = 0$
\begin{equation*}
I_{R} = 2\pi i \cdot \left[\Res \left(f, i\right) + \Res \left(f, 0\right)\right] = \pi + i\pi \left(3 + 2\pi\right)
\end{equation*}
\item se $R > \sqrt{5}$ tutte le singolarità sono all'interno di $\gamma_{R}$. $f$ è olomorfa all'interno di $\gamma_{R}$ tranne che in $z = \pm i$ e $z = 0$
\begin{equation*}
I_{R} = 2\pi i \cdot \left[\Res \left(f, 0\right) + \Res \left(f, i\right) + \Res \left(f, - i\right)\right] = i2\pi (2 + \pi)
\end{equation*}
\end{itemize}

Per $R = 1, \sqrt{2}, \sqrt{5}$, $I_{R}$ non può essere calcolato perché $\gamma_{R}$ passa attraverso una singolarità

\Soluzione

\begin{align*}
\int^{2\pi}_{0}\frac{1}{3 + \sin t} dt & = \int_{\left| z\right| = 1}\frac{1}{3 + \frac{z - z^{- 1}}{2i}}\frac{1}{iz} dz\\
 & = \int_{\left| z\right| = 1}\frac{2\cancel{iz}}{z^{2} + 6iz - 1}\frac{1}{\cancel{iz}} dz = 2\int_{\left| z\right| = 1}\frac{i}{z^{2} + 6iz - 1} dz
\end{align*}
Le singolarità sono
\begin{equation*}
z^{2} + 6iz - 1 = 0\ \ \iff \ \
\begin{cases}
z_{1} = i\left(- 3 - 2\sqrt{2}\right)\\
z_{2} = i\left(- 3 + 2\sqrt{2}\right)
\end{cases}
\end{equation*}
di cui solo $z_{2}$ sta dentro il cerchio unitario
\begin{equation*}
2\int_{\left| z\right| = 1}\frac{i}{z^{2} + 6iz - 1} dz = 2 \cdot 2\pi i \cdot \Res \left(f, i\left(- 3 + 2\sqrt{2}\right)\right)
\end{equation*}
ricordando che è un polo semplice
\begin{equation*}
4\pi i \cdot \Res \left(f, i\left(- 3 + 2\sqrt{2}\right)\right) = 4\pi i\lim\limits_{z\rightarrow i\left(- 3 + 2\sqrt{2}\right)}\left[ z - i\left(- 3 + 2\sqrt{2}\right)\right] f(z) = \dotsc = \frac{\pi}{\sqrt{2}}
\end{equation*}

\Soluzione

\begin{equation*}
I_{\alpha, k} = \int^{+ \infty}_{- \infty}\frac{x^{k}}{\alpha^{2} + x^{2k}} dx\ \ \ \ \alpha > 0, k\in \NN
\end{equation*}
Per $x\rightarrow \pm \infty $ si ha $f(x) \sim \frac{1}{x^{k}}$, che è integrabile in un intorno di $\pm \infty $ se e solo se $k > 1$. Quindi $I_{\alpha, k}$ esiste per $k\in \NN, k \geq 2$.

Se $k$ è dispari allora $f$ è dispari e $I_{\alpha, k} = 0$

Se $k$ è pari, con $k \geq 2$, poniamo
\begin{equation*}
f(z) = \frac{z^{k}}{\alpha^{2} + z^{2k}}
\end{equation*}
Le singolarità si trovano ponendo
\begin{equation*}
z^{2k} = - \alpha^{2} = \alpha^{2} e^{\pi i} \ \ \implies \ \ z_{j} = \alpha^{\frac{1}{k}} e^{i\frac{\pi + 2j\pi}{2k}} \ \ \ \ 0 \leq j \leq k - 1
\end{equation*}
allora per il teorema dei residui
\begin{equation*}
\int_{\gamma_{R}} f(z) dz = \int_{\sigma} f(z) dz + \int_{\varphi} f(z) dz = 2\pi i \cdot \sum\limits^{k - 1}_{j = 0}\Res \left(f, z_{j}\right)
\end{equation*}
Nello specifico il pezzo su $\sigma $ tende proprio al nostro integrale
\begin{equation*}
\int_{\sigma} f(z) dz = \left\{z = t, t\in \left[ - R, R\right]\right\} = \int^{R}_{- R} f(t) dt\xrightarrow{R\rightarrow + \infty} I_{\alpha, k}
\end{equation*}
Verifichiamo che
\begin{equation*}
\int_{\varphi} f(z) dz\rightarrow 0
\end{equation*}
Infatti
\begin{align*}
\left| \int_{\varphi} f(z) dz\right| & = \left\{z = Re^{it}, t\in \left[ 0, \pi \right]\right\} = \left| \int^{\pi}_{0}\frac{R^{k} e^{ikt}}{\alpha^{2} + R^{2k} e^{i2kt}} Rie^{it} dt\right| \\
 & \leq \int^{\pi}_{0}\frac{\left| R^{k} e^{ikt}\right|}{\left| \alpha^{2} + R^{2k} e^{i2kt}\right|}\left| Rie^{it}\right| dt \leq \int^{\pi}_{0}\frac{R^{k + 1}}{R^{2k} - \alpha^{2}} dt\xrightarrow{R\rightarrow + \infty} 0
\end{align*}
dove è stata usata la disuguaglianza triangolare
\begin{rem}
[Disuguaglianza triangolare] Per maggiorazioni e minorazioni ecco la forma più generale della disuguaglianza del triangolo
\begin{equation*}
\left| \left| a\right| - \left| b\right| \right| \leq \left| a\pm b\right| \leq \left| a\right| + \left| b\right|
\end{equation*}
\end{rem}

\Soluzione

\begin{equation*}
I = \int^{+ \infty}_{0}\frac{\sin x}{x} dx = \frac{1}{2}\int^{+ \infty}_{- \infty}\frac{\sin x}{x} dx = \frac{1}{2}\Im (J) \ \ \ \ \ \ \ \ J = \int^{+ \infty}_{- \infty}\frac{e^{ix}}{x} dx
\end{equation*}
Posto $f(z) = \frac{e^{iz}}{z}$, $f\in \Hc \left(\CC \setminus \left\{0\right\}\right)$. In questo caso devo fare un \textit{morso} in corrispondenza dello $0$.
\begin{equation*}
J = \lim_{\varepsilon \rightarrow 0^{+}, R\rightarrow + \infty}\int_{\left[ - R, R\right] \setminus \left(- \varepsilon, \varepsilon \right)}\frac{e^{iz}}{z} dz
\end{equation*}
Per il teorema dell'integrale nullo di Cauchy
\begin{equation*}
0 = \int_{\gamma_{R, \varepsilon}} f(z) dz = \int_{\sigma_{1} \cup \sigma_{2}} f(z) dz + \int_{\psi} f(z) dz + \int_{\varphi} f(z) dz
\end{equation*}
Analizziamo i vari pezzi
\begin{equation*}
\int_{\sigma_{1} \cup \sigma_{2}} f(z) dz = \left\{z = t, t\in \left[ - R, - \varepsilon \right) \cup \left(\varepsilon, R\right]\right\} = \int_{\left[ - R, R\right] \setminus \left(- \varepsilon, \varepsilon \right)}\frac{e^{it}}{t} dt\xrightarrow[\varepsilon \rightarrow 0^{+}]{R\rightarrow + \infty} J
\end{equation*}
il morso:
\begin{equation*}
\int_{\psi} f(z) dz\xrightarrow{\varepsilon \rightarrow 0^{+}}\left(\beta - \alpha \right) \cdot \Res \left(f, 0\right) = \left(0 - \pi \right) i \cdot \Res \left(f, 0\right) = - \pi i\frac{e^{0}}{1} = - \pi i
\end{equation*}
mentre il semicerchio grande:
\begin{align*}
\left| \int_{\varphi} f(z) dz\right| & = \left\{z = Re^{it}, dz = Rie^{it} dt\right\} = \left| \int^{\pi}_{0}\frac{e^{iRe^{it}}}{\cancel{Re^{it}}} i\cancel{Re^{it}} dt\right| \\
 & \leq \int^{\pi}_{0}\left| e^{iRe^{it}}\right| dt = \left\{\left| e^{z}\right| = e^{\Re (z)}\right\}\\
 & = \int^{\pi}_{0} e^{- R\sin t} dt = 2\int^{\frac{\pi}{2}}_{0} e^{- R\sin t} dt \leq 2\int^{\frac{\pi}{2}}_{0} e^{- \frac{2Rt}{\pi}} dt\\
 & = \frac{\pi}{R}\left(1 - e^{- R}\right) \leq \frac{\pi}{R}\xrightarrow{R\rightarrow + \infty} 0
\end{align*}
allora
\begin{equation*}
0 = J - \pi i + 0\ \ \ \ \implies \ \ \ \ J = \pi i\ \ \ \ \implies \ \ \ \ I = \frac{1}{2}\Im (J) = \frac{\pi}{2}
\end{equation*}

\Soluzione

Dobbiamo calcolare
\begin{equation*}
\int^{+ \infty}_{- \infty}\frac{\sin\left(\pi x\right)}{\left(x - 3\right)\left(x^{2} - 2x + 2\right)} dx
\end{equation*}
poniamo
\begin{equation*}
f(z) = \frac{e^{i\pi z}}{\left(z - 3\right)\left(z^{2} - 2z + 2\right)}
\end{equation*}
$f\in \Hc \left(\CC \setminus \left\{3, 1\pm i\right\}\right)$, uno di essi si trova proprio sulla retta reale, pertanto lì faremo un morso.
\begin{align*}
\int_{\gamma_{R, \varepsilon}} f(z) dz & = 2\pi i \cdot \Res \left(f, 1 + i\right) = 2\pi i \cdot \lim_{z\rightarrow 1 + i} f(z)\left(z - \left(1 + i\right)\right) = \\
 & = 2\pi i \cdot \lim_{z\rightarrow 1 + i}\frac{e^{\pi iz}}{\left(z - 3\right)\left(z - 1 + i\right)} = \dotsc = \frac{\pi e^{- \pi}}{5}\left(2 + i\right)
\end{align*}
ma questo integrale possiamo vederlo come la somma di tre pezzi
\begin{equation*}
\int_{\gamma_{R, \varepsilon}} f(z) dz = \int_{\left(- R, 3 - \varepsilon \right) \cup \left(3 + \varepsilon, R\right)} f(z) dz + \int_{\psi} f(z) dz + \int_{\varphi} f(z) dz
\end{equation*}
di cui il primo tende proprio all'integrale che vogliamo calcolare
\begin{equation*}
\int_{\left(- R, 3 - \varepsilon \right) \cup \left(3 + \varepsilon, R\right)} f(z) dz\xrightarrow[\varepsilon \rightarrow 0^{+}]{R\rightarrow + \infty}\int^{+ \infty}_{- \infty}\frac{e^{i\pi z}}{\left(x - 3\right)\left(x^{2} - 2x + 2\right)} dx
\end{equation*}
occupiamoci del pezzo sul morso sfruttando il Lemma del cerchio piccolo
\begin{align*}
\lim_{\varepsilon \rightarrow 0^{+}}\int_{\psi} f(z) dz & = - \pi i \cdot \Res \left(f, 3\right)\\
 & = - \pi i \cdot \lim_{z\rightarrow 3} f(z)\left(z - 3\right) = - \pi i \cdot \lim_{z\rightarrow 3}\frac{e^{i\pi z}}{z^{2} - 2z + 2} = \dotsc = \frac{\pi i}{5}
\end{align*}
ed infine del semicerchio grande
\begin{align*}
\left| \int_{\varphi} f(z) dz\right| & = \left\{z = Re^{it}, dz = Rie^{it} dt\right\}\\
 & = \left| \int^{\pi}_{0}\frac{e^{\pi iRe^{it}}}{\left(Re^{it} - 3\right)\left(R^{2} e^{i2t} - 2Re^{it} + 2\right)} Rie^{it} dt\right| \\
 & \leq \int^{\pi}_{0}\frac{R}{\left(R - 3\right)\left(R^{2} - 2R - 2\right)} dt = \frac{R\pi}{\left(R - 3\right)\left(R^{2} - 2R - 2\right)}\xrightarrow{R\rightarrow + \infty} 0
\end{align*}
quindi
\begin{equation*}
\int^{+ \infty}_{- \infty}\frac{e^{i\pi x}}{\left(x - 3\right)\left(x^{2} - 2x + 2\right)} dx = -\frac{\pi i}{5} + \frac{\pi e^{- \pi}}{5}\left(2 + i\right)
\end{equation*}
ma l'integrale di partenza è proprio la parte immaginaria di questo integrale
\begin{equation*}
\int^{+ \infty}_{- \infty}\frac{\sin\left(\pi x\right)}{\left(x - 3\right)\left(x^{2} - 2x + 2\right)} dx = \Im \left[ - \frac{\pi i}{5} + \frac{\pi e^{- \pi}}{5}\left(2 + i\right)\right] = \frac{\pi}{5}\left(e^{- \pi} - 1\right)
\end{equation*}

\Soluzione

Dobbiamo calcolare
\begin{equation*}
\int^{+ \infty}_{- \infty}\frac{\cos x}{\cosh x} dx
\end{equation*}
consideriamo
\begin{equation*}
f(z) = \frac{e^{iz}}{\cosh z}
\end{equation*}
il cui denominatore si annulla in
\begin{gather*}
\cosh z = 0 = \frac{e^{z} + e^{- z}}{2} \ \ \iff \ \ e^{2z} = - 1\ \ \iff \ \ 2z = \log\left| - 1\right| + i\arg\left(- 1\right)\\
\iff \ \ 2z = 0 + i\left(\pi + 2k\pi \right), k\in \ZZ \ \ \iff \ \ z = \frac{\pi i}{2} + k\pi
\end{gather*}
Integriamo sul rettangolo che contiene il residuo in $\pi i/2$
\begin{equation*}
\int_{\gamma_{R}} f(z) dz = 2\pi i \cdot \Res \left(f, \frac{\pi i}{2}\right) = 2\pi i \cdot \left. \frac{e^{iz}}{\sinh}\right|_{z = \frac{\pi i}{2}} = \dotsc = \frac{2\pi}{e^{\frac{\pi}{2}}}
\end{equation*}
il circuito lo possiamo vedere come $4$ lati di un rettangolo, numerati in senso antiorario partendo da quello in basso


\begin{figure}[htpb]
	\centering
\tikzset{every picture/.style = {line width = 0.75pt}} %set default line width to 0.75pt

\begin{tikzpicture}[x = 0.75pt, y = 0.75pt, yscale = -1, xscale = 1]
%uncomment if require: \path (0, 137); %set diagram left start at 0, and has height of 137

%Shape: Axis 2D [id:dp18751408917716894]
\draw (150, 90.75) - - (450, 90.75)(300, 10) - - (300, 130) (443, 85.75) - - (450, 90.75) - - (443, 95.75) (295, 17) - - (300, 10) - - (305, 17) ;
%Shape: Circle [id:dp11440595401764786]
\draw [fill = {rgb, 255:red, 0; green, 0; blue, 0}, fill opacity = 1 ] (297, 70.5) .. controls (297, 69.12) and (298.12, 68) .. (299.5, 68) .. controls (300.88, 68) and (302, 69.12) .. (302, 70.5) .. controls (302, 71.88) and (300.88, 73) .. (299.5, 73) .. controls (298.12, 73) and (297, 71.88) .. (297, 70.5) - - cycle ;
%Straight Lines [id:da54907639410711]
\draw [color = {rgb, 255:red, 208; green, 2; blue, 27}, draw opacity = 1 ][line width = 1.5] (200, 50) - - (200, 90) ;
\draw [shift = {(200, 70)}, rotate = 270] [fill = {rgb, 255:red, 208; green, 2; blue, 27}, fill opacity = 1 ][line width = 0.08] [draw opacity = 0] (13.4, - 6.43) - - (0, 0) - - (13.4, 6.44) - - (8.9, 0) - - cycle ;
%Straight Lines [id:da8839966596936082]
\draw [color = {rgb, 255:red, 208; green, 2; blue, 27}, draw opacity = 1 ][line width = 1.5] (400, 90) - - (400, 50) ;
\draw [shift = {(400, 70)}, rotate = 450] [fill = {rgb, 255:red, 208; green, 2; blue, 27}, fill opacity = 1 ][line width = 0.08] [draw opacity = 0] (13.4, - 6.43) - - (0, 0) - - (13.4, 6.44) - - (8.9, 0) - - cycle ;
%Straight Lines [id:da5768709170809461]
\draw [color = {rgb, 255:red, 208; green, 2; blue, 27}, draw opacity = 1 ][line width = 1.5] (400, 50) - - (200, 50) ;
\draw [shift = {(300, 50)}, rotate = 360] [fill = {rgb, 255:red, 208; green, 2; blue, 27}, fill opacity = 1 ][line width = 0.08] [draw opacity = 0] (13.4, - 6.43) - - (0, 0) - - (13.4, 6.44) - - (8.9, 0) - - cycle ;
%Straight Lines [id:da7918111958211875]
\draw [color = {rgb, 255:red, 208; green, 2; blue, 27}, draw opacity = 1 ][line width = 1.5] (200, 90) - - (400, 90) ;
\draw [shift = {(300, 90)}, rotate = 180] [fill = {rgb, 255:red, 208; green, 2; blue, 27}, fill opacity = 1 ][line width = 0.08] [draw opacity = 0] (13.4, - 6.43) - - (0, 0) - - (13.4, 6.44) - - (8.9, 0) - - cycle ;
%Shape: Circle [id:dp38938153564529987]
\draw [fill = {rgb, 255:red, 0; green, 0; blue, 0}, fill opacity = 1 ] (297, 110.5) .. controls (297, 109.12) and (298.12, 108) .. (299.5, 108) .. controls (300.88, 108) and (302, 109.12) .. (302, 110.5) .. controls (302, 111.88) and (300.88, 113) .. (299.5, 113) .. controls (298.12, 113) and (297, 111.88) .. (297, 110.5) - - cycle ;
%Shape: Circle [id:dp5781216178988735]
\draw [fill = {rgb, 255:red, 0; green, 0; blue, 0}, fill opacity = 1 ] (297, 30.5) .. controls (297, 29.12) and (298.12, 28) .. (299.5, 28) .. controls (300.88, 28) and (302, 29.12) .. (302, 30.5) .. controls (302, 31.88) and (300.88, 33) .. (299.5, 33) .. controls (298.12, 33) and (297, 31.88) .. (297, 30.5) - - cycle ;

% Text Node
\draw (391, 100.4) node [anchor = north west][inner sep = 0.75pt] {$R$};
% Text Node
\draw (181, 100.4) node [anchor = north west][inner sep = 0.75pt] {$ - R$};
% Text Node
\draw (254, 92.4) node [anchor = north west][inner sep = 0.75pt] {$l_{1}$};
% Text Node
\draw (411, 60.4) node [anchor = north west][inner sep = 0.75pt] {$l_{2}$};
% Text Node
\draw (254, 22.4) node [anchor = north west][inner sep = 0.75pt] {$l_{3}$};
% Text Node
\draw (171, 60.4) node [anchor = north west][inner sep = 0.75pt] {$l_{4}$};
% Text Node
\draw (308, 61.4) node [anchor = north west][inner sep = 0.75pt] {$i\pi /2$};
% Text Node
\draw (308, 101.4) node [anchor = north west][inner sep = 0.75pt] {$ - i\pi /2$};
% Text Node
\draw (308, 21.4) node [anchor = north west][inner sep = 0.75pt] {$i3\pi /2$};


\end{tikzpicture}
\end{figure}
\FloatBarrier

\begin{equation*}
\int_{\gamma_{R}} f(z) dz = \int_{l_{1}} f(z) dz + \int_{l_{2}} f(z) dz + \int_{l_{3}} f(z) dz + \int_{l_{4}} f(z) dz
\end{equation*}
il lato inferiore tende proprio al nostro integrale
\begin{equation*}
\int_{l_{1}} f(z) dz = \left\{z = t\right\} = \int^{R}_{- R}\frac{e^{it}}{\cosh t} dt\xrightarrow{R\rightarrow + \infty}\int^{+ \infty}_{- \infty}\frac{e^{it}}{\cosh t} dt
\end{equation*}
mentre il lato superiore tende al nostro integrale moltiplicato per una costante
\begin{align*}
\int_{l_{3}} f(z) dz & = \left\{z = t + \pi i\right\} = - \int^{R}_{- R}\frac{e^{i\left(t + \pi i\right)}}{\cosh\left(t + \pi i\right)} dt\\
 & = - \int^{R}_{- R}\frac{e^{it} e^{- \pi}}{\frac{e^{t + \pi i} + e^{t + \pi i}}{2}} dt = e^{- \pi}\int^{R}_{- R}\frac{e^{it}}{\cosh t} dt\xrightarrow{R\rightarrow + \infty} e^{- \pi}\int^{+ \infty}_{- \infty}\frac{e^{it}}{\cosh t} dt
\end{align*}
si può dimostrare che i lati verticali tendono a zero
\begin{align*}
\left| \int_{l_{2}} f(z) dz\right| & \leq \left\{z = R + it, t\in \left[ 0, \pi \right], dz = idt\right\} \leq \left| \int^{\pi}_{0}\frac{e^{iR} e^{- t}}{\cosh\left(R + it\right)} idt\right| \\
 & \leq \int^{\pi}_{0}\frac{2e^{- t}}{e^{R} + e^{- R}} dt = \frac{2}{e^{R} + e^{- R}}\int^{\pi}_{0} e^{- t} dt\xrightarrow{R\rightarrow + \infty} 0
\end{align*}
analogamente
\begin{equation*}
\int_{l_{4}} f(z) dz\xrightarrow{R\rightarrow + \infty} 0
\end{equation*}
quindi in conclusione
\begin{equation*}
\frac{2\pi}{e^{\frac{\pi}{2}}} = \int^{+ \infty}_{- \infty}\frac{e^{it}}{\cosh t} dt + 0 + e^{- \pi}\int^{+ \infty}_{- \infty}\frac{e^{it}}{\cosh t} dt + 0
\end{equation*}
isoliamo il nostro integrale
\begin{equation*}
\left(1 + e^{- \pi}\right)\int^{+ \infty}_{- \infty}\frac{e^{it}}{\cosh t} d = \frac{2\pi}{e^{\frac{\pi}{2}}} \ \ \implies \ \ \int^{+ \infty}_{- \infty}\frac{e^{it}}{\cosh t} dt = \frac{2\pi}{\left(1 + e^{- \pi}\right) e^{\frac{\pi}{2}}} = \frac{\pi}{\cosh\left(\frac{\pi}{2}\right)}
\end{equation*}
ma l'integrale di partenza è proprio la parte reale di questo integrale
\begin{equation*}
\int^{+ \infty}_{- \infty}\frac{\cos x}{\cosh x} dx = \Re \left[\int^{+ \infty}_{- \infty}\frac{e^{it}}{\cosh t} dt\right] = \frac{\pi}{\cosh\left(\frac{\pi}{2}\right)}
\end{equation*}
\chapter{Esercitazione 4 - Boella}

\ParteEsercizi

\Esercizio{}

Calcolare il seguente integrale:
\begin{equation*}
\int^{+ \infty}_{0}\frac{\cos (\alpha x)}{x^{2} + \beta^{2}} dx\ \ \ \ \alpha, \beta > 0
\end{equation*}

\Esercizio{}

Calcolare il seguente integrale:
\begin{equation*}
\int_{\RR}\frac{e^{ix}}{x^{3} + x^{2} + x + 1} dx
\end{equation*}

\Esercizio{}

Calcolare l'integrale della funzione polidroma:
\begin{equation*}
\int^{\infty}_{0}\frac{\sqrt[3]{x}}{x^{2} + 4} dx
\end{equation*}

\Esercizio{}

Studiare la convergenza della seguente successione di funzioni:
\begin{equation*}
f_{n} (x) = \frac{\sqrt{n}}{1 + (nx)^{2}}
\end{equation*}

\Esercizio{}

Studiare la convergenza della seguente successione di funzioni:
\begin{equation*}
f_{n} (x) = n^{\alpha} \chi_{\left(0, \frac{1}{n}\right)} (x)
\end{equation*}

\Esercizio{}

Studiare la convergenza della seguente successione di funzioni:
\begin{equation*}
f_{n} (x) = \frac{1 - e^{- x}}{x(1 - x)^{1/n}} \chi_{(0, 1)} (x)
\end{equation*}

\ParteSoluzioni

\Soluzione

Ho due vie:
\begin{itemize}
\item Calcolare l'integrale sfruttando le formule di Eulero
\item Farsi furbi (seguiremo questa soluzione)
\end{itemize}

Riconoscendo che la funzione integranda è pari posso scrivere che:
\begin{equation*}
\int^{+ \infty}_{0}\frac{\cos (\alpha x)}{x^{2} + \beta^{2}} dx\overset{\text{(pari)}}{=}\frac{1}{2}\int_{\RR}\frac{\cos (\alpha x)}{x^{2} + \beta^{2}} dx\overset{(\star)}{=}\frac{1}{2}\left(\int_{\RR}\frac{\cos (\alpha x)}{x^{2} + \beta^{2}} dx + i\int_{\RR}\frac{\sin (\alpha x)}{x^{2} + \beta^{2}} dx\right)
\end{equation*}
Nel passaggio $(\star)$ abbiamo sfruttato l'annullamento dell'integrale di una funzione dispari su tutto $\RR $. Riconosciamo ora che possiamo riscriverla come:

\begin{equation*}
\frac{1}{2}\int_{\RR}\frac{e^{i\alpha x}}{x^{2} + \beta^{2}} dx
\end{equation*}Prendiamo ora in esame $f(z) = \frac{e^{i\alpha x}}{x^{2} + \beta^{2}}$, sfruttiamo il Lemma di Jordan ($\alpha > 0$) e il Teorema dei residui, integrando sulla semicirconferenza superiore.
\begin{equation*}
\int_{\gamma_{R}} f(z) dz = 2\pi i \cdot \Res (f, z = i\beta) = 2\pi i \cdot \left. \frac{e^{i\alpha x}}{2z}\right|_{z = i\beta} = \frac{\pi}{\beta} e^{- \alpha \beta}
\end{equation*}
Per $R\rightarrow + \infty $
\begin{equation*}
\frac{\pi}{\beta} e^{- \alpha \beta} = \int_{\gamma_{R}} = \int^{R}_{- R} + \cancel{\int_{\sigma_{R}}}\xrightarrow{R\rightarrow + \infty}\int_{\RR} f(x) dx\ \ \implies \ \ \frac{1}{2}\int_{\RR}\frac{e^{i\alpha x}}{x^{2} + \beta^{2}} dx = \frac{\pi e^{- \alpha \beta}}{2\beta}
\end{equation*}

\Soluzione

C'è un asintoto verticale in $x = -1$, quindi non converge.
\begin{equation*}
\int_{\RR}\frac{e^{ix}}{x^{3} + x^{2} + x + 1} dx = \int_{\RR}\frac{e^{ix}}{(x^{2} + 1)(x + 1)} dx
\end{equation*}

Non calcoliamo lui, ma il suo \textit{Valore Principale:} ovvero dobbiamo fare i limiti tutti insieme, non separati. In caso di forme di indecisione $(\infty - \infty)$ potrebbe saltare fuori qualcosa di buono.
\begin{equation*}
(VP)\int_{\RR}\frac{e^{ix}}{(x^{2} + 1)(x + 1)} dx = \lim_{\varepsilon \rightarrow 0^{+} \ R\rightarrow \infty}\left[\int^{- 1 - \varepsilon}_{- R} f(x)dx + \int^{R}_{- 1 + \varepsilon} f(x)dx\right]
\end{equation*}
Rinfrescate le idee, ora consideriamo come dominio di integrazione una semicirconferenza come raffigurata e scomponiamo l'integrale sulle 4 curve (due tratti curvilinei e due rettilinei, prima e dopo la circonferenza interna).


\begin{figure}[htpb]
	\centering
\tikzset{every picture/.style = {line width = 0.75pt}} %set default line width to 0.75pt

\begin{tikzpicture}[x = 0.75pt, y = 0.75pt, yscale = -1, xscale = 1]
%uncomment if require: \path (0, 166); %set diagram left start at 0, and has height of 166

%Shape: Axis 2D [id:dp4597416392621234]
\draw (110, 120.09) - - (490, 120.09)(300.5, 10) - - (300.5, 160) (483, 115.09) - - (490, 120.09) - - (483, 125.09) (295.5, 17) - - (300.5, 10) - - (305.5, 17) ;
%Shape: Arc [id:dp48689113730354094]
\draw [draw opacity = 0][line width = 1.5] (231, 120.09) .. controls (231, 120.09) and (231, 120.09) .. (231, 120.09) .. controls (231, 81.71) and (262.12, 50.59) .. (300.5, 50.59) .. controls (338.88, 50.59) and (370, 81.71) .. (370, 120.09) - - (300.5, 120.09) - - cycle ; \draw [color = {rgb, 255:red, 208; green, 2; blue, 27}, draw opacity = 1 ][line width = 1.5] (231, 120.09) .. controls (231, 120.09) and (231, 120.09) .. (231, 120.09) .. controls (231, 81.71) and (262.12, 50.59) .. (300.5, 50.59) .. controls (338.88, 50.59) and (370, 81.71) .. (370, 120.09) ;
%Shape: Arc [id:dp3783656851334405]
\draw [draw opacity = 0][line width = 1.5] (250.48, 119.98) .. controls (250.48, 119.98) and (250.48, 119.98) .. (250.48, 119.98) .. controls (250.48, 108.93) and (259.44, 99.96) .. (270.5, 99.96) .. controls (281.56, 99.96) and (290.52, 108.93) .. (290.52, 119.98) - - (270.5, 119.98) - - cycle ; \draw [color = {rgb, 255:red, 208; green, 2; blue, 27}, draw opacity = 1 ][line width = 1.5] (250.48, 119.98) .. controls (250.48, 119.98) and (250.48, 119.98) .. (250.48, 119.98) .. controls (250.48, 108.93) and (259.44, 99.96) .. (270.5, 99.96) .. controls (281.56, 99.96) and (290.52, 108.93) .. (290.52, 119.98) ;
%Shape: Circle [id:dp8264521136482348]
\draw [fill = {rgb, 255:red, 0; green, 0; blue, 0}, fill opacity = 1 ] (269, 119.98) .. controls (269, 119.15) and (269.67, 118.48) .. (270.5, 118.48) .. controls (271.33, 118.48) and (272, 119.15) .. (272, 119.98) .. controls (272, 120.81) and (271.33, 121.48) .. (270.5, 121.48) .. controls (269.67, 121.48) and (269, 120.81) .. (269, 119.98) - - cycle ;
\draw [draw opacity = 0][fill = {rgb, 255:red, 208; green, 2; blue, 27}, fill opacity = 1 ] (347.82, 80) - - (342.34, 64.44) - - (358.08, 69.39) - - (347.64, 69.57) - - cycle ;
\draw [draw opacity = 0][fill = {rgb, 255:red, 208; green, 2; blue, 27}, fill opacity = 1 ] (325.24, 113) - - (340, 120.38) - - (325.24, 127.76) - - (332.62, 120.38) - - cycle ;
%Straight Lines [id:da6476958893771378]
\draw [color = {rgb, 255:red, 208; green, 2; blue, 27}, draw opacity = 1 ][line width = 1.5] (231, 120.09) - - (250.48, 119.98) ;
%Straight Lines [id:da8263168573202715]
\draw [color = {rgb, 255:red, 208; green, 2; blue, 27}, draw opacity = 1 ][line width = 1.5] (290, 120) - - (370, 120.09) ;
%Shape: Circle [id:dp7603714452226784]
\draw [fill = {rgb, 255:red, 0; green, 0; blue, 0}, fill opacity = 1 ] (298, 100.09) .. controls (298, 98.71) and (299.12, 97.59) .. (300.5, 97.59) .. controls (301.88, 97.59) and (303, 98.71) .. (303, 100.09) .. controls (303, 101.47) and (301.88, 102.59) .. (300.5, 102.59) .. controls (299.12, 102.59) and (298, 101.47) .. (298, 100.09) - - cycle ;
%Shape: Circle [id:dp46768054307943396]
\draw [fill = {rgb, 255:red, 0; green, 0; blue, 0}, fill opacity = 1 ] (298, 140.09) .. controls (298, 138.71) and (299.12, 137.59) .. (300.5, 137.59) .. controls (301.88, 137.59) and (303, 138.71) .. (303, 140.09) .. controls (303, 141.47) and (301.88, 142.59) .. (300.5, 142.59) .. controls (299.12, 142.59) and (298, 141.47) .. (298, 140.09) - - cycle ;

% Text Node
\draw (211, 122.4) node [anchor = north west][inner sep = 0.75pt] {$ - R$};
% Text Node
\draw (364, 122.4) node [anchor = north west][inner sep = 0.75pt] {$R$};
% Text Node
\draw (263, 127.4) node [anchor = north west][inner sep = 0.75pt] [font = \scriptsize] {$ - 1$};
% Text Node
\draw (409, 52.4) node [anchor = north west][inner sep = 0.75pt] [color = {rgb, 255:red, 208; green, 2; blue, 27}, opacity = 1 ] {$\gamma_{R, \varepsilon}$};
% Text Node
\draw (271, 72.4) node [anchor = north west][inner sep = 0.75pt] [color = {rgb, 255:red, 208; green, 2; blue, 27}, opacity = 1 ] {$\sigma_{\varepsilon}$};
% Text Node
\draw (340, 40.4) node [anchor = north west][inner sep = 0.75pt] [color = {rgb, 255:red, 208; green, 2; blue, 27}, opacity = 1 ] {$\sigma_{R}$};
% Text Node
\draw (307, 92.4) node [anchor = north west][inner sep = 0.75pt] [font = \scriptsize] {$i$};
% Text Node
\draw (307, 136.4) node [anchor = north west][inner sep = 0.75pt] [font = \scriptsize] {$ - i$};


\end{tikzpicture}
\end{figure}
\FloatBarrier

\begin{equation*}
\int_{\gamma_{R, \varepsilon}} f(z)dz = 2\pi i \cdot \Res (f, z = i) = 2\pi i \cdot \left. \frac{e^{iz}}{3z^{2} + 2z + 1}\right|_{z = i} = 2\pi i \cdot \frac{e^{- 1}}{2i - 2}
\end{equation*}
Quando $\varepsilon \rightarrow 0^{+}, R\rightarrow \infty $
\begin{equation*}
\int_{\gamma_{R, \varepsilon}} = \int^{- 1 - \varepsilon}_{- R} + \int_{\sigma_{\varepsilon}} + \int^{R}_{- 1 - \varepsilon} + \int_{\sigma_{R}}
\end{equation*}
\begin{thm}
[Lemma del cerchio piccolo] Sia $f\in \Hc (D(z_{0}, r) \setminus \{z_{0}\})$, sia $z_{0}$ un polo semplice per $f$, sia $\varphi_{\varepsilon}(t) = z_{0} + \varepsilon e^{it}$ con $t\in [ \alpha, \beta ]$. Allora
\begin{equation*}
\lim_{\varepsilon \rightarrow 0^{+}}\int_{\varphi_{\varepsilon}} f(z) dz = (\beta - \alpha) i \cdot \Res (f, z_{0})
\end{equation*}
\end{thm}
Per Jordan, quello su $\sigma_{R}$ tende a $0$.

Per il lemma del cerchio piccolo, quello su $\sigma_{\varepsilon}$ tende a $(0 - \pi) i \cdot \Res (f, - 1)$.

Applichiamo anche il teorema dei residui sull'integrale su $\gamma_{R, \varepsilon}$.
\begin{equation*}
\begin{aligned}
\int_{\gamma_{R, \varepsilon}} & = \int^{- 1 - \varepsilon}_{- R} + \int^{R}_{- 1 - \varepsilon} + \int_{\sigma_{\varepsilon}} + \int_{\sigma_{R}}\\
2\pi i \cdot \Res (f, i) & = \int_{\RR} f(x) dx - \pi i \cdot \Res (f, - 1) + 0
\end{aligned}
\end{equation*}
Ovvero
\begin{equation*}
\begin{aligned}
\int_{\RR} f(x) dx & = 2\pi i \cdot \Res (f, i) + \pi i \cdot \Res (f, - 1)\\
 & = 2\pi i \cdot \frac{e^{- 1}}{2i - 2} + \pi i \cdot \frac{e^{- i}}{2}\\
 & = \pi i\left(\frac{e^{- 1}}{i - 1} + \frac{e^{- i}}{2}\right)
\end{aligned}
\end{equation*}

\Soluzione

Siamo nella seguente situazione


\begin{figure}[htpb]
	\centering
\tikzset{every picture/.style = {line width = 0.75pt}} %set default line width to 0.75pt

\begin{tikzpicture}[x = 0.75pt, y = 0.75pt, yscale = -1, xscale = 1]
%uncomment if require: \path (0, 253); %set diagram left start at 0, and has height of 253

%Shape: Axis 2D [id:dp4998430184566036]
\draw (162.5, 129.13) - - (441.5, 129.13)(301.5, 13.63) - - (301.5, 245.13) (434.5, 124.13) - - (441.5, 129.13) - - (434.5, 134.13) (296.5, 20.63) - - (301.5, 13.63) - - (306.5, 20.63) ;
%Straight Lines [id:da677349311209462]
\draw [color = {rgb, 255:red, 208; green, 2; blue, 27}, draw opacity = 1 ][line width = 1.5] (331.47, 128.68) - - (400.59, 129.03) ;
\draw [shift = {(366.03, 128.85)}, rotate = 180.29] [fill = {rgb, 255:red, 208; green, 2; blue, 27}, fill opacity = 1 ][line width = 0.08] [draw opacity = 0] (13.4, - 6.43) - - (0, 0) - - (13.4, 6.44) - - (8.9, 0) - - cycle ;
%Shape: Arc [id:dp05736289743124079]
\draw [draw opacity = 0][line width = 1.5] (330.48, 137.92) .. controls (327.04, 150.71) and (315.37, 160.13) .. (301.5, 160.13) .. controls (284.93, 160.13) and (271.5, 146.69) .. (271.5, 130.13) .. controls (271.5, 113.56) and (284.93, 100.13) .. (301.5, 100.13) .. controls (317.58, 100.13) and (330.71, 112.78) .. (331.47, 128.68) - - (301.5, 130.13) - - cycle ; \draw [color = {rgb, 255:red, 208; green, 2; blue, 27}, draw opacity = 1 ][line width = 1.5] (330.48, 137.92) .. controls (327.04, 150.71) and (315.37, 160.13) .. (301.5, 160.13) .. controls (284.93, 160.13) and (271.5, 146.69) .. (271.5, 130.13) .. controls (271.5, 113.56) and (284.93, 100.13) .. (301.5, 100.13) .. controls (317.58, 100.13) and (330.71, 112.78) .. (331.47, 128.68) ;
%Shape: Arc [id:dp5246954139360556]
\draw [draw opacity = 0][line width = 1.5] (398.18, 138.07) .. controls (394.14, 187.93) and (352.4, 227.13) .. (301.5, 227.13) .. controls (247.93, 227.13) and (204.5, 183.7) .. (204.5, 130.13) .. controls (204.5, 76.55) and (247.93, 33.13) .. (301.5, 33.13) .. controls (354.77, 33.13) and (398.01, 76.06) .. (398.5, 129.21) - - (301.5, 130.13) - - cycle ; \draw [color = {rgb, 255:red, 208; green, 2; blue, 27}, draw opacity = 1 ][line width = 1.5] (398.18, 138.07) .. controls (394.14, 187.93) and (352.4, 227.13) .. (301.5, 227.13) .. controls (247.93, 227.13) and (204.5, 183.7) .. (204.5, 130.13) .. controls (204.5, 76.55) and (247.93, 33.13) .. (301.5, 33.13) .. controls (354.77, 33.13) and (398.01, 76.06) .. (398.5, 129.21) ;
%Straight Lines [id:da5817066561864661]
\draw [color = {rgb, 255:red, 208; green, 2; blue, 27}, draw opacity = 1 ][line width = 1.5] (397.11, 137.92) - - (330.48, 137.92) ;
\draw [shift = {(363.79, 137.92)}, rotate = 360] [fill = {rgb, 255:red, 208; green, 2; blue, 27}, fill opacity = 1 ][line width = 0.08] [draw opacity = 0] (13.4, - 6.43) - - (0, 0) - - (13.4, 6.44) - - (8.9, 0) - - cycle ;
\draw [draw opacity = 0][fill = {rgb, 255:red, 208; green, 2; blue, 27}, fill opacity = 1 ] (361.45, 62.08) - - (354.73, 49.18) - - (369.08, 51.55) - - (360, 53) - - cycle ;
%Shape: Circle [id:dp17752216243740282]
\draw [color = {rgb, 255:red, 0; green, 0; blue, 0}, draw opacity = 1 ][fill = {rgb, 255:red, 0; green, 0; blue, 0}, fill opacity = 1 ] (299, 80.13) .. controls (299, 78.74) and (300.12, 77.63) .. (301.5, 77.63) .. controls (302.88, 77.63) and (304, 78.74) .. (304, 80.13) .. controls (304, 81.51) and (302.88, 82.63) .. (301.5, 82.63) .. controls (300.12, 82.63) and (299, 81.51) .. (299, 80.13) - - cycle ;
%Shape: Circle [id:dp22166509967433679]
\draw [color = {rgb, 255:red, 0; green, 0; blue, 0}, draw opacity = 1 ][fill = {rgb, 255:red, 0; green, 0; blue, 0}, fill opacity = 1 ] (299, 180.13) .. controls (299, 178.74) and (300.12, 177.63) .. (301.5, 177.63) .. controls (302.88, 177.63) and (304, 178.74) .. (304, 180.13) .. controls (304, 181.51) and (302.88, 182.63) .. (301.5, 182.63) .. controls (300.12, 182.63) and (299, 181.51) .. (299, 180.13) - - cycle ;

% Text Node
\draw (400.18, 141.47) node [anchor = north west][inner sep = 0.75pt] {$R$};
% Text Node
\draw (448, 122.4) node [anchor = north west][inner sep = 0.75pt] {$x$};
% Text Node
\draw (279, 4.4) node [anchor = north west][inner sep = 0.75pt] {$y$};
% Text Node
\draw (331.48, 141.52) node [anchor = north west][inner sep = 0.75pt] {$\varepsilon $};
% Text Node
\draw (425.18, 45.47) node [anchor = north west][inner sep = 0.75pt] [color = {rgb, 255:red, 208; green, 2; blue, 27}, opacity = 1 ] {$\gamma_{R, \varepsilon}$};
% Text Node
\draw (309.48, 69.52) node [anchor = north west][inner sep = 0.75pt] {$2i$};
% Text Node
\draw (309.48, 171.52) node [anchor = north west][inner sep = 0.75pt] {$ - 2i$};


\end{tikzpicture}
\end{figure}
\FloatBarrier

Se prendiamo
\begin{equation*}
f(z) = \frac{\sqrt[3]{z}}{z^{2} + 4}
\end{equation*}
abbiamo diversi problemi. Dobbiamo assicurarci di prendere la determinazione reale della radice.
\begin{equation*}
f(\rho, \vartheta) = \frac{\sqrt[3]{\rho} e^{i\vartheta /3}}{\rho^{2} e^{2i\vartheta} + 4}
\end{equation*}
I poli sono in $\pm 2i$
\begin{equation*}
\begin{aligned}
\int_{\gamma_{R, \varepsilon}} f(z) dz & = 2\pi i \cdot \left(\frac{\sqrt[3]{2} e^{i\frac{\pi}{2} /3}}{4i} + \frac{\sqrt[3]{2} e^{i\frac{3\pi}{2} /3}}{- 4i}\right) = \frac{\pi \sqrt[3]{2}}{2}\left(e^{i\frac{\pi}{6}} - e^{i\frac{\pi}{2}}\right)\\
\int_{\gamma_{R, \varepsilon}} f(z) dz & = \int^{R}_{\varepsilon} + \int_{\sigma_{R}} + \int^{\varepsilon}_{R} + \int_{\sigma_{\varepsilon}}
\end{aligned}
\end{equation*}
I due integrali sulle circonferenze tendono a $0$, con opportune maggiorazioni.

L'integrale che ci interessa è
\begin{equation*}
\int^{R}_{\varepsilon} f(z) dz = \int^{R}_{\varepsilon}\frac{\sqrt[3]{\rho}}{\rho^{2} + 4} d\rho \rightarrow I
\end{equation*}
L'integrale contromano va fatto facendo \textbf{attenzione} che $\vartheta = 2\pi $, non più zero.
\begin{equation*}
\int^{\varepsilon}_{R} f(z) dz = -\int^{R}_{\varepsilon} f(z) dz = -\int^{R}_{\varepsilon}\frac{\sqrt[3]{\rho} e^{i2\pi /3}}{\rho^{2} + 4} d\rho \rightarrow - Ie^{i2\pi /3}
\end{equation*}
Allora deduciamo che
\begin{equation*}
\begin{aligned}
\int_{\gamma_{R, \varepsilon}} & = \int^{R}_{\varepsilon} + \int_{\sigma_{R}} + \int^{\varepsilon}_{R} + \int_{\sigma_{\varepsilon}}\\
\frac{\pi \sqrt[3]{2}}{2}\left(e^{i\frac{\pi}{6}} - e^{i\frac{\pi}{2}}\right) & = I + 0 - Ie^{i2\pi /3} + 0\\
 & \implies \ \ I = \frac{\frac{\pi \sqrt[3]{2}}{2}\left(e^{i\frac{\pi}{6}} - e^{i\frac{\pi}{2}}\right)}{1 - e^{i2\pi /3}}
\end{aligned}
\end{equation*}
Che possiamo riscrivere
\begin{equation*}
I = \frac{\frac{\pi \sqrt[3]{2}}{2}\left(\frac{\sqrt{3}}{2} + \frac{1}{2} i - i\right)}{1 - \left(- \frac{1}{2} + \frac{\sqrt{3}}{2} i\right)} = \frac{\frac{\pi \sqrt[3]{2}}{2}\left(\frac{\sqrt{3}}{2} - \frac{1}{2} i\right)}{\frac{3}{2} - \frac{\sqrt{3}}{2} i} = \frac{\frac{\pi \sqrt[3]{2}}{2}\left(\frac{\sqrt{3}}{2} - \frac{1}{2} i\right)}{\sqrt{3}\left(\frac{\sqrt{3}}{2} - \frac{1}{2} i\right)} = \frac{\pi \sqrt[3]{2}}{2\sqrt{3}}
\end{equation*}

\Soluzione

Vediamo il grafico

\fg{0.7}{07-4}

Calcoliamo il limite puntuale
\begin{equation*}
f_{n} (x) = \frac{\sqrt{n}}{1 + (nx)^{2}}\xrightarrow{n\rightarrow \infty} F(x) =
\begin{cases}
+ \infty, & x = 0\\
0, & x\neq 0
\end{cases} \ \ \implies \ \ f_{n}\xrightarrow[\text{q.o.}]{n\rightarrow \infty} 0
\end{equation*}
Controlliamo la convergenza in $L^{1}$
\begin{equation*}
\Vert f_{n} - F\Vert_{L^{1}} = \int_{\RR}\frac{\sqrt{n}}{1 + (nx)^{2}} dx = \left\{
\begin{array}{c}
nx = t\\
dx = \frac{1}{n} dt
\end{array}\right\} = \frac{1}{\sqrt{n}}\int_{\RR}\frac{1}{1 + t^{2}} dt\xrightarrow{n\rightarrow \infty} 0
\end{equation*}
Controlliamo la convergenza in $L^{p}$:
\begin{gather*}
\Vert f_{n} - F\Vert^{p}_{L^{p}} = \int_{\RR}\frac{n^{p/2}}{(1 + (nx)^{2})^{p}} = \left\{
\begin{array}{c}
nx = t\\
dx = \frac{1}{n} dt
\end{array}\right\} = n^{\frac{p}{2} - 1}\int_{\RR}\frac{1}{(1 + t^{2})^{p}} dt\\
\\
\implies \ \ \Vert f_{n} - F\Vert^{p}_{L^{p}}\xrightarrow{n\rightarrow \infty} 0\ \ \iff \ \ \frac{p}{2} - 1 < 0\ \ \iff \ \ p < 2
\end{gather*}
Non converge in $L^{\infty}$ perché in $0$ non sono limitate.

\Soluzione

Si può vedere che con $\alpha = 2$

\fg{0.6}{07-5}
\begin{equation*}
f_{n} (x) = n^{\alpha} \chi_{\left(0;\frac{1}{n}\right)} (x)\xrightarrow[\text{q.o}]{n\rightarrow \infty} F(x)\equiv 0
\end{equation*}
Non converge in $L^{\infty}$ perché non sono limitate.

Studiamo la convergenza in $L^{p}$:
\begin{equation*}
\Vert f_{n} - F\Vert^{p}_{L^{p}} = \int^{1/n}_{0} n^{\alpha p} dx = n^{\alpha p - 1}\rightarrow 0\ \ \iff \ \ \alpha p < 1\ \ \iff \ \ \alpha < \frac{1}{p}
\end{equation*}

\Soluzione

Vediamo il grafico

\fg{0.6}{07-6}

Limite puntuale
\begin{equation*}
\lim\limits_{n\rightarrow + \infty} f_{n}(x) = \frac{1 - e^{- x}}{x} \chi_{(0, 1)} (x) = F(x)
\end{equation*}
Non abbiamo problemi di integrabilità in $0$ perché
\begin{equation*}
x\rightarrow 0, \ \ f_{n}(x) \sim \frac{x}{x} = 1
\end{equation*}
Mentre in $1$
\begin{equation*}
x\rightarrow 1, \ \ f_{n}(x) \sim \frac{1 - e^{- 1}}{(1 - x)^{1/n}} \in L^{1}(0, 1) \ \ \iff \ \ n \geq 2
\end{equation*}
Quindi
\begin{equation*}
\Vert f_{n} - F\Vert_{L^{1}} = \int^{1}_{0}\left| \frac{1 - e^{- x}}{x(1 - x)^{1/n}} - \frac{1 - e^{- x}}{x}\right| dx = \int^{1}_{0}\frac{1 - e^{- x}}{x}\left(\frac{1}{(1 - x)^{1/n}} - 1\right) dx
\end{equation*}
Usiamo la convergenza dominata
\begin{equation*}
g_{n}(x) = \frac{1 - e^{- x}}{x}\left(\frac{1}{(1 - x)^{1/n}} - 1\right)
\end{equation*}
Prendiamo la più grande, $n = 2$
\begin{equation*}
| g_{n}(x)| \leq g_{2}(x) \in L^{1}
\end{equation*}
Allora possiamo scambiare limite e segno dell'integrale e dedurre che c'è convergenza in $L^{1}$.
\chapter{Esercitazione 4 - Potrich}

\ParteEsercizi

\Esercizio{}

Vediamo una quarta tipologia
\begin{equation*}
\boxed{\int^{+ \infty}_{0} x^{\alpha} R\left(x^{m}\right) dx} \ \ \ \ \alpha \in \RR, m\in \NN, m \geq 2
\end{equation*}
Se $\alpha $ è un intero, allora si integra su curve così


\begin{figure}[htpb]
	\centering
\tikzset{every picture/.style = {line width = 0.75pt}} %set default line width to 0.75pt

\begin{tikzpicture}[x = 0.75pt, y = 0.75pt, yscale = -1, xscale = 1]
%uncomment if require: \path (0, 137); %set diagram left start at 0, and has height of 137

%Shape: Axis 2D [id:dp5724429407600313]
\draw (190, 109.89) - - (390, 109.89)(210.5, 15.95) - - (210.5, 124) (383, 104.89) - - (390, 109.89) - - (383, 114.89) (205.5, 22.95) - - (210.5, 15.95) - - (215.5, 22.95) ;
%Straight Lines [id:da6909710830091127]
\draw [color = {rgb, 255:red, 208; green, 2; blue, 27}, draw opacity = 1 ][line width = 1.5] (210, 110) - - (273.5, 47.8) ;
\draw [shift = {(241.75, 78.9)}, rotate = 315.59] [fill = {rgb, 255:red, 208; green, 2; blue, 27}, fill opacity = 1 ][line width = 0.08] [draw opacity = 0] (13.4, - 6.43) - - (0, 0) - - (13.4, 6.44) - - (8.9, 0) - - cycle ;
%Curve Lines [id:da21980918672783578]
\draw [color = {rgb, 255:red, 208; green, 2; blue, 27}, draw opacity = 1 ][line width = 1.5] (273.5, 47.69) .. controls (288.5, 65.69) and (296.5, 78.69) .. (300, 110) ;
\draw [shift = {(292.32, 76.73)}, rotate = 65.41] [fill = {rgb, 255:red, 208; green, 2; blue, 27}, fill opacity = 1 ][line width = 0.08] [draw opacity = 0] (13.4, - 6.43) - - (0, 0) - - (13.4, 6.44) - - (8.9, 0) - - cycle ;
%Straight Lines [id:da9818804804141448]
\draw [color = {rgb, 255:red, 208; green, 2; blue, 27}, draw opacity = 1 ][line width = 1.5] (210, 110) - - (300, 110) ;
\draw [shift = {(255, 110)}, rotate = 180] [fill = {rgb, 255:red, 208; green, 2; blue, 27}, fill opacity = 1 ][line width = 0.08] [draw opacity = 0] (13.4, - 6.43) - - (0, 0) - - (13.4, 6.44) - - (8.9, 0) - - cycle ;

% Text Node
\draw (293, 114.4) node [anchor = north west][inner sep = 0.75pt] {$R$};
% Text Node
\draw (395, 104.4) node [anchor = north west][inner sep = 0.75pt] {$x$};
% Text Node
\draw (192, 6.4) node [anchor = north west][inner sep = 0.75pt] {$y$};
% Text Node
\draw (241, 82.4) node [anchor = north west][inner sep = 0.75pt] {$\vartheta $};


\end{tikzpicture}
\end{figure}
\FloatBarrier

Se $\alpha $ non è un intero, allora si integra su curve così


\begin{figure}[htpb]
	\centering
\tikzset{every picture/.style = {line width = 0.75pt}} %set default line width to 0.75pt

\begin{tikzpicture}[x = 0.75pt, y = 0.75pt, yscale = -1, xscale = 1]
%uncomment if require: \path (0, 137); %set diagram left start at 0, and has height of 137

%Shape: Axis 2D [id:dp029235818502451494]
\draw (190, 109.89) - - (390, 109.89)(210.5, 15.95) - - (210.5, 124) (383, 104.89) - - (390, 109.89) - - (383, 114.89) (205.5, 22.95) - - (210.5, 15.95) - - (215.5, 22.95) ;
%Straight Lines [id:da5818987952561847]
\draw [color = {rgb, 255:red, 208; green, 2; blue, 27}, draw opacity = 1 ][line width = 1.5] (240.05, 81.25) - - (273.5, 47.8) ;
\draw [shift = {(256.77, 64.52)}, rotate = 315] [fill = {rgb, 255:red, 208; green, 2; blue, 27}, fill opacity = 1 ][line width = 0.08] [draw opacity = 0] (13.4, - 6.43) - - (0, 0) - - (13.4, 6.44) - - (8.9, 0) - - cycle ;
%Curve Lines [id:da6921320122846151]
\draw [color = {rgb, 255:red, 208; green, 2; blue, 27}, draw opacity = 1 ][line width = 1.5] (273.5, 47.8) .. controls (288.5, 65.8) and (296.5, 78.8) .. (300, 110.11) ;
\draw [shift = {(292.31, 76.84)}, rotate = 65.41] [fill = {rgb, 255:red, 208; green, 2; blue, 27}, fill opacity = 1 ][line width = 0.08] [draw opacity = 0] (13.4, - 6.43) - - (0, 0) - - (13.4, 6.44) - - (8.9, 0) - - cycle ;
%Straight Lines [id:da44850949525670036]
\draw [color = {rgb, 255:red, 208; green, 2; blue, 27}, draw opacity = 1 ][line width = 1.5] (253, 110.11) - - (300, 110.11) ;
\draw [shift = {(276.5, 110.11)}, rotate = 180] [fill = {rgb, 255:red, 208; green, 2; blue, 27}, fill opacity = 1 ][line width = 0.08] [draw opacity = 0] (13.4, - 6.43) - - (0, 0) - - (13.4, 6.44) - - (8.9, 0) - - cycle ;
%Curve Lines [id:da2387592758227124]
\draw [color = {rgb, 255:red, 208; green, 2; blue, 27}, draw opacity = 1 ][line width = 1.5] (240.05, 81.25) .. controls (245.55, 88.25) and (252, 98.25) .. (253, 110.11) ;

% Text Node
\draw (293, 114.4) node [anchor = north west][inner sep = 0.75pt] {$R$};
% Text Node
\draw (395, 104.4) node [anchor = north west][inner sep = 0.75pt] {$x$};
% Text Node
\draw (192, 6.4) node [anchor = north west][inner sep = 0.75pt] {$y$};
% Text Node
\draw (244.5, 113.4) node [anchor = north west][inner sep = 0.75pt] {$\varepsilon $};


\end{tikzpicture}
\end{figure}
\FloatBarrier

L'ampiezza dell'angolo $\vartheta $ deve essere in entrambi i casi $\vartheta = \frac{2\pi}{m}$.



Integriamo la funzione $f(z) = e^{\alpha \log z} R\left(z^{m}\right)$.

\textbf{NB.} Se $m = 2$, l'ampiezza dell'angolo sarebbe $\pi $, tuttavia il dominio del logaritmo esclude che io possa integrare lungo la semiretta $x < 0$.

Se $m = 2$ allora tolgo un altro ramo.


\begin{figure}[htpb]
	\centering
\tikzset{every picture/.style = {line width = 0.75pt}} %set default line width to 0.75pt

\begin{tikzpicture}[x = 0.75pt, y = 0.75pt, yscale = -1, xscale = 1]
%uncomment if require: \path (0, 238); %set diagram left start at 0, and has height of 238

%Shape: Axis 2D [id:dp48898581387364404]
\draw (162.5, 140.33) - - (441.5, 140.33)(300.5, 24.63) - - (300.5, 179.33) (434.5, 135.33) - - (441.5, 140.33) - - (434.5, 145.33) (295.5, 31.63) - - (300.5, 24.63) - - (305.5, 31.63) ;
%Straight Lines [id:da5757902151460577]
\draw [color = {rgb, 255:red, 208; green, 2; blue, 27}, draw opacity = 1 ][line width = 1.5] (330.5, 140.75) - - (397.5, 140.75) ;
\draw [shift = {(364, 140.75)}, rotate = 180] [fill = {rgb, 255:red, 208; green, 2; blue, 27}, fill opacity = 1 ][line width = 0.08] [draw opacity = 0] (13.4, - 6.43) - - (0, 0) - - (13.4, 6.44) - - (8.9, 0) - - cycle ;
%Shape: Arc [id:dp28549627239777453]
\draw [draw opacity = 0][line width = 1.5] (270.5, 140.33) .. controls (270.5, 123.76) and (283.93, 110.33) .. (300.5, 110.33) .. controls (317.07, 110.33) and (330.5, 123.76) .. (330.5, 140.33) .. controls (330.5, 140.47) and (330.5, 140.61) .. (330.5, 140.75) - - (300.5, 140.33) - - cycle ; \draw [color = {rgb, 255:red, 208; green, 2; blue, 27}, draw opacity = 1 ][line width = 1.5] (270.5, 140.33) .. controls (270.5, 123.76) and (283.93, 110.33) .. (300.5, 110.33) .. controls (317.07, 110.33) and (330.5, 123.76) .. (330.5, 140.33) .. controls (330.5, 140.47) and (330.5, 140.61) .. (330.5, 140.75) ;
%Shape: Arc [id:dp39883767783687163]
\draw [draw opacity = 0][line width = 1.5] (203.5, 140.33) .. controls (203.5, 86.76) and (246.93, 43.33) .. (300.5, 43.33) .. controls (354.07, 43.33) and (397.5, 86.76) .. (397.5, 140.33) - - (300.5, 140.33) - - cycle ; \draw [color = {rgb, 255:red, 208; green, 2; blue, 27}, draw opacity = 1 ][line width = 1.5] (203.5, 140.33) .. controls (203.5, 86.76) and (246.93, 43.33) .. (300.5, 43.33) .. controls (354.07, 43.33) and (397.5, 86.76) .. (397.5, 140.33) ;
%Straight Lines [id:da500527919927189]
\draw [color = {rgb, 255:red, 208; green, 2; blue, 27}, draw opacity = 1 ][line width = 1.5] (203.5, 140.33) - - (270.5, 140.33) ;
\draw [shift = {(237, 140.33)}, rotate = 180] [fill = {rgb, 255:red, 208; green, 2; blue, 27}, fill opacity = 1 ][line width = 0.08] [draw opacity = 0] (13.4, - 6.43) - - (0, 0) - - (13.4, 6.44) - - (8.9, 0) - - cycle ;
\draw [draw opacity = 0][fill = {rgb, 255:red, 208; green, 2; blue, 27}, fill opacity = 1 ] (362.12, 76.27) - - (352.31, 57.43) - - (373.27, 60.88) - - (360, 63) - - cycle ;
%Straight Lines [id:da16629969721545534]
\draw [color = {rgb, 255:red, 74; green, 144; blue, 226}, draw opacity = 1 ][line width = 1.5] (302, 140.33) .. controls (303.67, 142) and (303.67, 143.66) .. (302, 145.33) .. controls (300.33, 147) and (300.33, 148.66) .. (302, 150.33) .. controls (303.67, 152) and (303.67, 153.66) .. (302, 155.33) .. controls (300.33, 157) and (300.33, 158.66) .. (302, 160.33) .. controls (303.67, 162) and (303.67, 163.66) .. (302, 165.33) .. controls (300.33, 167) and (300.33, 168.66) .. (302, 170.33) .. controls (303.67, 172) and (303.67, 173.66) .. (302, 175.33) .. controls (300.33, 177) and (300.33, 178.66) .. (302, 180.33) .. controls (303.67, 182) and (303.67, 183.66) .. (302, 185.33) .. controls (300.33, 187) and (300.33, 188.66) .. (302, 190.33) .. controls (303.67, 192) and (303.67, 193.66) .. (302, 195.33) .. controls (300.33, 197) and (300.33, 198.66) .. (302, 200.33) .. controls (303.67, 202) and (303.67, 203.66) .. (302, 205.33) .. controls (300.33, 207) and (300.33, 208.66) .. (302, 210.33) .. controls (303.67, 212) and (303.67, 213.66) .. (302, 215.33) .. controls (300.33, 217) and (300.33, 218.66) .. (302, 220.33) .. controls (303.67, 222) and (303.67, 223.66) .. (302, 225.33) - - (302, 229.33) - - (302, 229.33)(299, 140.33) .. controls (300.67, 142) and (300.67, 143.66) .. (299, 145.33) .. controls (297.33, 147) and (297.33, 148.66) .. (299, 150.33) .. controls (300.67, 152) and (300.67, 153.66) .. (299, 155.33) .. controls (297.33, 157) and (297.33, 158.66) .. (299, 160.33) .. controls (300.67, 162) and (300.67, 163.66) .. (299, 165.33) .. controls (297.33, 167) and (297.33, 168.66) .. (299, 170.33) .. controls (300.67, 172) and (300.67, 173.66) .. (299, 175.33) .. controls (297.33, 177) and (297.33, 178.66) .. (299, 180.33) .. controls (300.67, 182) and (300.67, 183.66) .. (299, 185.33) .. controls (297.33, 187) and (297.33, 188.66) .. (299, 190.33) .. controls (300.67, 192) and (300.67, 193.66) .. (299, 195.33) .. controls (297.33, 197) and (297.33, 198.66) .. (299, 200.33) .. controls (300.67, 202) and (300.67, 203.66) .. (299, 205.33) .. controls (297.33, 207) and (297.33, 208.66) .. (299, 210.33) .. controls (300.67, 212) and (300.67, 213.66) .. (299, 215.33) .. controls (297.33, 217) and (297.33, 218.66) .. (299, 220.33) .. controls (300.67, 222) and (300.67, 223.66) .. (299, 225.33) - - (299, 229.33) - - (299, 229.33) ;

% Text Node
\draw (392, 143.4) node [anchor = north west][inner sep = 0.75pt] {$R$};
% Text Node
\draw (448, 132.4) node [anchor = north west][inner sep = 0.75pt] {$x$};
% Text Node
\draw (279, 14.4) node [anchor = north west][inner sep = 0.75pt] {$y$};
% Text Node
\draw (325.5, 143.4) node [anchor = north west][inner sep = 0.75pt] {$\varepsilon $};
% Text Node
\draw (190, 143.4) node [anchor = north west][inner sep = 0.75pt] {$ - R$};
% Text Node
\draw (259.5, 143.4) node [anchor = north west][inner sep = 0.75pt] {$ - \varepsilon $};
% Text Node
\draw (311, 189) node [anchor = north west][inner sep = 0.75pt] [align = left] {escludo};


\end{tikzpicture}
\end{figure}
\FloatBarrier



Calcolare
\begin{equation*}
I = \int^{+ \infty}_{0}\frac{x^{\alpha}}{1 + x^{3}} dx\ \ \ \ \alpha \in \RR,
\end{equation*}

\Esercizio{}

Vediamo una quinta tipologia
\begin{equation*}
\boxed{
\begin{array}{c l}
A & \int^{+ \infty}_{0} x^{\alpha} R(x) dx\ \ \alpha \neq 0\\
B & \int^{+ \infty}_{0} R(x) dx\\
C & \int^{+ \infty}_{0}\ln(x) R(x) dx
\end{array}}
\end{equation*}
Bisogna integrare su


\begin{figure}[htpb]
	\centering
\tikzset{every picture/.style = {line width = 0.75pt}} %set default line width to 0.75pt

\begin{tikzpicture}[x = 0.75pt, y = 0.75pt, yscale = -1, xscale = 1]
%uncomment if require: \path (0, 253); %set diagram left start at 0, and has height of 253

%Shape: Axis 2D [id:dp376145381719732]
\draw (162.5, 130.13) - - (441.5, 130.13)(301.5, 14.63) - - (301.5, 246.13) (434.5, 125.13) - - (441.5, 130.13) - - (434.5, 135.13) (296.5, 21.63) - - (301.5, 14.63) - - (306.5, 21.63) ;
%Straight Lines [id:da6906588690572244]
\draw [color = {rgb, 255:red, 208; green, 2; blue, 27}, draw opacity = 1 ][line width = 1.5] (330, 130) - - (399.13, 130.35) ;
\draw [shift = {(364.56, 130.18)}, rotate = 180.29] [fill = {rgb, 255:red, 208; green, 2; blue, 27}, fill opacity = 1 ][line width = 0.08] [draw opacity = 0] (13.4, - 6.43) - - (0, 0) - - (13.4, 6.44) - - (8.9, 0) - - cycle ;
%Shape: Arc [id:dp11422143025607046]
\draw [draw opacity = 0][line width = 1.5] (330.48, 137.92) .. controls (327.04, 150.71) and (315.37, 160.13) .. (301.5, 160.13) .. controls (284.93, 160.13) and (271.5, 146.69) .. (271.5, 130.13) .. controls (271.5, 113.56) and (284.93, 100.13) .. (301.5, 100.13) .. controls (317.99, 100.13) and (331.37, 113.43) .. (331.5, 129.88) - - (301.5, 130.13) - - cycle ; \draw [color = {rgb, 255:red, 208; green, 2; blue, 27}, draw opacity = 1 ][line width = 1.5] (330.48, 137.92) .. controls (327.04, 150.71) and (315.37, 160.13) .. (301.5, 160.13) .. controls (284.93, 160.13) and (271.5, 146.69) .. (271.5, 130.13) .. controls (271.5, 113.56) and (284.93, 100.13) .. (301.5, 100.13) .. controls (317.99, 100.13) and (331.37, 113.43) .. (331.5, 129.88) ;
%Shape: Arc [id:dp7692032349738063]
\draw [draw opacity = 0][line width = 1.5] (398.18, 138.07) .. controls (394.14, 187.93) and (352.4, 227.13) .. (301.5, 227.13) .. controls (247.93, 227.13) and (204.5, 183.7) .. (204.5, 130.13) .. controls (204.5, 76.55) and (247.93, 33.13) .. (301.5, 33.13) .. controls (355.07, 33.13) and (398.5, 76.55) .. (398.5, 130.13) .. controls (398.5, 130.16) and (398.5, 130.2) .. (398.5, 130.24) - - (301.5, 130.13) - - cycle ; \draw [color = {rgb, 255:red, 208; green, 2; blue, 27}, draw opacity = 1 ][line width = 1.5] (398.18, 138.07) .. controls (394.14, 187.93) and (352.4, 227.13) .. (301.5, 227.13) .. controls (247.93, 227.13) and (204.5, 183.7) .. (204.5, 130.13) .. controls (204.5, 76.55) and (247.93, 33.13) .. (301.5, 33.13) .. controls (355.07, 33.13) and (398.5, 76.55) .. (398.5, 130.13) .. controls (398.5, 130.16) and (398.5, 130.2) .. (398.5, 130.24) ;
%Straight Lines [id:da4247309814315583]
\draw [color = {rgb, 255:red, 208; green, 2; blue, 27}, draw opacity = 1 ][line width = 1.5] (397.11, 137.92) - - (330.48, 137.92) ;
\draw [shift = {(363.79, 137.92)}, rotate = 360] [fill = {rgb, 255:red, 208; green, 2; blue, 27}, fill opacity = 1 ][line width = 0.08] [draw opacity = 0] (13.4, - 6.43) - - (0, 0) - - (13.4, 6.44) - - (8.9, 0) - - cycle ;
\draw [draw opacity = 0][fill = {rgb, 255:red, 208; green, 2; blue, 27}, fill opacity = 1 ] (361.45, 62.08) - - (354.73, 49.18) - - (369.08, 51.55) - - (360, 53) - - cycle ;

% Text Node
\draw (400.18, 141.47) node [anchor = north west][inner sep = 0.75pt] {$R$};
% Text Node
\draw (448, 122.4) node [anchor = north west][inner sep = 0.75pt] {$x$};
% Text Node
\draw (279, 4.4) node [anchor = north west][inner sep = 0.75pt] {$y$};
% Text Node
\draw (331.48, 141.52) node [anchor = north west][inner sep = 0.75pt] {$\varepsilon $};
% Text Node
\draw (208.18, 47.47) node [anchor = north west][inner sep = 0.75pt] {$\varphi $};
% Text Node
\draw (262.18, 94.47) node [anchor = north west][inner sep = 0.75pt] {$\psi $};
% Text Node
\draw (364.18, 93.47) node [anchor = north west][inner sep = 0.75pt] {$\sigma_{1}$};
% Text Node
\draw (361.18, 145.47) node [anchor = north west][inner sep = 0.75pt] {$\sigma_{2}$};
% Text Node
\draw (425.18, 45.47) node [anchor = north west][inner sep = 0.75pt] [color = {rgb, 255:red, 208; green, 2; blue, 27}, opacity = 1 ] {$\gamma_{R, \varepsilon}$};


\end{tikzpicture}
\end{figure}
\FloatBarrier

Si interpreta $\int_{\gamma_{R, \varepsilon}} = \lim\limits_{\alpha \rightarrow 0^{+}}\int_{\Gamma_{\varepsilon, R, \alpha}}$ dove $\Gamma_{\varepsilon, R, \alpha}$ è


\begin{figure}[htpb]
	\centering
\tikzset{every picture/.style = {line width = 0.75pt}} %set default line width to 0.75pt

\begin{tikzpicture}[x = 0.75pt, y = 0.75pt, yscale = -1, xscale = 1]
%uncomment if require: \path (0, 253); %set diagram left start at 0, and has height of 253

%Shape: Axis 2D [id:dp4977301214304819]
\draw (162.5, 130.13) - - (441.5, 130.13)(301.5, 14.63) - - (301.5, 246.13) (434.5, 125.13) - - (441.5, 130.13) - - (434.5, 135.13) (296.5, 21.63) - - (301.5, 14.63) - - (306.5, 21.63) ;
%Straight Lines [id:da9935131902630598]
\draw [color = {rgb, 255:red, 208; green, 2; blue, 27}, draw opacity = 1 ][line width = 1.5] (327.56, 117.36) - - (386.5, 83.73) ;
\draw [shift = {(357.03, 100.55)}, rotate = 510.29] [fill = {rgb, 255:red, 208; green, 2; blue, 27}, fill opacity = 1 ][line width = 0.08] [draw opacity = 0] (13.4, - 6.43) - - (0, 0) - - (13.4, 6.44) - - (8.9, 0) - - cycle ;
%Shape: Arc [id:dp13640780925563867]
\draw [draw opacity = 0][line width = 1.5] (330.48, 137.92) .. controls (327.04, 150.71) and (315.37, 160.13) .. (301.5, 160.13) .. controls (284.93, 160.13) and (271.5, 146.69) .. (271.5, 130.13) .. controls (271.5, 113.56) and (284.93, 100.13) .. (301.5, 100.13) .. controls (313.42, 100.13) and (323.72, 107.08) .. (328.56, 117.16) - - (301.5, 130.13) - - cycle ; \draw [color = {rgb, 255:red, 208; green, 2; blue, 27}, draw opacity = 1 ][line width = 1.5] (330.48, 137.92) .. controls (327.04, 150.71) and (315.37, 160.13) .. (301.5, 160.13) .. controls (284.93, 160.13) and (271.5, 146.69) .. (271.5, 130.13) .. controls (271.5, 113.56) and (284.93, 100.13) .. (301.5, 100.13) .. controls (313.42, 100.13) and (323.72, 107.08) .. (328.56, 117.16) ;
%Shape: Arc [id:dp11391386463291964]
\draw [draw opacity = 0][line width = 1.5] (383.66, 181.71) .. controls (366.5, 208.99) and (336.12, 227.13) .. (301.5, 227.13) .. controls (247.93, 227.13) and (204.5, 183.7) .. (204.5, 130.13) .. controls (204.5, 76.55) and (247.93, 33.13) .. (301.5, 33.13) .. controls (338.77, 33.13) and (371.13, 54.15) .. (387.38, 84.98) - - (301.5, 130.13) - - cycle ; \draw [color = {rgb, 255:red, 208; green, 2; blue, 27}, draw opacity = 1 ][line width = 1.5] (383.66, 181.71) .. controls (366.5, 208.99) and (336.12, 227.13) .. (301.5, 227.13) .. controls (247.93, 227.13) and (204.5, 183.7) .. (204.5, 130.13) .. controls (204.5, 76.55) and (247.93, 33.13) .. (301.5, 33.13) .. controls (338.77, 33.13) and (371.13, 54.15) .. (387.38, 84.98) ;
%Straight Lines [id:da9801209891079499]
\draw [color = {rgb, 255:red, 208; green, 2; blue, 27}, draw opacity = 1 ][line width = 1.5] (383.66, 181.71) - - (330.48, 137.92) ;
\draw [shift = {(357.07, 159.81)}, rotate = 399.46000000000004] [fill = {rgb, 255:red, 208; green, 2; blue, 27}, fill opacity = 1 ][line width = 0.08] [draw opacity = 0] (13.4, - 6.43) - - (0, 0) - - (13.4, 6.44) - - (8.9, 0) - - cycle ;
\draw [draw opacity = 0][fill = {rgb, 255:red, 208; green, 2; blue, 27}, fill opacity = 1 ] (361.45, 62.08) - - (354.73, 49.18) - - (369.08, 51.55) - - (360, 53) - - cycle ;
%Shape: Arc [id:dp2448048568760779]
\draw [draw opacity = 0][line width = 0.75] (386.82, 88.29) .. controls (391.83, 98.49) and (395.06, 109.72) .. (396.12, 121.58) - - (301.5, 130.13) - - cycle ; \draw [color = {rgb, 255:red, 0; green, 0; blue, 0}, draw opacity = 1 ][line width = 0.75] (386.82, 88.29) .. controls (391.83, 98.49) and (395.06, 109.72) .. (396.12, 121.58) ;

% Text Node
\draw (393.18, 138.47) node [anchor = north west][inner sep = 0.75pt] {$R$};
% Text Node
\draw (448, 122.4) node [anchor = north west][inner sep = 0.75pt] {$x$};
% Text Node
\draw (279, 4.4) node [anchor = north west][inner sep = 0.75pt] {$y$};
% Text Node
\draw (331.48, 141.52) node [anchor = north west][inner sep = 0.75pt] {$\varepsilon $};
% Text Node
\draw (425.18, 45.47) node [anchor = north west][inner sep = 0.75pt] [color = {rgb, 255:red, 208; green, 2; blue, 27}, opacity = 1 ] {$\Gamma_{\varepsilon, R, \alpha}$};
% Text Node
\draw (401.18, 94.47) node [anchor = north west][inner sep = 0.75pt] {$\alpha $};


\end{tikzpicture}
\end{figure}
\FloatBarrier

Dobbiamo usare la determinazione del logaritmo che esclude l'asse reale positivo
\begin{equation*}
\Log_{\left(2\pi \right)} \coloneqq \ln z + i\vartheta \ \ \ \ \vartheta \in \arg z\cap \left(0, 2\pi \right)
\end{equation*}
Si considera rispettivamente
\begin{equation*}
\begin{array}{c l}
A & f(z) = e^{\alpha \cdot \Log_{\left(2\pi \right)} z} R(z)\\
B & f(z) = \Log_{\left(2\pi \right)} z \cdot R(z)\\
C & f(z) = \left(\Log_{\left(2\pi \right)} z\right)^{2} R(z)
\end{array}
\end{equation*}


Calcolare


\begin{equation*}
I = \int^{+ \infty}_{0}\frac{1}{\sqrt{x}\left(1 + x\right)} dx = \int^{+ \infty}_{0}\frac{x^{- 1/2}}{\left(1 + x\right)} dx
\end{equation*}

\Esercizio{}

Calcolare
\begin{equation*}
I = \int^{+ \infty}_{0}\frac{1}{\left(x^{2} + 4\right)\left(x + 1\right)} dx
\end{equation*}

\Esercizio{(Spazi $L^{p}$)}

Si considerino le seguenti funzioni definite su $\left(0, + \infty \right)$ nel seguente modo
\begin{equation*}
f(x) = \frac{1}{\sqrt{x}} \ \ \ \ g(x) = \frac{1}{\sqrt{x + 1}}
\end{equation*}
\begin{enumerate}
\item Stabilire per quali $p\in \left[ 1, + \infty \right]$ le seguenti funzioni stanno in $L^{p}\left(1, + \infty \right)$:
\begin{equation*}
f\ \ \ \ g\ \ \ \ f - g\ \ \ \ \frac{g}{f}
\end{equation*}
\item Stessa domanda in $\left(0, 1\right)$
\end{enumerate}

\Esercizio{}

Sia $B = \left\{\left(x, y\right) \in \RR^{2} :x^{2} + y^{2} \leq 1\right\}$ e sia
\begin{equation*}
f\left(x, y\right) = \frac{1}{\sqrt{x^{2} + y^{2}}}
\end{equation*}
\begin{enumerate}
\item Stabilire per quali $p \geq 1$ si ha $f\in L^{p}(B)$
\item Stabilire per quali $q$ si ha $f\in L^{q}\left(\RR^{2} \setminus B\right)$.
\end{enumerate}

\Esercizio{}

Sia
\begin{equation*}
f_{n}(x) = \frac{n^{2}\sin\left(\frac{x^{2}}{n^{2}}\right)}{\left(x^{2} + n^{2}\right)^{3/2}\log^{3/2}\left(x^{2} + n^{2}\right)} \ \ n\in \NN
\end{equation*}
Rispondere ai seguenti quesiti
\begin{enumerate}
\item Per quali $n\in \NN $ si ha $f_{n} \in L^{1}\left(0, + \infty \right)$.
\item Determinare la funzione limite $f$ in $\left(0, + \infty \right)$.
\item Calcolare $\lim\limits_{n\rightarrow + \infty}\int^{+ \infty}_{0} f_{n}(x) dx$
\end{enumerate}

\Esercizio{}

Date le $f_{n}(x) = \arctan\left(nx\right)$ definite su $\left[ 0, + \infty \right)$.
\begin{enumerate}
\item Determinare il limite puntuale $f(x)$
\item Stabilire se $f_{n}\rightarrow f$ in $L^{p}\left(0, + \infty \right)$ per $p\in \left[ 1, + \infty \right)$.
\end{enumerate}

\ParteSoluzioni

\Soluzione

$f_{\alpha} \sim x^{\alpha - 3} \ \ x\rightarrow + \infty $.

$f_{\alpha}$ è integrabile in un intorno di $ + \infty $ se e solo se $3 - \alpha > 1$, cioè $\iff \alpha < 2$.

$f_{\alpha} \sim x^{\alpha} \ \ x\rightarrow 0^{+}$.

$f_{\alpha}$ è integrabile in senso proprio in un intorno destro di $0$ $\iff - \alpha < 1\iff \alpha > - 1$.

$\implies \exists I$ se e solo se $ - 1 < \alpha < 2$.



Sia $f(z) = \frac{e^{\alpha \log z}}{1 + z^{3}}$.

Considero un angolo $ = \frac{2\pi}{m} = \frac{2\pi}{3}$


\begin{figure}[htpb]
	\centering
\tikzset{every picture/.style = {line width = 0.75pt}} %set default line width to 0.75pt

\begin{tikzpicture}[x = 0.75pt, y = 0.75pt, yscale = -1, xscale = 1]
%uncomment if require: \path (0, 193); %set diagram left start at 0, and has height of 193

%Shape: Axis 2D [id:dp38802361743556224]
\draw (162.5, 140.33) - - (441.5, 140.33)(300.5, 24.63) - - (300.5, 179.33) (434.5, 135.33) - - (441.5, 140.33) - - (434.5, 145.33) (295.5, 31.63) - - (300.5, 24.63) - - (305.5, 31.63) ;
%Straight Lines [id:da7713867715877043]
\draw [color = {rgb, 255:red, 208; green, 2; blue, 27}, draw opacity = 1 ][line width = 1.5] (330.5, 140.75) - - (397.5, 140.75) ;
\draw [shift = {(364, 140.75)}, rotate = 180] [fill = {rgb, 255:red, 208; green, 2; blue, 27}, fill opacity = 1 ][line width = 0.08] [draw opacity = 0] (13.4, - 6.43) - - (0, 0) - - (13.4, 6.44) - - (8.9, 0) - - cycle ;
%Shape: Arc [id:dp12128862418563902]
\draw [draw opacity = 0][line width = 1.5] (283.84, 115.38) .. controls (288.61, 112.19) and (294.34, 110.33) .. (300.5, 110.33) .. controls (317.07, 110.33) and (330.5, 123.76) .. (330.5, 140.33) .. controls (330.5, 140.47) and (330.5, 140.61) .. (330.5, 140.75) - - (300.5, 140.33) - - cycle ; \draw [color = {rgb, 255:red, 208; green, 2; blue, 27}, draw opacity = 1 ][line width = 1.5] (283.84, 115.38) .. controls (288.61, 112.19) and (294.34, 110.33) .. (300.5, 110.33) .. controls (317.07, 110.33) and (330.5, 123.76) .. (330.5, 140.33) .. controls (330.5, 140.47) and (330.5, 140.61) .. (330.5, 140.75) ;
%Shape: Arc [id:dp7213752730044614]
\draw [draw opacity = 0][line width = 1.5] (244.75, 60.94) .. controls (260.52, 49.84) and (279.75, 43.33) .. (300.5, 43.33) .. controls (354.07, 43.33) and (397.5, 86.76) .. (397.5, 140.33) - - (300.5, 140.33) - - cycle ; \draw [color = {rgb, 255:red, 208; green, 2; blue, 27}, draw opacity = 1 ][line width = 1.5] (244.75, 60.94) .. controls (260.52, 49.84) and (279.75, 43.33) .. (300.5, 43.33) .. controls (354.07, 43.33) and (397.5, 86.76) .. (397.5, 140.33) ;
%Straight Lines [id:da09050365562806317]
\draw [color = {rgb, 255:red, 208; green, 2; blue, 27}, draw opacity = 1 ][line width = 1.5] (244.75, 60.94) - - (284.5, 115.89) ;
\draw [shift = {(264.62, 88.42)}, rotate = 234.12] [fill = {rgb, 255:red, 208; green, 2; blue, 27}, fill opacity = 1 ][line width = 0.08] [draw opacity = 0] (13.4, - 6.43) - - (0, 0) - - (13.4, 6.44) - - (8.9, 0) - - cycle ;
\draw [draw opacity = 0][fill = {rgb, 255:red, 208; green, 2; blue, 27}, fill opacity = 1 ] (362.12, 76.27) - - (352.31, 57.43) - - (373.27, 60.88) - - (360, 63) - - cycle ;

% Text Node
\draw (392, 143.4) node [anchor = north west][inner sep = 0.75pt] {$R$};
% Text Node
\draw (448, 132.4) node [anchor = north west][inner sep = 0.75pt] {$x$};
% Text Node
\draw (279, 14.4) node [anchor = north west][inner sep = 0.75pt] {$y$};
% Text Node
\draw (325.5, 143.4) node [anchor = north west][inner sep = 0.75pt] {$\varepsilon $};
% Text Node
\draw (381, 53.9) node [anchor = north west][inner sep = 0.75pt] [color = {rgb, 255:red, 208; green, 2; blue, 27}, opacity = 1 ] {$\varphi $};
% Text Node
\draw (314.5, 90.9) node [anchor = north west][inner sep = 0.75pt] [color = {rgb, 255:red, 208; green, 2; blue, 27}, opacity = 1 ] {$\psi $};
% Text Node
\draw (243.5, 91.4) node [anchor = north west][inner sep = 0.75pt] [color = {rgb, 255:red, 208; green, 2; blue, 27}, opacity = 1 ] {$\sigma_{2}$};
% Text Node
\draw (357, 146.4) node [anchor = north west][inner sep = 0.75pt] [color = {rgb, 255:red, 208; green, 2; blue, 27}, opacity = 1 ] {$\sigma_{1}$};


\end{tikzpicture}
\end{figure}
\FloatBarrier

Chiamo l'intera curva $\gamma_{R, \varepsilon} = \sigma_{1} \cup \varphi \cup \sigma_{2} \cup \psi $

Per il teorema dei residui
\begin{equation*}
\int_{\gamma_{R, \varepsilon}} f(z) dz = 2\pi i \cdot \Res \left(f, e^{i\pi /3}\right)
\end{equation*}
$e^{i\pi /3}$ è l'unico punto interno alla curva in cui si annulla il denominatore.
\begin{equation*}
\begin{aligned}
\int_{\gamma_{R, \varepsilon}} & = \int_{\sigma_{1}} + \int_{\varphi} + \int_{\sigma_{2}} + \int_{\psi}\\
 & \\
\int_{\sigma_{1}} & = \left\{
\begin{array}{c}
z = t\\
t\in [ \varepsilon, R]
\end{array}\right\} = \int^{R}_{\varepsilon}\frac{e^{\alpha \ln t}}{1 + t^{3}} dt\xrightarrow[R\rightarrow + \infty ]{\varepsilon \rightarrow 0^{+}} I\\
 & \\
\int_{\sigma_{2}} & = \left\{
\begin{array}{c}
z = te^{i2\pi /3}\\
t\in [ \varepsilon, R]
\end{array}\right\} = - \int^{R}_{\varepsilon}\frac{e^{\alpha \ln te^{i2\pi /3}}}{1 + t^{3}} e^{i2\pi /3} dt = -\int^{R}_{\varepsilon}\frac{e^{\alpha \left(\ln t + \ln e^{i2\pi /3}\right)}}{1 + t^{3}} e^{i2\pi /3} dt\\
 & = - \int^{R}_{\varepsilon}\frac{e^{\alpha (\ln t + i2\pi /3)}}{1 + t^{3}} e^{i2\pi /3} dt = -e^{i\frac{2}{3} \pi (1 + \alpha)}\int^{R}_{\varepsilon}\frac{e^{\alpha \ln t}}{1 + t^{3}} dt\\
 & \xrightarrow[R\rightarrow + \infty ]{\varepsilon \rightarrow 0^{+}} - e^{i\frac{2}{3} \pi (1 + \alpha)} I\\
 & \\
\left| \int_{\varphi}\right| & = \left\{
\begin{array}{c}
z = Re^{it}\\
t\in [ 0, 2\pi /3]
\end{array}\right\} = \left| \int^{2\pi /3}_{0}\frac{e^{\alpha (\ln R + it)}}{1 + R^{3} e^{i3t}}\textcolor[rgb]{0.29, 0.56, 0.89}{\underbrace{Rie^{it}}_{\gamma'}} dt\right| = \int^{2\pi /3}_{0}\frac{\left| e^{\alpha (\ln R + it)}\right|}{\left| 1 + R^{3} e^{i3t}\right|}\textcolor[rgb]{0.82, 0.01, 0.11}{\left| Rie^{it}\right|} dt\\
 & \textcolor[rgb]{0.96, 0.65, 0.14}{| | a| - | b| | \leq | a + b| \leq | a| + | b|}\\
 & \leq \int^{2\pi /3}_{0}\frac{\left| e^{\alpha (\ln R + it)}\right|}{\left| 1 - R^{3}\right|}\textcolor[rgb]{0.82, 0.01, 0.11}{R} dt \leq \frac{R^{\alpha + 1}}{\left| 1 - R^{3}\right|}\frac{2\pi}{3} dt\sim \frac{2\pi}{3}\frac{1}{R^{2 - \alpha}}\xrightarrow[\alpha \in (- 1, 2)]{R\rightarrow + \infty} 0\\
 & \\
\left| \int_{\psi}\right| & = \left\{
\begin{array}{c}
z = \varepsilon e^{it}\\
t\in [ 0, 2\pi /3]
\end{array}\right\} = \left| \int^{2\pi /3}_{0}\frac{e^{\alpha (\ln \varepsilon + it)}}{1 + \varepsilon^{3} e^{i3t}}\textcolor[rgb]{0.29, 0.56, 0.89}{\underbrace{\varepsilon ie^{it}}_{\gamma'}} dt\right| \leq \dotsc \leq \frac{2\pi}{3} \varepsilon^{\alpha + 1}\xrightarrow[\alpha \in (- 1, 2)]{\varepsilon \rightarrow 0^{+}} 0\\
 & \\
\implies & \ \ \frac{2\pi i \cdot \Res \left(f, e^{i\pi /3}\right)}{1 - e^{i\frac{2}{3} \pi (1 + \alpha)}} = I
\end{aligned}
\end{equation*}
Resta solo da calcolare il residuo.

\Soluzione

Devo considerare la forma $A$
\begin{gather*}
f(z) = e^{- \frac{1}{2}\Log_{(2\pi)} z} \cdot \frac{1}{1 + z}\\
\int_{\gamma_{\varepsilon, R}} f(z) dz = 2\pi i \cdot \sum \Res (f) = 2\pi i \cdot \Res (f, - 1)
\end{gather*}
Ugualmente
\begin{equation*}
\begin{aligned}
\int_{\gamma_{R, \varepsilon}} & = \int_{\sigma_{1}} + \int_{\varphi} + \int_{\sigma_{2}} + \int_{\psi}\\
 & \\
\int_{\sigma_{1} \cup \sigma_{2}} & = \left\{
\begin{array}{c}
z = t\\
t\in [ \varepsilon, R]
\end{array}\right\} = \int^{R}_{\varepsilon}\frac{1}{1 + t}\left[ e^{- \frac{1}{2}\ln t} - e^{- \frac{1}{2}(\ln t + 2\pi i)}\right] dt\\
 & = \underbrace{\left(1 - e^{- \pi i}\right)}_{1 + 1 = 2}\int^{R}_{\varepsilon}\frac{1}{\sqrt{t}(1 + t)} dt\xrightarrow[R\rightarrow + \infty ]{\varepsilon \rightarrow 0^{+}} 2I\\
 & \\
\int_{\varphi} & \xrightarrow{R\rightarrow + \infty} 0\\
\int_{\psi} & \xrightarrow{\varepsilon \rightarrow 0^{+}} 0\\
 & \\
\implies & \ \ 2I = 2\pi i \cdot \Res (f, - 1) \ \ \implies \ \ I = \pi i \cdot \Res (f, - 1)
\end{aligned}
\end{equation*}

\Soluzione

Siamo nel caso $B$
\begin{gather*}
f(z) = \Log_{\left(2\pi \right)} z \cdot R(z) = f(z) = \frac{\Log_{\left(2\pi \right)} z}{\left(z^{2} + 4\right)\left(z + 1\right)}\\
\int_{\gamma_{\varepsilon, R}} f(z) dz = 2\pi i \cdot \sum \Res \left(f, z_{i}\right)
\end{gather*}
Il denominatore si annulla per $z = -1, \pm 2i$.
\begin{equation*}
\begin{aligned}
\int_{\gamma_{R, \varepsilon}} & = \int_{\sigma_{1}} + \int_{\varphi} + \int_{\sigma_{2}} + \int_{\psi}\\
 & \\
\int_{\sigma_{1} \cup \sigma_{2}} & = \left\{
\begin{array}{c}
z = t\\
t\in \left[ \varepsilon, R\right]
\end{array}\right\} = \int^{R}_{\varepsilon}\frac{1}{\left(t + 1\right)\left(t^{2} + 4\right)}\left[\ln t - \left(\ln t + 2\pi i\right)\right] dt\xrightarrow[R\rightarrow + \infty ]{\varepsilon \rightarrow 0^{+}} - 2\pi iI\\
 & \\
\int_{\varphi} & \xrightarrow{R\rightarrow + \infty} 0\\
\int_{\psi} & \xrightarrow{\varepsilon \rightarrow 0^{+}} 0
\end{aligned}
\end{equation*}

\Soluzione

Rispondiamo al punto $1$.
\begin{equation*}
\
\begin{aligned}
\left| f(x)\right|^{p} & = \frac{1}{x^{p/2}} \ \ \implies \ \ f\in L^{p}\left(1, + \infty \right) \ \ \iff \ \ \frac{p}{2} > 1\ \ \iff \ \ p > 2\ \left(+ \infty \ \text{incluso}\right)\\
 & \\
\left| g(x)\right|^{p} & = \frac{1}{\left(x + 1\right)^{p/2}}\overset{x\rightarrow + \infty}{\sim}\frac{1}{x^{p/2}} \ \ \implies \ \ g\in L^{p}\left(1, + \infty \right) \ \ \iff \ \ p > 2\ \left(+ \infty \ \text{incluso}\right)\\
 & \\
\left| f - g\right|^{p} & = \left| \frac{1}{\sqrt{x}} - \frac{1}{\sqrt{x + 1}}\right|^{p} = \left| \frac{\sqrt{x + 1} - \sqrt{x}}{\sqrt{x}\sqrt{x + 1}}\right|^{p}\\
 & = \left| \frac{\sqrt{x + 1} - \sqrt{x}}{\sqrt{x^{2} + x}} \cdot \frac{\sqrt{x + 1} + \sqrt{x}}{\sqrt{x + 1} + \sqrt{x}}\right|^{p} = \left| \frac{x + 1 - x}{\sqrt{x^{2} + x}\left(\sqrt{x + 1} + \sqrt{x}\right)}\right|^{p}\\
 & = \left| \frac{x + 1 - x}{\sqrt{x^{2} + x}\left(\sqrt{x + 1} + \sqrt{x}\right)}\right|^{p}\overset{x\rightarrow + \infty}{\sim}\left| \frac{1}{x\left(2\sqrt{x}\right)}\right|^{p} = \frac{1}{2^{p} x^{3p/2}}\\
 & \implies \ \ f - g\in L^{p}\left(1, + \infty \right) \ \ \iff \ \ \frac{3}{2} p > 1\ \ \iff \ \ p > \frac{2}{3} \ \text{cioè il primo intero:} \ p \geq 1\\
 & \\
\frac{g(x)}{f(x)} & \overset{x\rightarrow + \infty}{\sim} 1\ \ \implies \ \ \frac{g}{f} \in L^{p}\left(1, + \infty \right) \ \ \iff \ \ p = +\infty
\end{aligned}
\end{equation*}
Rispondiamo al punto $2$.
\begin{equation*}
| f(x)|^{p}\overset{x\rightarrow 0^{+}}{\sim}\frac{1}{x^{p/2}} \ \ \implies \ \ f\in L^{p}(0, 1) \ \ \iff \ \ \frac{p}{2} < 1\ \ \iff \ \ 1 \leq p < 2
\end{equation*}
$g$ è limitata in $(0, 1) \implies g\in L^{p}(0, 1) \ \forall p \geq 1$, incluso $ + \infty $.

$f - g\in L^{p}(0, 1) \iff 1 \leq p < 2$ dato che so sommando due funzioni, di cui una limitata su intervallo limitato che non presenta problemi e una in $L^{p}$ per $1 \leq p < 2$.

$\frac{g}{f} \in L^{p}(0, 1) \ \forall p \geq 1$, incluso $ + \infty $.

\Soluzione

Rispondiamo al punto $1$.

Passiamo in coordinate polari centrate nell'origine
\begin{gather*}
\begin{cases}
x = \rho \cos \vartheta \\
y = \rho \sin \vartheta
\end{cases} \ \ \implies \ \ \iint_{B}| f(x, y)|^{p} dxdy = \iint_{B^{\star}}\left| \frac{1}{\rho}\right|^{p} \rho d\rho d\vartheta \\
\\
B^{\star} = \{(\rho, \vartheta) :0 \leq \rho \leq 1, 0 < \vartheta < 2\pi \}\\
\\
\int^{2\pi}_{0} d\vartheta \int^{1}_{0}\frac{1}{\rho^{p - 1}} d\rho < + \infty \ \ \iff \ \ p - 1 < 1\ \ \iff \ \ p < 2\land p \geq 1
\end{gather*}
Rispondiamo al punto $2$.
\begin{gather*}
\iint_{\RR^{2} \setminus B}| f(x, y)|^{q} dxdy = 2\pi \int^{+ \infty}_{1}\frac{1}{\rho^{q - 1}} d\rho < + \infty \ \ \iff \ \ q - 1 > 1\\
\iff q > 2\ \left(+ \infty \ \text{incluso}\right)
\end{gather*}

\Soluzione

\begin{thm}
[della convergenza dominata] Sia $\{f_{n}\}$ una successione di funzioni misurabili su un insieme $\Omega \subset \RR^{n}$ misurabile. Supponiamo che siano verificate le seguenti due condizioni:\footnote{È importante che la $g$ non dipenda da $n$.}
\begin{equation*}
\begin{cases}
f_{n}\xrightarrow{n\rightarrow + \infty} f\ \text{q.o. in} \ \Omega \\
\exists g\in L^{1}(\Omega) :| f_{n}| \leq g(x) \ \text{per q.o.} \ x\in \Omega, \ \forall n\in \NN
\end{cases}
\end{equation*}
Allora
\begin{equation*}
f\in L^{1}(\Omega), \ \ \ \ \lim_{n\rightarrow + \infty}\int_{\Omega} f_{n}(x) dx = \int_{\Omega}\lim_{n\rightarrow + \infty} f_{n}(x) dx
\end{equation*}
\end{thm}
\begin{enumerate}
\item Le $f_{n}$ sono continue su $E = (0, + \infty)$ e quindi misurabili su $E$

$\forall n > 0$ fissato si ha
\begin{equation*}
| f_{n}(x)| \leq \frac{n^{2}}{\left(x^{2}\right)^{3/2}\log^{3/2}\left(x^{2}\right)} = \frac{n^{2}}{x^{3}(2\log x)^{3/2}} = \frac{n^{2}}{2^{3/2} x^{3}(\log x)^{3/2}} \ \ x\rightarrow + \infty
\end{equation*}

$f_{n}$ è integrabile nell'intorno di $ + \infty, \forall n > 0$.
\begin{equation*}
\text{per} \ x\rightarrow 0\ \ f(x) \sim
\begin{cases}
\frac{x^{2}}{n^{3}\log^{3/2}\left(n^{2}\right)}, & n \geq 2\\
\frac{x^{2}}{\log^{3/2}\left(x^{2} + 1\right)} \sim \frac{x^{2}}{x^{3}} = \frac{1}{x}, & n = 1
\end{cases}
\end{equation*}

$f_{n}$ sono integrabili in $U\left(0^{+}\right), \forall n \geq 2$.

$\implies f_{n} \in L^{1}(0, + \infty) \iff n \geq 2$.
\item Calcoliamo
\begin{equation*}
\begin{aligned}
\lim_{n\rightarrow + \infty} f_{n}(x_{0}) & = \left\{x_{0} \ \text{è fissato}\right\}\\
 & = \lim_{n\rightarrow + \infty}\frac{n^{2}\sin\left(\frac{x^{2}_{0}}{n^{2}}\right)}{\left(x^{2}_{0} + n^{2}\right)^{3/2}\log^{3/2}\left(x^{2}_{0} + n^{2}\right)}\\
 & = \lim_{n\rightarrow + \infty}\frac{x^{2}_{0}}{n^{3}\log^{3/2}\left(n^{2}\right)} = 0
\end{aligned}
\end{equation*}

$f_{n}$ converge puntualmente alla funzione limite $f(x) \equiv 0$.
\item Osservo
\begin{equation*}
| \sin x| \leq x\ \ \forall x\in (0, + \infty) \ \ \implies \ \ n^{2}\sin\left(\frac{x^{2}}{n^{2}}\right) \leq n^{2}\frac{x^{2}}{n^{2}} = x^{2}
\end{equation*}

Per $n \geq 2$
\begin{equation*}
| f_{n}(x)| \leq g(x) \coloneqq \frac{x^{2}}{\left(x^{2} + 2^{2}\right)^{3/2}\log^{3/2}\left(x^{2} + 2^{2}\right)} \in L^{1}(0, + \infty)
\end{equation*}
\begin{enumerate}
\item diminuisco il denominatore
\item devo ottenere qualcosa che \underline{\textbf{non dipende}} da $n$
\end{enumerate}

$\overset{\text{Dom}}{\implies}\lim\limits_{n\rightarrow + \infty}\int^{+ \infty}_{0} f_{n}(x) dx = \int^{+ \infty}_{0}\lim\limits_{n\rightarrow + \infty} f_{n}(x) dx = 0$.
\end{enumerate}

\Soluzione

\begin{defn}
\begin{equation*}
f_{n}\xrightarrow[n\rightarrow + \infty ]{L^{p}} f\ (p\in [ 1, + \infty)) \ \ \iff \ \ \Vert f_{n} - f\Vert^{p}_{L^{p}}\xrightarrow{n\rightarrow + \infty} 0
\end{equation*}
\end{defn}
\begin{enumerate}
\item Fisso $x_{0} \in [ 0, + \infty)$
\begin{enumerate}
\item se $x_{0} = 0\ \ \implies \ \ f_{n}(0) = 0$
\item se $x_{0} \neq 0\ \ \implies \ \ \lim\limits_{n\rightarrow + \infty} f_{n}(x) = \frac{\pi}{2}$
\end{enumerate}

$\implies \ \ \{f_{n}\}_{n\in \NN}$ converge puntualmente a $f(x) =
\begin{cases}
0, & x = 0\\
\pi /2, & x\neq 0
\end{cases}$

\fg{0.7}{08-7}
\item Se $x > 0$
\begin{gather*}
f(x) - f_{n}(x) = \frac{\pi}{2} - \arctan(nx) = \arctan\left(\frac{1}{nx}\right)\\
\implies \ \ \Vert f_{n} - f\Vert^{p}_{L^{p}} = \int^{+ \infty}_{0}\left(\arctan\left(\frac{1}{nx}\right)\right)^{p} dx
\end{gather*}

Escludo $p = 1$, perché $\arctan\left(\frac{1}{nx}\right) \sim \frac{1}{nx}$ e $\frac{1}{x}$ non è integrabile a $ + \infty $.

Verifichiamo le ipotesi della convergenza dominata.
\begin{equation*}
\left| \left(\arctan\left(\frac{1}{nx}\right)\right)^{p}\right| \leq g(x) \coloneqq
\begin{cases}
\left(\frac{\pi}{2}\right)^{p}, & 0 < x < 1\ \left(\text{integrabile in zero} \ \forall p\right)\\
\left| \frac{1}{x}\right|^{p} & x \geq 1\ \left(\text{uso proprio} \ n = 1\ \text{per maggiorare}\right)
\end{cases}
\end{equation*}

$\overset{\text{Dom}}{\implies}\lim\limits_{n\rightarrow + \infty}\Vert f_{n} - f\Vert^{p}_{L^{p}} = \lim\limits_{n\rightarrow + \infty}\int^{+ \infty}_{0}\left(\arctan\left(\frac{1}{nx}\right)\right)^{p} dx = \int^{+ \infty}_{0}\lim\limits_{n\rightarrow + \infty}\left(\arctan\left(\frac{1}{nx}\right)\right)^{p} dx = 0$
\end{enumerate}
\chapter{Esercitazione 5 - Boella}

\ParteEsercizi

\Esercizio{}

Studiare la convergenza di
\begin{equation*}
f(x, y) = \frac{1}{(x^{2} + y^{2})^{\alpha}\sqrt{1 + x^{2} + y^{2}}}
\end{equation*}
scegliendo $\alpha $ tale per cui:
\begin{itemize}
\item $f\in L^{1} (\RR)$
\item $f\in L^{p} (\RR)$
\item $f\in L^{\infty} (\RR)$
\end{itemize}

\Esercizio{}

Studiare la convergenza (ovvero limite puntuale, limite in $L^{p}$, limite in $L^{\infty}$) della seguente serie di funzioni:
\begin{equation*}
f_{n} (x) = \pi - 2\arctan (|x|^{n})
\end{equation*}

\Esercizio{}

Analizzare la seguente serie:
\begin{equation*}
f(x) = \sum^{\infty}_{n = 1}\frac{n + 1}{n^{4} + n}\sin (n\pi x)
\end{equation*}
\begin{itemize}
\item $f_{n} (x)\in L^{2}$?
\item Che periodo ha?
\item Si può dire qualcosa anche sulla continuità?
\end{itemize}

\Esercizio{}

Data una funzione $2\pi $ - periodica
\begin{equation*}
f:[ 0, 2\pi)\rightarrow \RR \ \ \ \ f(x) = x
\end{equation*}
Determinare la sue serie di Fourier. Si riesce a ricondurre la serie ad una nota?

\Esercizio{}

Data una funzione $1$ - periodica dispari
\begin{equation*}
f:[ 0, 1]\rightarrow \RR \ \ \ \ f(x) = x - x^{2}
\end{equation*}
Determinare la sue serie di Fourier.

\ParteSoluzioni

\Soluzione

Sicuramente non appartiene a $L^{\infty}$ a meno che $\alpha $ non sia negativo dato che se $\alpha $ è positivo questa funzione ha un asintoto verticale.

Per vedere se $f\in L^{1}$, posto che sia sempre positiva, basta integrarla su tutto il piano e passare in coordinate polari. Quindi dovrà convergere l'integrale doppio
\begin{gather*}
\int^{2\pi}_{0} d\vartheta \int^{+ \infty}_{0}\underbrace{\frac{1}{\rho^{2}\sqrt{1 + \rho^{2}}} \rho}_{g(\rho)} d\rho \\
\begin{aligned}
\rho \rightarrow 0, & \ \ g(\rho)\sim \frac{1}{\rho^{2\alpha - 1}} \ \ \implies \ \ 2\alpha - 1 < 1\ \ \implies \ \ \alpha < 1\\
\rho \rightarrow + \infty, & \ \ g(\rho)\sim \frac{1}{\rho^{2\alpha}} \ \ \implies \ \ 2\alpha > 1\ \ \implies \ \ \alpha > \frac{1}{2}
\end{aligned}
\end{gather*}
Quindi
\begin{equation*}
f(x, y)\in L^{1} (\RR^{2})\ \ \iff \ \ \frac{1}{2} < \alpha < 1
\end{equation*}
Se invece consideriamo il generico $L^{p}$ dobbiamo elevare alla $p$, ma non il $\rho $ dello Jacobiano
\begin{equation*}
g(\rho) = \frac{\rho}{\rho^{2\alpha p}\left(\sqrt{1 + \rho^{2}}\right)^{p}}
\end{equation*}
Calcoliamo
\begin{equation*}
\begin{aligned}
\rho \rightarrow 0^{+}, & \ \ g(\rho)\sim \frac{1}{\rho^{2\alpha p - 1}} \ \ \implies \ \ 2\alpha p - 1 < 1\ \ \implies \ \ \alpha < \frac{1}{\rho}\\
\rho \rightarrow + \infty, & \ \ g(\rho)\sim \frac{1}{\rho^{(2\alpha + 1)p - 1}} \ \ \implies \ \ (2\alpha + 1)p - 1 > 1\ \ \implies \ \ \alpha > \frac{1}{\rho} - \frac{1}{2}
\end{aligned}
\end{equation*}
Allora
\begin{equation*}
f(x, y)\in L^{p} (\RR^{2})\ \ \iff \ \ \frac{1}{p} - \frac{1}{2} < \alpha < \frac{1}{p}
\end{equation*}

\Soluzione

Vediamo come sono fatte le $f_{n}$.

\fg{0.7}{09-2}
\begin{equation*}
F(x) = \lim_{n\rightarrow \infty}\left[ \pi - 2\arctan (|x|^{n})\right] =
\begin{cases}
\pi, & |x| < 1\\
\frac{\pi}{2}, & |x| = 1\\
0, & |x| > 1
\end{cases}
\end{equation*}
Il limite in $L^{\infty}$ è immediato in quanto tutte le funzioni $f_{n}$ sono continue e quindi anche il limite dovrebbe essere continuo. Visto che il limite, come abbiamo visto, non è quasi ovunque continuo la convergenza in $L^{\infty}$ non sussiste.

Inoltre, la norma della differenza
\begin{equation*}
\Vert f_{n} - F\Vert_{\infty} = \frac{\pi}{2} \nrightarrow 0
\end{equation*}
La funzione converge a $F(x)$ in $L^{1}, L^{p}$? Con la definizione
\begin{equation*}
\begin{aligned}
\lim_{x\rightarrow + \infty}\frac{f_{n} (x)}{\left(\frac{1}{x}\right)^{\alpha}} & = \lim_{x\rightarrow + \infty}\frac{\pi - 2\arctan (x^{n})}{x^{- \alpha}}\\
 & \overset{\text{H}}{=}\lim_{x\rightarrow \infty}\frac{- 2\frac{nx^{n - 1}}{1 + x^{2n}}}{- \alpha x^{- \alpha - 1}} = \lim_{x\rightarrow \infty}\frac{2n}{\alpha}\frac{x^{- n - 1}}{x^{- \alpha - 1}} =
\begin{cases}
\infty, & \alpha > n\\
2, & \alpha = n\\
0, & \alpha < n
\end{cases}
\end{aligned}
\end{equation*}
Quindi tutte le funzioni stanno in $L^{p}$, ma in realtà la funzione con $n = 1$ non sta in $L^{1}$ ma in \ $L^{p}, p > 1$.

Per la convergenza in norma $L^{p}$ controlliamo la norma
\begin{equation*}
\Vert f_{n} - F\Vert^{p}_{L^{p}} = 2\int^{1}_{0} (\pi - f_{n} (x))^{p} dx + 2\int^{+ \infty}_{1} (f_{n} (x))^{p} dx
\end{equation*}
Utilizziamo la convergenza dominata in maniera disinvolta, formalmente servirebbe una $h(x)$ che maggiori
\begin{equation*}
| g_{n}(x) - G(x)| \leq h(x)
\end{equation*}
in realtà basta meno, serve una $h(x)$ che maggiori
\begin{equation*}
| g_{n}(x)| < h(x)
\end{equation*}
dalla quale si può risalire a quella cercata con la disuguaglianza triangolare.

Scegliamo quindi
\begin{equation*}
h(x) =
\begin{cases}
\pi, & |x| \leq 1\\
f_{2} (x), & |x| > 1
\end{cases}
\end{equation*}

\Soluzione

\begin{itemize}
\item Se gli $a_{n}$ e i $b_{n}$ tendono a $0$ in maniera monotona, allora la serie converge salvo in $0, 2\pi, 4\pi \dotsc $
\item Vale
\begin{equation*}
\sum\limits^{\infty}_{n = 0}(| a_{n}| + | b_{n}|) < \infty \ \ \implies \ \ \text{la serie converge a una funzione continua}
\end{equation*}
\end{itemize}

Usiamo il secondo risultato
\begin{equation*}
f(x) \sim \frac{1}{n^{3}} \ \ \implies \ \ \text{converge}
\end{equation*}
Nella serie compaiono il seno ed il coseno di $2\pi nx/T$, quindi $T = 2$.

Essendo una serie di soli seni la funzione è una funzione dispari\footnote{se fosse stata una funzione di soli coseni allora la funzione sarebbe stata pari.}.

Se la deriviamo, spunta fuori un $n$, quindi va a zero come $\frac{1}{n^{2}}$, quindi anche la derivata è continua, ovvero la funzione è $C^{1}$. Non possiamo dire altro.

\Soluzione

Si può sviluppare in serie di Fourier perché $f\in L^{2}(0, 2\pi)$.\footnote{In alcuni libri una volta costruita la serie di Fourier si scrive
\begin{equation*}
f(x)\sim \frac{a_{o}}{2} + \sum \ldots
\end{equation*}
dove il simbolo $\sim $ indica che alla funzione \textit{si associa} la serie di Fourier.}

Scriviamo la serie di Fourier della funzione
\begin{equation*}
\begin{aligned}
a_{0} & = \frac{1}{\pi}\int^{2\pi}_{0} xdx = 2\pi \\
a_{n} & = \frac{1}{\pi}\int^{2\pi}_{0} x\cos (nx)dx\overset{\text{ipp}}{=}\frac{1}{\pi}\left\{\left[\frac{x}{n}\sin (nx)\right]^{2\pi}_{0} - \frac{1}{n}\int^{2\pi}_{0}\sin (nx)dx\right\} = 0\\
b_{n} & = \frac{1}{\pi}\int^{2\pi}_{0} x\sin (nx)dx\overset{\text{ipp}}{=}\frac{1}{\pi}\left\{\left[ - \frac{x}{n}\cos (nx)\right]^{2\pi}_{0} - \frac{1}{n}\int^{2\pi}_{0}\cos (nx)dx\right\} = - \frac{2}{n}
\end{aligned}
\end{equation*}
Possiamo allora dire che
\begin{equation*}
f(x) = \pi + \sum^{+ \infty}_{n = 1} - \frac{2}{n}\sin (nx)
\end{equation*}
\fg{0.6}{09-4}

A seguito di questo esempio possiamo dire se converge la serie
\begin{equation*}
\sum^{+ \infty}_{n = 1}\frac{\sin n}{n}
\end{equation*}
Basta calcolare
\begin{equation*}
1 = f(1) = \pi - 2\sum^{\infty}_{n = 1}\frac{\sin (n)}{n} \ \ \implies \ \ \sum^{+ \infty}_{n = 1}\frac{\sin n}{n} = \frac{\pi - 1}{2}
\end{equation*}
Altra cosa che possiamo utilizzare avendo la serie di Fourier è l'uguaglianza di Parseval: se una funzione sta in $L^{2}(0, 2\pi)$ \ sviluppata in serie di Fourier allora
\begin{equation*}
\int^{2\pi}_{0} f(x)^{2} dx = \pi \left[\frac{a^{2}_{0}}{2} + \sum^{+ \infty}_{n = 1}\left(a^{2}_{n} + b^{2}_{n}\right)\right]
\end{equation*}
Applicandola alla serie che abbiamo calcolato avremo che
\begin{equation*}
\begin{aligned}
\int^{2\pi}_{0} x^{2} dx & = \pi \left[\frac{(2\pi)^{2}}{2} + \sum^{+ \infty}_{n = 1}\frac{4}{n^{2}}\right]\\
\frac{8}{3} \pi^{3} & = \pi \left[\frac{(2\pi)^{2}}{2} + \sum^{+ \infty}_{n = 1}\frac{4}{n^{2}}\right]\\
 & \implies \ \ \sum^{+ \infty}_{n = 1}\frac{1}{n^{2}} = \frac{\frac{8}{3} \pi^{2} - 2\pi^{2}}{4} = \frac{\pi^{2}}{6}
\end{aligned}
\end{equation*}

\Soluzione

Dal grafico vediamo che la sua funzione è continua.

Lo sarà anche la sua derivata? Si vede dal grafico che quando $x\rightarrow 0^{-}$ la funzione è asintotica a $x$ mentre quando $x\rightarrow 0^{+}$ la funzione è asintotica a $x$ ergo la derivata sarà continua. Non si può certo dire della sua derivata seconda in quanto nell'origine ci sarà un'oscillazione del segno della costante. È ragionevole aspettarci che i coefficienti della serie convergeranno come $1/x^{3}$.

Visto che la funzione è dispari sappiamo già che $a_{0} = a_{n} = 0$
\begin{equation*}
f(x) = \sum^{+ \infty}_{n = 1} b_{n}\sin (n\pi x)
\end{equation*}
Calcoliamo $b_{n}$

Usando le proprietà delle funzioni pari e il fatto che l'integrale possiamo farlo dove ci pare, purché di ampiezza uguale al periodo
\begin{equation*}
\begin{aligned}
b_{n} & = \frac{1}{T/2}\int^{T}_{0} f(x)\sin\left(\frac{2\pi n}{T} x\right) dx = \int^{1}_{- 1} f(x)\sin(\pi nx) dx\\
 & = 2\int^{1}_{0} f(x)\sin(\pi nx) dx = 2\int^{1}_{0}\left(x - x^{2}\right)\sin(\pi nx) dx\\
 & \overset{\text{ipp}}{=} 2\left\{\cancel{\left[ - \frac{x - x^{2}}{n\pi}\cos(n\pi x)\right]^{1}_{0}} + \frac{1}{n\pi}\int^{1}_{0}(1 - 2x)\cos(n\pi x) dx\right\}\\
 & \overset{\text{ipp}}{=}\frac{2}{n\pi}\left\{\cancel{\left[\frac{1 - 2x}{n\pi}\sin(n\pi x)\right]^{1}_{0}} + \frac{2}{n\pi}\int^{1}_{0}\sin(n\pi x) dx\right\}\\
 & = \frac{4}{n^{2} \pi^{2}}\left[ - \frac{1}{n\pi}\cos (nx)\right]^{1}_{0} = \frac{4}{n^{3} \pi^{3}}\left[ 1 - (- 1)^{n}\right] =
\begin{cases}
0, & n\ \text{pari}\\
\frac{8}{n^{3} \pi^{3}}, & n\ \text{dispari}
\end{cases}
\end{aligned}
\end{equation*}
In definitiva avremo, con $n = 2k + 1$
\begin{equation*}
f(x) = \sum^{+ \infty}_{n = 1}\frac{4}{n^{3} \pi^{3}} (1 - (- 1)^{n}\sin (n\pi x) = \sum^{+ \infty}_{k = 0}\frac{8}{\pi^{3} (2k + 1)^{3}}\sin (\pi (2k + 1)x)
\end{equation*}
\chapter{Esercitazione 5 - Potrich}

\ParteEsercizi

\Esercizio{}

Data la successione di funzioni $f_{n} :[ 0, + \infty)\rightarrow \RR $
\begin{equation*}
f_{n}(x) = \sqrt{n} e^{- nx}
\end{equation*}
determinarne il limite puntuale e nelle norme $L^{1}, L^{2}, L^{\infty}$.

\Esercizio{}

Si consideri la successione di funzioni
\begin{equation*}
f_{n} (x) = \frac{2n^{2}\left(1 - \cos\frac{x}{n}\right)}{\left(1 + \frac{x}{n}\right)^{n}}, \ \ x\in (0, + \infty)
\end{equation*}
\begin{enumerate}
\item Stabilire per quale $n\in \NN $ si ha $f_{n} \in L^{1}((1, + \infty))$ e per quale $n\in \NN $ si ha

$f_{n} \in L^{2}((1, + \infty))$
\item Calcolare
\begin{equation*}
\lim\limits_{n\rightarrow + \infty}\int^{+ \infty}_{0} f_{n}(x) dx\ \ \ \ \ \ \ \ \lim\limits_{n\rightarrow + \infty}\int^{+ \infty}_{0}[ f_{n}(x)]^{2} dx
\end{equation*}
\end{enumerate}

\Esercizio{}

Fissato $\alpha \in [ 0, 4]$, consideriamo
\begin{equation*}
f_{n, \alpha}(x) = \frac{n^{\alpha}}{\left(x^{2} + n^{2}\right)\left(x^{2} + 4n^{2}\right)}
\end{equation*}
\begin{enumerate}
\item Determinare il limite puntuale $F_{\alpha}(x)$.
\item Stabilire al variare di $\alpha $ se $f_{n, \alpha}\rightarrow F_{\alpha}$ in $L^{1}(\RR)$ e in $L^{\infty}(\RR)$.
\end{enumerate}

\Esercizio{}

Sia $f:\RR \rightarrow \RR $ la funzione $2\pi $ - periodica che sull'intervallo $(- \pi, \pi ]$ è definita da
\begin{equation*}
f(x) = x + | x| =
\begin{cases}
2x, & x\in [ 0, \pi ]\\
0, & x\in (- \pi, 0)
\end{cases}
\end{equation*}
\begin{enumerate}
\item Disegnare il grafico di $f$ in $[ - 3\pi, 3\pi ]$
\item Stabilire per quali valori di $x\in \RR $ la serie di Fourier $F$ di $f$ converge puntualmente, precisandone l'eventuale limite.
\item Calcolare la serie di Fourier $F$ di $f$.
\item Stabilire se $F$ converge in media quadratica a $f$.
\item Studiare la convergenza uniforme di $F$ a $f$
\end{enumerate}

\Esercizio{}

Si consideri la funzione
\begin{equation*}
f(x) = \cos\left(\sqrt{2} \cdot x\right) \ \ \ \ \forall x\in \left[ - \frac{\pi}{2}, \frac{\pi}{2}\right]
\end{equation*}
ripetuta periodicamente
\begin{enumerate}
\item Calcolare i coefficienti di Fourier di $f$.
\item Calcolare il valore di
\begin{equation*}
\sum\limits^{\infty}_{k = 1}\frac{1}{2k^{2} - 1}
\end{equation*}
\item Studiare la convergenza della serie (puntuale, in media quadratica, uniforme).
\end{enumerate}

\ParteSoluzioni

\Soluzione

Sia $x_{0} \in [ 0, + \infty)$ fissato
\begin{itemize}
\item Se $x_{0} = 0, \ f_{n}(0) = \sqrt{n}\xrightarrow{n\rightarrow + \infty} + \infty $
\item Se $x_{0} > 0, \ f_{n}(x_{0}) = \sqrt{n} e^{- nx_{0}}\xrightarrow{n\rightarrow + \infty} 0$
\end{itemize}
allora $\{f_{n}\}_{n\in \NN}$ converge puntualmente alla funzione limite
\begin{equation*}
f(x) =
\begin{cases}
+ \infty, & x = 0\\
0, & x \geq 0
\end{cases}
\end{equation*}
\fg{0.4}{10-1}
Studiamo la norma $L^{1}$
\begin{equation*}
\begin{aligned}
\| f_{n} - f\|_{1} & = \| f_{n} \|_{1} = \int^{+ \infty}_{0}\sqrt{n} e^{- nx} dx = \\
 & = \sqrt{n}\int^{+ \infty}_{0} e^{- nx} dx = \sqrt{n}\left[\frac{e^{- nx}}{- n}\right]^{+ \infty}_{0} = \frac{1}{\sqrt{n}}\xrightarrow{n\rightarrow + \infty} 0\\
 & \implies \ \ f_{n}\xrightarrow[n\rightarrow + \infty ]{L^{1}} f
\end{aligned}
\end{equation*}
Studiamo la norma $L^{2}$
\begin{equation*}
\begin{aligned}
\| f_{n} - f\|_{2} & = \| f_{n} \|_{2} = \int^{+ \infty}_{0} ne^{- 2nx} dx = \\
 & = n\int^{+ \infty}_{0} e^{- 2nx} dx = n\left[\frac{e^{- 2nx}}{- 2n}\right]^{+ \infty}_{0} = \frac{1}{2}\cancel{\xrightarrow{n\rightarrow + \infty}} 0\\
 & \implies \ \ f_{n}\cancel{\xrightarrow[n\rightarrow + \infty ]{L^{2}}} f
\end{aligned}
\end{equation*}
Studiamo la norma $L^{\infty}$

Le $f_{n}$ sono positive e continue su $\RR^{+}$, quindi
\begin{equation*}
\begin{aligned}
\| f_{n} - f\|_{\infty} & = \| f_{n} \|_{\infty} = \max_{x\in [ 0, + \infty)} f_{n}(x)\\
 & = f_{n}(0) = \sqrt{n}\xrightarrow{n\rightarrow + \infty} + \infty \neq 0\\
 & \implies \ \ f_{n}\cancel{\xrightarrow[n\rightarrow + \infty ]{L^{\infty}}} f
\end{aligned}
\end{equation*}

\Soluzione

Vediamo il grafico

\fg{0.7}{10-2}
\begin{enumerate}
\item Per $x\rightarrow 0^{+}, \ f_{n}(x) \sim \frac{2n^{2} \cdot \frac{x^{2}}{2n^{2}}}{1} = x^{2}$ è integrabile in $U\left(0^{+}\right) \ \forall n\in \NN $

Per $x\rightarrow + \infty, \ | f_{n}(x)| \leq \frac{4n^{2}}{\left(\frac{x}{n}\right)^{n}}$ è integrabile in $U(+ \infty)$ per $n > 1$

$
\begin{array}{l}
\implies f_{n} \in L^{1}((1, + \infty)) \ \ \forall n \geq 2\\
\implies f_{n} \in L^{2}((1, + \infty)) \ \ \forall n \geq 1
\end{array}$
\item $0 \leq 1 - \cos t \leq \frac{t^{2}}{2} \ \forall t$

$n\mapsto \left(1 + \frac{x}{n}\right)^{n}$ è crescente $\forall x > 0$
\begin{equation*}
\implies \ \ 0 \leq | f_{n} (x)| \leq \frac{x^{2}}{\left(1 + \frac{x}{4}\right)^{4}} \ \ \forall x > 0, \forall n \geq 4
\end{equation*}

scegliamo il $4$ perché così la maggiorante $\in L^{1}((0, + \infty))$ e non dipende da $n$

\begin{equation*}
\begin{aligned}
\overset{\text{Dom}}{\implies}\lim\limits_{n\rightarrow + \infty}\int^{+ \infty}_{0} f_{n}(x) dx & = \int^{+ \infty}_{0}\lim\limits_{n\rightarrow + \infty} f_{n}(x) dx\\
 & \left\{\lim\limits_{n\rightarrow + \infty} f_{n}(x_{0}) = \frac{x^{2}_{0}}{e^{x_{0}}}\right\}\\
 & = \int^{+ \infty}_{0} x^{2} e^{- x}\overset{\text{ipp}}{=} 2
\end{aligned}
\end{equation*}

Analogamente si trova una maggiorante $L^{1}$ per $f^{2}_{n}$
\begin{equation*}
\overset{\text{Dom}}{\implies}\lim\limits_{n\rightarrow + \infty}\int^{+ \infty}_{0} f^{2}_{n}(x) dx = \int^{+ \infty}_{0}\lim\limits_{n\rightarrow + \infty} f^{2}_{n}(x) dx = \int^{+ \infty}_{0} x^{4} e^{- 2x}\overset{\text{ipp}}{=}\frac{3}{4}
\end{equation*}
\end{enumerate}

\Soluzione

\begin{enumerate}
\item Sia $x_{0} \in \RR $ fissato
\begin{equation*}
f_{n, \alpha}(x_{0}) \sim \frac{n^{\alpha}}{4n^{4}} = \frac{1}{4n^{4 - \alpha}}\xrightarrow{n\rightarrow + \infty}
\begin{cases}
0, & \alpha \in [ 0, 4)\\
\frac{1}{4}, & \alpha = 4
\end{cases}
\end{equation*}
\item $\forall n\in \NN, \forall \alpha \in [ 0, 4], f_{n, \alpha} \in L^{1}(\RR)$

Tuttavia se $\alpha = 4$, $F_{4} \notin L^{1}(\RR) \implies f_{n, 4} \nrightarrow F_{4}$ in $L^{1}(\RR)$ per $n\rightarrow + \infty $.

Mentre se $\alpha \in [ 0, 4)$
\begin{equation*}
\begin{aligned}
\Vert f_{n, \alpha} - F_{\alpha}\Vert_{L^{1}} & = \Vert f_{n, \alpha}\Vert = \int_{\RR}\frac{n^{\alpha}}{\left(x^{2} + n^{2}\right)\left(x^{2} + 4n^{2}\right)} dx\\
 & = n^{\alpha}\int_{\RR}\frac{1}{\left(x^{2} + n^{2}\right)\left(x^{2} + 4n^{2}\right)} dx = n^{\alpha} \cdot I_{n}
\end{aligned}
\end{equation*}

Calcoliamo $I_{n}$ usando la teoria dei residui.

Poniamo
\begin{equation*}
g_{n}(z) = \frac{1}{\left(z^{2} + n^{2}\right)\left(z^{2} + 4n^{2}\right)}
\end{equation*}

ha $4$ poli semplici in $z = \pm ni$ e $z = \pm 2ni$.

\begin{figure}[htpb]
	\centering
\tikzset{every picture/.style = {line width = 0.75pt}} %set default line width to 0.75pt

\begin{tikzpicture}[x = 0.75pt, y = 0.75pt, yscale = -1, xscale = 1]
%uncomment if require: \path (0, 224); %set diagram left start at 0, and has height of 224

%Shape: Axis 2D [id:dp7020002747782936]
\draw (112.5, 140.17) - - (391.5, 140.17)(250.33, 24.63) - - (250.33, 209.5) (384.5, 135.17) - - (391.5, 140.17) - - (384.5, 145.17) (245.33, 31.63) - - (250.33, 24.63) - - (255.33, 31.63) ;
%Straight Lines [id:da8245049527209181]
\draw [color = {rgb, 255:red, 208; green, 2; blue, 27}, draw opacity = 1 ][line width = 1.5] (250.5, 140.75) - - (347.5, 140.75) ;
\draw [shift = {(299, 140.75)}, rotate = 180] [fill = {rgb, 255:red, 208; green, 2; blue, 27}, fill opacity = 1 ][line width = 0.08] [draw opacity = 0] (13.4, - 6.43) - - (0, 0) - - (13.4, 6.44) - - (8.9, 0) - - cycle ;
%Shape: Arc [id:dp6724290607980123]
\draw [draw opacity = 0][line width = 1.5] (153.5, 140.33) .. controls (153.5, 86.76) and (196.93, 43.33) .. (250.5, 43.33) .. controls (304.07, 43.33) and (347.5, 86.76) .. (347.5, 140.33) - - (250.5, 140.33) - - cycle ; \draw [color = {rgb, 255:red, 208; green, 2; blue, 27}, draw opacity = 1 ][line width = 1.5] (153.5, 140.33) .. controls (153.5, 86.76) and (196.93, 43.33) .. (250.5, 43.33) .. controls (304.07, 43.33) and (347.5, 86.76) .. (347.5, 140.33) ;
%Straight Lines [id:da5477015911612453]
\draw [color = {rgb, 255:red, 208; green, 2; blue, 27}, draw opacity = 1 ][line width = 1.5] (153.5, 140.33) - - (250.5, 140.33) ;
\draw [shift = {(202, 140.33)}, rotate = 180] [fill = {rgb, 255:red, 208; green, 2; blue, 27}, fill opacity = 1 ][line width = 0.08] [draw opacity = 0] (13.4, - 6.43) - - (0, 0) - - (13.4, 6.44) - - (8.9, 0) - - cycle ;
\draw [draw opacity = 0][fill = {rgb, 255:red, 208; green, 2; blue, 27}, fill opacity = 1 ] (312.12, 76.27) - - (302.31, 57.43) - - (323.27, 60.88) - - (310, 63) - - cycle ;
%Shape: Circle [id:dp3205475558675792]
\draw [draw opacity = 0][fill = {rgb, 255:red, 0; green, 0; blue, 0}, fill opacity = 1 ] (248, 80.83) .. controls (248, 79.45) and (249.12, 78.33) .. (250.5, 78.33) .. controls (251.88, 78.33) and (253, 79.45) .. (253, 80.83) .. controls (253, 82.21) and (251.88, 83.33) .. (250.5, 83.33) .. controls (249.12, 83.33) and (248, 82.21) .. (248, 80.83) - - cycle ;
%Shape: Circle [id:dp21100228306693403]
\draw [draw opacity = 0][fill = {rgb, 255:red, 0; green, 0; blue, 0}, fill opacity = 1 ] (248, 118.83) .. controls (248, 117.45) and (249.12, 116.33) .. (250.5, 116.33) .. controls (251.88, 116.33) and (253, 117.45) .. (253, 118.83) .. controls (253, 120.21) and (251.88, 121.33) .. (250.5, 121.33) .. controls (249.12, 121.33) and (248, 120.21) .. (248, 118.83) - - cycle ;
%Shape: Circle [id:dp731209382058088]
\draw [draw opacity = 0][fill = {rgb, 255:red, 0; green, 0; blue, 0}, fill opacity = 1 ] (248, 163.83) .. controls (248, 162.45) and (249.12, 161.33) .. (250.5, 161.33) .. controls (251.88, 161.33) and (253, 162.45) .. (253, 163.83) .. controls (253, 165.21) and (251.88, 166.33) .. (250.5, 166.33) .. controls (249.12, 166.33) and (248, 165.21) .. (248, 163.83) - - cycle ;
%Shape: Circle [id:dp8736348536667242]
\draw [draw opacity = 0][fill = {rgb, 255:red, 0; green, 0; blue, 0}, fill opacity = 1 ] (248, 201.83) .. controls (248, 200.45) and (249.12, 199.33) .. (250.5, 199.33) .. controls (251.88, 199.33) and (253, 200.45) .. (253, 201.83) .. controls (253, 203.21) and (251.88, 204.33) .. (250.5, 204.33) .. controls (249.12, 204.33) and (248, 203.21) .. (248, 201.83) - - cycle ;

% Text Node
\draw (342, 143.4) node [anchor = north west][inner sep = 0.75pt] {$R$};
% Text Node
\draw (398, 132.4) node [anchor = north west][inner sep = 0.75pt] {$x$};
% Text Node
\draw (229, 14.4) node [anchor = north west][inner sep = 0.75pt] {$y$};
% Text Node
\draw (140, 143.4) node [anchor = north west][inner sep = 0.75pt] {$ - R$};
% Text Node
\draw (338, 67.4) node [anchor = north west][inner sep = 0.75pt] {$\sigma_{R}$};
% Text Node
\draw (390, 40.4) node [anchor = north west][inner sep = 0.75pt] {$\gamma_{R} = [ - R, R] \cup \sigma_{R}$};
% Text Node
\draw (259, 69.4) node [anchor = north west][inner sep = 0.75pt] {$2ni$};
% Text Node
\draw (258.5, 108.4) node [anchor = north west][inner sep = 0.75pt] {$ni$};
% Text Node
\draw (259, 153.9) node [anchor = north west][inner sep = 0.75pt] {$ - ni$};
% Text Node
\draw (258.5, 190.9) node [anchor = north west][inner sep = 0.75pt] {$ - 2ni$};


\end{tikzpicture}
\end{figure}
\FloatBarrier
\begin{equation*}
\begin{aligned}
I^{\star}_{n} & = 2\pi i \cdot \{\Res (g_{n}, 2ni) + \Res (g_{n}, ni)\} = 2\pi i \cdot \left\{- \frac{i}{6n^{3}} + \frac{i}{12n^{3}}\right\}\\
I^{\star}_{n} & = \underbrace{\int_{\sigma_{R}} g(z) dz}_{\xrightarrow{R\rightarrow + \infty} 0} + \underbrace{\int^{R}_{- R} g(z) dz}_{\xrightarrow{R\rightarrow + \infty} I_{n}}\\
 & \implies I_{n} = \frac{\pi}{6n^{3}}\\
 & \implies \Vert f_{n, \alpha} - F_{\alpha}\Vert_{L^{1}} = n^{\alpha} \cdot \frac{\pi}{6n^{3}}\xrightarrow{n\rightarrow + \infty} 0\ \ \iff \ \ \alpha < 3\land \alpha \in [ 0, 4)
\end{aligned}
\end{equation*}

Studiamo ora la norma $L^{\infty}$

Per $\alpha \in [ 0, 4)$
\begin{equation*}
\begin{aligned}
\Vert f_{n, \alpha} - F_{\alpha}\Vert_{L^{\infty}} & = \Vert f_{n, \alpha}\Vert_{L^{\infty}} & \\
 & = \max_{x\in \RR} f_{n, \alpha}(x) & \left(f_{n} \ \text{sono continue e positive}\right)\\
 & = f_{n, \alpha}(0) & \text{(simmetria)}\\
 & = \frac{1}{4n^{4 - \alpha}}\xrightarrow{n\rightarrow + \infty} 0 & \\
 & \implies \ \ f_{n, \alpha}\xrightarrow[n\rightarrow + \infty ]{L^{\infty}(\RR)} F_{\alpha} &
\end{aligned}
\end{equation*}Per $\alpha = 4$
\begin{equation*}
\begin{aligned}
\Vert f_{n, 4} - F_{4}\Vert_{L^{\infty}} & = \left\Vert f_{n, 4} - \frac{1}{4}\right\Vert_{L^{\infty}}\\
 & = \max_{x\in \RR}\left[\frac{n^{4}}{\left(x^{2} + n^{2}\right)\left(x^{2} + 4n^{2}\right)} - \frac{1}{4}\right]\\
 & = \frac{1}{4}\cancel{\xrightarrow{n\rightarrow + \infty}} 0\\
 & \implies \ \ f_{n, 4}\cancel{\xrightarrow[n\rightarrow + \infty ]{L^{\infty}(\RR)}} F_{4}
\end{aligned}
\end{equation*}
\end{enumerate}

\Soluzione

\begin{thm}
Sia $f:\RR \rightarrow \RR $, $T$ - periodica, allora la serie di Fourier associata ad $f$ è
\begin{equation*}
F(x) = \frac{a_{0}}{2} + \sum\limits^{\infty}_{n = 1}\left[ a_{n}\cos\left(\frac{2\pi n}{T} x\right) + b_{n}\sin\left(\frac{2\pi n}{T} x\right)\right]
\end{equation*}

dove
\begin{gather*}
a_{0} = \frac{2}{T}\int^{\frac{T}{2}}_{- \frac{T}{2}} f(x) dx\\
a_{n} = \frac{2}{T}\int^{\frac{T}{2}}_{- \frac{T}{2}} f(x)\cos\left(\frac{2\pi n}{T} x\right) dx\ \ \ \ b_{n} = \frac{2}{T}\int^{\frac{T}{2}}_{- \frac{T}{2}} f(x)\sin\left(\frac{2\pi n}{T} x\right) dx
\end{gather*}
\textbf{NB.} Se $f$ è pari, il seno è dispari, $P \cdot D = D\implies b_{n} = 0\ \forall n \geq 1$.

\textbf{NB.} Se $f$ è dispari, il coseno è pari, $D \cdot P = D\implies a_{n} = 0\ \forall n \geq 0$.
\end{thm}
\begin{enumerate}
\item Disegniamo la funzione

\begin{figure}[htpb]
	\centering
\tikzset{every picture/.style = {line width = 0.75pt}} %set default line width to 0.75pt

\begin{tikzpicture}[x = 0.75pt, y = 0.75pt, yscale = -1, xscale = 1]
%uncomment if require: \path (0, 198); %set diagram left start at 0, and has height of 198

%Shape: Axis 2D [id:dp24320808205526556]
\draw (100.11, 120.97) - - (500.89, 120.97)(300.5, 6.41) - - (300.5, 182.63) (493.89, 115.97) - - (500.89, 120.97) - - (493.89, 125.97) (295.5, 13.41) - - (300.5, 6.41) - - (305.5, 13.41) ;
%Straight Lines [id:da6107470839046387]
\draw [color = {rgb, 255:red, 208; green, 2; blue, 27}, draw opacity = 1 ][line width = 1.5] (300.5, 120.67) - - (340.04, 39.87) ;
\draw [shift = {(340.04, 39.87)}, rotate = 296.08] [color = {rgb, 255:red, 208; green, 2; blue, 27}, draw opacity = 1 ][fill = {rgb, 255:red, 208; green, 2; blue, 27}, fill opacity = 1 ][line width = 1.5] (0, 0) circle [x radius = 2.61, y radius = 2.61] ;
%Straight Lines [id:da4254536093640342]
\draw [color = {rgb, 255:red, 208; green, 2; blue, 27}, draw opacity = 1 ][line width = 1.5] (300.5, 120.67) - - (261.9, 120.67) ;
\draw [shift = {(260.29, 120.67)}, rotate = 180] [color = {rgb, 255:red, 208; green, 2; blue, 27}, draw opacity = 1 ][line width = 1.5] (0, 0) circle [x radius = 2.61, y radius = 2.61] ;

%Straight Lines [id:da07742835835652251]
\draw [color = {rgb, 255:red, 208; green, 2; blue, 27}, draw opacity = 1 ][line width = 1.5] (380.92, 120.67) - - (420.46, 39.87) ;
\draw [shift = {(420.46, 39.87)}, rotate = 296.08] [color = {rgb, 255:red, 208; green, 2; blue, 27}, draw opacity = 1 ][fill = {rgb, 255:red, 208; green, 2; blue, 27}, fill opacity = 1 ][line width = 1.5] (0, 0) circle [x radius = 2.61, y radius = 2.61] ;
%Straight Lines [id:da9261529019653281]
\draw [color = {rgb, 255:red, 208; green, 2; blue, 27}, draw opacity = 1 ][line width = 1.5] (380.92, 120.67) - - (342.32, 120.67) ;
\draw [shift = {(340.71, 120.67)}, rotate = 180] [color = {rgb, 255:red, 208; green, 2; blue, 27}, draw opacity = 1 ][line width = 1.5] (0, 0) circle [x radius = 2.61, y radius = 2.61] ;

%Straight Lines [id:da5018794931536248]
\draw [color = {rgb, 255:red, 208; green, 2; blue, 27}, draw opacity = 1 ][line width = 1.5] (220.08, 120.67) - - (259.62, 39.87) ;
\draw [shift = {(259.62, 39.87)}, rotate = 296.08] [color = {rgb, 255:red, 208; green, 2; blue, 27}, draw opacity = 1 ][fill = {rgb, 255:red, 208; green, 2; blue, 27}, fill opacity = 1 ][line width = 1.5] (0, 0) circle [x radius = 2.61, y radius = 2.61] ;
%Straight Lines [id:da10630889376486596]
\draw [color = {rgb, 255:red, 208; green, 2; blue, 27}, draw opacity = 1 ][line width = 1.5] (220.08, 120.67) - - (181.48, 120.67) ;
\draw [shift = {(179.87, 120.67)}, rotate = 180] [color = {rgb, 255:red, 208; green, 2; blue, 27}, draw opacity = 1 ][line width = 1.5] (0, 0) circle [x radius = 2.61, y radius = 2.61] ;

%Shape: Ellipse [id:dp5794600318678271]
\draw [draw opacity = 0][fill = {rgb, 255:red, 208; green, 2; blue, 27}, fill opacity = 1 ] (174.5, 39.87) .. controls (174.5, 37.28) and (176.6, 35.18) .. (179.2, 35.18) .. controls (181.79, 35.18) and (183.89, 37.28) .. (183.89, 39.87) .. controls (183.89, 42.46) and (181.79, 44.57) .. (179.2, 44.57) .. controls (176.6, 44.57) and (174.5, 42.46) .. (174.5, 39.87) - - cycle ;

% Text Node
\draw (161.5, 132.13) node [anchor = north west][inner sep = 0.75pt] [font = \normalsize] {$ - 3\pi $};
% Text Node
\draw (206.07, 132.13) node [anchor = north west][inner sep = 0.75pt] [font = \normalsize] {$ - 2\pi $};
% Text Node
\draw (246.6, 132.13) node [anchor = north west][inner sep = 0.75pt] [font = \normalsize] {$ - \pi $};
% Text Node
\draw (334.04, 132.13) node [anchor = north west][inner sep = 0.75pt] [font = \normalsize] {$\pi $};
% Text Node
\draw (376.1, 132.13) node [anchor = north west][inner sep = 0.75pt] [font = \normalsize] {$2\pi $};
% Text Node
\draw (414.3, 132.13) node [anchor = north west][inner sep = 0.75pt] [font = \normalsize] {$3\pi $};
% Text Node
\draw (276.25, 29.26) node [anchor = north west][inner sep = 0.75pt] [font = \normalsize] {$2\pi $};


\end{tikzpicture}
\end{figure}
\FloatBarrier

\item $f$ è regolare a tratti in $[ - \pi, \pi ] \implies $ la serie di Fourier $F(x)$ converge puntualmente $\forall x\in \RR $.
\begin{enumerate}
\item se $x$ è un punto di continuità, $F(x)$ converge a $f(x)$
\item se $x$ non è punto di continuità, $F(x)$ converge alla media tra il limite destro e sinistro
\end{enumerate}

nel nostro caso $f$ è continua in ogni punto $x\neq (2k + 1) \pi, k\in \ZZ $ e presenta delle discontinuità di I specie (tipo salto) nei punti $x = (2k + 1) \pi, k\in \ZZ $.
\begin{enumerate}
\item $F(x)$ converge puntualmente a
\begin{equation*}
f(x) \ \ \ \ \forall x\neq (2k + 1) \pi, k\in \ZZ
\end{equation*}
\item $F(x)$ converge puntualmente a
\begin{equation*}
\frac{f\left(x^{+}\right) + f\left(x^{-}\right)}{2} = \frac{0 + 2\pi}{2} = \pi \ \ \ \ \forall x = (2k + 1) \pi, k\in \ZZ
\end{equation*}
\end{enumerate}
\item Calcoliamo i coefficienti
\begin{align*}
a_{0} & = \frac{2}{2\pi}\int^{\pi}_{- \pi} f(x) dx = \frac{1}{\pi}\left\{\int^{0}_{- \pi} f(x) dx + \int^{\pi}_{0} f(x) dx\right\}\\
 & = \frac{1}{\pi}\int^{\pi}_{0} f(x) dx = \frac{1}{\pi}\int^{\pi}_{0} 2xdx = \frac{1}{\pi}\left[ x^{2}\right]^{\pi}_{0} = \frac{1}{\pi} \pi^{2} = \textcolor[rgb]{0.82, 0.01, 0.11}{\pi}\\
 & \\
a_{n} & \overset{n \geq 1}{=}\frac{1}{\pi}\int^{\pi}_{- \pi} f(x)\cos(nx) dx = \frac{1}{\pi}\int^{\pi}_{0} 2x\cos(nx) dx\\
 & \\
 &
\begin{array}{l l}
h(x) = 2x & h'(x) = 2\\
g(x) = \frac{1}{n}\sin(nx) & g'(x) = \cos(nx)
\end{array}\\
 & \\
 & \overset{\text{ipp}}{=}\frac{1}{\pi}\left\{\left[\frac{2x}{n}\sin(nx)\right]^{\pi}_{0} - \int^{\pi}_{0}\frac{2}{n}\sin(nx) dx\right\}\\
 & = \frac{1}{\pi}\left\{\cancel{\frac{2\pi}{n}\sin(n\pi)} - 0 + \left[\frac{2\cos(nx)}{n^{2}}\right]^{\pi}_{0}\right\}\\
 & = \frac{1}{\pi}\left\{\frac{2\cos(n\pi)}{n^{2}} - \frac{2}{n^{2}}\right\} = \frac{1}{\pi}\left\{\frac{2(- 1)^{n}}{n^{2}} - \frac{2}{n^{2}}\right\}\\
 & = \textcolor[rgb]{0.82, 0.01, 0.11}{\frac{2}{\pi}}\textcolor[rgb]{0.82, 0.01, 0.11}{\frac{(- 1)^{n} - 1}{n^{2}}}\\
 & \\
b_{n} & = \frac{1}{\pi}\int^{\pi}_{- \pi} f(x)\sin(nx) dx = \frac{1}{\pi}\int^{\pi}_{0} 2x\sin(nx) dx\\
 & \\
 &
\begin{array}{l l}
k(x) = 2x & k'(x) = 2\\
r(x) = - \frac{1}{n}\cos(nx) & r'(x) = \sin(nx)
\end{array}\\
 & \\
 & \overset{\text{ipp}}{=}\frac{1}{\pi}\left\{\left[ - \frac{2x}{n}\cos(nx)\right]^{\pi}_{0} + \int^{\pi}_{0}\frac{2}{n}\cos(nx) dx\right\}\\
 & = \frac{1}{\pi}\left\{- \frac{2\pi}{n}\cos(n\pi) + 0 + \left[\frac{2\sin(nx)}{n^{2}}\right]^{\pi}_{0}\right\}\\
 & = \frac{1}{\pi}\left\{- \frac{2\pi}{n}(- 1)^{n}\right\} = \textcolor[rgb]{0.82, 0.01, 0.11}{\frac{2}{n}}\textcolor[rgb]{0.82, 0.01, 0.11}{(}\textcolor[rgb]{0.82, 0.01, 0.11}{- 1}\textcolor[rgb]{0.82, 0.01, 0.11}{)}\textcolor[rgb]{0.82, 0.01, 0.11}{^{n + 1}}
\end{align*}

La serie di Fourier di $f$ è
\begin{equation*}
F(x) = \frac{\pi}{2} + 2\sum\limits^{\infty}_{n = 1}\left[\frac{(- 1)^{n} - 1}{\pi n^{2}}\cos(nx) + \frac{(- 1)^{n + 1}}{n}\sin(nx)\right]
\end{equation*}
\item Poiché $f\in L^{2}([ - \pi, \pi ])$
\begin{equation*}
\int^{\pi}_{- \pi}| f(x)|^{2} dx = \int^{\pi}_{0}(2x)^{2} dx = \int^{\pi}_{0} 4x^{2} dx = \frac{4}{3} \pi^{3} < + \infty
\end{equation*}

allora $F(x)$ di $f$ converge in media quadratica.
\item Poiché $f$ \underline{\textbf{non}} è continua, allora la convergenza \underline{\textbf{non}} è uniforme su $\RR $.
\end{enumerate}

\Soluzione

\begin{enumerate}
\item $f$ è pari $\implies b_{n} = 0\ \forall n\in \NN $
\begin{equation*}
a_{k} = \frac{2}{\pi}\int^{\frac{\pi}{2}}_{- \frac{\pi}{2}}\cos\left(\sqrt{2} x\right)\cos(2kx) dx\ \ \forall k\in \NN
\end{equation*}

modo $A$: calcolo per parti e ottengo un integrale ciclico.

modo $B$: usare la formula di Werner, usiamo questo
\begin{equation*}
\cos \alpha \cos \beta = \frac{1}{2}[\cos(\alpha + \beta) + \cos(\alpha - \beta)]
\end{equation*}

Procediamo al calcolo
\begin{align*}
a_{k} & = \frac{\cancel{2}}{\pi}\int^{\frac{\pi}{2}}_{- \frac{\pi}{2}}\frac{1}{\cancel{2}}\left\{\cos\left(\sqrt{2} x + 2kx\right) + \cos\left(\sqrt{2} x - 2kx\right)\right\} dx\\
 & = \frac{1}{\pi}\int^{\frac{\pi}{2}}_{- \frac{\pi}{2}}\left\{\cos\left(\left(\sqrt{2} + 2k\right) x\right) + \cos\left(\left(\sqrt{2} - 2k\right) x\right)\right\} dx\\
 & = \frac{1}{\pi}\left\{\left[\frac{\sin\left(\left(\sqrt{2} + 2k\right) x\right)}{\sqrt{2} + 2k}\right]^{\frac{\pi}{2}}_{- \frac{\pi}{2}} + \left[\frac{\sin\left(\left(\sqrt{2} - 2k\right) x\right)}{\sqrt{2} - 2k}\right]^{\frac{\pi}{2}}_{- \frac{\pi}{2}}\right\}\\
 & = \frac{1}{\pi}\left\{\frac{\sin\left(\frac{\pi}{\sqrt{2}} + k\pi \right)}{\sqrt{2} + 2k} + \frac{\sin\left(\frac{\pi}{\sqrt{2}} + k\pi \right)}{\sqrt{2} + 2k} + \left[\frac{\sin\left(\left(\sqrt{2} - 2k\right) x\right)}{\sqrt{2} - 2k}\right]^{\frac{\pi}{2}}_{- \frac{\pi}{2}}\right\}\\
 & = \frac{1}{\pi}\left\{\frac{2\sin\left(\frac{\pi}{\sqrt{2}} + k\pi \right)}{\sqrt{2} + 2k} + \frac{2\sin\left(\frac{\pi}{\sqrt{2}} - k\pi \right)}{\sqrt{2} - 2k}\right\}\\
 & = \frac{2(- 1)^{k}\sin\left(\frac{\pi}{\sqrt{2}}\right)}{\pi}\left(\frac{1}{\sqrt{2} + 2k} + \frac{1}{\sqrt{2} - 2k}\right)\\
 & = \frac{2(- 1)^{k}\sin\left(\frac{\pi}{\sqrt{2}}\right)}{\pi}\left(\frac{1}{2k + \sqrt{2}} - \frac{1}{2k - \sqrt{2}}\right)\\
 & = - \frac{2\sqrt{2}(- 1)^{k}\sin\left(\frac{\pi}{\sqrt{2}}\right)}{\pi \left(2k^{2} - 1\right)}
\end{align*}
\item La serie di Fourier di $f$ è
\begin{equation*}
F(x) = \frac{a_{0}}{2} + \sum\limits^{\infty}_{k = 1} a_{k}\cos(2kx)
\end{equation*}

per $x = \frac{\pi}{2}$
\begin{equation*}
\begin{aligned}
F\left(\frac{\pi}{2}\right) & = \cos\left(\sqrt{2} \cdot \frac{\pi}{2}\right) = \textcolor[rgb]{0.25, 0.46, 0.02}{\cos\left(\frac{\pi}{\sqrt{2}}\right)}\\
F\left(\frac{\pi}{2}\right) & = \frac{\sqrt{2}}{\pi}\sin\left(\frac{\pi}{\sqrt{2}}\right) + \sum\limits^{\infty}_{k = 1} a_{k}(- 1)^{k}\\
 & = \textcolor[rgb]{0.74, 0.06, 0.88}{\frac{\sqrt{2}}{\pi}\sin\left(\frac{\pi}{\sqrt{2}}\right)}\textcolor[rgb]{0.96, 0.65, 0.14}{- \frac{2\sqrt{2}\sin\left(\frac{\pi}{\sqrt{2}}\right)}{\pi}}\sum\limits^{\infty}_{k = 1}\frac{(- 1)^{k}(- 1)^{k}}{\left(2k^{2} - 1\right)}\\
 & \\
\implies \ \ \sum\limits^{\infty}_{k = 1}\frac{1}{\left(2k^{2} - 1\right)} & = \frac{\textcolor[rgb]{0.25, 0.46, 0.02}{\cos\left(\frac{\pi}{\sqrt{2}}\right)} - \textcolor[rgb]{0.74, 0.06, 0.88}{\frac{\sqrt{2}}{\pi}\sin\left(\frac{\pi}{\sqrt{2}}\right)}}{\textcolor[rgb]{0.96, 0.65, 0.14}{- \frac{2\sqrt{2}\sin\left(\frac{\pi}{\sqrt{2}}\right)}{\pi}}} = - \frac{\textcolor[rgb]{0.25, 0.46, 0.02}{\cos\left(\frac{\pi}{\sqrt{2}}\right)}}{\textcolor[rgb]{0.96, 0.65, 0.14}{\frac{2\sqrt{2}\sin\left(\frac{\pi}{\sqrt{2}}\right)}{\pi}}} + \frac{\textcolor[rgb]{0.74, 0.06, 0.88}{\frac{\sqrt{2}}{\pi}\sin\left(\frac{\pi}{\sqrt{2}}\right)}}{\textcolor[rgb]{0.96, 0.65, 0.14}{\frac{2\sqrt{2}\sin\left(\frac{\pi}{\sqrt{2}}\right)}{\pi}}}\\
 & = - \frac{\pi}{2\sqrt{2}}\cot\left(\frac{\pi}{\sqrt{2}}\right) + \frac{1}{2}
\end{aligned}
\end{equation*}
\item $f\in L^{2}\left(\left[ - \frac{\pi}{2}, \frac{\pi}{2}\right]\right) \implies $ ho convergenza in media quadratica

Criterio di Weierstrass, prendo il termine generale in modulo
\begin{gather*}
| a_{k}\cos(2k\pi)| \leq \frac{2\sqrt{2}}{\pi \left(2k^{2} - 1\right)}\\
\sum\limits^{\infty}_{k = 1}\frac{2\sqrt{2}}{\pi \left(2k^{2} - 1\right)} < + \infty \ \ \implies \ \ \text{convergenza è uniforme su} \ \RR
\end{gather*}
\end{enumerate}
\chapter{Esercitazione 6 - Boella}

\ParteEsercizi

\Esercizio{}

Data una funzione $2\pi $ - periodica
\begin{equation*}
f:[ - \pi, \pi ]\rightarrow \RR \ \ \ \ f(x) = x^{2}
\end{equation*}
Determinare la sue serie di Fourier. Si riesce a ricondurre la serie ad una nota?

\Esercizio{}

Data una funzione $3$ - periodica
\begin{equation*}
f(x) =
\begin{cases}
0, & 0 \leq x < 1\\
1, & 1 \leq x < 2\\
2, & 2 \leq x < 3
\end{cases}
\end{equation*}
Determinare la sue serie di Fourier. Si riesce a ricondurre la serie ad una nota?

\Esercizio{}

\begin{defn}
[Spazio delle funzioni test] $\varphi \in D(A)$ se $\varphi \in C^{\infty}(A)$, $\exists K$ compatto tale per cui $\varphi (x) = 0, \forall x\notin K$
\end{defn}
\textit{Esempio.}
\begin{equation*}
\varphi (x) =
\begin{cases}
e^{- \frac{1}{1 - x^{2}}}, & | x| < 1\\
0, & | x| \geq 1
\end{cases}
\end{equation*}
\fg{0.7}{11-3-1}
\begin{defn}
[Distribuzione] Si chiamano distribuzioni gli oggetti $T\in D'(A)$. $T$ è un funzionale lineare continuo su $D(A)$. Quindi so dire cosa fa $T$ a una funzione test $\varphi $ tramite il prodotto di dualità
\begin{equation*}
T(\varphi) = \langle T, \varphi \rangle
\end{equation*}
\end{defn}
\begin{defn}
[Convergenza di funzioni test] Diciamo che $\varphi_{k}\xrightarrow{D(A)} \varphi $ se $\forall \alpha, D^{\alpha} \varphi_{k}\rightarrow D^{\alpha} \varphi $.
\end{defn}
\begin{defn}
[Prodotto di dualità] Data $T\in D'(A)$, se $T\in L^{1}_{\loc}(A) \implies \langle T, \varphi \rangle = \int_{A} T(x) \varphi (x) dx$. Se invece $T\notin L^{1}_{\loc} \implies \langle T, \varphi \rangle $ lo dobbiamo definire manualmente, per esempio la Delta di Dirac che è definita come
\begin{equation*}
\langle \delta, \varphi \rangle = \varphi (0)
\end{equation*}
Possiamo anche traslarla
\begin{equation*}
\langle \delta_{x_{0}}, \varphi \rangle = \varphi (x_{0})
\end{equation*}
\end{defn}
\begin{defn}
[Convergenza di distribuzioni] Diciamo che $T_{n}\xrightarrow{D'(A)} T$ se $\langle T_{n}, \varphi \rangle \rightarrow \langle T, \varphi \rangle $, $\forall \varphi \in D(A)$.
\end{defn}
Studiare la convergenza in distribuzione della seguente successione
\begin{equation*}
f_{n} (x) = ne^{- n|x|}
\end{equation*}

\Esercizio{}

Studiare la convergenza in distribuzione della seguente successione
\begin{equation*}
f_{n} (x) =
\begin{cases}
- 3n, & |x| \leq \frac{1}{n}\\
2n, & \frac{1}{n} < |x| \leq \frac{2}{n}\\
0, & |x| > \frac{2}{n}
\end{cases}
\end{equation*}

\Esercizio{}

Studiare la convergenza in distribuzione della seguente successione
\begin{equation*}
f_{n} (x) = nx^{n} \chi_{(0, 1)} (x)
\end{equation*}

\ParteSoluzioni

\Soluzione

Osserviamo che la funzione $f(x) = x^{2}$ converge in $L^{2}$ e puntualmente su $\RR $.

Osserviamo che $b_{n} = 0, \forall n$ perché la funzione è pari.

Calcoliamo ora $a_{0}$ e $a_{n}$:
\begin{align*}
a_{0} & = \frac{1}{\pi}\int^{\pi}_{- \pi} f(x)dx = \frac{2}{\pi}\int^{\pi}_{0} f(x)dx = \frac{2}{3} \pi^{2}\\
 & \\
a_{n} & = \frac{1}{\pi}\int^{\pi}_{- \pi} f(x)\cos(nx) dx\\
 & = \frac{2}{\pi}\int^{\pi}_{0} f(x)\cos (nx)dx\\
 & = \frac{2}{\pi}\int^{\pi}_{0} x^{2}\cos (nx)dx\\
 & \overset{\text{ipp}}{=}\frac{2}{\pi}\left\{\cancel{\left[ x^{2}\frac{\sin (nx)}{n}\right]^{\pi}_{0}} - \frac{2}{n}\int^{\pi}_{0} x\sin (nx)dx\right\}\\
 & \overset{\text{ipp}}{=} - \frac{4}{n\pi}\left\{\left[ - \frac{x}{n}\cos(nx)\right]^{\pi}_{0} + \cancel{\frac{1}{n}\int^{\pi}_{0}\cos(nx) dx}\right\}\\
 & = \frac{4}{n\pi} \cdot \frac{\pi}{n}\cos(n\pi) = \frac{4}{n^{2}}(- 1)^{n}
\end{align*}
La serie di Fourier cercata è:
\begin{equation*}
f(x) = \frac{\pi^{2}}{3} + \sum^{\infty}_{n = 1} (- 1)^{n}\frac{4}{n^{2}}\cos (nx)
\end{equation*}
Inoltre
\begin{equation*}
\sum^{\infty}_{n = 1}(|a_{n} | + |b_{n} |) < \infty \ \ \implies \ \ \text{c'è anche convergenza uniforme}
\end{equation*}
Ricaviamo alcune serie notevoli.
\begin{itemize}
\item in $x = \pi $
\begin{gather*}
\pi^{2} = f(\pi) = \frac{\pi^{2}}{3} + \sum^{\infty}_{n = 1} (- 1)^{n}\frac{4}{n^{2}}\cos (n\pi) = \frac{\pi^{2}}{3} + 4\sum^{\infty}_{n = 1}\frac{1}{n^{2}}\\
\implies \ \ \boxed{\sum^{\infty}_{n = 1}\frac{1}{n^{2}} = \frac{\pi^{2}}{6}}
\end{gather*}
\item in $x = 0$
\begin{gather*}
0 = f(0) = \frac{\pi^{2}}{3} + \sum^{\infty}_{n = 1} (- 1)^{n}\frac{4}{n^{2}}\cos (0) = \frac{\pi^{2}}{3} + 4\sum^{\infty}_{n = 1}\frac{(- 1)^{n}}{n^{2}}\\
\implies \ \ \boxed{\sum^{\infty}_{n = 1}\frac{(- 1)^{n}}{n^{2}} = - \frac{\pi^{2}}{12}}
\end{gather*}
\item Parseval
\begin{equation*}
\int^{2\pi}_{0}[ f(x)]^{2} dx = \pi \left[\frac{a^{2}_{0}}{2} + \sum^{\infty}_{n = 1} (a^{2}_{n} + b^{2}_{n})\right]
\end{equation*}

Allora
\begin{equation*}
\begin{aligned}
\int^{\pi}_{- \pi}\left(x^{2}\right)^{2} dx & = \pi \left[\frac{\left(\frac{2}{3} \pi^{2}\right)^{2}}{2} + \sum\limits^{\infty}_{n = 1}\frac{16}{n^{4}}\right]\\
2 \cdot \frac{\pi^{5}}{5} & = \pi \left[\frac{2}{9} \pi^{4} + 16\sum\limits^{\infty}_{n = 1}\frac{1}{n^{4}}\right]\\
\left(\frac{2}{5} - \frac{2}{9}\right) \pi^{4} & = 16\sum\limits^{\infty}_{n = 1}\frac{1}{n^{4}} \ \ \implies \ \ \boxed{\sum\limits^{\infty}_{n = 1}\frac{1}{n^{4}} = \frac{\pi^{4}}{90}}
\end{aligned}
\end{equation*}
\end{itemize}

\Soluzione

Iniziamo osservando che la funzione
\begin{equation*}
f(x) =
\begin{cases}
0, & 0 \leq x < 1\\
1, & 1 \leq x < 2\\
2, & 2 \leq x < 3
\end{cases}
\end{equation*}
converge in $L^{2}(0, 3)$. Converge inoltre puntualmente q.o. su $\RR $ (a meno dell'insieme $x\notin \ZZ $).


\begin{figure}[htpb]
	\centering
\tikzset{every picture/.style = {line width = 0.75pt}} %set default line width to 0.75pt

\begin{tikzpicture}[x = 0.75pt, y = 0.75pt, yscale = -1, xscale = 1]
%uncomment if require: \path (0, 214); %set diagram left start at 0, and has height of 214

%Shape: Axis 2D [id:dp9173052200245366]
\draw (120, 150) - - (480, 150)(299.33, 20) - - (299.33, 200) (473, 145) - - (480, 150) - - (473, 155) (294.33, 27) - - (299.33, 20) - - (304.33, 27) ;
%Straight Lines [id:da6842800458327545]
\draw [color = {rgb, 255:red, 208; green, 2; blue, 27}, draw opacity = 1 ][line width = 1.5] (300, 150) - - (338.39, 150) ;
\draw [shift = {(340, 150)}, rotate = 0] [color = {rgb, 255:red, 208; green, 2; blue, 27}, draw opacity = 1 ][line width = 1.5] (0, 0) circle [x radius = 2.61, y radius = 2.61] ;
\draw [shift = {(300, 150)}, rotate = 0] [color = {rgb, 255:red, 208; green, 2; blue, 27}, draw opacity = 1 ][fill = {rgb, 255:red, 208; green, 2; blue, 27}, fill opacity = 1 ][line width = 1.5] (0, 0) circle [x radius = 2.61, y radius = 2.61] ;
%Straight Lines [id:da3441135935225803]
\draw [color = {rgb, 255:red, 208; green, 2; blue, 27}, draw opacity = 1 ][line width = 1.5] (340, 110) - - (378.39, 110) ;
\draw [shift = {(380, 110)}, rotate = 0] [color = {rgb, 255:red, 208; green, 2; blue, 27}, draw opacity = 1 ][line width = 1.5] (0, 0) circle [x radius = 2.61, y radius = 2.61] ;
\draw [shift = {(340, 110)}, rotate = 0] [color = {rgb, 255:red, 208; green, 2; blue, 27}, draw opacity = 1 ][fill = {rgb, 255:red, 208; green, 2; blue, 27}, fill opacity = 1 ][line width = 1.5] (0, 0) circle [x radius = 2.61, y radius = 2.61] ;
%Straight Lines [id:da03802847326309822]
\draw [color = {rgb, 255:red, 208; green, 2; blue, 27}, draw opacity = 1 ][line width = 1.5] (380, 70) - - (418.39, 70) ;
\draw [shift = {(420, 70)}, rotate = 0] [color = {rgb, 255:red, 208; green, 2; blue, 27}, draw opacity = 1 ][line width = 1.5] (0, 0) circle [x radius = 2.61, y radius = 2.61] ;
\draw [shift = {(380, 70)}, rotate = 0] [color = {rgb, 255:red, 208; green, 2; blue, 27}, draw opacity = 1 ][fill = {rgb, 255:red, 208; green, 2; blue, 27}, fill opacity = 1 ][line width = 1.5] (0, 0) circle [x radius = 2.61, y radius = 2.61] ;
%Straight Lines [id:da9565199441833196]
\draw [color = {rgb, 255:red, 208; green, 2; blue, 27}, draw opacity = 1 ][line width = 1.5] (180, 150) - - (218.39, 150) ;
\draw [shift = {(220, 150)}, rotate = 0] [color = {rgb, 255:red, 208; green, 2; blue, 27}, draw opacity = 1 ][line width = 1.5] (0, 0) circle [x radius = 2.61, y radius = 2.61] ;
\draw [shift = {(180, 150)}, rotate = 0] [color = {rgb, 255:red, 208; green, 2; blue, 27}, draw opacity = 1 ][fill = {rgb, 255:red, 208; green, 2; blue, 27}, fill opacity = 1 ][line width = 1.5] (0, 0) circle [x radius = 2.61, y radius = 2.61] ;
%Straight Lines [id:da9765401043379787]
\draw [color = {rgb, 255:red, 208; green, 2; blue, 27}, draw opacity = 1 ][line width = 1.5] (220, 110) - - (258.39, 110) ;
\draw [shift = {(260, 110)}, rotate = 0] [color = {rgb, 255:red, 208; green, 2; blue, 27}, draw opacity = 1 ][line width = 1.5] (0, 0) circle [x radius = 2.61, y radius = 2.61] ;
\draw [shift = {(220, 110)}, rotate = 0] [color = {rgb, 255:red, 208; green, 2; blue, 27}, draw opacity = 1 ][fill = {rgb, 255:red, 208; green, 2; blue, 27}, fill opacity = 1 ][line width = 1.5] (0, 0) circle [x radius = 2.61, y radius = 2.61] ;
%Straight Lines [id:da7209581982756688]
\draw [color = {rgb, 255:red, 208; green, 2; blue, 27}, draw opacity = 1 ][line width = 1.5] (260, 70) - - (298.39, 70) ;
\draw [shift = {(300, 70)}, rotate = 0] [color = {rgb, 255:red, 208; green, 2; blue, 27}, draw opacity = 1 ][line width = 1.5] (0, 0) circle [x radius = 2.61, y radius = 2.61] ;
\draw [shift = {(260, 70)}, rotate = 0] [color = {rgb, 255:red, 208; green, 2; blue, 27}, draw opacity = 1 ][fill = {rgb, 255:red, 208; green, 2; blue, 27}, fill opacity = 1 ][line width = 1.5] (0, 0) circle [x radius = 2.61, y radius = 2.61] ;
%Straight Lines [id:da15781035361747886]
\draw [color = {rgb, 255:red, 208; green, 2; blue, 27}, draw opacity = 1 ][line width = 1.5] (140, 70) - - (178.39, 70) ;
\draw [shift = {(180, 70)}, rotate = 0] [color = {rgb, 255:red, 208; green, 2; blue, 27}, draw opacity = 1 ][line width = 1.5] (0, 0) circle [x radius = 2.61, y radius = 2.61] ;
\draw [shift = {(140, 70)}, rotate = 0] [color = {rgb, 255:red, 208; green, 2; blue, 27}, draw opacity = 1 ][fill = {rgb, 255:red, 208; green, 2; blue, 27}, fill opacity = 1 ][line width = 1.5] (0, 0) circle [x radius = 2.61, y radius = 2.61] ;
%Straight Lines [id:da6022039956275556]
\draw [color = {rgb, 255:red, 208; green, 2; blue, 27}, draw opacity = 1 ][line width = 1.5] (420, 150) - - (458.39, 150) ;
\draw [shift = {(460, 150)}, rotate = 0] [color = {rgb, 255:red, 208; green, 2; blue, 27}, draw opacity = 1 ][line width = 1.5] (0, 0) circle [x radius = 2.61, y radius = 2.61] ;
\draw [shift = {(420, 150)}, rotate = 0] [color = {rgb, 255:red, 208; green, 2; blue, 27}, draw opacity = 1 ][fill = {rgb, 255:red, 208; green, 2; blue, 27}, fill opacity = 1 ][line width = 1.5] (0, 0) circle [x radius = 2.61, y radius = 2.61] ;

% Text Node
\draw (335, 158.9) node [anchor = north west][inner sep = 0.75pt] {$1$};
% Text Node
\draw (377, 158.9) node [anchor = north west][inner sep = 0.75pt] {$2$};
% Text Node
\draw (417, 158.9) node [anchor = north west][inner sep = 0.75pt] {$3$};
% Text Node
\draw (307, 102.4) node [anchor = north west][inner sep = 0.75pt] {$1$};
% Text Node
\draw (307, 62.4) node [anchor = north west][inner sep = 0.75pt] {$2$};
% Text Node
\draw (287, 152.4) node [anchor = north west][inner sep = 0.75pt] {$0$};


\end{tikzpicture}
\end{figure}
\FloatBarrier

\begin{equation*}
f(x) \sim \frac{a_{0}}{2} + \sum\limits^{\infty}_{n = 1}\left[ a_{n}\cos\left(\frac{2n\pi}{3} x\right) + b_{n}\sin\left(\frac{2n\pi}{3} x\right)\right]
\end{equation*}
Calcoliamo ora i coefficienti:
\begin{align*}
a_{0} & = \frac{1}{3/2}\int^{3}_{0} f(x)dx = \frac{2}{3} \cdot 3 = 2\\
 & \\
a_{n} & = \frac{1}{3/2}\int^{3}_{0} f(x)\cos\left(\frac{2n\pi}{3}\right) dx = \frac{2}{3}\left\{\int^{2}_{1} 1\cos\left(\frac{2n\pi}{3}\right) dx + \int^{3}_{2} 2\cos\left(\frac{2n\pi}{3}\right) dx\right\}\\
 & = \frac{2}{3}\left\{\left[\frac{3}{2n\pi}\sin\left(\frac{2n\pi}{3} x\right)\right]^{2}_{1} + 2\left[\frac{3}{2n\pi}\sin\left(\frac{2n\pi}{3} x\right)\right]^{3}_{2}\right\}\\
 & = \frac{2}{3}\frac{3}{2n\pi}\left\{\sin\left(\frac{4n\pi}{3}\right) - \sin\left(\frac{2n\pi}{3}\right) + \cancel{2\sin(2n\pi)} - 2\sin\left(\frac{4n\pi}{3}\right)\right\} = 0, \ \forall n\\
 & \\
b_{n} & = \frac{2}{3}\int^{3}_{0} f(x)\sin\left(\frac{2n\pi}{3}\right) dx = \frac{2}{3}\left\{\int^{2}_{1} 1\sin\left(\frac{2n\pi}{3}\right) dx + \int^{3}_{2} 2\sin\left(\frac{2n\pi}{3}\right) dx\right\}\\
 & = \frac{2}{3}\frac{3}{2n\pi}\left\{\left[ - \cos\left(\frac{2n\pi}{3}\right)\right]^{2}_{1} - 2\left[\cos\left(\frac{2n\pi}{3}\right)\right]^{3}_{2}\right\}\\
 & = - \frac{1}{n\pi}\left\{\cos\left(\frac{4n\pi}{3}\right) - \cos\left(\frac{2n\pi}{3}\right) + 2\cos(2n\pi) - 2\cos\left(\frac{4n\pi}{3}\right)\right\}\\
 & = - \frac{1}{n\pi} \cdot
\begin{cases}
1, & n = 1, 4, 7, \dotsc \\
1, & n = 2, 5, 8, \dotsc \\
0, & \text{altrimenti}
\end{cases} =
\begin{cases}
0, & n = 3k, k\in \ZZ \\
- \frac{1}{n\pi}, & \text{altrimenti}
\end{cases}
\end{align*}
La serie cercata è:
\begin{equation*}
f(x) = 1 + \sum^{\infty}_{n = 1} b_{n}\sin\left(\frac{2n\pi}{3} x\right)
\end{equation*}
La funzione è una funzione dispari traslata di $1$ verso l'alto.

\Soluzione

Limite puntuale
\begin{equation*}
f_{n} (x) = ne^{- n|x|} \ \ \ \ \lim\limits_{n\rightarrow + \infty} f_{n}(x) =
\begin{cases}
0, & x\neq 0\\
+ \infty, & x = 0
\end{cases} \ \ \implies \ \ f_{n}\xrightarrow{\text{q.o.}} F(x) \equiv 0
\end{equation*}
\fg{0.7}{11-3}

Notiamo che non c'è convergenza in $L^{1}$:
\begin{equation*}
\Vert f_{n} - F\Vert_{L^{1}(\RR)} = 2\int^{\infty}_{0} ne^{- nx} dx = 2\left[ - e^{- nx}\right]^{\infty}_{0} = 2\nrightarrow 0
\end{equation*}
Tuttavia $f_{n} \in L^{1}_{\loc}(\RR)$, quindi sono anche associate a una distribuzione.
\begin{equation*}
\begin{aligned}
\langle f_{n}, \varphi \rangle & = \int_{\RR} ne^{- n| x|} \varphi (x) dx = \int^{0}_{- \infty} ne^{nx} \varphi (x) dx + \int^{\infty}_{0} ne^{- nx} \varphi (x) dx\\
 & \overset{\text{ipp}}{=}\left[ e^{nx} \varphi (x)\right]^{0}_{- \infty} - \int^{0}_{- \infty} e^{nx} \varphi'(x) dx + \left[ - e^{- nx}\right]^{\infty}_{0} + \int^{\infty}_{0} e^{- nx} \varphi'(x) dx\\
 & = \varphi (0) - \underbrace{\int^{0}_{- \infty} e^{nx} \varphi'(x) dx}_{\xrightarrow{\text{Dom}} 0} + \varphi (0) + \underbrace{\int^{\infty}_{0} e^{- nx} \varphi'(x) dx}_{\xrightarrow{\text{Dom}} 0}\rightarrow 2\varphi (0)
\end{aligned}
\end{equation*}
Prendiamo $g_{n}(x) = e^{- nx} \varphi'(x)$ su $(0, + \infty)$, $g_{n}(x)\rightarrow 0$ puntualmente, inoltre
\begin{equation*}
| \varphi'(x)| \leq K\ \ \implies \ \ | g_{n}(x)| \leq Ke^{- x} \in L^{1}(0, + \infty)
\end{equation*}
Quindi
\begin{equation*}
\langle f_{n}, \varphi \rangle \rightarrow 2\varphi (0) = 2\langle \delta, \varphi \rangle \ \ \implies \ \ f_{n}\xrightarrow{D'(\RR)} 2\delta
\end{equation*}

\Soluzione

Limite puntuale
\begin{equation*}
\lim\limits_{n\rightarrow + \infty} f_{n}(x) =
\begin{cases}
- \infty, & x = 0\\
0, & x\neq 0
\end{cases} \ \ \implies \ \ f_{n}\xrightarrow{\text{q.o.}} F(x) \equiv 0
\end{equation*}
\fg{0.7}{11-4}

In $D'(\RR)$
\begin{equation*}
\begin{aligned}
\langle f_{n}, \varphi \rangle & = \int^{- 1/n}_{- 2/n} 2n\varphi (x) dx + \int^{1/n}_{- 1/n}(- 3n) \varphi (x) dx + \int^{2/n}_{1/n} 2n\varphi (x) dx\\
 & = 2n\int^{- 1/n}_{- 2/n} \varphi (x) dx - 3n\int^{1/n}_{- 1/n} \varphi (x) dx + 2n\int^{2/n}_{1/n} \varphi (x) dx
\end{aligned}
\end{equation*}
Per le funzioni continue, e le $\varphi $ lo sono, vale il teorema della media
\begin{equation*}
\int^{b}_{a} \varphi (x) dx = (b - a) \varphi (c)
\end{equation*}
Allora
\begin{equation*}
\begin{aligned}
\langle f_{n}, \varphi \rangle & = 2n\int^{- 1/n}_{- 2/n} \varphi (x) dx - 3n\int^{1/n}_{- 1/n} \varphi (x) dx + 2n\int^{2/n}_{1/n} \varphi (x) dx\\
 & = 2n\left(\frac{- 1}{n} - \frac{- 2}{n}\right) \varphi (\alpha_{n}) - 3n\left(\frac{1}{n} - \frac{- 1}{n}\right) \varphi (\beta_{n}) + 2n\left(\frac{2}{n} - \frac{1}{n}\right) \varphi (\gamma_{n})
\end{aligned}
\end{equation*}
dove
\begin{equation*}
- \frac{2}{n} \leq \alpha_{n} \leq - \frac{1}{n} \leq \beta_{n} \leq \frac{1}{n} \leq \gamma_{n} \leq \frac{2}{n}
\end{equation*}
procediamo coi calcoli
\begin{equation*}
\begin{aligned}
\langle f_{n}, \varphi \rangle & = 2n\left(\frac{- 1}{n} - \frac{- 2}{n}\right) \varphi (\alpha_{n}) - 3n\left(\frac{1}{n} - \frac{- 1}{n}\right) \varphi (\beta_{n}) + 2n\left(\frac{2}{n} - \frac{1}{n}\right) \varphi (\gamma_{n})\\
 & = 2\varphi (\alpha_{n}) - 6\varphi (\beta_{n}) + 2\varphi (\gamma_{n})\xrightarrow{n\rightarrow + \infty} 2\varphi (0) - 6\varphi (0) + 2\varphi (0) = - 2\varphi (0)
\end{aligned}
\end{equation*}
Facendo lo stesso gioco su una funzione
\begin{equation*}
g_{n} (x) =
\begin{cases}
- n, & |x| \leq \frac{1}{n}\\
2n, & \frac{1}{n} < |x| \leq \frac{2}{n}\\
0, & |x| > \frac{2}{n}
\end{cases} \ \ \implies \ \ G(x) \equiv 0
\end{equation*}
viene
\begin{equation*}
\langle g_{n}, \varphi \rangle = 2\varphi (\alpha_{n}) - 2\varphi (\beta_{n}) + 2\varphi (\gamma_{n})\rightarrow 2\varphi (0)
\end{equation*}
Ovvero
\begin{equation*}
f_{n}\xrightarrow{D'(\RR)} - 2\delta \ \ \ \ g_{n}\xrightarrow{D'(\RR)} 2\delta
\end{equation*}

\Soluzione

Limite puntuale
\begin{equation*}
f_{n} (x) = nx^{n} \chi_{(0, 1)} (x)\ \ \ \ \lim\limits_{n\rightarrow + \infty} f_{n}(x) = 0, \ \forall x\in \RR
\end{equation*}
\fg{0.3}{11-5}
\begin{equation*}
\begin{aligned}
\langle f_{n}, \varphi \rangle & = \int^{1}_{0} nx^{n} \varphi (x) dx\overset{\text{ipp}}{=}\left[\frac{n}{n + 1} x^{n + 1} \varphi (x)\right]^{1}_{0} - \int^{1}_{0}\frac{n}{n + 1} x^{n + 1} \varphi'(x) dx\\
 & = \frac{n}{n + 1} \varphi (1) - \frac{n}{n + 1}\underbrace{\int^{1}_{0} x^{n + 1} \varphi'(x) dx}_{\xrightarrow{\text{Dom}} 0}\xrightarrow{n\rightarrow + \infty} \varphi (1)
\end{aligned}
\end{equation*}
Dato che $g_{n}(x) = x^{n + 1} \varphi'(x)\xrightarrow{n\rightarrow + \infty} 0, \ \forall x\in (0, 1)$
\begin{equation*}
| g_{n}(x)| \leq \varphi'(x) \in L^{1}
\end{equation*}
In conclusione
\begin{equation*}
\langle f_{n}, \varphi \rangle \rightarrow \langle \delta_{1}, \varphi \rangle \ \ \implies \ \ f_{n}\xrightarrow{D'(\RR)} \delta_{1}
\end{equation*}
\textit{Esercizio per casa.}

Date $f_{n}(x) = nx^{n} \chi_{(0, 1)}(x)$, calcolare il limite in $D'(0, 1)$.
\chapter{Esercitazione 6 - Potrich}

\ParteEsercizi

\Esercizio{}

Sia $f:\RR \rightarrow \RR $ la funzione $2\pi $ - periodica dispari definita da
\begin{equation*}
f(x) =
\begin{cases}
\sin x, & x\in \left[ 0, \frac{\pi}{2}\right)\\
0, & x\in \left[\frac{\pi}{2}, \pi \right]
\end{cases}
\end{equation*}
\begin{enumerate}
\item Disegnare il grafico di $f$ su $[ - 3\pi, 3\pi ]$
\item Stabilire se la serie di Fourier di $f$ converge in media quadratica.
\item Stabilire se la serie di Fourier converge puntualmente.
\item Stabilire se la serie di Fourier converge uniformemente.
\item Scrivere la serie di Fourier di $f$.
\item Dire che cosa accade per $x = \frac{\pi}{4}$ e calcolare
\begin{equation*}
\sum\limits^{\infty}_{n = 0}(- 1)^{n}\frac{2n + 1}{4(2n + 1)^{2} - 1}
\end{equation*}
\item Scrivere l'identità di Parseval per $f$ e sfruttarla per calcolare
\begin{equation*}
\sum\limits^{\infty}_{n = 1}\frac{n^{2}}{\left(4n^{2} - 1\right)^{2}}
\end{equation*}
\end{enumerate}

\Esercizio{}

Sia $Q = [ - \pi, \pi)$ e $F(x)$ lo sviluppo in serie di Fourier di $f(x) = | x|, \ \forall x\in Q$. Calcolare
\begin{equation*}
\sum\limits^{\infty}_{n = 0}\frac{1}{(2n + 1)^{4}} \ \ \ \ \sum\limits^{\infty}_{n = 1}\frac{1}{n^{4}}
\end{equation*}

\Esercizio{}

Sulla falsa riga dell'esercizio precedente, sviluppare in serie di Fourier
\begin{equation*}
f(x) = x(\pi - | x|), \ \ x\in [ - \pi, \pi)
\end{equation*}
estesa con $2\pi $ - periodicità all'asse reale. Calcolare
\begin{equation*}
\sum\limits^{\infty}_{n = 0}\frac{1}{(2n + 1)^{6}} \ \ \ \ \left(= \frac{\pi^{6}}{960}\right) \ \ \ \ \ \ \ \ \ \ \ \ \sum\limits^{\infty}_{n = 1}\frac{1}{n^{6}} \ \ \ \ \left(= \frac{\pi^{6}}{945}\right)
\end{equation*}

\Esercizio{}

Data la successione di funzioni
\begin{equation*}
f_{n}(x) =
\begin{cases}
ne^{n(x - 1)}, & x \leq 1\\
0, & x > 1
\end{cases}
\end{equation*}
\begin{enumerate}
\item Determinare il limite puntuale
\item Determinare il limite nel senso delle distribuzioni
\end{enumerate}

\Esercizio{}

Sia $\alpha > 0$ un valore fissato, si consideri la successione di funzioni
\begin{equation*}
f_{n, \alpha}(x) = n^{\alpha}\sqrt{1 - n| x|} \cdot \chi_{\left[ - \frac{1}{n}, \frac{1}{n}\right]}(x)
\end{equation*}
dove
\begin{equation*}
\chi_{\left[ - \frac{1}{n}, \frac{1}{n}\right]}(x) =
\begin{cases}
1, & x\in \left[ - \frac{1}{n}, \frac{1}{n}\right]\\
0, & \text{altrove}
\end{cases}
\end{equation*}
\begin{enumerate}
\item Determinare il limite puntuale $f(x)$.
\item Stabilire al variare di $\alpha $ se $f_{n, \alpha}\rightarrow f$ in $L^{p}(\RR)$.
\item Se $\alpha = 1$, calcolare $\lim\limits_{n\rightarrow + \infty} f_{n, 1}$ in $D'(\RR)$.
\end{enumerate}

\ParteSoluzioni

\Soluzione

\begin{enumerate}
\item Grafico

\begin{figure}[htpb]
	\centering
\tikzset{every picture/.style = {line width = 0.75pt}} %set default line width to 0.75pt

\begin{tikzpicture}[x = 0.75pt, y = 0.75pt, yscale = -1, xscale = 1]
%uncomment if require: \path (0, 126); %set diagram left start at 0, and has height of 126

%Straight Lines [id:da10741259255713742]
\draw [dash pattern = {on 0.84pt off 2.51pt}] (526.5, 59.75) - - (153, 59.75) ;
%Shape: Axis 2D [id:dp7770976156065474]
\draw (37, 79.83) - - (564, 79.83)(300.5, 13.5) - - (300.5, 115.53) (557, 74.83) - - (564, 79.83) - - (557, 84.83) (295.5, 20.5) - - (300.5, 13.5) - - (305.5, 20.5) ;
%Straight Lines [id:da9173549601591173]
\draw [color = {rgb, 255:red, 208; green, 2; blue, 27}, draw opacity = 1 ][line width = 1.5] (380.92, 80.67) - - (340.71, 80.67) ;
\draw [shift = {(340.71, 80.67)}, rotate = 180] [color = {rgb, 255:red, 208; green, 2; blue, 27}, draw opacity = 1 ][fill = {rgb, 255:red, 208; green, 2; blue, 27}, fill opacity = 1 ][line width = 1.5] (0, 0) circle [x radius = 2.61, y radius = 2.61] ;
\draw [shift = {(380.92, 80.67)}, rotate = 180] [color = {rgb, 255:red, 208; green, 2; blue, 27}, draw opacity = 1 ][fill = {rgb, 255:red, 208; green, 2; blue, 27}, fill opacity = 1 ][line width = 1.5] (0, 0) circle [x radius = 2.61, y radius = 2.61] ;
%Curve Lines [id:da4725268586811755]
\draw [color = {rgb, 255:red, 208; green, 2; blue, 27}, draw opacity = 1 ][line width = 1.5] (300.5, 80.97) .. controls (308.75, 70.09) and (323.11, 60.82) .. (338.56, 59.82) ;
\draw [shift = {(340, 59.75)}, rotate = 358.21] [color = {rgb, 255:red, 208; green, 2; blue, 27}, draw opacity = 1 ][line width = 1.5] (0, 0) circle [x radius = 2.61, y radius = 2.61] ;
\draw [shift = {(300.5, 80.97)}, rotate = 307.15] [color = {rgb, 255:red, 208; green, 2; blue, 27}, draw opacity = 1 ][fill = {rgb, 255:red, 208; green, 2; blue, 27}, fill opacity = 1 ][line width = 1.5] (0, 0) circle [x radius = 2.61, y radius = 2.61] ;
%Straight Lines [id:da4808528317246443]
\draw [color = {rgb, 255:red, 208; green, 2; blue, 27}, draw opacity = 1 ][line width = 1.5] (260.29, 80.67) - - (220.08, 80.67) ;
\draw [shift = {(220.08, 80.67)}, rotate = 180] [color = {rgb, 255:red, 208; green, 2; blue, 27}, draw opacity = 1 ][fill = {rgb, 255:red, 208; green, 2; blue, 27}, fill opacity = 1 ][line width = 1.5] (0, 0) circle [x radius = 2.61, y radius = 2.61] ;
\draw [shift = {(260.29, 80.67)}, rotate = 180] [color = {rgb, 255:red, 208; green, 2; blue, 27}, draw opacity = 1 ][fill = {rgb, 255:red, 208; green, 2; blue, 27}, fill opacity = 1 ][line width = 1.5] (0, 0) circle [x radius = 2.61, y radius = 2.61] ;
%Curve Lines [id:da8687455493649499]
\draw [color = {rgb, 255:red, 208; green, 2; blue, 27}, draw opacity = 1 ][line width = 1.5] (300.5, 80.97) .. controls (292.26, 91.85) and (277.89, 101.11) .. (262.44, 102.12) ;
\draw [shift = {(261, 102.18)}, rotate = 178.21] [color = {rgb, 255:red, 208; green, 2; blue, 27}, draw opacity = 1 ][line width = 1.5] (0, 0) circle [x radius = 2.61, y radius = 2.61] ;
\draw [shift = {(300.5, 80.97)}, rotate = 127.15] [color = {rgb, 255:red, 208; green, 2; blue, 27}, draw opacity = 1 ][fill = {rgb, 255:red, 208; green, 2; blue, 27}, fill opacity = 1 ][line width = 1.5] (0, 0) circle [x radius = 2.61, y radius = 2.61] ;

%Straight Lines [id:da599685016533257]
\draw [color = {rgb, 255:red, 208; green, 2; blue, 27}, draw opacity = 1 ][line width = 1.5] (540.92, 80.67) - - (500.71, 80.67) ;
\draw [shift = {(500.71, 80.67)}, rotate = 180] [color = {rgb, 255:red, 208; green, 2; blue, 27}, draw opacity = 1 ][fill = {rgb, 255:red, 208; green, 2; blue, 27}, fill opacity = 1 ][line width = 1.5] (0, 0) circle [x radius = 2.61, y radius = 2.61] ;
\draw [shift = {(540.92, 80.67)}, rotate = 180] [color = {rgb, 255:red, 208; green, 2; blue, 27}, draw opacity = 1 ][fill = {rgb, 255:red, 208; green, 2; blue, 27}, fill opacity = 1 ][line width = 1.5] (0, 0) circle [x radius = 2.61, y radius = 2.61] ;
%Curve Lines [id:da4402562566796744]
\draw [color = {rgb, 255:red, 208; green, 2; blue, 27}, draw opacity = 1 ][line width = 1.5] (460.5, 80.97) .. controls (468.75, 70.09) and (483.11, 60.82) .. (498.56, 59.82) ;
\draw [shift = {(500, 59.75)}, rotate = 358.21] [color = {rgb, 255:red, 208; green, 2; blue, 27}, draw opacity = 1 ][line width = 1.5] (0, 0) circle [x radius = 2.61, y radius = 2.61] ;
\draw [shift = {(460.5, 80.97)}, rotate = 307.15] [color = {rgb, 255:red, 208; green, 2; blue, 27}, draw opacity = 1 ][fill = {rgb, 255:red, 208; green, 2; blue, 27}, fill opacity = 1 ][line width = 1.5] (0, 0) circle [x radius = 2.61, y radius = 2.61] ;
%Straight Lines [id:da08510756351773896]
\draw [color = {rgb, 255:red, 208; green, 2; blue, 27}, draw opacity = 1 ][line width = 1.5] (420.29, 80.67) - - (380.08, 80.67) ;
\draw [shift = {(380.08, 80.67)}, rotate = 180] [color = {rgb, 255:red, 208; green, 2; blue, 27}, draw opacity = 1 ][fill = {rgb, 255:red, 208; green, 2; blue, 27}, fill opacity = 1 ][line width = 1.5] (0, 0) circle [x radius = 2.61, y radius = 2.61] ;
\draw [shift = {(420.29, 80.67)}, rotate = 180] [color = {rgb, 255:red, 208; green, 2; blue, 27}, draw opacity = 1 ][fill = {rgb, 255:red, 208; green, 2; blue, 27}, fill opacity = 1 ][line width = 1.5] (0, 0) circle [x radius = 2.61, y radius = 2.61] ;
%Curve Lines [id:da9678260771297018]
\draw [color = {rgb, 255:red, 208; green, 2; blue, 27}, draw opacity = 1 ][line width = 1.5] (460.5, 80.97) .. controls (452.26, 91.85) and (437.89, 101.11) .. (422.44, 102.12) ;
\draw [shift = {(421, 102.18)}, rotate = 178.21] [color = {rgb, 255:red, 208; green, 2; blue, 27}, draw opacity = 1 ][line width = 1.5] (0, 0) circle [x radius = 2.61, y radius = 2.61] ;
\draw [shift = {(460.5, 80.97)}, rotate = 127.15] [color = {rgb, 255:red, 208; green, 2; blue, 27}, draw opacity = 1 ][fill = {rgb, 255:red, 208; green, 2; blue, 27}, fill opacity = 1 ][line width = 1.5] (0, 0) circle [x radius = 2.61, y radius = 2.61] ;

%Straight Lines [id:da6685458363363093]
\draw [color = {rgb, 255:red, 208; green, 2; blue, 27}, draw opacity = 1 ][line width = 1.5] (220.92, 80.67) - - (180.71, 80.67) ;
\draw [shift = {(180.71, 80.67)}, rotate = 180] [color = {rgb, 255:red, 208; green, 2; blue, 27}, draw opacity = 1 ][fill = {rgb, 255:red, 208; green, 2; blue, 27}, fill opacity = 1 ][line width = 1.5] (0, 0) circle [x radius = 2.61, y radius = 2.61] ;
\draw [shift = {(220.92, 80.67)}, rotate = 180] [color = {rgb, 255:red, 208; green, 2; blue, 27}, draw opacity = 1 ][fill = {rgb, 255:red, 208; green, 2; blue, 27}, fill opacity = 1 ][line width = 1.5] (0, 0) circle [x radius = 2.61, y radius = 2.61] ;
%Curve Lines [id:da31702343634185315]
\draw [color = {rgb, 255:red, 208; green, 2; blue, 27}, draw opacity = 1 ][line width = 1.5] (140.5, 80.97) .. controls (148.75, 70.09) and (163.11, 60.82) .. (178.56, 59.82) ;
\draw [shift = {(180, 59.75)}, rotate = 358.21] [color = {rgb, 255:red, 208; green, 2; blue, 27}, draw opacity = 1 ][line width = 1.5] (0, 0) circle [x radius = 2.61, y radius = 2.61] ;
\draw [shift = {(140.5, 80.97)}, rotate = 307.15] [color = {rgb, 255:red, 208; green, 2; blue, 27}, draw opacity = 1 ][fill = {rgb, 255:red, 208; green, 2; blue, 27}, fill opacity = 1 ][line width = 1.5] (0, 0) circle [x radius = 2.61, y radius = 2.61] ;
%Straight Lines [id:da9353356461505657]
\draw [color = {rgb, 255:red, 208; green, 2; blue, 27}, draw opacity = 1 ][line width = 1.5] (100.29, 80.67) - - (60.08, 80.67) ;
\draw [shift = {(60.08, 80.67)}, rotate = 180] [color = {rgb, 255:red, 208; green, 2; blue, 27}, draw opacity = 1 ][fill = {rgb, 255:red, 208; green, 2; blue, 27}, fill opacity = 1 ][line width = 1.5] (0, 0) circle [x radius = 2.61, y radius = 2.61] ;
\draw [shift = {(100.29, 80.67)}, rotate = 180] [color = {rgb, 255:red, 208; green, 2; blue, 27}, draw opacity = 1 ][fill = {rgb, 255:red, 208; green, 2; blue, 27}, fill opacity = 1 ][line width = 1.5] (0, 0) circle [x radius = 2.61, y radius = 2.61] ;
%Curve Lines [id:da990279944542586]
\draw [color = {rgb, 255:red, 208; green, 2; blue, 27}, draw opacity = 1 ][line width = 1.5] (140.5, 80.97) .. controls (132.26, 91.85) and (117.89, 101.11) .. (102.44, 102.12) ;
\draw [shift = {(101, 102.18)}, rotate = 178.21] [color = {rgb, 255:red, 208; green, 2; blue, 27}, draw opacity = 1 ][line width = 1.5] (0, 0) circle [x radius = 2.61, y radius = 2.61] ;
\draw [shift = {(140.5, 80.97)}, rotate = 127.15] [color = {rgb, 255:red, 208; green, 2; blue, 27}, draw opacity = 1 ][fill = {rgb, 255:red, 208; green, 2; blue, 27}, fill opacity = 1 ][line width = 1.5] (0, 0) circle [x radius = 2.61, y radius = 2.61] ;


% Text Node
\draw (41.5, 92.13) node [anchor = north west][inner sep = 0.75pt] [font = \normalsize] {$ - 3\pi $};
% Text Node
\draw (126.07, 92.13) node [anchor = north west][inner sep = 0.75pt] [font = \normalsize] {$ - 2\pi $};
% Text Node
\draw (203.6, 92.13) node [anchor = north west][inner sep = 0.75pt] [font = \normalsize] {$ - \pi $};
% Text Node
\draw (374.04, 92.13) node [anchor = north west][inner sep = 0.75pt] [font = \normalsize] {$\pi $};
% Text Node
\draw (454.1, 92.13) node [anchor = north west][inner sep = 0.75pt] [font = \normalsize] {$2\pi $};
% Text Node
\draw (532.3, 92.13) node [anchor = north west][inner sep = 0.75pt] [font = \normalsize] {$3\pi $};
% Text Node
\draw (289.25, 39.76) node [anchor = north west][inner sep = 0.75pt] [font = \normalsize] {$1$};


\end{tikzpicture}
\end{figure}
\FloatBarrier

\item Poiché

\begin{equation*}
\int^{\pi}_{- \pi} f^{2}(x) dx < + \infty \ \ \implies \ \ f\in L^{2}([ - \pi, \pi ])
\end{equation*}

allora la serie di Fourier converge in media quadratica.
\item $f$ è una funzione regolare a tratti in $[ - \pi, \pi ]$, allora converge puntualmente $\forall x\in \RR $.

Più precisamente converge puntualmente a $f(x), \ \forall x\neq (2k + 1)\frac{\pi}{2}, \ k\in \ZZ $, mentre converge a $\frac{1}{2}, \ \forall x = (2k + 1)\frac{\pi}{2}, \ k\in \ZZ $.
\item Non c'è convergenza uniforme perché $f$ non è continua su $\RR $.
\item $f$ è dispari, allora $a_{n} = 0, \ \forall n\in \NN $.

\begin{equation*}
\begin{aligned}
b_{n} & = \frac{2}{\pi}\int^{\pi}_{0} f(x)\sin(nx) dx\\
 & = \frac{2}{\pi}\int^{\frac{\pi}{2}}_{0} f(x)\sin(nx) dx = \frac{2}{\pi}\int^{\frac{\pi}{2}}_{0}\sin x\sin(nx) dx
\end{aligned}
\end{equation*}

Possiamo fare un integrale ciclico o usare le formule di Werner. Usiamo il primo
\begin{align*}
\frac{2}{\pi}\int^{\frac{\pi}{2}}_{0}\sin x\sin(nx) dx & \overset{\text{ipp}}{=}\frac{2}{\pi}\left\{\cancel{[ - \cos x\sin(nx)]^{\frac{\pi}{2}}_{0}} + n\int^{\frac{\pi}{2}}_{0}\cos x\cos(nx) dx\right\}\\
 & = \frac{2n}{\pi}\int^{\frac{\pi}{2}}_{0}\cos x\cos(nx) dx\\
 & \overset{\text{ipp}}{=}\frac{2n}{\pi}\left\{[\sin x\cos(nx)]^{\frac{\pi}{2}}_{0} + n\int^{\frac{\pi}{2}}_{0}\sin x\sin(nx) dx\right\}\\
 & = \frac{2n}{\pi}\cos\left(n\frac{\pi}{2}\right) + \frac{2n^{2}}{\pi}\int^{\frac{\pi}{2}}_{0}\sin x\sin(nx) dx\\
 & \\
\implies \ \ b_{n} & = \frac{2n}{\pi}\cos\left(n\frac{\pi}{2}\right) + n^{2} b_{n}\\
\left(1 - n^{2}\right) b_{n} & = \frac{2n}{\pi}\cos\left(n\frac{\pi}{2}\right)
\end{align*}

per $n\neq 1$
\begin{equation*}
b_{n} = \frac{2}{\pi}\frac{n}{1 - n^{2}}\cos\left(n\frac{\pi}{2}\right) =
\begin{cases}
(- 1)^{\frac{n}{2}}\frac{2}{\pi}\frac{n}{1 - n^{2}}, & n\ \text{pari}\\
0, & n\ \text{dispari}
\end{cases}
\end{equation*}

per $n = 1$
\begin{equation*}
\begin{aligned}
b_{1} & = \frac{2}{\pi}\int^{\pi}_{0} f(x)\sin(x) dx = \frac{2}{\pi}\int^{\frac{\pi}{2}}_{0}\sin^{2}(x) dx = \frac{2}{\pi}\int^{\frac{\pi}{2}}_{0}\frac{1 - \cos 2x}{2} dx\\
 & = \frac{1}{\pi}\left[ x - \frac{1}{2}\sin(2x)\right]^{\frac{\pi}{2}}_{0} = \frac{1}{\pi}\left[\frac{\pi}{2} - 0 + 0 - 0\right] = \frac{1}{2}
\end{aligned}
\end{equation*}

Allora $(n = 2k, \ k\in \NN)$:

\begin{equation*}
F(x) = \frac{1}{2}\sin x + \frac{2}{\pi}\sum^{+ \infty}_{k = 1}(- 1)^{k}\frac{2k}{1 - 4k^{2}}\sin(2kx)
\end{equation*}
\item In $x = \frac{\pi}{4}$ la $f$ è continua, quindi la $F(x)$ converge puntualmente a $f(x)$.
\begin{align*}
F\left(\frac{\pi}{4}\right) & = \sin\left(\frac{\pi}{4}\right) = \frac{1}{\sqrt{2}}\\
F\left(\frac{\pi}{4}\right) & = \frac{1}{2}\sin\left(\frac{\pi}{4}\right) + \frac{4}{\pi}\sum^{+ \infty}_{k = 1}(- 1)^{k - 1}\frac{k}{4k^{2} - 1}\sin\left(k\frac{\pi}{2}\right)\\
\implies \ \ \frac{1}{\sqrt{2}} & = \frac{1}{2\sqrt{2}} + \frac{4}{\pi}\sum^{+ \infty}_{k = 0}(- 1)^{k - 1}\frac{k}{4k^{2} - 1}\sin\left(k\frac{\pi}{2}\right)
\end{align*}

Notiamo che la serie si annulla ogni volta che $k$ è pari, essendo il seno di $\pi, 2\pi, 3\pi \dotsc $, quindi possiamo sommare direttamente sui dispari

\begin{equation*}
\begin{aligned}
\frac{1}{\sqrt{2}} & = \frac{1}{2\sqrt{2}} + \frac{4}{\pi}\sum^{+ \infty}_{k = 0}\cancel{(- 1)^{(2k + 1) - 1}}\frac{2k + 1}{4(2k + 1)^{2} - 1}\underbrace{\sin\left((2k + 1)\frac{\pi}{2}\right)}_{(- 1)^{k}}\\
\frac{1}{2\sqrt{2}} & = \frac{4}{\pi}\sum^{+ \infty}_{k = 0}\frac{2k + 1}{4(2k + 1)^{2} - 1}(- 1)^{k}\\
 & \implies \ \ \sum^{+ \infty}_{k = 0}\frac{2k + 1}{4(2k + 1)^{2} - 1}(- 1)^{k} = \frac{\pi}{8\sqrt{2}}
\end{aligned}
\end{equation*}
\item $f\in L^{2}([ - \pi, \pi ])$ allora vale l'identità di Parseval

\begin{equation*}
\frac{1}{\pi}\int^{\pi}_{- \pi}| f(x)|^{2} dx = \frac{a^{2}_{0}}{2} + \sum\limits^{\infty}_{n = 1}\left(a^{2}_{n} + b^{2}_{n}\right)
\end{equation*}

Nel nostro caso $f$ è dispari, $a_{n} = 0, \ \forall n\in \NN $, e si ha

\begin{equation*}
\underbrace{\frac{1}{\pi}\int^{\pi}_{- \pi}| f(x)|^{2} dx}_{A} = \underbrace{\sum\limits^{\infty}_{n = 1} b^{2}_{n}}_{B}
\end{equation*}

Termine $A$
\begin{equation*}
\begin{aligned}
\frac{1}{\pi}\int^{\pi}_{- \pi}| f(x)|^{2} dx & = \frac{1}{\pi}\int^{\frac{\pi}{2}}_{- \frac{\pi}{2}}\sin^{2}(x) dx = \frac{1}{\pi}\int^{\frac{\pi}{2}}_{- \frac{\pi}{2}}\frac{1 - \cos(2x)}{2} dx\\
 & = \frac{1}{2\pi}\left[ x - \frac{1}{2}\sin(2x)\right]^{\pi /2}_{- \pi /2} = \frac{1}{2\pi}\left[\frac{\pi}{2} + \frac{\pi}{2}\right] = \frac{1}{2}
\end{aligned}
\end{equation*}

Termine $B$
\begin{equation*}
\begin{aligned}
\sum\limits^{\infty}_{n = 1} b^{2}_{n} & = b^{2}_{1} + \sum\limits^{\infty}_{n = 1} b^{2}_{2n}\\
 & = \left(\frac{1}{2}\right)^{2} + \frac{4}{\pi^{2}}\sum\limits^{\infty}_{n = 1}\frac{4n^{2}}{\left(4n^{2} - 1\right)^{2}}
\end{aligned}
\end{equation*}

Uguagliamo i termini

\begin{equation*}
\frac{1}{2} = \frac{1}{4} + \frac{4}{\pi^{2}}\sum\limits^{\infty}_{n = 1}\frac{4n^{2}}{\left(4n^{2} - 1\right)^{2}} \ \ \implies \ \ \sum\limits^{\infty}_{n = 1}\frac{4n^{2}}{\left(4n^{2} - 1\right)^{2}} = \frac{\pi^{2}}{64}
\end{equation*}
\end{enumerate}

\Soluzione

Notiamo che sono serie convergenti, perché asintotiche a $\frac{1}{n^{\alpha}}$ con $\alpha > 1$.

$f$ è pari, allora $b_{n} = 0, \ \forall n \geq 1$.
\begin{align*}
a_{0} & = \frac{2}{\pi}\int^{\pi}_{0} xdx = \frac{2}{\pi}\frac{\pi^{2}}{2} = \pi \\
 & \\
a_{n} & \overset{n \geq 1}{=}\frac{1}{\pi}\int^{\pi}_{- \pi} f(x)\cos(nx) dx = \frac{2}{\pi}\int^{\pi}_{0} x\cos(nx) dx\\
 & \overset{\text{ipp}}{=}\frac{2}{\pi}\left\{\cancel{\left[\frac{x\sin(nx)}{n}\right]^{\pi}_{0}} - \frac{1}{n}\int^{\pi}_{0}\sin(nx) dx\right\}\\
 & = - \frac{2}{\pi n}\left[ - \frac{\cos(nx)}{n}\right]^{\pi}_{0} = \frac{2}{\pi n^{2}}\left[(- 1)^{n} - 1\right] =
\begin{cases}
0, & n\ \text{pari}\\
- \frac{4}{\pi n^{2}}, & n\ \text{dispari}
\end{cases}
\end{align*}
Allora
\begin{equation*}
F(x) = \frac{\pi}{2} - \frac{4}{\pi}\sum\limits^{\infty}_{n = 0}\frac{\cos[(2n + 1) x]}{(2n + 1)^{2}}
\end{equation*}
$f\in L^{2}(Q) \implies $ vale l'identità di Parseval
\begin{equation*}
\underbrace{\frac{1}{\pi}\int^{\pi}_{- \pi}| f(x)|^{2} dx}_{A} = \underbrace{\frac{a^{2}_{0}}{2} + \sum\limits^{\infty}_{n = 1}\left(a^{2}_{n} + b^{2}_{n}\right)}_{B}
\end{equation*}
Termine $A$
\begin{equation*}
A = \frac{1}{\pi}\int^{\pi}_{- \pi} x^{2} dx = \frac{1}{\pi}\left[\frac{x^{3}}{3}\right]^{\pi}_{- \pi} = \frac{1}{\pi}\left\{\frac{\pi^{3}}{3} + \frac{\pi^{3}}{3}\right\} = \frac{2\pi^{2}}{3}
\end{equation*}
Termine $B$
\begin{equation*}
B = \frac{a^{2}_{0}}{2} + \sum\limits^{\infty}_{n = 1}\left(a^{2}_{n} + \cancel{b^{2}_{n}}\right) = \frac{\pi^{2}}{2} + \frac{16}{\pi^{2}}\sum\limits^{\infty}_{\textcolor[rgb]{0.82, 0.01, 0.11}{n = 0}}\frac{1}{(2n + 1)^{4}}
\end{equation*}
Uguagliamo
\begin{equation*}
\frac{2\pi^{2}}{3} - \frac{\pi^{2}}{2} = \frac{16}{\pi^{2}}\sum\limits^{\infty}_{n = 0}\frac{1}{(2n + 1)^{4}} \ \ \implies \ \ \sum\limits^{\infty}_{n = 0}\frac{1}{(2n + 1)^{4}} = \frac{\pi^{4}}{96}
\end{equation*}
Osserviamo che
\begin{equation*}
\underbrace{\sum\limits^{\infty}_{n = 0}\frac{1}{(2n + 1)^{4}}}_{\text{sui dispari}} = \underbrace{\sum\limits^{\infty}_{n = 1}\frac{1}{n^{4}}}_{\text{su tutti}} - \underbrace{\sum\limits^{\infty}_{n = 1}\frac{1}{(2n)^{4}}}_{\text{sui pari}} = \sum\limits^{\infty}_{n = 1}\frac{1}{n^{4}} - \frac{1}{2^{4}}\sum\limits^{\infty}_{n = 1}\frac{1}{n^{4}} = \left(1 - \frac{1}{2^{4}}\right)\sum\limits^{\infty}_{n = 1}\frac{1}{n^{4}}
\end{equation*}
allora
\begin{equation*}
\sum\limits^{\infty}_{n = 1}\frac{1}{n^{4}} = \frac{\sum\limits^{\infty}_{n = 0}\frac{1}{(2n + 1)^{4}}}{1 - \frac{1}{2^{4}}} = \frac{\frac{\pi^{4}}{96}}{1 - \frac{1}{2^{4}}} = \frac{\pi^{4}}{90}
\end{equation*}

\Soluzione

La funzione è dispari, per cui $a_{n} = 0, \forall n$.
\begin{align*}
b_{n} & = \frac{1}{\pi}\int^{\pi}_{- \pi} f(x)\sin(nx) dx\\
 & = \frac{2}{\pi}\int^{\pi}_{0} f(x)\sin(nx) dx\\
 & = \frac{2}{\pi}\int^{\pi}_{0}\left(- x^{2} + x\pi \right)\sin(nx) dx\\
 & = \frac{2}{\pi}\left\{\cancel{\left[\left(- x^{2} + x\pi \right) \cdot \left(- \frac{\cos(nx)}{n}\right)\right]^{\pi}_{0}} + \frac{1}{n}\int^{\pi}_{0}\cos(nx)(\pi - 2x) dx\right\}\\
 & = \frac{2}{\pi n}\left\{\cancel{\left[(\pi - 2x) \cdot \frac{\sin(nx)}{n}\right]^{\pi}_{0}} + 2\int^{\pi}_{0}\frac{\sin(nx)}{n} dx\right\}\\
 & = \frac{4}{\pi n^{2}}\left\{\int^{\pi}_{0}\sin(nx) dx\right\}\\
 & = \frac{4}{\pi n^{2}}\left[ - \frac{\cos(nx)}{n}\right]^{\pi}_{0}\\
 & = \frac{4}{\pi n^{3}}[ 1 - \cos(n\pi)]\\
 & = \frac{4}{\pi n^{3}}\left[ 1 - (- 1)^{n}\right] =
\begin{cases}
\frac{8}{\pi n^{3}}, & n\ \text{dispari}\\
0, & n\ \text{pari}
\end{cases}
\end{align*}
La serie di Fourier cercata è
\begin{align*}
F(x) & = \sum\limits^{\infty}_{n = 1} b_{n}\sin(nx)\\
 & = \sum\limits^{\infty}_{n = 1}\frac{4}{\pi n^{3}}\left[ 1 - (- 1)^{n}\right]\sin(nx)\\
 & = \sum\limits^{\infty}_{n = 0}\frac{8}{\pi (2n + 1)^{3}}\sin((2n + 1) x)
\end{align*}
Scriviamo l'identità di Parseval
\begin{align*}
\frac{1}{\pi}\int^{\pi}_{- \pi}[ f(x)]^{2} dx & = \cancel{\frac{a^{2}_{0}}{2}} + \sum\limits^{\infty}_{n = 1}\left(\cancel{a^{2}_{n}} + b^{2}_{n}\right)\\
\frac{1}{\pi} \cdot 2\int^{\pi}_{0}\left(x\pi - x^{2}\right)^{2} dx & = \sum\limits^{\infty}_{n = 1} b^{2}_{n}\\
\frac{2}{\pi}\int^{\pi}_{0}\left[ x^{4} + x^{2} \pi^{2} - 2\pi x^{3}\right) dx & = \sum\limits^{\infty}_{n = 0}\frac{8^{2}}{\pi^{2}(2n + 1)^{6}}\\
\frac{2}{\pi}\left[\frac{\pi^{5}}{5} + \frac{\pi^{3}}{3} \pi^{2} - 2\pi \frac{\pi^{4}}{4}\right] & = \sum\limits^{\infty}_{n = 0}\frac{64}{\pi^{2}(2n + 1)^{6}}\\
\frac{\pi^{4}}{15} & = \frac{64}{\pi^{2}}\sum\limits^{\infty}_{n = 0}\frac{1}{(2n + 1)^{6}}\\
 & \implies \ \ \sum\limits^{\infty}_{n = 0}\frac{1}{(2n + 1)^{6}} = \frac{\pi^{6}}{960}
\end{align*}
Per il secondo risultato sappiamo che la somma su tutti i naturali è come sommare tutti i dispari e tutti i pari. Chiamiamo $J$ la serie cercata
\begin{align*}
\sum\limits^{\infty}_{n = 1}\frac{1}{n^{6}} & = \sum\limits^{\infty}_{n = 0}\frac{1}{(2n + 1)^{6}} + \sum\limits^{\infty}_{n = 1}\frac{1}{(2n)^{6}}\\
J & = \frac{\pi^{6}}{960} + \frac{1}{2^{6}} J\\
 & \implies \ \ J = \frac{\frac{\pi^{6}}{960}}{1 - \frac{1}{2^{6}}} = \frac{\pi^{6}}{945}
\end{align*}

\Soluzione

Disegniamo il grafico di
\begin{equation*}
f_{n}(x) =
\begin{cases}
ne^{n(x - 1)}, & x \leq 1\\
0, & x > 1
\end{cases}
\end{equation*}
\fg{0.7}{12-4}
\begin{enumerate}
\item Fissiamo $x_{0} \in \RR $
\begin{enumerate}
\item Se $x_{0} = 1\implies f_{n}(1) = n\xrightarrow{n\rightarrow + \infty} + \infty $.
\item Se $x_{0} \neq 1\implies f_{n}(x_{0})\xrightarrow{n\rightarrow + \infty} 0$.
\end{enumerate}

Quindi converge puntualmente alla funzione limite $f(x) \equiv 0\ \forall x\neq 1$.
\item $\forall \varphi \in D(\RR)$, cioè per ogni $\varphi $ a supporto compatto in $\RR $,
\begin{align*}
\langle f_{n}, \varphi \rangle & = \int_{\RR} f_{n}(x) \varphi (x) dx\\
 & = \int^{1}_{- \infty} ne^{n(x - 1)} \varphi (x) dx\\
 & \overset{\text{ipp}}{=}\left[ e^{n(x - 1)} \varphi (x)\right]^{1}_{- \infty} - \int^{1}_{- \infty} e^{n(x - 1)} \varphi'(x) dx\\
 & = \varphi (1) - \int^{1}_{- \infty} e^{n(x - 1)} \varphi'(x) dx
\end{align*}

Obiettivo: calcolare il limite per $n\rightarrow + \infty $
\begin{align*}
\lim\limits_{n\rightarrow + \infty} \langle f_{n}, \varphi \rangle & = \lim\limits_{n\rightarrow + \infty}\left[ \varphi (1) - \int^{1}_{- \infty} e^{n(x - 1)} \varphi'(x) dx\right]\\
 & = \varphi (1) - \lim\limits_{n\rightarrow + \infty}\int^{1}_{- \infty} e^{n(x - 1)} \varphi'(x) dx\\
 & \overset{\text{Dom}}{=} \varphi (1) - \int^{1}_{- \infty}\underbrace{\lim\limits_{n\rightarrow + \infty} e^{n(x - 1)} \varphi'(x)}_{= 0} dx = \varphi (1)
\end{align*}

Posso applicare il teorema della convergenza dominata perché $\varphi \in D(\RR)$. $\exists M$ tale che $| \varphi'(x)| < M, \ \forall x\in \RR $.

\begin{equation*}
\left| e^{n(x - 1)} \varphi'(x)\right| \leq Me^{x - 1} \in L^{1}((- \infty, 1))
\end{equation*}

Quindi concludiamo che

\begin{equation*}
f_{n}\xrightarrow[n\rightarrow + \infty ]{D'(\RR)} \delta_{1} \
\end{equation*}

nel senso delle distribuzioni.
\end{enumerate}

\Soluzione

\begin{enumerate}
\item Studiamo il limite puntuale di

\begin{gather*}
f_{n, \alpha}(x) = n^{\alpha}\sqrt{1 - n| x|} \cdot \chi_{\left[ - \frac{1}{n}, \frac{1}{n}\right]}(x)\\
\lim\limits_{n\rightarrow + \infty} f_{n, \alpha}(x) =
\begin{cases}
+ \infty, & x = 0\\
0, & x\neq 0
\end{cases} \ \ \implies \ \ f_{n, \alpha}(x)\xrightarrow{n\rightarrow + \infty} f(x) \equiv 0\ \text{q.o.}
\end{gather*}
\item Calcoliamo il limite della norma elevata alla $p$ della differenza e per quali condizioni tende a zero
\begin{align*}
 & \int_{\RR}[ f_{n, \alpha}(x) - 0]^{p} d\xrightarrow{n\rightarrow + \infty} 0\\
 & \iff \ \ 2\int^{1/n}_{0} n^{\alpha p}(1 - nx)^{p/2} dx\xrightarrow{n\rightarrow + \infty} 0\\
 & \iff \ \ 2n^{\alpha p}\int^{1/n}_{0}(1 - nx)^{p/2} dx\xrightarrow{n\rightarrow + \infty} 0\\
 & \iff \ \ 2n^{\alpha p}\left[\frac{(1 - nx)^{p/2 + 1}}{\frac{p}{2} + 1} \cdot \left(\frac{1}{- n}\right)\right]^{1/n}_{0}\xrightarrow{n\rightarrow + \infty} 0\\
 & \iff \ \ 2n^{\alpha p} \cdot \frac{1}{n\left(\frac{p}{2} + 1\right)}\xrightarrow{n\rightarrow + \infty} 0\\
 & \iff \ \ C \cdot n^{\alpha p - 1}\xrightarrow{n\rightarrow + \infty} 0\\
 & \iff \ \ \alpha p - 1 < 0\\
 & \iff \ \ \alpha < \frac{1}{p} \land \alpha > 0
\end{align*}

Notiamo infine che le $f_{n, \alpha}$ non sono limitate, quindi
\begin{equation*}
f_{n, \alpha}\cancel{\xrightarrow[n\rightarrow + \infty ]{L^{\infty}(\RR)}} 0
\end{equation*}
\item Per $\alpha = 1$

\begin{equation*}
f_{n, 1}(x) = n\sqrt{1 - n| x|} \cdot \chi_{\left[ - \frac{1}{n}, \frac{1}{n}\right]}(x)
\end{equation*}

Per ogni $\varphi \in D(\RR)$
\begin{align*}
\langle f_{n, 1}, \varphi \rangle & = \int_{\RR} f_{n, 1}(x) \varphi (x) dx = n\int^{\frac{1}{n}}_{- \frac{1}{n}}\sqrt{1 - n| x|} \cdot \varphi (x) dx\\
 & = n\left\{\int^{0}_{- \frac{1}{n}}\sqrt{1 + nx} \cdot \varphi (x) dx + \int^{\frac{1}{n}}_{0}\sqrt{1 - nx} \cdot \varphi (x) dx\right\}\\
 &
\begin{array}{c c}
\varphi (x) & \varphi'(x)\\
\frac{1}{n}\frac{(1 + nx)^{3/2}}{\frac{3}{2}} & \sqrt{1 + nx}\\
- \frac{1}{n}\frac{(1 - nx)^{3/2}}{\frac{3}{2}} & \sqrt{1 - nx}
\end{array}\\
 & \overset{\text{ipp}}{=} n\left\{\left[\frac{2}{3n}\sqrt{(1 + nx)^{3}} \varphi (x)\right]^{0}_{- 1/n} - \frac{2}{3n}\int^{0}_{- \frac{1}{n}}\sqrt{(1 + nx)^{3}} \varphi'(x)dx + \right. \\
 & \left. + \left[ - \frac{2}{3n}\sqrt{(1 - nx)^{3}} \varphi (x)\right]^{1/n}_{0} + \frac{2}{3n}\int^{1/n}_{0}\sqrt{(1 + nx)^{3}} \varphi'(x)dx\right\}\\
 & = 2 \cdot \frac{2}{3} \varphi (0) + \frac{2}{3}\left\{- \int^{0}_{- \frac{1}{n}}\sqrt{(1 + nx)^{3}} \varphi'(x)dx + \int^{1/n}_{0}\sqrt{(1 + nx)^{3}} \varphi'(x)dx\right\}
\end{align*}

Quindi

\begin{equation*}
\lim\limits_{n\rightarrow + \infty} \langle f_{n, 1}, \varphi \rangle = \frac{4}{3} \varphi (0) + \frac{2}{3}\lim\limits_{n\rightarrow + \infty}\left\{- \int^{0}_{- \frac{1}{n}} \dotsc dx + \int^{1/n}_{0} \dotsc dx\right\}
\end{equation*}

A questo punto osserviamo che

\begin{equation*}
\varphi \in D(\RR) \ \ \implies \ \ M = \max_{x\in [ - 1, 1]} \varphi (x)
\end{equation*}

Con un ragionamento analogo a prima
\begin{equation*}
\implies \ \ f_{n, 1}\xrightarrow[n\rightarrow + \infty ]{D'(\RR)}\frac{4}{3} \delta_{0}
\end{equation*}
\end{enumerate}
\chapter{Esercitazione 7 - Boella}

\ParteEsercizi

\Esercizio{}

Calcolare la derivata nel senso delle distribuzioni di
\begin{equation*}
f(x) = \arctan\left(\frac{1}{x}\right)
\end{equation*}

\Esercizio{}

Calcolare la derivata prima e seconda nel senso delle distribuzioni in $D'(0, 5)$ di
\begin{equation*}
f(x) =
\begin{cases}
x + x^{2}, & 0 < x < 2\\
6, & 2 < x < 5
\end{cases}
\end{equation*}

\Esercizio{}

Trovare l'equivalente in $D'(\RR)$ di
\begin{equation*}
u(x) = x\delta'_{0} (x)
\end{equation*}

\Esercizio{}

Calcolare il limite delle $f_{n}$ in $L^{p} (\RR)$, $D'(\RR)$ e $\Sc'(\RR)$ di\footnote{Dove $H$ indica la funzione di Heaviside.}
\begin{equation*}
g(x) = H(x)e^{- x} \ \ \ \ f_{n} (x) = g(x)*\chi_{(0, n)}
\end{equation*}

\Esercizio{}

Calcolare la serie di Fourier associata alla funzione $2\pi $ - periodica dispari
\begin{equation*}
f(x) =
\begin{cases}
1, & 0 < x < \pi \\
- 1, & - \pi < x < 0
\end{cases}
\end{equation*}

\ParteSoluzioni

\Soluzione

Dato che $f\in L^{1}_{\loc} (\RR)\implies f\in D'(\RR)$.

Sappiamo anche che $f\in \Sc'(\RR)$, lo spazio delle distribuzioni temperate.

\fg{0.7}{13-1}

Calcolo quindi la derivata nel senso delle distribuzioni. Ricordiamo innanzitutto che
\begin{equation*}
\frac{d}{dx}\arctan\left(\frac{1}{x}\right) = - \frac{1}{1 + \left(\frac{1}{x}\right)^{2}}\frac{1}{x^{2}} = - \frac{1}{1 + x^{2}}
\end{equation*}
Calcoliamo
\begin{align*}
\langle f', \varphi \rangle & = - \langle f, \varphi'\rangle = - \int_{\RR}\arctan\left(\frac{1}{x}\right) \varphi'(x)dx = \\
 & = - \int^{0}_{- \infty}\arctan\left(\frac{1}{x}\right) \varphi'(x)dx - \int^{+ \infty}_{0}\arctan\left(\frac{1}{x}\right) \varphi'(x)dx = \\
 & \overset{\text{ipp}}{=} - \left[\arctan\left(\frac{1}{x}\right) \varphi (x)\right]^{0}_{- \infty} + \int^{+ \infty}_{0}\frac{- 1}{1 + x^{2}} \varphi (x) dx\\
 & \ \ \ \ \ - \left[\arctan\left(\frac{1}{x}\right) \varphi (x)\right]^{+ \infty}_{0} + \int^{+ \infty}_{0}\frac{- 1}{1 + x^{2}} \varphi (x) dx\\
 & = \frac{\pi}{2} \varphi (0) + \frac{\pi}{2} \varphi (0) + \int_{\RR}\frac{- 1}{1 + x^{2}} \varphi (x)dx\\
 & = \pi \langle \delta_{0}, \varphi \rangle + \langle \frac{- 1}{1 + x^{2}}, \varphi \rangle = \langle \pi \delta_{0} - \frac{1}{1 + x^{2}}, \varphi \rangle
\end{align*}
Quindi
\begin{equation*}
f'(x)\overset{D'}{=} \pi \delta_{0} - \frac{1}{1 + x^{2}}
\end{equation*}
che può essere rappresentata come la funzione $1/\left(1 + x^{2}\right)$ con un salto in $x = 0$ verso l'alto di ampiezza $\pi $.

\Soluzione

Dato che $f\in L^{1}_{\loc} (0, 5)\implies f\in D'(0, 5)$.

\fg{0.7}{13-2}

Questo significa che la funzione $\varphi (x)$ sarà identicamente nulla su qualsiasi intervallo esterno a $(0, 5)$, cioè in particolare $\varphi (0) = \varphi (5) = 0$.
\begin{align*}
\langle f', \varphi \rangle & = - \langle f, \varphi'\rangle = - \int^{5}_{0} f(x)\varphi'(x)dx\\
 & = - \left\{\int^{2}_{0} (x + x^{2})\varphi'(x)dx + \int^{5}_{2} 6\varphi'(x)dx\right\}\\
 & \overset{\text{ipp}}{=} - \left\{\left[ (x + x^{2})\varphi (x)\right]^{2}_{0} - \int^{2}_{0} (1 + 2x)\varphi (x)dx + [ 6\varphi (x)]^{5}_{2}\right\}\\
 & = - \left\{\cancel{6\varphi (2)} - 0 - \int^{2}_{0}(1 + 2x) \varphi (x) dx + \cancel{6\varphi (5)} - \cancel{6\varphi (2)}\right\}\\
 & = \int^{2}_{0}(1 + 2x) \varphi (x) dx
\end{align*}
Definiamo
\begin{equation*}
g(x) =
\begin{cases}
2x + 1, & 0 < x < 2\\
0, & 2 < x < 5
\end{cases} \ \ \implies \ \ \langle f', \varphi \rangle = \int_{\RR} g(x) \varphi (x) dx = \langle g, \varphi \rangle
\end{equation*}
ovvero
\begin{equation*}
f'(x)\overset{D'(0, 5)}{=} g(x)
\end{equation*}
Derivata seconda. Si prosegue in maniera analoga:
\begin{equation*}
\begin{aligned}
\langle f'', \varphi \rangle & = - \langle f', \varphi'\rangle = - \langle g, \varphi'\rangle = - \int^{2}_{0} (1 + 2x)\varphi'(x)dx & \\
 & = - \left\{[ (1 + 2x)\varphi (x)]^{2}_{0} - \int^{2}_{0} 2\varphi (x)dx\right\} & \\
 & = - \left\{5\varphi (2) - 0 - \int^{2}_{0} 2\varphi (x)dx\right\} & h(x) =
\begin{cases}
2, & 0 < x < 2\\
0, & 2 < x < 5
\end{cases}\\
 & \overset{}{=} - 5\varphi (2) + \int^{5}_{0} h(x)\varphi (x)dx & \\
 & = - 5\langle \delta_{2}, \varphi \rangle + \langle h, \varphi \rangle & \\
 & = \langle h - 5\delta_{2}, \varphi \rangle &
\end{aligned}
\end{equation*}
ovvero
\begin{equation*}
f''(x)\overset{D'(0, 5)}{=} h(x) - 5\delta_{2}
\end{equation*}
Ciò è coerente con come era stata definita la $g$


\begin{figure}[htpb]
	\centering
\tikzset{every picture/.style = {line width = 0.75pt}} %set default line width to 0.75pt

\begin{tikzpicture}[x = 0.75pt, y = 0.75pt, yscale = -1, xscale = 1]
%uncomment if require: \path (0, 203); %set diagram left start at 0, and has height of 203

%Shape: Axis 2D [id:dp46013604632113503]
\draw (200, 160) - - (380, 160)(200, 20) - - (200, 160) - - cycle (373, 155) - - (380, 160) - - (373, 165) (195, 27) - - (200, 20) - - (205, 27) (220, 155) - - (220, 165)(240, 155) - - (240, 165)(260, 155) - - (260, 165)(280, 155) - - (280, 165)(300, 155) - - (300, 165)(320, 155) - - (320, 165)(340, 155) - - (340, 165)(360, 155) - - (360, 165)(195, 140) - - (205, 140)(195, 120) - - (205, 120)(195, 100) - - (205, 100)(195, 80) - - (205, 80)(195, 60) - - (205, 60)(195, 40) - - (205, 40) ;
\draw ;
%Straight Lines [id:da5583914209314376]
\draw [color = {rgb, 255:red, 208; green, 2; blue, 27}, draw opacity = 1 ][line width = 1.5] (240, 60) - - (200, 140) ;
%Straight Lines [id:da8784587512983899]
\draw [color = {rgb, 255:red, 208; green, 2; blue, 27}, draw opacity = 1 ][line width = 1.5] (300, 160) - - (240, 160) ;
%Shape: Brace [id:dp22057370394693132]
\draw (243, 154.53) .. controls (247.67, 154.53) and (250, 152.2) .. (250, 147.53) - - (250, 121.53) .. controls (250, 114.86) and (252.33, 111.53) .. (257, 111.53) .. controls (252.33, 111.53) and (250, 108.2) .. (250, 101.53)(250, 104.53) - - (250, 75.53) .. controls (250, 70.86) and (247.67, 68.53) .. (243, 68.53) ;
%Curve Lines [id:da9997704212540879]
\draw (280, 20) .. controls (226.44, 17.62) and (216.19, 30.57) .. (219.58, 57.06) ;
\draw [shift = {(220, 60)}, rotate = 261.02] [fill = {rgb, 255:red, 0; green, 0; blue, 0} ][line width = 0.08] [draw opacity = 0] (10.72, - 5.15) - - (0, 0) - - (10.72, 5.15) - - (7.12, 0) - - cycle ;

% Text Node
\draw (181, 132.4) node [anchor = north west][inner sep = 0.75pt] {$1$};
% Text Node
\draw (181, 52.4) node [anchor = north west][inner sep = 0.75pt] {$5$};
% Text Node
\draw (237, 172.4) node [anchor = north west][inner sep = 0.75pt] {$2$};
% Text Node
\draw (187, 162.4) node [anchor = north west][inner sep = 0.75pt] {$0$};
% Text Node
\draw (297, 172.4) node [anchor = north west][inner sep = 0.75pt] {$5$};
% Text Node
\draw (437, 32.4) node [anchor = north west][inner sep = 0.75pt] [color = {rgb, 255:red, 208; green, 2; blue, 27}, opacity = 1 ] {$g$};
% Text Node
\draw (261, 100) node [anchor = north west][inner sep = 0.75pt] [align = left] {salto di $5$ unità verso il basso in $x = 2$};
% Text Node
\draw (291, 12) node [anchor = north west][inner sep = 0.75pt] [align = left] {pendenza di $2$};


\end{tikzpicture}
\end{figure}
\FloatBarrier

Il salto verso l'alto in $x = 0$ non lo possiamo vedere perché le funzioni test sono nulle prima di zero.

\Soluzione

\begin{rem}
Si può dimostrare che se $v\in D'(\RR)$, $\psi \in C^{\infty}(\RR) \implies \psi v\in D'(\RR)$ e il suo effetto su una funzione test $\varphi $ è
\begin{equation*}
\langle \psi v, \varphi \rangle \coloneqq \langle v, \psi \varphi \rangle
\end{equation*}
\end{rem}
Quindi
\begin{equation*}
\langle x\delta'_{0} (x), \varphi \rangle = \langle \delta'_{0} (x), x\varphi (x) \rangle = - \langle \delta_{0}, x\varphi'(x) + \varphi (x) \rangle = - \varphi (0) = \langle - \delta_{0}, \varphi \rangle
\end{equation*}
Cioè in $D'(\RR)$
\begin{equation*}
x\delta'_{0} = - \delta_{0}
\end{equation*}
\textit{Esercizio per casa.}

$\forall n, m\in \NN $ trovare la distribuzione di $u(x) = x^{m} \delta^{(n)}_{0}$.

\Soluzione

\begin{equation*}
f_{n} (x) = g(x)*\chi_{(0, n)} = \int_{\RR} e^{- (x - t)} H(x - t)\chi_{(0, n)} (t)dt = \int^{n}_{0} e^{- x} e^{t} H(x - t)dt
\end{equation*}
\fg{0.7}{13-3}
\begin{gather*}
\begin{cases}
0, & x < 0\\
\int^{x}_{0} e^{- x} e^{t} dt, & 0 \leq x \leq n\\
\int^{n}_{0} e^{- x} e^{t} dt, & x > n
\end{cases} =
\begin{cases}
0, & x < 0\\
e^{- x}\left(e^{x} - 1\right) = 1 - e^{- x} & 0 \leq x \leq n\\
e^{- x} (e^{n} - 1) & x > n
\end{cases}\\
\\
\xrightarrow{n\rightarrow \infty} F(x) =
\begin{cases}
0, & x < 0\\
1 - e^{- x}, & x \geq 0
\end{cases}
\end{gather*}
\begin{itemize}
\item Convergenza in $L^{p}(\RR)$?

No perché l'ipotetico limite $F(x)$ non è integrabile a $ + \infty $.
\item Convergenza in $L^{\infty}(\RR)$?

Notiamo che

\begin{equation*}
\begin{aligned}
\Vert F\Vert_{L^{\infty}(\RR)} & = \sup_{\RR}| F(x)| = 1\xrightarrow{n\rightarrow + \infty} 1\\
\Vert f_{n}\Vert_{L^{\infty}(\RR)} & = \sup_{\RR}| f_{n}(x)| = 1 - e^{- n}\xrightarrow{n\rightarrow + \infty} 1
\end{aligned}
\end{equation*}

Tuttavia non possiamo concludere da queste due affermazioni che c'è convergenza in $L^{\infty}$, quello che ci interessa è la \textit{differenza} in norma infinito.

\begin{equation*}
\Vert f_{n} - F\Vert_{L^{\infty}(\RR)} =
\begin{cases}
0, & x < n\\
F(x) - f_{n}(x), & x \geq n
\end{cases} =
\begin{cases}
0, & x < n\\
1 - e^{n - x}, & x \geq n
\end{cases}
\end{equation*}

È un esponenziale traslato di $n$

\begin{equation*}
\Vert f_{n} - F\Vert_{L^{\infty}(\RR)} = \sup_{\RR}| f_{n}(x) - F(x)| = 1\nrightarrow 0
\end{equation*}

Quindi non c'è convergenza in $L^{\infty}(\RR)$.
\item Convergenza in $D'(\RR)$?

Sono distribuzioni in quanto $f_{n}, F\in L^{1}_{\loc}(\RR)$.

\begin{equation*}
\langle f_{n}, \varphi \rangle \rightarrow \langle F, \varphi \rangle \ \ \forall \varphi \in D(\RR) \ \ \iff \ \ \langle F - f_{n}, \varphi \rangle \rightarrow 0
\end{equation*}

nel nostro caso

\begin{equation*}
\langle F - f_{n}, \varphi \rangle = \int^{\infty}_{n}\left(1 - e^{n - x}\right) \varphi (x)dx = 0\ \ \forall n \geq n_{\varphi}
\end{equation*}

perché $\varphi $ è a supporto compatto, cioè $\forall \varphi \ \exists n_{\varphi} :\varphi (x) = 0, \forall x \geq n$. Il dominio di integrazione prima o poi andrà oltre il dominio di $\varphi $, che poi si annullerà.

Quindi

\begin{equation*}
f_{n}\xrightarrow{D'(\RR)} F
\end{equation*}
\item Convergenza in $\Sc'(\RR)$?
\begin{equation*}
\langle f_{n}, \psi \rangle \rightarrow \langle F, \psi \rangle \ \ \forall \psi \in S(\RR) \ \ \iff \ \ \langle F - f_{n}, \psi \rangle \rightarrow 0
\end{equation*}

dove $S$ è lo spazio di Schwartz, lo spazio delle funzioni a decrescita rapida
\begin{equation*}
\psi \in C^{\infty} \ \ \land \ \ \lim\limits_{x\rightarrow + \infty} x^{n} \psi (x) = 0
\end{equation*}

nel nostro caso

\begin{equation*}
\langle F - f_{n}, \psi \rangle = \int^{+ \infty}_{n}\left(1 - e^{n - x}\right) \psi (x) dx = \int^{+ \infty}_{0}\left(1 - e^{n - x}\right) \chi_{(n, + \infty)}(x) \psi (x) dx
\end{equation*}

Se chiamiamo $h_{n}(x)$ l'integranda

\begin{equation*}
\lim\limits_{n\rightarrow + \infty} h_{n}(x) = 0
\end{equation*}

Usiamo la convergenza dominata

\begin{equation*}
| h_{n}(x)| \leq | \psi (x)| \ \ | \psi (x)| \in L^{1}(\RR)
\end{equation*}

Allora

\begin{equation*}
\langle F - f_{n}, \psi \rangle \rightarrow 0
\end{equation*}
\end{itemize}

\Soluzione

\begin{equation*}
f(x) \sim \sum\limits^{\infty}_{n = 1} b_{n}\sin(nx)
\end{equation*}
Calcoliamo i coefficienti
\begin{equation*}
b_{n} = \frac{2}{\pi}\int^{\pi}_{0} 1 \cdot \sin(nx) dx = \frac{2}{\pi}\left[ - \frac{1}{n}\cos(nx)\right]^{\pi}_{0} = \frac{2}{\pi n}\left(1 - (- 1)^{n}\right) =
\begin{cases}
\frac{4}{n\pi}, & n\ \text{dispari}\\
0, & n\ \text{pari}
\end{cases}
\end{equation*}
La serie cercata è
\begin{equation*}
f(x) \sim \sum\limits^{\infty}_{k = 0}\frac{4}{(2k + 1) \pi}\sin((2k + 1) x)
\end{equation*}
Questa funzione è anche una distribuzione? Sì, $f\in L^{1}_{\loc}(\RR)$.
\begin{equation*}
\langle u, \varphi \rangle = \langle u, \varphi (x + T) \rangle
\end{equation*}
In $D'(\RR)$, a occhio possiamo dedurre la derivata\footnote{Si noti la somiglianza col Pettine di Dirac.}


\begin{figure}[htpb]
	\centering
\tikzset{every picture/.style = {line width = 0.75pt}} %set default line width to 0.75pt

\begin{tikzpicture}[x = 0.75pt, y = 0.75pt, yscale = -1, xscale = 1]
%uncomment if require: \path (0, 132); %set diagram left start at 0, and has height of 132

%Shape: Axis 2D [id:dp6130811625741925]
\draw (80, 69.4) - - (520, 69.4)(300, 10) - - (300, 130) (513, 64.4) - - (520, 69.4) - - (513, 74.4) (295, 17) - - (300, 10) - - (305, 17) ;
%Straight Lines [id:da4688738915321422]
\draw [color = {rgb, 255:red, 208; green, 2; blue, 27}, draw opacity = 1 ][line width = 1.5] (300, 30) - - (350, 30) ;
%Straight Lines [id:da564644152513289]
\draw [color = {rgb, 255:red, 208; green, 2; blue, 27}, draw opacity = 1 ][line width = 1.5] (250, 110) - - (300, 110) ;
%Straight Lines [id:da1288587788441018]
\draw [color = {rgb, 255:red, 208; green, 2; blue, 27}, draw opacity = 1 ][line width = 1.5] (400, 30) - - (450, 30) ;
%Straight Lines [id:da42001293411482954]
\draw [color = {rgb, 255:red, 208; green, 2; blue, 27}, draw opacity = 1 ][line width = 1.5] (350, 110) - - (400, 110) ;
%Straight Lines [id:da41879142834815575]
\draw [color = {rgb, 255:red, 208; green, 2; blue, 27}, draw opacity = 1 ][line width = 1.5] (200, 30) - - (250, 30) ;
%Straight Lines [id:da34094039721778624]
\draw [color = {rgb, 255:red, 208; green, 2; blue, 27}, draw opacity = 1 ][line width = 1.5] (150, 110) - - (200, 110) ;
%Straight Lines [id:da6751056390959596]
\draw [color = {rgb, 255:red, 126; green, 211; blue, 33}, draw opacity = 1 ] (350, 30) - - (350, 107) ;
\draw [shift = {(350, 110)}, rotate = 270] [fill = {rgb, 255:red, 126; green, 211; blue, 33}, fill opacity = 1 ][line width = 0.08] [draw opacity = 0] (10.72, - 5.15) - - (0, 0) - - (10.72, 5.15) - - (7.12, 0) - - cycle ;
%Straight Lines [id:da8152221642203317]
\draw [color = {rgb, 255:red, 126; green, 211; blue, 33}, draw opacity = 1 ] (250, 30) - - (250, 107) ;
\draw [shift = {(250, 110)}, rotate = 270] [fill = {rgb, 255:red, 126; green, 211; blue, 33}, fill opacity = 1 ][line width = 0.08] [draw opacity = 0] (10.72, - 5.15) - - (0, 0) - - (10.72, 5.15) - - (7.12, 0) - - cycle ;
%Straight Lines [id:da21495238477473233]
\draw [color = {rgb, 255:red, 126; green, 211; blue, 33}, draw opacity = 1 ] (200, 110) - - (200, 33) ;
\draw [shift = {(200, 30)}, rotate = 450] [fill = {rgb, 255:red, 126; green, 211; blue, 33}, fill opacity = 1 ][line width = 0.08] [draw opacity = 0] (10.72, - 5.15) - - (0, 0) - - (10.72, 5.15) - - (7.12, 0) - - cycle ;
%Straight Lines [id:da5779103582115537]
\draw [color = {rgb, 255:red, 126; green, 211; blue, 33}, draw opacity = 1 ] (400, 110) - - (400, 33) ;
\draw [shift = {(400, 30)}, rotate = 450] [fill = {rgb, 255:red, 126; green, 211; blue, 33}, fill opacity = 1 ][line width = 0.08] [draw opacity = 0] (10.72, - 5.15) - - (0, 0) - - (10.72, 5.15) - - (7.12, 0) - - cycle ;

% Text Node
\draw (281, 22.4) node [anchor = north west][inner sep = 0.75pt] {$1$};
% Text Node
\draw (307, 102.4) node [anchor = north west][inner sep = 0.75pt] {$ - 1$};
% Text Node
\draw (224, 72.4) node [anchor = north west][inner sep = 0.75pt] {$ - \pi $};
% Text Node
\draw (336, 72.4) node [anchor = north west][inner sep = 0.75pt] {$\pi $};


\end{tikzpicture}
\end{figure}
\FloatBarrier

\begin{equation*}
f'(x) = \sum\limits^{+ \infty}_{k = -\infty}(- 1)^{k} \cdot 2\delta_{k\pi}(x)
\end{equation*}
Possiamo derivare termine a termine
\begin{equation*}
f'(x)\overset{D'(\RR)}{\sim}\sum\limits^{\infty}_{k = 0}\frac{4}{\pi}\cos((2k + 1) x)
\end{equation*}
\chapter{Esercitazione 7 - Potrich}

\ParteEsercizi

\begin{defn}
[Distribuzione] Una distribuzione è un operatore $u$ che a ogni funzione test $\varphi $ (funzione $C^{\infty}(\RR)$ il cui supporto è un compatto $\Omega $ contenuto in $\RR $) associa un numero, reale o complesso, $\langle u, \varphi \rangle $ chiamato \textbf{dualità} tra $u$ e $\varphi $ tale che:
\begin{equation*}
\begin{array}{l}
\langle u, a\varphi + b\psi \rangle = a\langle u, \varphi \rangle + b\langle u, \psi \rangle \ \ \forall a, b\in \CC, \ \forall \varphi, \psi \in D(\Omega)\\
\varphi_{j}\rightarrow \varphi \ \text{in} \ D(\Omega) \ \ \implies \ \ \langle u, \varphi_{j} \rangle \rightarrow \langle u, \varphi \rangle
\end{array}
\end{equation*}
\end{defn}
\begin{rem}
[Delta di Dirac] L'esempio fondamentale di distribuzione in $D'\left(\RR^{n}\right)$ è la Delta di Dirac, denotata $\delta $, e così definita
\begin{equation*}
\langle \delta, \varphi \rangle = \varphi (0) \ \ \forall \varphi \in D\left(\RR^{n}\right)
\end{equation*}
\end{rem}
\begin{defn}
Se \ $u\in D'(\RR)$, allora la derivata distribuzionale di $u$ è la distribuzione $u'\in D'(\RR)$ definita univocamente da
\begin{equation*}
\int_{\RR} u'\varphi = - \int_{\RR} u\varphi'\ \ \forall \varphi \in D(\RR)
\end{equation*}
\end{defn}
\begin{defn}
Si definisce prodotto di una distribuzione in $\Omega $ per una funzione $C^{\infty}(\Omega)$: se $\psi \in C^{\infty}(\Omega)$ e $u\in D'(\RR)$, allora $(\psi u) \in D'(\Omega)$ è la distribuzione tale che
\begin{equation*}
\langle \psi u, \varphi \rangle = \langle u, \psi \varphi \rangle \ \ \forall \varphi \in D\left(\RR^{n}\right)
\end{equation*}
\end{defn}
\begin{rem}
\begin{equation*}
\langle u, \varphi \rangle \coloneqq \int_{\Omega} u(x) \varphi (x) dx, \ \ u\in D'(\Omega), \ \varphi \in D(\Omega)
\end{equation*}
\end{rem}

\Esercizio{}

Provare che valgono le seguenti identità in $D'(\RR)$:
\begin{enumerate}
\item $x\delta (x) = 0$
\item $x\delta'(x) = - \delta (x)$
\end{enumerate}

\Esercizio{(TDE 6/07/2015)}

Mostrare che, per ogni $n\in \NN^{+}$, si ha
\begin{equation*}
\boxed{x^{n} \delta^{(n)} = (- 1)^{n} n!\delta \ \ \text{in} \ D'}
\end{equation*}

\Esercizio{(Teorema di derivazione di funzioni con discontinuità di tipo salto)}

Sia $a\in \RR $, $f\in C^{1}((- \infty, a))$, $g\in C^{1}((a, + \infty))$ con $f'\in L^{1}((- \infty, a))$ e $g'\in L^{1}((a, + \infty))$ e definiamo
\begin{equation*}
u(x) =
\begin{cases}
f(x), & x < a\\
g(x), & x > a
\end{cases}
\end{equation*}
allora $u$ risulta definita q.o. su $\RR $ e la sua derivata distribuzionale $u'$ è la somma di una funzione localmente integrabile e che coincide con $f'$ in $(- \infty, a)$ e con $g'$ in $(a, + \infty)$ più una parte singolare, data dalla Delta di Dirac concentrata in $x = a$ e moltiplicata per l'ampiezza del salto
\begin{equation*}
[ u(a_{+}) - u(a_{-})] \delta_{a}
\end{equation*}

\Esercizio{}

Si consideri
\begin{equation*}
f_{n}(x) = \frac{n}{1 + n^{2} x^{2}}
\end{equation*}
provare che $f_{n}\rightarrow \pi \delta $ in $\Sc'(\RR)$ e in $D'(\RR)$.

\Esercizio{}

Si considerino
\begin{equation*}
f_{n}(x) = \frac{1}{\sqrt{n}} \chi_{[ 0, n]}(x) \ \ \ \ \ \ \ \ g_{n}(x) = \sin(nx) \chi_{[ 0, 7]}(x)
\end{equation*}
Stabilire se le successioni sono convergenti in $L^{1}(\RR), L^{2}(\RR), L^{\infty}(\RR), D'(\RR), \Sc'(\RR)$.

\Esercizio{}

Si consideri
\begin{equation*}
f_{n}(x) = n^{2} xe^{- n| x|}
\end{equation*}
\begin{enumerate}
\item Determinare il limite puntuale
\item Stabilire se $f_{n}$ converge in $L^{p}$, con $p\in [ 1, + \infty ]$
\item Determinare il limite di
\begin{equation*}
g_{n}(x) = f_{n}(x) H(x) \ \ \text{in} \ D'(\RR)
\end{equation*}

dove
\begin{equation*}
H(x) =
\begin{cases}
1, & x \geq 0\\
0, & x < 0
\end{cases}
\end{equation*}

detta \textbf{Funzione di Heaviside}.
\item Determinare il limite di $f_{n}$ in $D'(\RR)$
\end{enumerate}

\ParteSoluzioni

\Soluzione

\begin{enumerate}
\item $\forall \varphi \in D(\RR)$
\begin{equation*}
\langle x\delta, \varphi \rangle = \langle \delta, x\varphi \rangle = 0 \cdot \varphi (0) = 0\qed
\end{equation*}
\item $\forall \varphi \in D(\RR)$
\begin{gather*}
\begin{aligned}
\langle x\delta', \varphi \rangle & = \langle \delta', x\varphi \rangle = - \langle \delta, (x\varphi)'\rangle \\
 & = - \langle \delta, \varphi + x\varphi'\rangle \\
 & = - \langle \delta, \varphi \rangle - \langle \delta, x\varphi'\rangle = - \varphi (0)
\end{aligned}\\
\qed
\end{gather*}
\end{enumerate}

\Soluzione

Per induzione
\begin{itemize}
\item La formula è vera per $n = 1$, per l'esercizio precedente
\begin{equation*}
x\delta'(x) = - \delta (x)
\end{equation*}
\item Supponiamo vera la tesi per $n$ generico
\begin{equation*}
\langle x^{n} \delta^{(n)}, \varphi \rangle = (- 1)^{n} n!\langle \delta, \varphi \rangle \ \ \forall \varphi \in D
\end{equation*}

dobbiamo mostrare che vale anche per $n + 1$
\begin{gather*}
\begin{aligned}
\langle x^{n + 1} \delta^{(n + 1)}, \varphi \rangle & = \langle \delta^{(n + 1)}, x^{n + 1} \varphi \rangle = - \langle \delta^{(n)}, \left(x^{n + 1} \varphi \right)'\rangle \\
 & = - \langle \delta^{(n)}, (n + 1) x^{n} \varphi + x^{n + 1} \varphi'\rangle \\
 & = - \langle x^{n} \delta^{(n)}, (n + 1) \varphi + x\varphi'\rangle \\
 & \overset{\text{induz.}}{=} - \langle (- 1)^{n} n!\delta, (n + 1) \varphi + x\varphi'\rangle \\
 & = (- 1)^{n + 1}(n + 1) !\langle \delta, \varphi \rangle \underbrace{- (- 1)^{n} n!\langle \delta, x\varphi'\rangle}_{= 0}\\
 & = (- 1)^{n + 1}(n + 1) !\langle \delta, \varphi \rangle
\end{aligned}\\
\qed
\end{gather*}
\end{itemize}

\Soluzione

Per comodità $u(a_{-}) = f(a) \coloneqq f(a_{-})$ e $u(a_{+}) = g(a) \coloneqq g(a_{+})$.

Questi termini esistono finiti perché $f', g'$ sono integrabili.

$\forall \varphi \in D(\RR)$
\begin{equation*}
\langle u', \varphi \rangle = - \langle u, \varphi'\rangle = - \int^{+ \infty}_{- \infty} u(x) \varphi'(x) dx
\end{equation*}
la funzione $u$ è definita a tratti
\begin{equation*}
- \int^{+ \infty}_{- \infty} u(x) \varphi'(x) dx = -\int^{a}_{- \infty} f(x) \varphi'(x) dx - \int^{+ \infty}_{a} g(x) \varphi'(x) dx = (\star)
\end{equation*}
integro per parti
\begin{gather*}
\begin{aligned}
(\star) & = - [ f(x) \varphi (x)]^{a}_{- \infty} + \int^{a}_{- \infty} f'(x) \varphi (x) dx - [ g(x) \varphi (x)]^{+ \infty}_{a} + \int^{+ \infty}_{a} g'(x) \varphi (x) dx\\
 & = - f(a) \varphi (a) + \int^{a}_{- \infty} f'(x) \varphi (x) dx + g(a) \varphi (a) + \int^{+ \infty}_{a} g'(x) \varphi (x) dx\\
 & = \varphi (a)[ g(a) - f(a)] + \int^{a}_{- \infty} f'(x) \varphi (x) dx + \int^{+ \infty}_{a} g'(x) \varphi (x) dx\\
 & = [ u(a_{+}) - u(a_{-})] \langle \delta_{x = a}, \varphi \rangle + \int^{a}_{- \infty} f'(x) \varphi (x) dx + \int^{+ \infty}_{a} g'(x) \varphi (x) dx
\end{aligned}\\
\qed
\end{gather*}
\textit{Esempio.}
\begin{equation*}
\sgn(x) =
\begin{cases}
1, & x > 0\\
- 1, & x < 0
\end{cases}
\end{equation*}
Per $x > 0$, $\sgn'(x) = 0$.

Per $x < 0$, $\sgn'(x) = 0$.

$\sgn(x)$ presenta un salto in $x = 0$ di ampiezza $2$.
\begin{equation*}
\implies \ \ \sgn'(x) = 0 + 0 + 2\delta (x)
\end{equation*}
\textit{Esempio.}
\begin{equation*}
f(x) = \frac{1}{2}\sgn xe^{- | x|} =
\begin{cases}
\frac{1}{2} e^{- x}, & x > 0\\
- \frac{1}{2} e^{x}, & x < 0
\end{cases}
\end{equation*}
Per $x > 0$, $f'(x) = - \frac{1}{2} e^{- x}$.

Per $x < 0$, $f'(x) = - \frac{1}{2} e^{x}$.

$f(x)$ presenta un salto in $x = 0$ di ampiezza $1$.
\begin{equation*}
\implies \ \ f'(x) = - \frac{1}{2} e^{- | x|} + \delta (x)
\end{equation*}

\Soluzione

\fg{0.45}{14-4}

Se dimostro in $\Sc'(\RR)$, quella in $D'(\RR)$ segue automaticamente.

Devo provare che
\begin{equation*}
\int_{\RR}\frac{n\varphi (x)}{1 + n^{2} x^{2}} dx\xrightarrow{n\rightarrow + \infty} \pi \varphi (0) \ \ \forall \varphi \in S(\RR)
\end{equation*}
pongo $nx = t\implies ndx = dt$
\begin{equation*}
\int_{\RR}\frac{n\varphi \left(\frac{t}{n}\right)}{1 + t^{2}}\frac{dt}{n} = \int_{\RR} \varphi \left(\frac{t}{n}\right)\frac{1}{1 + t^{2}} dt
\end{equation*}
Proviamo che tale convergenza ha luogo per ogni $\varphi $ limitata in $0$ stimando la differenza
\begin{equation*}
\left| \int_{\RR} \varphi \left(\frac{t}{n}\right)\frac{1}{1 + t^{2}} dt - \pi \varphi (0)\right| \leq \underbrace{\left| \int_{| t| \leq T}\frac{\varphi \left(\frac{t}{n}\right) - \varphi (0)}{1 + t^{2}} dt\right|}_{I_{1}} + \underbrace{\left| \int_{| t| > T}\frac{\varphi \left(\frac{t}{n}\right) - \varphi (0)}{1 + t^{2}} dt\right|}_{I_{2}}
\end{equation*}
Per mostrare la convergenza di questi due integrali posso applicare la convergenza dominata
\begin{equation*}
\left| \varphi \left(\frac{t}{n}\right)\frac{1}{1 + t^{2}}\right| \leq \frac{\Vert \varphi \Vert_{\infty}}{1 + t^{2}} \in L^{1}(\RR)
\end{equation*}
\begin{rem}
\begin{gather*}
\frac{1}{1 + t^{2}} \in L^{1}(\RR) \ \ \ \ \ \ \ \ \int_{\RR}\frac{1}{1 + t^{2}} dt = [\arctan t]^{+ \infty}_{- \infty} = \frac{\pi}{2} - \left(- \frac{\pi}{2}\right) = \pi \\
\\
\implies \ \ \forall \varepsilon > 0\ \ \exists T = T(\varepsilon) :\ \int_{| t| > T}\frac{1}{1 + t^{2}} dt \leq \varepsilon
\end{gather*}
Fissati $\varepsilon, T, \ \exists n$ tale che
\begin{equation*}
\left| \varphi \left(\frac{t}{n}\right) - \varphi (0)\right| \leq \varepsilon \ \ \forall t:\ | t| \leq T
\end{equation*}
Allora
\begin{equation*}
I_{1} + I_{2} \leq \pi \varepsilon + 2\Vert \varphi \Vert_{\infty} \varepsilon
\end{equation*}
Per l'arbitrarietà di $\varepsilon $ segue la tesi.
\end{rem}

\Soluzione

Studiamo le $f_{n}(x)$
\begin{equation*}
| f_{n}(x)| \leq \frac{1}{\sqrt{n}} \ \ \forall x\in \RR \ \ \implies \ \ f_{n}\xrightarrow[n\rightarrow + \infty ]{L^{\infty}(\RR)} f\equiv 0
\end{equation*}
ma allora
\begin{equation*}
f_{n}\xrightarrow[n\rightarrow + \infty ]{L^{1}_{\loc}(\RR)} f\equiv 0\ \
\begin{cases}
\implies \ \ f_{n}\xrightarrow[n\rightarrow + \infty ]{D'(\RR)} f\equiv 0\\
\implies \ \ f_{n}\xrightarrow[n\rightarrow + \infty ]{\Sc'(\RR)} f\equiv 0
\end{cases}
\end{equation*}
Stimiamo
\begin{equation*}
\Vert f_{n} - 0\Vert_{L^{1}} = \Vert f_{n}\Vert_{L^{1}} = \frac{n}{\sqrt{n}} = \sqrt{n}\cancel{\xrightarrow{n\rightarrow + \infty}} 0\ \ \implies \ \ f_{n}\cancel{\xrightarrow[n\rightarrow + \infty ]{L^{1}(\RR)}} 0
\end{equation*}
Poi
\begin{equation*}
\Vert f_{n} - 0\Vert_{L^{2}} = \Vert f_{n}\Vert_{L^{2}} = 1\cancel{\xrightarrow{n\rightarrow + \infty}} 0\ \ \implies \ \ f_{n}\cancel{\xrightarrow[n\rightarrow + \infty ]{L^{2}(\RR)}} 0
\end{equation*}


Studiamo le $g_{n}(x)$

Dimostro che $g_{n}$ converge in $\Sc'(\RR)$, da cui si ottiene che $g_{n}$ converge anche in $D'(\RR)$.
\begin{equation*}
\int_{\RR} g_{n}(x) \varphi (x) dx = \overset{\text{ipp}}{=} - \frac{1}{n}[\cos(nx) \varphi (x)]^{7}_{0} + \int^{7}_{0}\frac{1}{n}\cos(nx) \varphi'(x) dx\xrightarrow{n\rightarrow + \infty} 0
\end{equation*}
Analizziamo in $L^{\infty}$
\begin{equation*}
\Vert g_{n} - g\Vert_{\infty} = \Vert g_{n}\Vert_{\infty} = 1\cancel{\xrightarrow{n\rightarrow + \infty}} 0\ \ \implies \ \ g_{n}\cancel{\xrightarrow[n\rightarrow + \infty ]{L^{\infty}(\RR)}} 0
\end{equation*}
Analizziamo in $L^{p}$
\begin{equation*}
\Vert g_{n} - g\Vert^{p}_{L^{p}} = \Vert g_{n}\Vert^{p}_{L^{p}} = \int^{7}_{0}| \sin(nx)|^{p} dx\overset{nx = y}{=}\frac{1}{n}\int^{7n}_{0}| \sin(y)|^{p} dy
\end{equation*}
\begin{rem}
$a > 0$
\begin{equation*}
a\int^{ka}_{0}| \sin y|^{p} dy = \frac{2h\pi}{k}\int^{2h\pi}_{0}| \sin y|^{p} dy + \frac{1}{k}\int^{ka}_{2h\pi}| \sin y|^{p} dy
\end{equation*}
dove $h = h(k)$ è l'unico intero tale che $2h\pi \leq ka \leq 2(h + 1) \pi $.

Se $k\rightarrow + \infty $ allora $\frac{2h\pi}{k}\rightarrow 1$ e
\begin{equation*}
\int^{2h\pi}_{0}| \sin y|^{p} dy = \int^{2\pi}_{0}| \sin y|^{p} dy\ \ \forall h
\end{equation*}
mentre l'altro addendo è infinitesimo.
\end{rem}
Quindi
\begin{equation*}
\frac{1}{n}\int^{7n}_{0}| \sin(y)|^{p} dy\xrightarrow{n\rightarrow + \infty} 7\int^{2\pi}_{0}| \sin y|^{p} dy\ \ \implies \ \ g_{n}\cancel{\xrightarrow[n\rightarrow + \infty ]{L^{p}(\RR)}} 0
\end{equation*}

\Soluzione

\begin{enumerate}
\item Fissiamo $x_{0} \in \RR $.
\begin{enumerate}
\item se $x_{0} = 0$ allora $f_{n}(0) \equiv 0\ \forall n\in \NN $
\item se $x_{0} \neq 0$ allora $\lim\limits_{n\rightarrow + \infty} f_{n}(x_{0}) = 0$
\item quindi $f_{n}$ converge puntualmente su tutto $\RR $ a $f(x) \equiv 0$.
\end{enumerate}

\fg{0.7}{14-6}
\item Se $p\in [ 1, + \infty)$
\begin{equation*}
\Vert f_{n}\Vert^{p}_{L^{p}} = \int_{\RR} n^{2p} x^{p} e^{- np| x|} dx = 2\int^{+ \infty}_{0} n^{2p} x^{p} e^{- npx} dx = (\star)
\end{equation*}

poniamo $npx = t\implies npdx = dt$
\begin{equation*}
(\star) = 2\int^{+ \infty}_{0} n^{2p}\frac{t^{p}}{n^{p} p^{p}} e^{- t}\frac{dt}{np} = \frac{2}{p^{p + 1}} n^{p - 1}\int^{+ \infty}_{0} t^{p} e^{- t} dt\cancel{\xrightarrow{n\rightarrow + \infty}} 0
\end{equation*}

l'integrale non dipende da $n$, siamo nel caso $p \geq 1$, quindi non c'è convergenza.

Se $p = \infty $
\begin{equation*}
\Vert f_{n} - f\Vert_{\infty} = \Vert f_{n}\Vert_{\infty} = \max_{\RR}| f_{n}(x)| = \left| f_{n}\left(\frac{1}{n}\right)\right| = n^{2}\frac{1}{n} e^{- n\frac{1}{n}} = ne^{- 1}\cancel{\xrightarrow{n\rightarrow + \infty}} 0
\end{equation*}

quindi non c'è convergenza.
\item La successione $g_{n}$ è
\begin{equation*}
g_{n}(x) = f_{n}(x) H(x) =
\begin{cases}
n^{2} xe^{- nx}, & x \geq 0\\
0, & x < 0
\end{cases}
\end{equation*}

$\forall n\in \NN, \ g_{n} \in L^{1}_{\loc}(\RR)$ e $\forall \varphi \in D(\RR)$
\begin{equation*}
\langle g_{n}, \varphi \rangle = \int_{\RR} g_{n}(x) \varphi (x) dx = \int^{+ \infty}_{0} n^{2}\textcolor[rgb]{0.82, 0.01, 0.11}{x} e^{- nx}\textcolor[rgb]{0.82, 0.01, 0.11}{\varphi}\textcolor[rgb]{0.82, 0.01, 0.11}{(}\textcolor[rgb]{0.82, 0.01, 0.11}{x}\textcolor[rgb]{0.82, 0.01, 0.11}{)} dx
\end{equation*}

integriamo per parti con
\begin{gather*}
\textcolor[rgb]{0.82, 0.01, 0.11}{f}\textcolor[rgb]{0.82, 0.01, 0.11}{(}\textcolor[rgb]{0.82, 0.01, 0.11}{x}\textcolor[rgb]{0.82, 0.01, 0.11}{)} = x\varphi (x) \ \ \rightarrow \ \ f'(x) = \varphi (x) + x\varphi'(x)\\
g(x) = - ne^{- nx} \ \ \rightarrow \ \ g'(x) = n^{2} e^{- nx}
\end{gather*}

da cui
\begin{equation*}
\overset{\text{ipp}}{=}\underbrace{\left[ - ne^{- nx} x\varphi (x)\right]^{+ \infty}_{0}}_{= 0} + \int^{+ \infty}_{0} ne^{- nx}[\textcolor[rgb]{0.25, 0.46, 0.02}{\varphi}\textcolor[rgb]{0.25, 0.46, 0.02}{(}\textcolor[rgb]{0.25, 0.46, 0.02}{x}\textcolor[rgb]{0.25, 0.46, 0.02}{)}\textcolor[rgb]{0.25, 0.46, 0.02}{+ x\varphi'}\textcolor[rgb]{0.25, 0.46, 0.02}{(}\textcolor[rgb]{0.25, 0.46, 0.02}{x}\textcolor[rgb]{0.25, 0.46, 0.02}{)}] dx
\end{equation*}

integriamo per parti con
\begin{gather*}
\textcolor[rgb]{0.25, 0.46, 0.02}{f}\textcolor[rgb]{0.25, 0.46, 0.02}{(}\textcolor[rgb]{0.25, 0.46, 0.02}{x}\textcolor[rgb]{0.25, 0.46, 0.02}{)} = \varphi (x) + x\varphi'(x) \ \ \rightarrow \ \ f'(x) = 2\varphi'(x) + x\varphi''(x)\\
g(x) = - e^{- nx} \ \ \rightarrow \ \ g'(x) = ne^{- nx}
\end{gather*}

da cui
\begin{gather*}
\overset{\text{ipp}}{=}\underbrace{\left[ - e^{- nx}(\varphi (x) + x\varphi'(x))\right]^{+ \infty}_{0}}_{\varphi (0)} + \int^{+ \infty}_{0} e^{- nx}[ 2\varphi'(x) + x\varphi''(x)] dx\\
\begin{aligned}
\implies & \langle g_{n}, \varphi \rangle = \varphi (0) + \int^{+ \infty}_{0} e^{- nx}[ 2\varphi'(x) + x\varphi''(x)] dx\\
\implies & \lim\limits_{n\rightarrow + \infty} \langle g_{n}, \varphi \rangle = \varphi (0) + \lim\limits_{n\rightarrow + \infty}\int^{+ \infty}_{0} e^{- nx}[ 2\varphi'(x) + x\varphi''(x)] dx\\
\overset{\text{Dom}}{\implies} & \lim\limits_{n\rightarrow + \infty} \langle g_{n}, \varphi \rangle = \varphi (0) + \int^{+ \infty}_{0}\lim\limits_{n\rightarrow + \infty} e^{- nx}[ 2\varphi'(x) + x\varphi''(x)] dx = \varphi (0)
\end{aligned}
\end{gather*}

essendo $\varphi \in D(\RR) \implies \exists M > 0$ tale che $| 2\varphi'(x) + x\varphi''(x)| < M, \ \forall x\in \RR $ e
\begin{equation*}
\left| e^{- nx}[ 2\varphi'(x) + x\varphi''(x)]\right| \leq Me^{- x} \in L^{1}([ 0, + \infty))
\end{equation*}

quindi
\begin{equation*}
g_{n}\xrightarrow[n\rightarrow + \infty ]{D'(\RR)} \delta_{0}
\end{equation*}
\item $\forall \varphi \in D(\RR)$
\begin{equation*}
\begin{aligned}
\langle f_{n}, \varphi \rangle & = \int_{\RR} n^{2} xe^{- n| x|} \varphi (x) dx\\
 & = \int^{0}_{- \infty} n^{2} xe^{nx} \varphi (x) dx + \int^{+ \infty}_{0} n^{2} xe^{- nx} \varphi (x) dx
\end{aligned}
\end{equation*}

integrazione per parti due volte per entrambi gli integrali e teorema della convergenza dominata
\begin{align*}
\langle f_{n}, \varphi \rangle & = \int_{\RR} n^{2} xe^{- n| x|} dx\\
 & = \int^{0}_{- \infty} n^{2} xe^{nx} \varphi (x) dx + \int^{\infty}_{0} n^{2} xe^{- nx} \varphi (x) dx
\end{align*}

ponendo $f = x\varphi (x)$ e $g'$ l'altro pezzo per entrambi gli integrali
\begin{align*}
 & \overset{\text{ipp}}{=}\cancel{\left[ ne^{nx} \cdot x\varphi (x)\right]^{0}_{- \infty}} - \int^{0}_{- \infty} ne^{nx}[ \varphi (x) + x\varphi'(x)] dx\\
 & \ \ \ \ + \cancel{\left[ - ne^{- nx} \cdot x\varphi (x)\right]^{\infty}_{0}} + \int^{\infty}_{0} ne^{- nx}[ \varphi (x) + x\varphi'(x)] dx\\
 & \overset{\text{ipp}}{=} - \left\{\left[ e^{nx} \cdot (\varphi (x) + x\varphi'(x))\right]^{0}_{- \infty} - \int^{0}_{- \infty} e^{nx}[ \varphi'(x) - \varphi'(x) - x\varphi''(x)] dx\right\}\\
 & \ \ \ \ + \left\{\left[ - e^{- nx}(\varphi (x) + x\varphi'(x))\right]^{\infty}_{0} + \int^{\infty}_{0} e^{- nx}[ \varphi'(x) + \varphi'(x) + x\varphi''(x)] dx\right\}
\end{align*}

al limite i due integrali si cancellano per convergenza dominata, rimane
\begin{equation*}
- [ \varphi (0) - 0] + [ 0 - (- \varphi (0))] = - \varphi (0) + \varphi (0) = - \delta_{0} + \delta_{0} = 0
\end{equation*}
\end{enumerate}
\chapter{Esercitazione 8 - Boella}

\ParteEsercizi

\Esercizio{}

Calcolare la trasformata di Fourier di $f(x) = e^{- |x|}$

\Esercizio{}

Calcolare la trasformata di Fourier di $f(x) = \chi_{[ - 1, 1]} (x)$

\Esercizio{}

Calcolare la trasformata di Fourier di $f(x) = e^{- x^{2}}$

\Esercizio{}

Calcolare la trasformata di Fourier di $f(x) = \frac{1}{\cosh (x)}$

\Esercizio{}

Calcolare la trasformata di Fourier di $f(x) = \frac{1}{\left(1 + x^{2}\right)^{2}}$

\Esercizio{}

Calcolare la trasformata di Fourier di $u(x) = \frac{1}{\sqrt{|x|}}$

\ParteSoluzioni

\Soluzione

Grafico di $f(x) = e^{- | x|}$

\fg{0.7}{15-1}
\begin{equation*}
f\ \text{reale pari} \ \ \implies \ \ \hat{f} \ \text{reale pari}
\end{equation*}
Inoltre $f\in L^{1}(\RR)$ quindi per il Teorema di Riemann - Lebesgue
\begin{equation*}
f\in L^{1}(\RR) \ \ \implies \ \ \hat{f} \in C^{0}_{0}(\RR) \cap L^{\infty}(\RR)
\end{equation*}
La trasformata di Fourier è derivabile? In $0$ c'è un punto angoloso quindi non possiamo utilizzare la proprietà della trasformata di Fourier per le derivate.

Poiché all'infinito la funzione va a zero come un'esponenziale anche se la moltiplichiamo per una qualsiasi potenza di $x$ quello che otteniamo è ancora una funzione in $L^{1} (\RR)$
\begin{equation*}
x^{n} f(x)\in L^{1} (\RR)\ \ \implies \ \ \frac{d^{n}}{d\xi^{n}}\hat{f} \in C^{0}(\RR) \ \ \implies \ \ \hat{f} \in C^{\infty} (\RR)
\end{equation*}
Inoltre
\begin{equation*}
f\notin C^{\infty} (\RR)\ \ \implies \ \ f\notin \Sc (\RR) \ \ \implies \ \ \hat{f} \notin \Sc (\RR)
\end{equation*}
In compenso visto che $f$ è sicuramente una distribuzione temperata anche $\hat{f}$ lo sarà
\begin{equation*}
f\in \Sc'(\RR) \ \ \implies \ \ \hat{f} \in \Sc'(\RR)
\end{equation*}
Inoltre
\begin{equation*}
f\in L^{2}(\RR) \cap L^{1}(\RR) \ \ \implies \ \ \hat{f} \in L^{2}(\RR)
\end{equation*}
Visto che c'è un valore assoluto la cosa migliore da fare è quella di spezzare l'integrale, quindi:
\begin{equation*}
\begin{aligned}
\hat{f} (\xi) & = \int e^{- i\xi x} \ e^{- |x|} \ dx = \int^{0}_{- \infty} e^{- i\xi x} \ e^{x} dx + \int^{+ \infty}_{0} e^{- i\xi x} \ e^{- x} dx\\
 & = \left[\frac{1}{1 - i\xi} e^{(1 - i\xi) x}\right]^{0}_{- \infty} + \left[\frac{- 1}{1 + i\xi} e^{- (1 + i\xi) x}\right]^{\infty}_{0}\\
 & = \frac{1}{1 - i\xi} + \frac{1}{1 + i\xi} = \frac{2}{1 + \xi^{2}}
\end{aligned}
\end{equation*}
Ricordiamo la regola
\begin{equation*}
\Fc \{u(\alpha x)\} = \frac{1}{| \alpha |}\hat{u}\left(\frac{\xi}{\alpha}\right)
\end{equation*}
Sapendo questo, deduciamo quanto vale, dato $\alpha > 0$
\begin{equation*}
\Fc \left\{e^{- \alpha | x|}\right\} = \frac{1}{\alpha} \cdot \frac{2}{1 + \left(\frac{\xi}{\alpha}\right)^{2}} = \frac{2\alpha}{\alpha^{2} + \xi^{2}}
\end{equation*}
Ricordiamo la formula dell'antitrasformata
\begin{equation*}
f(x) = \frac{1}{2\pi}\int_{\RR} e^{i\xi x}\left(\frac{2}{1 + \xi^{2}}\right) d\xi = e^{- |x|}
\end{equation*}
Sapendo questo, deduciamo quanto vale
\begin{equation*}
\Fc \left\{\frac{1}{1 + x^{2}}\right\} = \int_{\RR} e^{- i\xi x}\left(\frac{1}{1 + x^{2}}\right) dx\overset{\text{parità}}{=} \pi e^{- | \xi |}
\end{equation*}

\Soluzione

\begin{equation*}
\begin{aligned}
\hat{f} (\xi) & = \int_{\RR} e^{- i\xi x} \chi_{[ - 1, 1]} dx = \int^{1}_{- 1} e^{- i\xi x} dx = \left[ - \frac{1}{i\xi} e^{- i\xi x}\right]^{1}_{- 1} = \\
 & = \frac{1}{i\xi}\left(e^{i\xi} - e^{- i\xi}\right) = \frac{2}{\xi}\frac{e^{i\xi} - e^{- i\xi}}{2i} = \frac{2\sin \xi}{\xi}
\end{aligned}
\end{equation*}
Questa volta $\hat{f} \notin L^{1} (\RR)$ ma $\hat{f} \in L^{2} (\RR)$

Ragionando come prima
\begin{gather*}
f(x) = \frac{1}{2\pi}\int_{\RR} e^{i\xi x}\frac{2\sin(\xi)}{\xi} d\xi \\
\\
\implies \ \ g(x) = \frac{\sin x}{x} \ \ \implies \ \ \hat{g}(\xi) = \pi \chi_{[ - 1, 1]} (\xi)
\end{gather*}

\Soluzione

Abbiamo due diverse
\begin{enumerate}
\item Analisi complessa
\item Equazione differenziale
\end{enumerate}

\textbf{Analisi complessa.}

$f$ reale pari, allora $\hat{f}$ reale pari. Completando il quadrato\footnote{Aggiungiamo e togliamo $\xi^{2} /4$
\begin{equation*}
- i\xi x - x^{2} = - \left(i\xi x + x^{2}\right) = - \left[\left(x + \frac{i\xi}{2}\right)^{2} + \frac{\xi^{2}}{4}\right]
\end{equation*}
}
\begin{equation*}
\hat{f} (\xi) = \int_{\RR} e^{- i\xi x} e^{- x^{2}} dx = e^{- \frac{\xi^{2}}{4}}\int_{\RR} e^{- \left(x + \frac{i\xi}{2}\right)^{2}} dx
\end{equation*}
Noto che la funzione integranda ha la forma di una gaussiana traslata nel piano complesso, integriamo sul seguente percorso.


\begin{figure}[htpb]
	\centering
\tikzset{every picture/.style = {line width = 0.75pt}} %set default line width to 0.75pt

\begin{tikzpicture}[x = 0.75pt, y = 0.75pt, yscale = -1, xscale = 1]
%uncomment if require: \path (0, 178); %set diagram left start at 0, and has height of 178

%Shape: Axis 2D [id:dp3537451156705087]
\draw (162.5, 130.33) - - (441.5, 130.33)(300.5, 14.63) - - (300.5, 169.33) (434.5, 125.33) - - (441.5, 130.33) - - (434.5, 135.33) (295.5, 21.63) - - (300.5, 14.63) - - (305.5, 21.63) ;
%Straight Lines [id:da6355301965678406]
\draw [color = {rgb, 255:red, 208; green, 2; blue, 27}, draw opacity = 1 ][line width = 1.5] (203.5, 130.33) - - (397.5, 130.33) ;
\draw [shift = {(300.5, 130.33)}, rotate = 180] [fill = {rgb, 255:red, 208; green, 2; blue, 27}, fill opacity = 1 ][line width = 0.08] [draw opacity = 0] (13.4, - 6.43) - - (0, 0) - - (13.4, 6.44) - - (8.9, 0) - - cycle ;
%Straight Lines [id:da04659866719258687]
\draw [color = {rgb, 255:red, 208; green, 2; blue, 27}, draw opacity = 1 ][line width = 1.5] (397.5, 70.33) - - (203.5, 70.33) ;
\draw [shift = {(300.5, 70.33)}, rotate = 360] [fill = {rgb, 255:red, 208; green, 2; blue, 27}, fill opacity = 1 ][line width = 0.08] [draw opacity = 0] (13.4, - 6.43) - - (0, 0) - - (13.4, 6.44) - - (8.9, 0) - - cycle ;
%Straight Lines [id:da2855621428556525]
\draw [color = {rgb, 255:red, 208; green, 2; blue, 27}, draw opacity = 1 ][line width = 1.5] (397.5, 130.33) - - (397.5, 70.33) ;
\draw [shift = {(397.5, 100.33)}, rotate = 450] [fill = {rgb, 255:red, 208; green, 2; blue, 27}, fill opacity = 1 ][line width = 0.08] [draw opacity = 0] (13.4, - 6.43) - - (0, 0) - - (13.4, 6.44) - - (8.9, 0) - - cycle ;
%Straight Lines [id:da09072689237424836]
\draw [color = {rgb, 255:red, 208; green, 2; blue, 27}, draw opacity = 1 ][line width = 1.5] (203.5, 70.33) - - (203.5, 130.33) ;
\draw [shift = {(203.5, 100.33)}, rotate = 270] [fill = {rgb, 255:red, 208; green, 2; blue, 27}, fill opacity = 1 ][line width = 0.08] [draw opacity = 0] (13.4, - 6.43) - - (0, 0) - - (13.4, 6.44) - - (8.9, 0) - - cycle ;

% Text Node
\draw (392, 143.4) node [anchor = north west][inner sep = 0.75pt] {$R$};
% Text Node
\draw (190, 143.4) node [anchor = north west][inner sep = 0.75pt] {$ - R$};
% Text Node
\draw (279, 39.4) node [anchor = north west][inner sep = 0.75pt] {$i\frac{\pi}{2}$};
% Text Node
\draw (434, 39.4) node [anchor = north west][inner sep = 0.75pt] [color = {rgb, 255:red, 208; green, 2; blue, 27}, opacity = 1 ] {$\gamma_{R}$};
% Text Node
\draw (317, 134.4) node [anchor = north west][inner sep = 0.75pt] {$\gamma_{1}$};
% Text Node
\draw (411, 86.4) node [anchor = north west][inner sep = 0.75pt] {$\gamma_{2}$};
% Text Node
\draw (320, 42.4) node [anchor = north west][inner sep = 0.75pt] {$\gamma_{3}$};
% Text Node
\draw (175, 91.4) node [anchor = north west][inner sep = 0.75pt] {$\gamma_{4}$};


\end{tikzpicture}
\end{figure}
\FloatBarrier

La funzione non ha singolarità
\begin{equation*}
\int_{\gamma_{R}} e^{- z^{2}} \ dz = 0
\end{equation*}
che è uguale alla somma degli integrali
\begin{equation*}
\int_{\gamma_{R}} e^{- z^{2}} \ dz = \int_{\gamma_{1}} + \int_{\gamma_{2}} + \int_{\gamma_{3}} + \int_{\gamma_{4}}\xrightarrow{R\rightarrow + \infty}\int_{\RR} e^{- x^{2}} dx - \int_{\RR} e^{- \left(x + \frac{i\xi}{2}\right)^{2}} dx
\end{equation*}
quindi questi due integrali sono uguali al limite
\begin{equation*}
\int_{\RR} e^{- \left(x + \frac{i\xi}{2}\right)^{2}} dx = \int_{\RR} e^{- x^{2}} dx = \sqrt{\pi}
\end{equation*}
Quindi
\begin{equation*}
\boxed{\Fc \left\{e^{- x^{2}}\right\} = \sqrt{\pi} e^{- \frac{\xi^{2}}{4}}}
\end{equation*}
Deduciamo anche, per $\alpha > 0$
\begin{equation*}
\boxed{\Fc \left\{e^{- \alpha x^{2}}\right\} = \Fc \left\{e^{- (\sqrt{\alpha} x)^{2}}\right\} = \sqrt{\frac{\pi}{\alpha}} e^{- \frac{\xi^{2}}{4\alpha}}}
\end{equation*}
\textbf{Equazione differenziale.}

Costruiamo ad hoc un'equazione differenziale
\begin{equation*}
u(x) = e^{- x^{2}} \ \ \implies \ \ u^{'} (x) = -2xe^{- x^{2}} \ \ \implies \ \ u^{'} + 2xu = 0
\end{equation*}
Ora
\begin{equation*}
\Fc \{u\} = \hat{u} \ \ \ \ \Fc \{xu\} = i\frac{d}{d\xi}\hat{u} \ \ \ \ \Fc \{u'\} = i\xi \hat{u}'
\end{equation*}
L'equazione differenziale si riscrive
\begin{equation*}
i\xi \hat{u} + 2i\hat{u}' = 0\ \ \implies \ \ \hat{u}' + \frac{\xi}{2}\hat{u} = 0\ \ \implies \ \ \hat{u}(\xi) = Ce^{- \int \frac{\xi}{2} d\xi} = Ce^{- \xi^{2} /4}
\end{equation*}
Determiniamo la costante
\begin{equation*}
C = \hat{u}(0) = \int_{\RR} e^{- i0x} u(x) dx = \int_{\RR} u(x) dx = \int_{\RR} e^{- x^{2}} dx = \sqrt{\pi}
\end{equation*}
Allora
\begin{equation*}
\boxed{\hat{u}(\xi) = \sqrt{\pi} e^{- \xi^{2} /4}}
\end{equation*}

\Soluzione

\fg{0.7}{15-4}
\begin{equation*}
\Fc \left\{\frac{1}{\cosh x}\right\} = \int_{\RR}\frac{e^{- i\xi x}}{\cosh x} dx = \int_{\RR}\frac{2e^{- i\xi x}}{e^{x} + e^{- x}} dx\ \ \implies \ \ f(x) = \frac{2e^{- i\xi z}}{e^{z} + e^{- z}}
\end{equation*}
Tutte le singolarità sono tutti i punti del piano complesso in cui si annulla il denominatore
\begin{equation*}
e^{z} + e^{- z} = e^{- z} (e^{2z} + 1) = 0\ \ \iff \ \ 2z = i\pi + 2k\pi \ \ \iff \ \ z = i\frac{\pi}{2} + k\pi, \ \ k\in \ZZ
\end{equation*}
Disegno il grafico con il solito rettangolo della gamma che contenga solo un residuo.


\begin{figure}[htpb]
	\centering
\tikzset{every picture/.style = {line width = 0.75pt}} %set default line width to 0.75pt

\begin{tikzpicture}[x = 0.75pt, y = 0.75pt, yscale = -1, xscale = 1]
%uncomment if require: \path (0, 178); %set diagram left start at 0, and has height of 178

%Shape: Axis 2D [id:dp7510536935747429]
\draw (162.5, 130.33) - - (441.5, 130.33)(300.5, 14.63) - - (300.5, 169.33) (434.5, 125.33) - - (441.5, 130.33) - - (434.5, 135.33) (295.5, 21.63) - - (300.5, 14.63) - - (305.5, 21.63) ;
%Straight Lines [id:da06806358271411694]
\draw [color = {rgb, 255:red, 208; green, 2; blue, 27}, draw opacity = 1 ][line width = 1.5] (203.5, 130.33) - - (397.5, 130.33) ;
\draw [shift = {(300.5, 130.33)}, rotate = 180] [fill = {rgb, 255:red, 208; green, 2; blue, 27}, fill opacity = 1 ][line width = 0.08] [draw opacity = 0] (13.4, - 6.43) - - (0, 0) - - (13.4, 6.44) - - (8.9, 0) - - cycle ;
%Straight Lines [id:da47939213158777694]
\draw [color = {rgb, 255:red, 208; green, 2; blue, 27}, draw opacity = 1 ][line width = 1.5] (397.5, 70.33) - - (203.5, 70.33) ;
\draw [shift = {(300.5, 70.33)}, rotate = 360] [fill = {rgb, 255:red, 208; green, 2; blue, 27}, fill opacity = 1 ][line width = 0.08] [draw opacity = 0] (13.4, - 6.43) - - (0, 0) - - (13.4, 6.44) - - (8.9, 0) - - cycle ;
%Straight Lines [id:da9055165704816575]
\draw [color = {rgb, 255:red, 208; green, 2; blue, 27}, draw opacity = 1 ][line width = 1.5] (397.5, 130.33) - - (397.5, 70.33) ;
\draw [shift = {(397.5, 100.33)}, rotate = 450] [fill = {rgb, 255:red, 208; green, 2; blue, 27}, fill opacity = 1 ][line width = 0.08] [draw opacity = 0] (13.4, - 6.43) - - (0, 0) - - (13.4, 6.44) - - (8.9, 0) - - cycle ;
%Straight Lines [id:da5009557145102899]
\draw [color = {rgb, 255:red, 208; green, 2; blue, 27}, draw opacity = 1 ][line width = 1.5] (203.5, 70.33) - - (203.5, 130.33) ;
\draw [shift = {(203.5, 100.33)}, rotate = 270] [fill = {rgb, 255:red, 208; green, 2; blue, 27}, fill opacity = 1 ][line width = 0.08] [draw opacity = 0] (13.4, - 6.43) - - (0, 0) - - (13.4, 6.44) - - (8.9, 0) - - cycle ;
%Shape: Circle [id:dp4183664382061272]
\draw [fill = {rgb, 255:red, 0; green, 0; blue, 0}, fill opacity = 1 ] (299, 100.33) .. controls (299, 99.5) and (299.67, 98.83) .. (300.5, 98.83) .. controls (301.33, 98.83) and (302, 99.5) .. (302, 100.33) .. controls (302, 101.16) and (301.33, 101.83) .. (300.5, 101.83) .. controls (299.67, 101.83) and (299, 101.16) .. (299, 100.33) - - cycle ;

% Text Node
\draw (392, 143.4) node [anchor = north west][inner sep = 0.75pt] {$R$};
% Text Node
\draw (190, 143.4) node [anchor = north west][inner sep = 0.75pt] {$ - R$};
% Text Node
\draw (277, 86.4) node [anchor = north west][inner sep = 0.75pt] {$i\frac{\pi}{2}$};
% Text Node
\draw (373, 39.4) node [anchor = north west][inner sep = 0.75pt] [color = {rgb, 255:red, 208; green, 2; blue, 27}, opacity = 1 ] {$\gamma_{R}$};
% Text Node
\draw (317, 134.4) node [anchor = north west][inner sep = 0.75pt] {$\gamma_{1}$};
% Text Node
\draw (409, 84.4) node [anchor = north west][inner sep = 0.75pt] {$\gamma_{2}$};
% Text Node
\draw (317, 44.4) node [anchor = north west][inner sep = 0.75pt] {$\gamma_{3}$};
% Text Node
\draw (175, 89.4) node [anchor = north west][inner sep = 0.75pt] {$\gamma_{4}$};


\end{tikzpicture}
\end{figure}
\FloatBarrier

Calcolo
\begin{equation*}
\begin{aligned}
\int_{\gamma_{R}} f(z)dz & = 2\pi i \cdot \Res \left(f, z = \frac{i\pi}{2}\right) = 2\pi i\left. \frac{2e^{- i\xi z}}{e^{z} - e^{- z}}\right|_{z = i\pi /2}\\
 & = 2\pi i \cdot \frac{2e^{\xi \frac{\pi}{2}}}{e^{\frac{i\pi}{2}} - e^{- \frac{i\pi}{2}}} = 2\pi i \cdot \frac{2e^{\xi \frac{\pi}{2}}}{i - (- i)} = 2\pi e^{\xi \frac{\pi}{2}}
\end{aligned}
\end{equation*}
Scriviamo i vari pezzi
\begin{align*}
\int_{\gamma_{R}} f(z) dz & = 2\pi e^{\frac{\xi \pi}{2}} = \int_{\gamma_{1}} + \int_{\gamma_{2}} + \int_{\gamma_{3}} + \int_{\gamma_{4}}\\
 & \\
\int_{\gamma_{1}} & = \int^{R}_{- R}\frac{e^{- i\xi x}}{\cosh (x)} dx\xrightarrow{R\rightarrow + \infty}\Fc \left\{\frac{1}{\cosh (x)}\right\}\\
 & \\
\int_{\gamma_{2}} & = 2\int^{\pi i}_{0}\frac{e^{- i\xi (R + iy)}}{e^{R + iy} + e^{- R - iy}} idy\\
\left| \int_{\gamma_{2}}\right| & = 2\left| \int^{\pi}_{0}\frac{e^{- i\xi R} e^{y}}{e^{R}\left(e^{iy} + e^{- 2R - iy}\right)} idy\right| \leq \frac{2}{e^{R}}\underbrace{\int^{\pi}_{0}\left| \frac{e^{y}}{e^{iy} + e^{- 2R - iy}}\right| dy}_{\text{limitata}}\xrightarrow{R\rightarrow + \infty} 0\\
 & \\
\int_{\gamma_{3}} = & - \int^{R}_{- R}\frac{2e^{- i\xi (x + i\pi)}}{e^{x + i\pi} + e^{- x - i\pi}} dx = \int^{R}_{- R}\frac{2e^{- i\xi x} e^{\pi \xi}}{e^{x} + e^{- x}} dx\\
 & = e^{\pi \xi}\int^{R}_{- R}\frac{e^{- i\xi x}}{\frac{e^{x} + e^{- x}}{2}} dx\xrightarrow{R\rightarrow + \infty} e^{\pi \xi}\Fc \left\{\frac{1}{\cosh (x)}\right\}\\
 & \\
\left| \int_{\gamma_{4}}\right| & \xrightarrow{R\rightarrow + \infty} 0
\end{align*}
In totale avremo
\begin{equation*}
2\pi e^{\xi \frac{\pi}{2}} = \int_{\gamma_{R}} f(z) dz\xrightarrow{R\rightarrow + \infty}\Fc \left\{\frac{1}{\cosh (x)}\right\}\left(1 + e^{\pi \xi}\right)
\end{equation*}
Dunque
\begin{equation*}
\Fc \left\{\frac{1}{\cosh (x)}\right\} = \frac{2\pi e^{\xi \frac{\pi}{2}}}{e^{\pi \xi} + 1} = \frac{2\pi}{e^{\xi \frac{\pi}{2}} + e^{- \xi \frac{\pi}{2}}} = \frac{\pi}{\cosh\left(\frac{\pi}{2} \xi \right)}
\end{equation*}

\Soluzione

\fg{0.7}{15-5}
\begin{equation*}
\Fc \left\{\frac{1}{\left(1 + x^{2}\right)^{2}}\right\} = \int_{\RR}\frac{e^{- i\xi x}}{\left(1 + x^{2}\right)^{2}} dx
\end{equation*}
$f$ pari, allora $\hat{f}$ pari.

Poniamo\footnote{Con $f$ si indica sia la funzione data che quella integranda, ma sono due funzioni che differiscono per il numeratore. Non facciamo troppo i pignoli.}
\begin{equation*}
f(z) = \frac{e^{- i\xi z}}{\left(1 + z^{2}\right)^{2}}
\end{equation*}
$z = \pm i$ sono poli del II ordine.

Giochiamo con i residui utilizzando il Lemma di Jordan
\begin{equation*}
\int_{\gamma_{R}} g(z) e^{i\alpha z} dz
\end{equation*}
\textit{Quale circonferenza bisogna prendere?}

Se $\alpha $ è positivo il lemma di Jordan parla della semicirconferenza superiore mentre se $\alpha $ è negativo della semicirconferenza inferiore.

Nel nostro caso si parla di $ - \xi $ per calcolare quanto vale questa trasformata avremo che se $\xi $ è negativo avremo bisogno del semipiano superiore, se $\xi $ è positivo di quello inferiore. Essendo simmetrici si sfrutta il risultato teorico della parità.


\begin{figure}[htpb]
	\centering
\tikzset{every picture/.style = {line width = 0.75pt}} %set default line width to 0.75pt

\begin{tikzpicture}[x = 0.75pt, y = 0.75pt, yscale = -1, xscale = 1]
%uncomment if require: \path (0, 172); %set diagram left start at 0, and has height of 172

%Shape: Axis 2D [id:dp752436124621402]
\draw (110, 80.25) - - (490, 80.25)(300, 10) - - (300, 160) (483, 75.25) - - (490, 80.25) - - (483, 85.25) (295, 17) - - (300, 10) - - (305, 17) ;
%Shape: Arc [id:dp579891374669032]
\draw [draw opacity = 0][line width = 1.5] (369.5, 80.77) .. controls (369.22, 118.92) and (338.21, 149.75) .. (300, 149.75) .. controls (261.62, 149.75) and (230.5, 118.63) .. (230.5, 80.25) .. controls (230.5, 80.16) and (230.5, 80.07) .. (230.5, 79.98) - - (300, 80.25) - - cycle ; \draw [color = {rgb, 255:red, 208; green, 2; blue, 27}, draw opacity = 1 ][line width = 1.5] (369.5, 80.77) .. controls (369.22, 118.92) and (338.21, 149.75) .. (300, 149.75) .. controls (261.62, 149.75) and (230.5, 118.63) .. (230.5, 80.25) .. controls (230.5, 80.16) and (230.5, 80.07) .. (230.5, 79.98) ;
%Shape: Circle [id:dp9196698365325671]
\draw [fill = {rgb, 255:red, 0; green, 0; blue, 0}, fill opacity = 1 ] (298.75, 109.99) .. controls (298.75, 109.16) and (299.42, 108.49) .. (300.25, 108.49) .. controls (301.08, 108.49) and (301.75, 109.16) .. (301.75, 109.99) .. controls (301.75, 110.82) and (301.08, 111.49) .. (300.25, 111.49) .. controls (299.42, 111.49) and (298.75, 110.82) .. (298.75, 109.99) - - cycle ;
\draw [draw opacity = 0][fill = {rgb, 255:red, 208; green, 2; blue, 27}, fill opacity = 1 ] (326.2, 86.94) - - (311.45, 79.56) - - (326.2, 72.19) - - (318.83, 79.56) - - cycle ;
\draw [draw opacity = 0][fill = {rgb, 255:red, 208; green, 2; blue, 27}, fill opacity = 1 ] (335.24, 131.55) - - (351.34, 127.96) - - (344.55, 143) - - (345.62, 132.62) - - cycle ;
%Straight Lines [id:da3950283299762536]
\draw [color = {rgb, 255:red, 208; green, 2; blue, 27}, draw opacity = 1 ][line width = 1.5] (230.5, 79.98) - - (370, 80) ;

% Text Node
\draw (211, 57.4) node [anchor = north west][inner sep = 0.75pt] {$ - R$};
% Text Node
\draw (363, 57.4) node [anchor = north west][inner sep = 0.75pt] {$R$};
% Text Node
\draw (274, 99.4) node [anchor = north west][inner sep = 0.75pt] [font = \normalsize] {$ - i$};


\end{tikzpicture}
\end{figure}
\FloatBarrier

Per $\xi > 0$
\begin{equation*}
\int_{R}\frac{e^{- i\xi x}}{\left(1 + x^{2}\right)^{2}} dx = -2\pi i \cdot \Res (f, z = -i)
\end{equation*}
Il segno meno è giustificato dal fatto che percorriamo la semicirconferenza in senso \textit{antiorario}.

Il residuo allora sarà pari a
\begin{equation*}
\begin{aligned}
\Res (f, z = -i) & = \left. \frac{d}{dz}\frac{e^{- i\xi z}}{(z - i)^{2}}\right|_{z = -i}\\
 & = \left. \frac{- i\xi e^{- i\xi z}(z - i)^{2} - e^{- i\xi z} 2(z - i)}{(z - i)^{4}}\right|_{z = -i}\\
 & = \frac{e^{- \xi}(4i + 4i\xi)}{16} = i\frac{e^{- \xi}}{4} (1 + \xi)
\end{aligned}
\end{equation*}
La trasformata vale quindi
\begin{equation*}
\Fc \left\{\frac{1}{\left(1 + x^{2}\right)^{2}}\right\} = - 2\pi i \cdot i\frac{e^{- \xi}}{4} (1 + \xi) = \frac{\pi}{2} e^{- \xi}(1 + \xi) \ \ \ \ \xi > 0
\end{equation*}
Per le $\xi < 0$ basta raddoppiare
\begin{equation*}
\hat{f}(\xi) = \frac{\pi}{2} e^{- | \xi |} (1 + | \xi |)
\end{equation*}

\Soluzione

Vediamo il grafico di $u(x) = 1/\sqrt{|x|}$

\fg{0.7}{15-6}

Notiamo che $u\notin L^{1}$ perché all'infinito va a zero troppo lentamente e neanche $u\notin L^{2}$, in tal caso sarebbe asintotica a $1/x$ per $x\rightarrow \infty $, che non è integrabile.

Possiamo considerare $u$ come una distribuzione temperata, cioè $u\in \Sc^{'}(\RR)$.

Possiamo prendere una successione di funzioni $u_{n}$ che l'approssimano nel senso delle distribuzioni temperate
\begin{equation*}
u_{n}(x) = u(x) \chi_{(- n, n)}(x) \ \ \ \ u_{n} \in L^{1} (\RR)
\end{equation*}
Considerando la dualità con una qualsiasi funzione dello spazio di Schwartz questa sarà pari a:
\begin{gather*}
\langle (u - u_{n}), \psi \rangle = \int^{- n}_{- \infty}\frac{1}{\sqrt{| x|}} \psi (x) dx + \ \int^{+ \infty}_{n}\frac{1}{\sqrt{x}} \psi (x) dx\\
\\
| \langle (u - u_{n}), \ \psi \rangle | \leq \frac{1}{\sqrt{n}} \ \int_{\RR}| \psi (x)| dx\xrightarrow{n\rightarrow \infty} 0\ \ \implies \ \ u_{n}\xrightarrow[\Sc']{n\rightarrow \infty} u
\end{gather*}
Ne consegue che possiamo scrivere:
\begin{equation*}
\Fc \{u\} = \lim_{n\rightarrow \infty}\Fc \{u_{n}\}
\end{equation*}
Calcoliamo le trasformate delle $u_{n}$ in senso classico
\begin{equation*}
\Fc \{u_{n}\} = \int_{\RR}\frac{e^{- i\xi x}}{\sqrt{|} x|} \chi_{[ - n, n]} (u)dx = \int^{n}_{- n}\frac{\cos(\xi x) - i\sin (\xi x)}{\sqrt{| x|}} dx = 2\int^{n}_{0}\frac{\cos(\xi x)}{\sqrt{x}} dx
\end{equation*}
Sostituzione
\begin{equation*}
\xi x = t^{2} \ \ \implies \ \ dx = \frac{2}{\xi} tdt
\end{equation*}
Allora
\begin{equation*}
= 2\int^{\sqrt{\xi n}}_{0}\frac{\cos\left(t^{2}\right)}{\sqrt{\frac{t^{2}}{\xi}}}\frac{2t}{\xi} dt = \frac{4}{\sqrt{\xi}} \ \int^{\sqrt{\xi n}}_{0}\cos\left(t^{2}\right) dt =
\end{equation*}
Ricordiamo l'integrale di Fresnel
\begin{equation*}
\int^{+ \infty}_{0}\cos\left(x^{2}\right) dx = \sqrt{\frac{\pi}{8}}
\end{equation*}
Quindi, la trasformata sarà pari a
\begin{equation*}
\Fc \{u\} = \lim_{n\rightarrow \infty}\Fc \{u_{n}\} = \lim_{n\rightarrow \infty}\frac{4}{\sqrt{| \xi |}} \ \int^{\sqrt{| \xi | n}}_{0}\cos\left(t^{2}\right) dt = \frac{4}{\sqrt{| \xi |}} \ \sqrt{\frac{\pi}{8}} = \ \sqrt{\frac{2\pi}{| \xi |}}
\end{equation*}
\chapter{Esercitazione 8 - Potrich}

\ParteEsercizi

\begin{defn}
Sia $u\in L^{1}(\RR)$. Si dice \textbf{Trasformata di Fourier} di $u$ la funzione $\hat{u} :\RR \rightarrow \CC $ definita da
\begin{equation*}
\hat{u}(\xi) = \int_{\RR} e^{- i\xi x} u(x) dx
\end{equation*}
\end{defn}
\begin{thm}
[Proprietà] Ecco alcune utili proprietà della trasformata di Fourier.
\begin{itemize}
\item Scaling
\begin{equation*}
u(ax) \ \ \rightarrow \ \ \frac{1}{| a|}\hat{u}\left(\frac{\xi}{a}\right)
\end{equation*}
\item Shifting
\begin{equation*}
u(x - a) \ \ \rightarrow \ \ e^{- ia\xi}\hat{u}(\xi)
\end{equation*}
\item Modulazione
\begin{gather*}
u(x) e^{iax} \ \ \rightarrow \ \ \hat{u}(\xi - a)\\
u(x)\cos(\xi_{0} x) \ \ \rightarrow \ \ \frac{1}{2}[\hat{u}(\xi - \xi_{0}) + \hat{u}(\xi + \xi_{0})]
\end{gather*}
\item Linearità
\begin{equation*}
\Fc \{\alpha u + \beta v\} = \alpha \Fc \{u\} + \beta \Fc \{v\} \ \ \forall \alpha, \beta \in \RR, \ \forall u, v\in L^{1}(\RR)
\end{equation*}
\item Derivazione
\begin{equation*}
\widehat{u'}(\xi) = i\xi \hat{u}(\xi)
\end{equation*}
\end{itemize}
\end{thm}

\Esercizio{Una trasformata notevole}

Calcolare la trasformata di Fourier di
\begin{equation*}
u(x) = \frac{1}{1 + x^{2}}, \ \ x\in \RR
\end{equation*}

\Esercizio{Trasformata della Gaussiana}

Calcolare la trasformata di Fourier della funzione Gaussiana
\begin{equation*}
u(x) = e^{- x^{2}}, \ x\in \RR
\end{equation*}

\Esercizio{}

Calcolare la trasformata di Fourier di
\begin{equation*}
f(x) = \frac{\sin x}{\left(1 + x^{2}\right)\left(4 + x^{2}\right)}
\end{equation*}

\Esercizio{}

Calcolare la trasformata di Fourier di
\begin{equation*}
u(x) = \frac{2\cos x}{1 + x^{2}}
\end{equation*}

\Esercizio{}

Determinare l'espressione analitica di
\begin{equation*}
f(\tau) = \int_{\RR} e^{- i\left(\tau^{2} - 2\right) x}\frac{\cos(2x)}{1 + x^{2}} dx
\end{equation*}

\Esercizio{}

Sia
\begin{equation*}
f(x) =
\begin{cases}
1, & 0 < x < 1\\
- 1, & - 1 < x < 0
\end{cases}
\end{equation*}
\begin{enumerate}
\item Calcolare la trasformata di Fourier di $f(x)$
\item Calcolare la trasformata di Fourier di $g(x) = xf(x)$
\end{enumerate}

\Esercizio{}

Sia $H$ la funzione di Heaviside
\begin{equation*}
H(t) =
\begin{cases}
1, & t \geq 0\\
0, & t < 0
\end{cases}
\end{equation*}
\begin{enumerate}
\item Calcolare la trasformata di Fourier di
\begin{equation*}
f(x) = \left(1 - x^{2}\right) H\left(1 - x^{2}\right)
\end{equation*}
\item Utilizzare il risultato trovato per calcolare
\begin{equation*}
\int^{+ \infty}_{0}\frac{x\cos x - \sin x}{x^{3}}\cos\left(\frac{x}{2}\right) dx
\end{equation*}
\end{enumerate}

\ParteSoluzioni

\Soluzione

Per definizione
\begin{equation*}
\hat{u}(\xi) = \int_{\RR} e^{- i\xi x} u(x) dx
\end{equation*}
Consideriamo la funzione analitica
\begin{equation*}
f(z) = \frac{1}{1 + z^{2}}
\end{equation*}
le cui singolarità si trovano in
\begin{equation*}
1 + z^{2} = 0\ \ \iff \ \ z = \pm i
\end{equation*}
e sono dei poli semplici.


\begin{figure}[htpb]
	\centering
\tikzset{every picture/.style = {line width = 0.75pt}} %set default line width to 0.75pt

\begin{tikzpicture}[x = 0.75pt, y = 0.75pt, yscale = -1, xscale = 1]
%uncomment if require: \path (0, 181); %set diagram left start at 0, and has height of 181

%Shape: Axis 2D [id:dp6994241881147589]
\draw (162.5, 130.33) - - (441.5, 130.33)(300.5, 14.63) - - (300.5, 169.33) (434.5, 125.33) - - (441.5, 130.33) - - (434.5, 135.33) (295.5, 21.63) - - (300.5, 14.63) - - (305.5, 21.63) ;
%Straight Lines [id:da47956119812522835]
\draw [color = {rgb, 255:red, 208; green, 2; blue, 27}, draw opacity = 1 ][line width = 1.5] (300.5, 130.75) - - (397.5, 130.75) ;
\draw [shift = {(349, 130.75)}, rotate = 180] [fill = {rgb, 255:red, 208; green, 2; blue, 27}, fill opacity = 1 ][line width = 0.08] [draw opacity = 0] (13.4, - 6.43) - - (0, 0) - - (13.4, 6.44) - - (8.9, 0) - - cycle ;
%Shape: Arc [id:dp5355059210550983]
\draw [draw opacity = 0][line width = 1.5] (203.5, 130.33) .. controls (203.5, 76.76) and (246.93, 33.33) .. (300.5, 33.33) .. controls (354.07, 33.33) and (397.5, 76.76) .. (397.5, 130.33) - - (300.5, 130.33) - - cycle ; \draw [color = {rgb, 255:red, 208; green, 2; blue, 27}, draw opacity = 1 ][line width = 1.5] (203.5, 130.33) .. controls (203.5, 76.76) and (246.93, 33.33) .. (300.5, 33.33) .. controls (354.07, 33.33) and (397.5, 76.76) .. (397.5, 130.33) ;
%Straight Lines [id:da2269425139728658]
\draw [color = {rgb, 255:red, 208; green, 2; blue, 27}, draw opacity = 1 ][line width = 1.5] (203.5, 130.33) - - (300.5, 130.33) ;
\draw [shift = {(252, 130.33)}, rotate = 180] [fill = {rgb, 255:red, 208; green, 2; blue, 27}, fill opacity = 1 ][line width = 0.08] [draw opacity = 0] (13.4, - 6.43) - - (0, 0) - - (13.4, 6.44) - - (8.9, 0) - - cycle ;
\draw [draw opacity = 0][fill = {rgb, 255:red, 208; green, 2; blue, 27}, fill opacity = 1 ] (362.12, 66.27) - - (352.31, 47.43) - - (373.27, 50.88) - - (360, 53) - - cycle ;
%Shape: Circle [id:dp54653528391848]
\draw [draw opacity = 0][fill = {rgb, 255:red, 0; green, 0; blue, 0}, fill opacity = 1 ] (297.5, 100.75) .. controls (297.5, 99.1) and (298.84, 97.75) .. (300.5, 97.75) .. controls (302.16, 97.75) and (303.5, 99.1) .. (303.5, 100.75) .. controls (303.5, 102.41) and (302.16, 103.75) .. (300.5, 103.75) .. controls (298.84, 103.75) and (297.5, 102.41) .. (297.5, 100.75) - - cycle ;
%Shape: Circle [id:dp23929766105583639]
\draw [draw opacity = 0][fill = {rgb, 255:red, 0; green, 0; blue, 0}, fill opacity = 1 ] (297.5, 160.75) .. controls (297.5, 159.1) and (298.84, 157.75) .. (300.5, 157.75) .. controls (302.16, 157.75) and (303.5, 159.1) .. (303.5, 160.75) .. controls (303.5, 162.41) and (302.16, 163.75) .. (300.5, 163.75) .. controls (298.84, 163.75) and (297.5, 162.41) .. (297.5, 160.75) - - cycle ;

% Text Node
\draw (392, 133.4) node [anchor = north west][inner sep = 0.75pt] {$R$};
% Text Node
\draw (190, 133.4) node [anchor = north west][inner sep = 0.75pt] {$ - R$};
% Text Node
\draw (310.5, 90.9) node [anchor = north west][inner sep = 0.75pt] {$i$};
% Text Node
\draw (307, 150.4) node [anchor = north west][inner sep = 0.75pt] {$ - i$};


\end{tikzpicture}
\end{figure}
\FloatBarrier

Osserviamo che $| f(z)| \sim \frac{1}{z^{2}}$ per $| z| \rightarrow + \infty $, allora col teorema dei residui e il lemma di Jordan
\begin{equation*}
\int^{+ \infty}_{- \infty} e^{- i\xi x}\frac{1}{1 + x^{2}} dx = 2\pi i \cdot \Res \left(\frac{e^{- i\xi z}}{1 + z^{2}}, z = i\right) = \pi e^{\xi} \ \ \forall \xi < 0
\end{equation*}
\begin{rem}
Se $f$ è pari, allora $\hat{f}$ è pari.
\end{rem}
I valori per $\xi > 0$ si ottengono da quelli negativi imponendo che $\hat{f}$ sia pari
\begin{rem}
$f\in L^{1} \implies \hat{f}$ è continua.
\end{rem}
Il valore in $0$ si ottiene per continuità.
\begin{equation*}
\implies \ \ \boxed{\hat{f}(\xi) = \pi e^{- | \xi |} \ \ \forall \xi \in \RR}
\end{equation*}

\Soluzione

\begin{equation*}
\hat{u}(\xi) = \int_{\RR} e^{- i\xi x} u(x) dx = \int_{\RR} e^{- i\xi x} e^{- x^{2}} dx
\end{equation*}
\textbf{\underline{Metodo con l'analisi complessa}}
\begin{equation*}
\hat{u}(\xi) = \int_{\RR} e^{- i\xi x} e^{- x^{2}} dx = \int_{\RR} e^{- \left(i\xi x + x^{2}\right)} dx
\end{equation*}
Completiamo il quadrato
\begin{equation*}
\int_{\RR} e^{- \left(i\xi x + x^{2}\right)} dx = e^{- \frac{\xi^{2}}{4}}\int_{\RR} e^{- \left(x + \frac{i\xi}{2}\right)^{2}} dx
\end{equation*}
Sfruttiamo la seguente uguaglianza
\begin{equation*}
\int^{+ \infty}_{- \infty} e^{- (x + ib)^{2}} dx\overset{(\star)}{=}\int^{+ \infty}_{- \infty} e^{- x^{2}} dx = \sqrt{\pi} \ \ \implies \ \ \boxed{\hat{u}(\xi) = \sqrt{\pi} e^{- \frac{\xi^{2}}{4}}}
\end{equation*}
Resta da provare $(\star)$. Consideriamo la funzione analitica intera (senza singolarità) $z\mapsto e^{- z^{2}}$, prolungamento analitico di $e^{- x^{2}}, x\in \RR $.


\begin{figure}[htpb]
	\centering
\tikzset{every picture/.style = {line width = 0.75pt}} %set default line width to 0.75pt

\begin{tikzpicture}[x = 0.75pt, y = 0.75pt, yscale = -1, xscale = 1]
%uncomment if require: \path (0, 178); %set diagram left start at 0, and has height of 178

%Shape: Axis 2D [id:dp08896652720756437]
\draw (162.5, 130.33) - - (441.5, 130.33)(300.5, 14.63) - - (300.5, 169.33) (434.5, 125.33) - - (441.5, 130.33) - - (434.5, 135.33) (295.5, 21.63) - - (300.5, 14.63) - - (305.5, 21.63) ;
%Straight Lines [id:da3929751532760877]
\draw [color = {rgb, 255:red, 208; green, 2; blue, 27}, draw opacity = 1 ][line width = 1.5] (203.5, 130.33) - - (397.5, 130.33) ;
\draw [shift = {(300.5, 130.33)}, rotate = 180] [fill = {rgb, 255:red, 208; green, 2; blue, 27}, fill opacity = 1 ][line width = 0.08] [draw opacity = 0] (13.4, - 6.43) - - (0, 0) - - (13.4, 6.44) - - (8.9, 0) - - cycle ;
%Straight Lines [id:da9171312565843239]
\draw [color = {rgb, 255:red, 208; green, 2; blue, 27}, draw opacity = 1 ][line width = 1.5] (397.5, 70.33) - - (203.5, 70.33) ;
\draw [shift = {(300.5, 70.33)}, rotate = 360] [fill = {rgb, 255:red, 208; green, 2; blue, 27}, fill opacity = 1 ][line width = 0.08] [draw opacity = 0] (13.4, - 6.43) - - (0, 0) - - (13.4, 6.44) - - (8.9, 0) - - cycle ;
%Straight Lines [id:da891206314069279]
\draw [color = {rgb, 255:red, 208; green, 2; blue, 27}, draw opacity = 1 ][line width = 1.5] (397.5, 130.33) - - (397.5, 70.33) ;
\draw [shift = {(397.5, 100.33)}, rotate = 450] [fill = {rgb, 255:red, 208; green, 2; blue, 27}, fill opacity = 1 ][line width = 0.08] [draw opacity = 0] (13.4, - 6.43) - - (0, 0) - - (13.4, 6.44) - - (8.9, 0) - - cycle ;
%Straight Lines [id:da5821046363194966]
\draw [color = {rgb, 255:red, 208; green, 2; blue, 27}, draw opacity = 1 ][line width = 1.5] (203.5, 70.33) - - (203.5, 130.33) ;
\draw [shift = {(203.5, 100.33)}, rotate = 270] [fill = {rgb, 255:red, 208; green, 2; blue, 27}, fill opacity = 1 ][line width = 0.08] [draw opacity = 0] (13.4, - 6.43) - - (0, 0) - - (13.4, 6.44) - - (8.9, 0) - - cycle ;

% Text Node
\draw (392, 143.4) node [anchor = north west][inner sep = 0.75pt] {$R$};
% Text Node
\draw (190, 143.4) node [anchor = north west][inner sep = 0.75pt] {$ - R$};
% Text Node
\draw (182, 107.4) node [anchor = north west][inner sep = 0.75pt] {$A$};
% Text Node
\draw (182, 49.4) node [anchor = north west][inner sep = 0.75pt] {$D$};
% Text Node
\draw (402, 51.4) node [anchor = north west][inner sep = 0.75pt] {$C$};
% Text Node
\draw (406, 109.4) node [anchor = north west][inner sep = 0.75pt] {$B$};
% Text Node
\draw (283, 49.4) node [anchor = north west][inner sep = 0.75pt] {$ib$};
% Text Node
\draw (348, 25.4) node [anchor = north west][inner sep = 0.75pt] [color = {rgb, 255:red, 208; green, 2; blue, 27}, opacity = 1 ] {$\Gamma_{R}$};


\end{tikzpicture}
\end{figure}
\FloatBarrier

con
\begin{equation*}
\begin{aligned}
A & = - R\\
B & = R\\
C & = R + ib\\
D & = - R + ib
\end{aligned} \ \ \ \ \forall R > 0, \forall b > 0
\end{equation*}
Per il teorema dell'integrale nullo di Cauchy
\begin{equation*}
\oint_{\Gamma_{R}} e^{- z^{2}} dz = 0
\end{equation*}
Sul pezzo $BC$ affermiamo
\begin{equation*}
\lim\limits_{R\rightarrow + \infty}\int^{C}_{B} e^{- z^{2}} dz = 0
\end{equation*}
dato che se $R > 2b$ si ottiene, per ogni $z\in BC$,
\begin{gather*}
z = | z| (\cos \vartheta + i\sin \vartheta)\\
\left| e^{- z^{2}}\right| = e^{\Re \left(- z^{2}\right)} = e^{- | z|^{2}\cos 2\vartheta} \leq e^{- R^{2}\cos 2\vartheta} = e^{- R^{2}\left(\cos^{2} \vartheta - \sin^{2} \vartheta \right)} \leq e^{- \frac{R^{2}}{4}}\xrightarrow{R\rightarrow + \infty} 0
\end{gather*}
Analogamente
\begin{equation*}
\lim\limits_{R\rightarrow + \infty}\int^{A}_{D} e^{- z^{2}} dz = 0
\end{equation*}
Allora possiamo dire
\begin{equation*}
\implies \ \ \int^{B}_{A} e^{- z^{2}} dz = -\int^{D}_{C} e^{- z^{2}} dz + o\left(\frac{1}{R}\right)
\end{equation*}
Passando al limite per $R\rightarrow + \infty $ si ottiene $(\star)$.



\underline{\textbf{Metodo alternativo con equazioni differenziali.}}

Consideriamo l'equazione differenziale lineare omogenea del primo ordine
\begin{equation*}
u'(x) = - 2xu(x), \ \ x\in \RR
\end{equation*}
in quanto $e^{- x^{2}}$ è soluzione di questa equazione differenziale. Andiamo a trasformare tutti i termini di questa equazione.
\begin{equation*}
i\xi \hat{u}(\xi) = - 2i\hat{u}'(\xi) \ \ \implies \ \ \hat{u}(\xi) = Ce^{- \frac{\xi^{2}}{4}}
\end{equation*}
per cui si deve scegliere
\begin{equation*}
C = \hat{u}(0) = \int_{\RR} e^{- i\xi x} e^{- t^{2}} dt\overset{\xi = 0}{=}\int_{\RR} e^{- t^{2}} dt = \sqrt{\pi}
\end{equation*}
Concludiamo che
\begin{equation*}
\boxed{\hat{u}(\xi) = \sqrt{\pi} e^{- \frac{\xi^{2}}{4}}}
\end{equation*}

\Soluzione

Poniamo
\begin{gather*}
u(x) = \frac{1}{\left(1 + x^{2}\right)\left(4 + x^{2}\right)}\\
\implies \ \ \hat{u}(\xi) = \int_{\RR} e^{- i\xi x} u(x) dx = \int^{+ \infty}_{- \infty}\frac{e^{- i\xi x}}{\left(1 + x^{2}\right)\left(4 + x^{2}\right)}
\end{gather*}
Poniamo
\begin{equation*}
u(z) = \frac{1}{\left(1 + z^{2}\right)\left(4 + z^{2}\right)} \in \Hc (\CC \setminus \{\pm i, \pm 2i\})
\end{equation*}
Le singolarità sono poli semplici.


\begin{figure}[htpb]
	\centering
\tikzset{every picture/.style = {line width = 0.75pt}} %set default line width to 0.75pt

\begin{tikzpicture}[x = 0.75pt, y = 0.75pt, yscale = -1, xscale = 1]
%uncomment if require: \path (0, 181); %set diagram left start at 0, and has height of 181

%Shape: Axis 2D [id:dp663384449293001]
\draw (162.5, 130.33) - - (441.5, 130.33)(300.5, 14.63) - - (300.5, 169.33) (434.5, 125.33) - - (441.5, 130.33) - - (434.5, 135.33) (295.5, 21.63) - - (300.5, 14.63) - - (305.5, 21.63) ;
%Straight Lines [id:da95452190554064]
\draw [color = {rgb, 255:red, 208; green, 2; blue, 27}, draw opacity = 1 ][line width = 1.5] (300.5, 130.75) - - (397.5, 130.75) ;
\draw [shift = {(349, 130.75)}, rotate = 180] [fill = {rgb, 255:red, 208; green, 2; blue, 27}, fill opacity = 1 ][line width = 0.08] [draw opacity = 0] (13.4, - 6.43) - - (0, 0) - - (13.4, 6.44) - - (8.9, 0) - - cycle ;
%Shape: Arc [id:dp42287292414644995]
\draw [draw opacity = 0][line width = 1.5] (203.5, 130.33) .. controls (203.5, 76.76) and (246.93, 33.33) .. (300.5, 33.33) .. controls (354.07, 33.33) and (397.5, 76.76) .. (397.5, 130.33) - - (300.5, 130.33) - - cycle ; \draw [color = {rgb, 255:red, 208; green, 2; blue, 27}, draw opacity = 1 ][line width = 1.5] (203.5, 130.33) .. controls (203.5, 76.76) and (246.93, 33.33) .. (300.5, 33.33) .. controls (354.07, 33.33) and (397.5, 76.76) .. (397.5, 130.33) ;
%Straight Lines [id:da4448454277513003]
\draw [color = {rgb, 255:red, 208; green, 2; blue, 27}, draw opacity = 1 ][line width = 1.5] (203.5, 130.33) - - (300.5, 130.33) ;
\draw [shift = {(252, 130.33)}, rotate = 180] [fill = {rgb, 255:red, 208; green, 2; blue, 27}, fill opacity = 1 ][line width = 0.08] [draw opacity = 0] (13.4, - 6.43) - - (0, 0) - - (13.4, 6.44) - - (8.9, 0) - - cycle ;
\draw [draw opacity = 0][fill = {rgb, 255:red, 208; green, 2; blue, 27}, fill opacity = 1 ] (362.12, 66.27) - - (352.31, 47.43) - - (373.27, 50.88) - - (360, 53) - - cycle ;
%Shape: Circle [id:dp1636459542824935]
\draw [draw opacity = 0][fill = {rgb, 255:red, 0; green, 0; blue, 0}, fill opacity = 1 ] (297.5, 100.75) .. controls (297.5, 99.1) and (298.84, 97.75) .. (300.5, 97.75) .. controls (302.16, 97.75) and (303.5, 99.1) .. (303.5, 100.75) .. controls (303.5, 102.41) and (302.16, 103.75) .. (300.5, 103.75) .. controls (298.84, 103.75) and (297.5, 102.41) .. (297.5, 100.75) - - cycle ;
%Shape: Circle [id:dp3693616304321441]
\draw [draw opacity = 0][fill = {rgb, 255:red, 0; green, 0; blue, 0}, fill opacity = 1 ] (297.5, 70.75) .. controls (297.5, 69.1) and (298.84, 67.75) .. (300.5, 67.75) .. controls (302.16, 67.75) and (303.5, 69.1) .. (303.5, 70.75) .. controls (303.5, 72.41) and (302.16, 73.75) .. (300.5, 73.75) .. controls (298.84, 73.75) and (297.5, 72.41) .. (297.5, 70.75) - - cycle ;

% Text Node
\draw (392, 133.4) node [anchor = north west][inner sep = 0.75pt] {$R$};
% Text Node
\draw (190, 133.4) node [anchor = north west][inner sep = 0.75pt] {$ - R$};
% Text Node
\draw (310.5, 90.9) node [anchor = north west][inner sep = 0.75pt] {$i$};
% Text Node
\draw (310.5, 60.9) node [anchor = north west][inner sep = 0.75pt] {$2i$};


\end{tikzpicture}
\end{figure}
\FloatBarrier

Notiamo che $| u(z)| \sim \frac{1}{z^{4}}$ per $| z| \rightarrow + \infty $. Per il teorema dei residui e il lemma di Jordan
\begin{equation*}
\begin{aligned}
\hat{u}(\xi) & = \int^{+ \infty}_{- \infty}\frac{e^{- i\xi x}}{\left(1 + x^{2}\right)\left(4 + x^{2}\right)}\overset{\xi < 0}{=} 2\pi i \cdot \{\Res (u, i) + \Res (u, 2i)\}\\
 & = 2\pi i \cdot \left\{\frac{e^{\xi}}{6i} - \frac{e^{2\xi}}{12i}\right\} = \frac{\pi}{3} e^{\xi} - \frac{\pi}{6} e^{2\xi} \ \ (\xi < 0)
\end{aligned}
\end{equation*}
$u(x)$ è pari, allora $\hat{u}(\xi)$ è pari e reale.

$\implies $ i valori per $\xi > 0$ si ottengono imponendo $\hat{u}(- \xi) = \hat{u}(\xi)$.

$u\in L^{1}(\RR)$, allora la sua trasformata è continua.

$\implies $ il valore per $\xi = 0$ si ottiene per continuità.
\begin{equation*}
\hat{u}(\xi) = \frac{\pi}{3} e^{- | \xi |} - \frac{\pi}{6} e^{- 2| \xi |} \ \ \ \ \forall \xi \in \RR
\end{equation*}
A questo punto ricordiamo la proprietà della \textit{modulazione}
\begin{equation*}
u(x) e^{iax} \mapsto \hat{u}(\xi - a)
\end{equation*}
e che possiamo scrivere le identità
\begin{equation*}
\cos \vartheta = \frac{e^{i\vartheta} + e^{- i\vartheta}}{2} \ \ \ \ \sin \vartheta = \frac{e^{i\vartheta} - e^{- i\vartheta}}{2i}
\end{equation*}
in modo da far apparire il termine per usare la proprietà
\begin{equation*}
\begin{aligned}
\Fc (f(x), \xi) & = \Fc \left(\frac{\sin x}{\left(1 + x^{2}\right)\left(4 + x^{2}\right)}, \xi \right)\\
 & = \Fc \left(\frac{e^{ix} - e^{- ix}}{2i\left(1 + x^{2}\right)\left(4 + x^{2}\right)}, \xi \right)\\
 & = \frac{1}{2i}\Fc \left(\frac{e^{ix}}{\left(1 + x^{2}\right)\left(4 + x^{2}\right)}, \xi \right) + \frac{1}{2i}\Fc \left(\frac{- e^{- ix}}{\left(1 + x^{2}\right)\left(4 + x^{2}\right)}, \xi \right)\\
 & = \frac{1}{2i}\left\{\Fc \left(\frac{1}{\left(1 + x^{2}\right)\left(4 + x^{2}\right)}, \xi - 1\right) - \Fc \left(\frac{1}{\left(1 + x^{2}\right)\left(4 + x^{2}\right)}, \xi + 1\right)\right\}\\
 & = \frac{1}{2i}\left\{\frac{\pi}{3} e^{- | \xi - 1|} - \frac{\pi}{6} e^{- 2| \xi - 1|} - \frac{\pi}{3} e^{- | \xi + 1|} + \frac{\pi}{6} e^{- 2| \xi + 1|}\right\}
\end{aligned}
\end{equation*}

\Soluzione

Conviene riscrivere $u$
\begin{equation*}
u(x) = \frac{2\cos x}{1 + x^{2}} = u(x) = \frac{2\frac{e^{ix} + e^{- ix}}{2}}{1 + x^{2}} = \frac{e^{ix} + e^{- ix}}{1 + x^{2}}
\end{equation*}
Sappiamo dal primo esercizio che
\begin{equation*}
\Fc \left\{\frac{1}{1 + x^{2}}, \xi \right\} = \pi e^{- | \xi |}
\end{equation*}
Allora
\begin{equation*}
\begin{aligned}
\Fc \{u(x), \xi \} & = \Fc \left\{\frac{e^{ix} + e^{- ix}}{1 + x^{2}}, \xi \right\}\\
 & = \Fc \left\{\frac{e^{ix}}{1 + x^{2}}, \xi \right\} + \Fc \left\{\frac{e^{- ix}}{1 + x^{2}}, \xi \right\}\\
 & = \pi e^{- | \xi - 1|} + \pi e^{- | \xi + 1|}
\end{aligned}
\end{equation*}

\Soluzione

Posto
\begin{equation*}
g(x) = \frac{\cos(2x)}{1 + x^{2}}
\end{equation*}
Per definizione
\begin{equation*}
\hat{g}(\xi) = \int_{\RR} e^{- i\xi x} g(x) dx
\end{equation*}
e dunque
\begin{equation*}
f(\tau) = \hat{g}\left(\tau^{2} - 2\right)
\end{equation*}
Sappiamo per il primo esercizio che
\begin{equation*}
\Fc \left\{\frac{1}{1 + x^{2}}, \xi \right\} = \pi e^{- | \xi |}
\end{equation*}
Inoltre
\begin{equation*}
g(x) = \frac{\cos(2x)}{1 + x^{2}} = \frac{e^{2ix} + e^{- 2ix}}{2\left(1 + x^{2}\right)}
\end{equation*}
Allora
\begin{equation*}
\hat{g}(\xi) = \Fc \{g(x), \xi \} = \frac{\pi}{2}\left(e^{- | \xi - 2|} + e^{- | \xi + 2|}\right)
\end{equation*}
E determiniamo $f(\tau)$ come
\begin{equation*}
f(\tau) = \hat{g}\left(\tau^{2} - 2\right) = \frac{\pi}{2}\left(e^{- \left| \tau^{2} - 4\right|} + e^{- \left| \tau^{2}\right|}\right)
\end{equation*}

\Soluzione

\begin{enumerate}
\item Calcoliamo
\begin{equation*}
\begin{aligned}
\hat{f}(\xi) & = \int_{\RR} e^{- i\xi x} f(x) dx\\
 & = \int^{0}_{- 1} - e^{- i\xi x} dx + \int^{1}_{0} e^{- i\xi x} dx\\
 & = \left[\frac{1}{i\xi} e^{- i\xi x}\right]^{0}_{- 1} + \left[ - \frac{1}{i\xi} e^{- i\xi x}\right]^{1}_{0}\\
 & = \frac{1}{i\xi}\left(1 - e^{i\xi} - e^{- i\xi} + 1\right)\\
 & = \frac{1}{i\xi}\left(2 - \textcolor[rgb]{0.29, 0.56, 0.89}{\left(e^{i\xi} + e^{- i\xi}\right)}\right) = \frac{1}{i\xi}(2 - \textcolor[rgb]{0.29, 0.56, 0.89}{2\cos\xi})
\end{aligned}
\end{equation*}
\item Calcoliamo
\begin{equation*}
\hat{g}(\xi) = i\frac{d}{d\xi}\hat{f}(\xi) = i\frac{d}{d\xi}\frac{2 - 2\cos \xi}{i\xi} = \frac{2(1 - \cos \xi - \xi \sin \xi)}{\xi^{2}}
\end{equation*}
\end{enumerate}

\Soluzione

$f$ può essere riscritta
\begin{equation*}
f(x) =
\begin{cases}
1 - x^{2}, & - 1 \leq x \leq 1\\
0, & \text{altrove}
\end{cases}
\end{equation*}
$f$ è pari, allora $\hat{f}$ è pari e reale.
\begin{equation*}
\hat{f}(\xi) = \int_{\RR} e^{- i\xi x} f(x) dx = \int^{1}_{- 1} e^{- i\xi x}\left(1 - x^{2}\right) dx = (\star)
\end{equation*}
Usando l'analisi complessa
\begin{equation*}
\rho e^{i\vartheta} = \rho [\cos \vartheta + i\sin \vartheta ]
\end{equation*}
allora
\begin{align*}
(\star) & = \int^{1}_{- 1}[\cos(\xi x) - i\sin(\xi x)]\left(1 - x^{2}\right) dx\\
 & = \int^{1}_{- 1}\underbrace{\cos(\xi x)\left(1 - x^{2}\right)}_{\text{pari}} dx - i\int^{1}_{- 1}\underbrace{\sin(\xi x)\left(1 - x^{2}\right)}_{\text{dispari}} dx\\
 & = 2\int^{1}_{0}\underbrace{\cos(\xi x)}_{g'}\underbrace{\left(1 - x^{2}\right)}_{f} - i \cdot 0\\
 & \overset{\text{ipp}}{=} 2\left\{\cancel{\left[\left(1 - x^{2}\right)\frac{1}{\xi}\sin(2\xi)\right]^{1}_{0}} - \int^{1}_{0}\frac{- 2x}{\xi}\sin(\xi x) dx\right\}\\
 & = \frac{4}{\xi}\int^{1}_{0}\underbrace{x}_{f}\underbrace{\sin(\xi x)}_{g'} dx\\
 & \overset{\text{ipp}}{=}\frac{4}{\xi}\left\{\left[ - \frac{x}{\xi}\cos(\xi x)\right]^{1}_{0} - \int^{1}_{0} - \frac{1}{\xi}\cos(\xi x) dx\right\}\\
 & = \frac{4}{\xi}\left\{- \frac{1}{\xi}\cos \xi + \left[\frac{1}{\xi^{2}}\sin(\xi x)\right]^{1}_{0}\right\}\\
 & = \frac{4}{\xi}\left\{- \frac{1}{\xi}\cos \xi + \frac{1}{\xi^{2}}\sin(\xi)\right\}\\
 & = - \frac{4}{\xi^{2}}\cos \xi + \frac{4}{\xi^{3}}\sin \xi = 4\left(\frac{\sin \xi - \xi \cos \xi}{\xi^{3}}\right)
\end{align*}
\begin{thm}
[Antitrasformata di Fourier] Siano $u, \hat{u} \in L^{1}(\RR)$
\begin{equation*}
\boxed{u(x) = \frac{1}{2\pi}\int_{\RR} e^{i\xi x}\hat{u}(\xi) d\xi \ \ \forall x\in \RR}
\end{equation*}
\end{thm}
Allora, ricordando che $\hat{f}$ è pari
\begin{align*}
f(x) & = \frac{1}{2\pi}\int_{\RR} e^{i\xi x}\hat{f}(\xi) d\xi \\
 & = \frac{1}{2\pi}\int_{\RR}[\cos(\xi x) + i\sin(\xi x)]\hat{f}(\xi) \xi \\
 & = \frac{1}{2\pi}\int_{\RR}\cos(\xi x) \cdot \hat{f}(\xi) d\xi \\
 & = \frac{1}{2\pi} \cdot 2\int^{\infty}_{0}\cos(\xi x) \cdot \hat{f}(\xi) d\xi \\
 & = \frac{1}{2\pi} \cdot 2\int^{\infty}_{0}\cos(\xi x) \cdot 4\left(\frac{\sin \xi - \xi \cos \xi}{\xi^{3}}\right) d\xi \\
 & = \frac{4}{\pi}\int^{+ \infty}_{0}\frac{\sin \xi - \xi \cos \xi}{\xi^{3}}\cos(\xi x) d\xi
\end{align*}
Per avere l'integrale richiesto dobbiamo prendere $x = \frac{1}{2}$
\begin{equation*}
f\left(\frac{1}{2}\right) = \frac{4}{\pi}\int^{+ \infty}_{0}\frac{\sin \xi - \xi \cos \xi}{\xi^{3}}\cos\left(\frac{\xi}{2}\right) d\xi \ \ \ \ \ \ \ \ f\left(\frac{1}{2}\right) = 1 - \frac{1}{4} = \frac{3}{4}
\end{equation*}
Troviamo quindi il l'integrale richiesto, a meno di un segno
\begin{equation*}
\int^{+ \infty}_{0}\frac{x\cos x - \sin x}{x^{3}}\cos\left(\frac{x}{2}\right) dx = -\frac{3}{4} \cdot \frac{\pi}{4} = - \frac{3\pi}{16}
\end{equation*}
\chapter{Esercitazione 9 - Boella}

\ParteEsercizi

Richiami di teoria sulla Trasformata di Laplace.
\begin{defn}
[Funzione di Heaviside] Si definisce
\begin{equation*}
H(t) =
\begin{cases}
1, & t \geq 0\\
0, & t < 0
\end{cases}
\end{equation*}
\end{defn}
\begin{figure}[htpb]
	\centering
\tikzset{every picture/.style = {line width = 0.75pt}} %set default line width to 0.75pt

\begin{tikzpicture}[x = 0.75pt, y = 0.75pt, yscale = -1, xscale = 1]
%uncomment if require: \path (0, 158); %set diagram left start at 0, and has height of 158

%Shape: Axis 2D [id:dp10122351594887924]
\draw (150.5, 120) - - (448.5, 120)(299.5, 15) - - (299.5, 152) (441.5, 115) - - (448.5, 120) - - (441.5, 125) (294.5, 22) - - (299.5, 15) - - (304.5, 22) ;
%Straight Lines [id:da9543821452612189]
\draw [line width = 1.5] (142.5, 120) - - (296.15, 120) ;
\draw [shift = {(299.5, 120)}, rotate = 0] [color = {rgb, 255:red, 0; green, 0; blue, 0} ][line width = 1.5] (0, 0) circle [x radius = 4.36, y radius = 4.36] ;
%Straight Lines [id:da289644330277681]
\draw [line width = 1.5] (429.5, 80) - - (299.5, 80) ;
\draw [shift = {(299.5, 80)}, rotate = 180] [color = {rgb, 255:red, 0; green, 0; blue, 0} ][fill = {rgb, 255:red, 0; green, 0; blue, 0} ][line width = 1.5] (0, 0) circle [x radius = 4.36, y radius = 4.36] ;

% Text Node
\draw (281, 68.4) node [anchor = north west][inner sep = 0.75pt] {$1$};
% Text Node
\draw (282, 125.4) node [anchor = north west][inner sep = 0.75pt] {$0$};
% Text Node
\draw (452, 122.4) node [anchor = north west][inner sep = 0.75pt] {$t$};


\end{tikzpicture}
\end{figure}
\FloatBarrier
Diciamo che $\Lc\{f(t)\} = F(s)$.

\textbf{Alcune trasformate notevoli.}
\begin{itemize}
\item $\Lc\{H(t)\} = \int^{+ \infty}_{0} e^{- st} dt = \frac{1}{s}$, in questo caso $\lambda = 0$
\item $\Lc\left\{t^{k} f(t)\right\} = (- 1)^{k}\frac{d^{k}}{ds^{k}} F(s)$
\item $\Lc\{tH(t)\} = \frac{1}{s^{2}}$

$\Lc\left\{t^{2} H(t)\right\} = - \left(- \frac{2}{s^{3}}\right) = \frac{2}{s^{3}}$

$\Lc\left\{t^{3} H(t)\right\} = - \frac{d}{ds}\left(\frac{2}{s^{3}}\right) = \frac{6}{s^{4}}$

$\Lc\left\{t^{n} H(t)\right\} = \frac{n!}{s^{n + 1}}$
\item $\Gamma (x) = \int^{+ \infty}_{0} t^{x - 1} e^{- t} dt$
\begin{itemize}
\item $\Gamma (n) = (n - 1) !$ se $n$ è intero
\item $\Gamma (x + 1) = x\Gamma (x)$ per ogni $x > 0$
\item $\Gamma \left(\frac{1}{2}\right) = \sqrt{\pi}$
\end{itemize}

$\Lc\left\{t^{\alpha} H(t)\right\} = \frac{\Gamma (\alpha + 1)}{s^{\alpha + 1}}$ se $\Re (\alpha) > - 1$
\item $\Lc\left\{\frac{1}{\sqrt{t}} H(t)\right\} = \frac{\Gamma \left(\frac{1}{2}\right)}{\sqrt{s}} = \frac{\sqrt{\pi}}{\sqrt{s}}$
\end{itemize}

\textbf{Proprietà.}
\begin{itemize}
\item $\Lc\{f(\alpha t)\} = \frac{1}{\alpha} F\left(\frac{s}{\alpha}\right)$
\item $\Lc\{f(t - \alpha)\} = e^{- \alpha s} F(s)$
\item $\Lc\left\{e^{\alpha t} f(t)\right\} = F(s - \alpha)$
\item $\Lc\{f'(t)\} = sF(s) - f\left(0^{+}\right)$
\item $\Lc\{f''(t)\} = s^{2} F(s) - sf\left(0^{+}\right) - f'\left(0^{+}\right)$
\item $\Lc\left\{\int^{t}_{0} f(\tau) d\tau \right\} = \frac{1}{s} F(s)$
\item $\Lc\left\{\frac{1}{t} f(t)\right\} = \int^{+ \infty}_{s} F(\sigma) d\sigma $
\end{itemize}

\textbf{Altre trasformate notevoli.}
\begin{itemize}
\item $\Lc\left\{e^{\alpha t} H(t)\right\} = \frac{1}{s - \alpha}$
\item $\Lc\{\sin(\omega t) H(t)\} = \frac{\omega}{s^{2} + \omega^{2}}$, pari in $s$
\item $\Lc\{\cos(\omega t) H(t)\} = \frac{s}{s^{2} + \omega^{2}}$, dispari in $s$
\item $\Lc\{\sinh(\omega t) H(t)\} = \frac{\omega}{s^{2} - \omega^{2}}$
\item $\Lc\{\cosh(\omega t) H(t)\} = \frac{s}{s^{2} - \omega^{2}}$
\end{itemize}

\Esercizio{}

Calcolare
\begin{equation*}
\Lc\left\{e^{- \alpha t}\sin(\omega t) H(t)\right\} = \frac{\omega}{\omega^{2} + (s + \alpha)^{2}}
\end{equation*}
Calcolare
\begin{equation*}
\begin{aligned}
\Lc\left\{e^{- t}\sin^{2}(t) H(t)\right\} & = \frac{1}{2}\Lc\left\{e^{- t}(1 - \cos 2t) H(t)\right\} = \frac{1}{2}\left[\frac{1}{s + 1} - \frac{s + 1}{4 + (s + 1)^{2}}\right]\\
 & = \frac{1}{2}\frac{4}{(s + 1)\left(4 + (s + 1)^{2}\right)} = \frac{2}{(s + 1)\left(4 + (s + 1)^{2}\right)}
\end{aligned}
\end{equation*}
Calcolare
\begin{equation*}
\Lc\left\{\int^{t}_{0}\sin(2\tau) d\tau \right\} = \frac{2}{s\left(4 + s^{2}\right)}
\end{equation*}
Calcolare
\begin{equation*}
\begin{aligned}
\Lc\left\{t^{2} e^{2t} H(t)\right\} & = \Lc\left\{e^{2t}\left[ t^{2} H(t)\right]\right\} = \frac{2}{(s - 2)^{3}}\\
 & = \Lc\left\{t^{2}\left[ e^{2t} H(t)\right]\right\} = \frac{d^{2}}{ds^{2}}\frac{1}{s - 2} = \frac{2}{(s - 2)^{3}} \
\end{aligned}
\end{equation*}
Calcolare
\begin{equation*}
\begin{aligned}
\Lc\left\{\frac{\sin t}{t} H(t)\right\} & = \Lc\left\{\frac{1}{t}[\sin(t) H(t)]\right\} = \int^{+ \infty}_{s}\frac{1}{1 + \sigma^{2}} d\sigma \\
 & = [\arctan \sigma ]^{+ \infty}_{s} = \frac{\pi}{2} - \arctan s
\end{aligned}
\end{equation*}

\Esercizio{}

Sia
\begin{equation*}
f(t) =
\begin{cases}
t, & 0 < t < 2\\
4 - t, & 2 \leq t < 4\\
0, & \text{altrimenti}
\end{cases}
\end{equation*}


\begin{figure}[htpb]
	\centering
\tikzset{every picture/.style = {line width = 0.75pt}} %set default line width to 0.75pt

\begin{tikzpicture}[x = 0.75pt, y = 0.75pt, yscale = -1, xscale = 1]
%uncomment if require: \path (0, 156); %set diagram left start at 0, and has height of 156

%Shape: Axis 2D [id:dp387811944250509]
\draw (150.5, 110) - - (448.5, 110)(299.5, 5) - - (299.5, 142) (441.5, 105) - - (448.5, 110) - - (441.5, 115) (294.5, 12) - - (299.5, 5) - - (304.5, 12) ;
%Straight Lines [id:da44163391987628864]
\draw [line width = 1.5] (142.5, 110) - - (299.5, 110) ;
%Straight Lines [id:da9344407308358857]
\draw [line width = 1.5] (434.5, 110) - - (379.5, 110) ;
%Straight Lines [id:da9599366268835559]
\draw [line width = 1.5] (339.5, 70) - - (299.5, 110) ;
%Straight Lines [id:da4331170712538084]
\draw [line width = 1.5] (379.5, 110) - - (339.5, 70) ;
%Straight Lines [id:da332148534363488]
\draw [line width = 0.75] [dash pattern = {on 0.84pt off 2.51pt}] (339.5, 70) - - (299.5, 70) ;
%Straight Lines [id:da8792189758357065]
\draw [line width = 0.75] [dash pattern = {on 0.84pt off 2.51pt}] (339.5, 109.39) - - (339.5, 70) ;

% Text Node
\draw (281, 58.4) node [anchor = north west][inner sep = 0.75pt] {$2$};
% Text Node
\draw (282, 115.4) node [anchor = north west][inner sep = 0.75pt] {$0$};
% Text Node
\draw (452, 112.4) node [anchor = north west][inner sep = 0.75pt] {$t$};
% Text Node
\draw (332, 115.4) node [anchor = north west][inner sep = 0.75pt] {$2$};
% Text Node
\draw (375, 115.4) node [anchor = north west][inner sep = 0.75pt] {$4$};


\end{tikzpicture}
\end{figure}
\FloatBarrier

Scriviamo diversamente la $f$
\begin{equation*}
\begin{aligned}
f(t) & = t[ H(t) - H(t - 2)] + (4 - t)[ H(t - 2) - H(t - 4)]\\
 & = tH(t) - 2(t - 2) H(t - 2) + (t - 4) H(t - 4)\\
 & \\
 & \implies \ \ \Lc\{f(t)\} = \frac{1}{s^{2}} - 2\frac{e^{- 2s}}{s^{2}} + \frac{e^{- 4s}}{s^{2}}
\end{aligned}
\end{equation*}

\Esercizio{}

Sia
\begin{equation*}
g(t) =
\begin{cases}
t, & 0 < t < 1\\
2 - t, & 1 \leq t < 2\\
t - 2, & 2 \leq t
\end{cases}
\end{equation*}


\begin{figure}[htpb]
	\centering
\tikzset{every picture/.style = {line width = 0.75pt}} %set default line width to 0.75pt

\begin{tikzpicture}[x = 0.75pt, y = 0.75pt, yscale = -1, xscale = 1]
%uncomment if require: \path (0, 156); %set diagram left start at 0, and has height of 156

%Shape: Axis 2D [id:dp16516737574733087]
\draw (150.5, 110) - - (448.5, 110)(299.5, 5) - - (299.5, 142) (441.5, 105) - - (448.5, 110) - - (441.5, 115) (294.5, 12) - - (299.5, 5) - - (304.5, 12) ;
%Straight Lines [id:da9317289947100813]
\draw [line width = 1.5] (142.5, 110) - - (299.5, 110) ;
%Straight Lines [id:da695102507830573]
\draw [line width = 1.5] (466.48, 23.02) - - (379.5, 110) ;
%Straight Lines [id:da44710495183723054]
\draw [line width = 1.5] (339.5, 70) - - (299.5, 110) ;
%Straight Lines [id:da5381477224930931]
\draw [line width = 1.5] (379.5, 110) - - (339.5, 70) ;
%Straight Lines [id:da018036722756767043]
\draw [line width = 0.75] [dash pattern = {on 0.84pt off 2.51pt}] (339.5, 70) - - (299.5, 70) ;
%Straight Lines [id:da9481675028243701]
\draw [line width = 0.75] [dash pattern = {on 0.84pt off 2.51pt}] (339.5, 109.39) - - (339.5, 70) ;

% Text Node
\draw (281, 58.4) node [anchor = north west][inner sep = 0.75pt] {$1$};
% Text Node
\draw (282, 115.4) node [anchor = north west][inner sep = 0.75pt] {$0$};
% Text Node
\draw (452, 112.4) node [anchor = north west][inner sep = 0.75pt] {$t$};
% Text Node
\draw (332, 115.4) node [anchor = north west][inner sep = 0.75pt] {$1$};
% Text Node
\draw (375, 115.4) node [anchor = north west][inner sep = 0.75pt] {$2$};


\end{tikzpicture}
\end{figure}
\FloatBarrier

Scriviamo diversamente la $g$
\begin{equation*}
\begin{aligned}
g(t) & = t[ H(t) - H(t - 1)] + (2 - t)[ H(t - 1) - H(t - 2)] + (t - 2)[ H(t - 2)\\
 & \\
 & \implies \ \ \Lc\{g(t)\} = \frac{1}{s^{2}} - 2\frac{e^{- s}}{s^{2}} + 2\frac{e^{- 2s}}{s^{2}}
\end{aligned}
\end{equation*}

\Esercizio{}

Calcolare l'antitrasformata
\begin{equation*}
\Lc^{- 1}\left\{\frac{2s^{2} - 4}{(s + 1)(s - 2)(s - 3)}\right\} = \Lc^{- 1}\{(\star)\}
\end{equation*}
Scomponiamo in fratti semplici
\begin{equation*}
\begin{aligned}
(\star) & = \frac{A}{s + 1} + \frac{B}{s - 2} + \frac{C}{s - 3}\\
 & = \frac{A(s - 2)(s - 3) + B(s + 1)(s - 3) + C(s + 1)(s - 2)}{(s + 1)(s - 2)(s - 3)}\\
 & \implies \ \ A = -\frac{1}{6} \ \ \ \ B = -\frac{3}{8} \ \ \ \ C = \frac{7}{2}
\end{aligned}
\end{equation*}
Quindi
\begin{equation*}
\begin{aligned}
\Lc^{- 1}\{(\star)\} & = \Lc^{- 1}\left\{- \frac{1}{6}\frac{1}{s + 1} - \frac{3}{8}\frac{1}{s - 2} + \frac{7}{2}\frac{1}{s - 3}\right\}\\
 & = \left[ - \frac{1}{6} e^{- t} - \frac{3}{8} e^{2t} + \frac{7}{2} e^{3t}\right] H(t)
\end{aligned}
\end{equation*}

\Esercizio{}

Calcolare l'antitrasformata
\begin{equation*}
\Lc^{- 1}\left\{\frac{3s + 1}{(s - 1)\left(s^{2} + 1\right)}\right\} = \Lc^{- 1}\{(\star)\}
\end{equation*}
Scomponiamo in fratti semplici
\begin{equation*}
\begin{aligned}
(\star) & = \frac{A}{s - 1} + \frac{Bs + C}{\left(s^{2} + 1\right)}\\
 & = \frac{A\left(s^{2} + 1\right) + (Bs + C)(s - 1)}{(s - 1)\left(s^{2} + 1\right)}\\
 & \implies \ \ A = 2\ \ \ \ C = 1\ \ \ \ B = -2
\end{aligned}
\end{equation*}
Quindi
\begin{equation*}
\begin{aligned}
\Lc^{- 1}\{(\star)\} & = \Lc^{- 1}\left\{\frac{2}{s - 1} + \frac{- 2s + 1}{s^{2} + 1}\right\}\\
 & = \Lc^{- 1}\left\{\frac{2}{s - 1} - \frac{2s}{s^{2} + 1} + \frac{1}{s^{2} + 1}\right\}\\
 & = \left[ 2e^{t} + \sin t - 2\cos t\right] H(t)
\end{aligned}
\end{equation*}

\Esercizio{}

Calcolare
\begin{equation*}
\Lc^{- 1}\left\{\frac{1}{(s + 3)^{2}}\right\} = e^{- 3t} tH(t)
\end{equation*}
Calcolare
\begin{equation*}
\begin{aligned}
\Lc^{- 1}\left\{\frac{e^{- 2s}}{s^{3} + s}\right\} & = \Lc^{- 1}\left\{\frac{1}{s}\frac{e^{- 2s}}{s^{2} + 1}\right\}\\
 & = \int^{t}_{0}\sin(\tau - 2) H(\tau - 2) d\tau \\
 & = H(t - 2)\int^{t}_{2}\sin(\tau - 2) d\tau \\
 & = H(t - 2)[ - \cos(\tau - 2)]^{t}_{2}\\
 & = H(t - 2)[ 1 - \cos(t - 2)]
\end{aligned}
\end{equation*}
Calcolare
\begin{equation*}
\begin{aligned}
\Lc^{- 1}\left\{\frac{4e^{- 2s}\sinh s}{s^{2} + 4}\right\} & = \Lc^{- 1}\left\{\frac{2\left(e^{- s} - e^{- 3s}\right)}{s^{2} + 4}\right\}\\
 & = \sin[ 2(t - 1)] H(t - 1) - \sin[ 2(t - 3)] H(t - 3)
\end{aligned}
\end{equation*}

\Esercizio{}

Determinare le soluzioni $u\in L^{1}(\RR)$ di
\begin{equation*}
u' - \alpha u = e^{- \alpha x} H(x) \ \ \ \ \alpha > 0
\end{equation*}

\Esercizio{}

Risolvere
\begin{equation*}
u'' + 2xu' + 2u = 0
\end{equation*}

\Esercizio{}

Determinare le soluzioni $u\in L^{1}(\RR)$ di
\begin{equation*}
xu'' + 2u' - xu = xe^{- | x|}
\end{equation*}

\ParteSoluzioni

\Soluzione

Soluzione nella parte esercizi.

\Soluzione

Soluzione nella parte esercizi.

\Soluzione

Soluzione nella parte esercizi.

\Soluzione

Soluzione nella parte esercizi.

\Soluzione

Soluzione nella parte esercizi.

\Soluzione

Soluzione nella parte esercizi.

\Soluzione

\underline{\textbf{Con l'Analisi 2.}}
\begin{thm}
Ricordiamo questo risultato di Analisi 2.
\begin{equation*}
y' - \alpha y = f
\end{equation*}
L'integrale generale è
\begin{equation*}
y(x) = Ce^{\alpha x} + y^{\star}(x)
\end{equation*}
\end{thm}
Nel nostro caso
\begin{equation*}
f(x) =
\begin{cases}
0, & x < 0\\
e^{- \alpha x}, & x \geq 0
\end{cases}
\end{equation*}
Quindi come equazione particolare
\begin{equation*}
y^{\star}(x) =
\begin{cases}
0, & x < 0\\
Ae^{- \alpha x} = - \frac{1}{2\alpha} e^{- \alpha x}, & x \geq 0
\end{cases}
\end{equation*}
Quindi l'integrale generale è
\begin{equation*}
\begin{aligned}
y(x) & =
\begin{cases}
C_{1} e^{\alpha x}, & x < 0\\
C_{2} e^{\alpha x} - \frac{1}{2\alpha} e^{- \alpha x}, & x \geq 0
\end{cases} & C_{1} = C_{2} - \frac{1}{2\alpha}\\
 & =
\begin{cases}
Ce^{\alpha x}, & x < 0\\
\left(C + \frac{1}{2\alpha}\right) e^{\alpha x} - \frac{1}{2\alpha} e^{- \alpha x}, & x \geq 0
\end{cases} &
\end{aligned}
\end{equation*}
\underline{\textbf{Con Fourier.}}

Trasformiamo secondo Fourier
\begin{equation*}
\begin{aligned}
\Fc \left\{e^{- \alpha x} H(x)\right\} & = \int^{+ \infty}_{0} e^{- i\xi x} e^{- \alpha x} dx = \int^{+ \infty}_{0} e^{- (\alpha + i\xi) x} dx\\
 & = \left[ - \frac{1}{\alpha + i\xi} e^{- (\alpha + i\xi) x}\right]^{+ \infty}_{0} = \frac{1}{\alpha + i\xi}
\end{aligned}
\end{equation*}
Torniamo all'equazione
\begin{equation*}
\begin{aligned}
i\xi \hat{u} - \alpha \hat{u} = \frac{1}{\alpha + i\xi} \ \ & \implies \ \ (i\xi - \alpha)\hat{u} = \frac{1}{\alpha + i\xi}\\
 & \implies \ \ \hat{u} = - \frac{1}{\alpha - i\xi} \cdot \frac{1}{\alpha + i\xi} = - \frac{1}{\alpha^{2} + \xi^{2}}
\end{aligned}
\end{equation*}
Ci ricordiamo che
\begin{equation*}
\Fc \left\{e^{- \alpha | x|}\right\} = \frac{2\alpha}{\alpha^{2} + \xi^{2}} \ \ \implies \ \ u(x) = - \frac{1}{2\alpha} e^{- \alpha | x|}
\end{equation*}
Ma quelle trovate con l'Analisi 2 dove sono finite?
\begin{equation*}
y(x) \notin L^{1} \ \text{se} \ \left(C + \frac{1}{2\alpha}\right) \neq 0
\end{equation*}
Con Fourier richiediamo che stiano in $L^{1}$ e l'esponenziale positivo causa problemi, quindi per trovare, tra le infinite soluzioni, quella secondo Fourier, dobbiamo porre
\begin{equation*}
C + \frac{1}{2\alpha} = 0\ \ \implies \ \ u(x) = - \frac{1}{2\alpha} e^{- | x|}
\end{equation*}

\Soluzione

È un'equazione lineare del II ordine omogenea, ma \textit{\underline{non a coefficienti costanti}}.
\begin{itemize}
\item $\Fc \{u\} = \hat{u}$
\item $\Fc \{u''\} = - \xi^{2}\hat{u}$
\item $\Fc \{xu'\} = i\frac{d}{d\xi}\Fc \{u'\} = i\frac{d}{d\xi}(i\xi \hat{u}) = - (\hat{u} + \xi \hat{u}')$
\end{itemize}

Allora
\begin{equation*}
\begin{array}{l}
\implies \ \ - \xi^{2}\hat{u} - 2\left(\cancel{\hat{u}} + \xi \hat{u}'\right) + \cancel{2\hat{u}} = 0\ \ \implies \ \ 2\xi \hat{u}' + \xi^{2}\hat{u} = 0\\
\implies \ \ \hat{u}' + \frac{\xi}{2}\hat{u} = 0\ \ \implies \ \ \hat{u}(\xi) = Ce^{- \frac{\xi^{2}}{4}}
\end{array}
\end{equation*}
ma
\begin{equation*}
\Fc \left\{e^{- x^{2}}\right\} = \sqrt{\pi} e^{- \frac{\xi^{2}}{4}} \ \ \implies \ \ u(x) = De^{- x^{2}}
\end{equation*}
$u(x)$ dovrebbe essere la combinazione lineare di due integrali particolari, qui abbiamo trovato solo quello che sta in $L^{1}$.

\Soluzione

Trasformiamo
\begin{itemize}
\item $\Fc \{xu''\} = i\frac{d}{d\xi}\left(- \xi^{2}\hat{u}\right) = i\left(- 2\xi \hat{u} - \xi^{2}\hat{u}'\right)$
\item $\Fc \{u'\} = i\xi \hat{u}$
\item $\Fc \{xu\} = i\hat{u}'$
\item $\Fc \left\{xe^{- | x|}\right\} = i\frac{d}{d\xi}\frac{2}{1 + \xi^{2}} = i \cdot \left[\frac{- 4\xi}{\left(1 + \xi^{2}\right)^{2}}\right]$
\end{itemize}

Allora
\begin{equation*}
\begin{aligned}
\cancel{- 2i\xi \hat{u}} - i\xi^{2}\hat{u}' + \cancel{2i\xi \hat{u}} - i\hat{u}' & = - i\frac{4\xi}{\left(1 + \xi^{2}\right)^{2}}\\
\left(1 + \xi^{2}\right)\hat{u}' & = \frac{4\xi}{\left(1 + \xi^{2}\right)^{2}}\\
\hat{u}' & = \frac{4\xi}{\left(1 + \xi^{2}\right)^{3}} \ \ \implies \ \ \hat{u}(\xi) = - \frac{1}{\left(1 + \xi^{2}\right)^{2}} + c
\end{aligned}
\end{equation*}
dove $c = 0$ perché altrimenti non è infinitesima all'infinito e non è la trasformata di alcuna funzione $L^{1}$.

$\hat{u}$ è pari, quindi $u$ è pari.
\chapter{Esercitazione 9 - Potrich}

\ParteEsercizi

\Esercizio{}

Sia
\begin{equation*}
f(x) = H(x) e^{- 2x} =
\begin{cases}
e^{- 2x}, & x \geq 0\\
0, & x < 0
\end{cases}
\end{equation*}
calcolare $\displaystyle f*f$ e la sua trasformata di Fourier $\displaystyle \Fc (f*f)$.
\begin{rem}
[Prodotto di Convoluzione] È definito da
\begin{equation*}
(u*v)(x) = \int_{\RR} u(x - y) v(y) dy = \int_{\RR} u(y) v(x - y) dy
\end{equation*}
\end{rem}

\Esercizio{}

Determinare la trasformata di Fourier di
\begin{equation*}
f(x) = \ \left(1 - x^{2}\right) H\left(1 - x^{2}\right)
\end{equation*}
Utilizzando il risultato ottenuto calcolare
\begin{equation*}
I = \int^{+ \infty}_{0}\frac{(x\cos x - \sin x)^{2}}{x^{6}} dx
\end{equation*}

\Esercizio{}

\begin{enumerate}
\item Calcolare la trasformata di Fourier delle funzioni:
\begin{equation*}
f_{1}(x) = \frac{1}{1 + x^{2}} \ \ \ \ f_{2}(x) = \frac{x}{\left(1 + x^{2}\right)^{2}}
\end{equation*}
\item Considerare la funzione:
\begin{equation*}
f_{0}(x) = \arctan\left(\frac{1}{x}\right)
\end{equation*}

Stabilire se $\displaystyle f_{0} \in L^{1}(\RR)$ e/o se $\displaystyle f_{0} \in L^{2}(\RR)$.
\item Calcolare la derivata di $\displaystyle f_{0}$ nel senso delle distribuzioni.
\item Calcolare $\Fc \{f'_{0}\}$ e $\Fc \{f_{0}\}$.
\end{enumerate}

\Esercizio{}

\begin{defn}
Sia $\displaystyle f:\RR \ \rightarrow \CC $ si dice Laplace - trasformabile se:
\begin{equation*}
\supp f\subset [ 0, + \infty) \ \ \ \ \land \ \ \ \ \exists \lambda \in \RR :e^{- \lambda t} f(t) \in L^{1}(\RR)
\end{equation*}
\end{defn}
\begin{defn}
Si dice Ascissa di convergenza di $\displaystyle f$:
\begin{equation*}
\lambda_{f} \coloneqq \inf\left\{\lambda \in \RR :e^{- \lambda t} f(t) \in L^{1}(\RR)\right\}
\end{equation*}
\end{defn}
\begin{defn}
Sia $\displaystyle f$ una funzione Laplace - trasformabile, allora la funzione:
\begin{equation*}
\boxed{F(s) \coloneqq \int^{+ \infty}_{0} e^{- st} f(t) dt\ \ \ \ \forall s\in \CC :\Re (s) > \lambda_{f}}
\end{equation*}
si dice Trasformata di Laplace di $\displaystyle f$.
\end{defn}
\begin{rem}
[Trasformate di Laplace notevoli] Si ha
\begin{equation*}
\Lc\{H(t)\} = \frac{1}{s} \ \ \ \ \forall s\in \CC :\Re (s) > \lambda_{f} = 0
\end{equation*}
\end{rem}
\begin{thm}
[Proprietà] Valgono
\begin{itemize}
\item Riscalamento
\begin{equation*}
\Lc\{f(at)\} = \frac{1}{a} F\left(\frac{s}{a}\right) \ \ \ \ \Re (s) > a\lambda_{f}
\end{equation*}
\item Ritardo
\begin{equation*}
\Lc\{f(t - a)\} = e^{- as} F(s) \ \ \ \ \Re (s) > \lambda_{f}
\end{equation*}
\item Prodotto con esponenziale
\begin{equation*}
\Lc\left\{e^{at} f(t)\right\} = F(s - a) \ \ \ \ \Re (s) > \lambda_{f} + \Re (a)
\end{equation*}
\item Linearità
\begin{equation*}
\Lc\{af + bg\} = a\Lc\{f\} + b\Lc\{g\} \ \ \ \ \forall a, b\in \CC, \forall s:\Re (s) > \max\{\lambda_{f}, \lambda_{g}\}
\end{equation*}
\item Derivata della trasformata
\begin{equation*}
\frac{d^{n}}{ds^{n}} F(s) = \Lc\left\{(- t)^{n} f(t)\right\}
\end{equation*}
\item La primitiva di una funzione $\Lc$ - trasformabile è anch'essa una funzione $\Lc$ - trasformabile
\begin{equation*}
\Lc\left\{\int^{t}_{0} f(\tau) d\tau \right\} = \frac{1}{s}\Lc\{f(t)\} \ \ \ \ \forall s\in \CC :\Re (s) > \max\{\lambda_{f}, 0\}
\end{equation*}
\end{itemize}
\end{thm}
Detta $\displaystyle H(x)$ la funzione di Heaviside, calcolare la trasformata di Laplace della funzione
\begin{equation*}
f(t) = H(t - 1) te^{2t}
\end{equation*}
precisando l'ascissa di convergenza.

\Esercizio{}

Calcolare la trasformata di Laplace di
\begin{equation*}
f(t) = H(t)\sum^{m}_{k = 0} a_{k} t^{k} \ \ \ \ a_{k} \in \RR
\end{equation*}
$\displaystyle \lambda_{f} = 0$, perché se moltiplichiamo $\displaystyle f$ per $\displaystyle e^{- \lambda t}$ con $\displaystyle \lambda > 0$ le $\displaystyle f$ sono integrabili,

mentre con $\displaystyle \lambda \leq 0$ non lo sono.

\Esercizio{}

Stabilire per quali valori di $\displaystyle a\in \RR $ la funzione
\begin{equation*}
f_{a}(x) = H(x) e^{ax^{2} + 2x}\sin^{2}(x)
\end{equation*}
è Laplace - trasformabile.

Calcolare poi la trasformata di Laplace di $\displaystyle f_{0}(x)$.

\Esercizio{}

Calcolare la trasformata di Laplace di
\begin{equation*}
f(t) = H(t)\sin(\omega t)
\end{equation*}
studiandone in particolare le trasformate di Laplace per $\displaystyle \omega = 1$ e dedurre le antitrasformate di Laplace delle seguenti funzioni:
\begin{equation*}
J(s) = \frac{1}{\left(1 + s^{2}\right)^{2}} \ \ \ \ G(s) = \frac{s}{\left(1 + s^{2}\right)^{2}} \ \ \ \ K(s) = \frac{1}{s\left(s^{2} + 4\right)}
\end{equation*}

\ParteSoluzioni

\Soluzione

Se $x < 0$
\begin{equation*}
(f*f)(x) = 0
\end{equation*}
Se $x \geq 0$ ricordando che l'integrale è non nullo per
\begin{equation*}
x - y \geq 0\ \ \land \ \ x \geq 0\ \ \ \ \iff \ \ \ \ y \leq x\ \ \land \ \ x \geq 0
\end{equation*}
quindi
\begin{align*}
(f*f)(x) & = \int_{\RR} f(x - y) f(x) dy = \int^{x}_{0} e^{- 2(x - y)} e^{- 2y} dy\\
 & = \int^{x}_{0} e^{- 2x} \ dy = e^{- 2x}\int^{x}_{0} dy = xe^{- 2x}
\end{align*}
\begin{thm}
Date $\displaystyle u, v\in \Sc (\RR) \implies \ \widehat{u*v} = \hat{u} \cdot \hat{v}$.
\end{thm}
\begin{align*}
\hat{f}(\xi) & = \int_{\RR} e^{- i\xi x} f(x) dx = \int_{\RR} e^{- i\xi x} H(x) e^{- 2x} dx = \int^{+ \infty}_{0} e^{- i\xi x} e^{- 2x} dx\\
 & = \int^{+ \infty}_{0} e^{- x(i\xi + 2)} dx = \left[ - \frac{e^{- x(i\xi + 2)}}{i\xi + 2}\right]^{+ \infty}_{0} = \frac{1}{2 + i\xi}\\
 & \\
 & \implies \ \ \widehat{f*f} = \hat{f} \cdot \hat{f} = \frac{1}{(2 + i\xi)^{2}}
\end{align*}

\Soluzione

Nella scorsa esercitazione:

$f$ pari, allora $\hat{f}$ pari.
\begin{equation*}
\hat{f}(\xi) = \int_{\RR} e^{- i\xi x} f(x) dx = \cdots = 4\frac{\sin(\xi) - \xi \cos(\xi)}{\xi^{3}}
\end{equation*}
\begin{thm}
[Identità di Plancherel] Se $\displaystyle u\in L^{2}\left(\RR^{n}\right)$
\begin{equation*}
\Vert \hat{u}\Vert_{L^{2}} = (2\pi)^{n/2}\Vert u\Vert_{L^{2}}
\end{equation*}
oppure elevando tutto alla seconda
\begin{equation*}
\Vert \hat{u}\Vert^{2}_{L^{2}} = (2\pi)^{n}\Vert u\Vert^{2}_{L^{2}} \ \ \implies \ \ \int_{\RR^{n}}| \hat{u}(\xi)|^{2} d\xi = (2\pi)^{n}\int_{\RR^{n}}| u(x)|^{2} dx
\end{equation*}
\end{thm}
Sfrutto il fatto che $f$ e $\hat{f}$ sono pari:
\begin{align*}
\int_{\RR}| f(x)|^{2} dx & = \int^{1}_{- 1}\left(1 - x^{2}\right)^{2} dx = 2\int^{1}_{0}\left(1 - x^{2}\right)^{2} dx\\
 & = 2\int^{1}_{0}\left(1 - 2x^2 + x^{4}\right) dx = 2\left[ x - \frac{2x^{3}}{3} + \frac{x^{5}}{5}\right]^{1}_{0} = \frac{16}{15}\\
 & \\
\int_{\RR}| \hat{f}(\xi)|^{2} d\xi & = \int^{+ \infty}_{- \infty}\left| 4\frac{(\sin \xi - \xi \ \cos \xi)}{\xi^{3}}\right|^{2} d\xi \\
 & = 16\int^{+ \infty}_{- \infty}\frac{(\sin \xi - \xi \ \cos \xi)^{2}}{\xi^{6}} d\xi \\
 & = 32\int^{+ \infty}_{0}\frac{(\sin \xi - \xi \ \cos \xi)^{2}}{\xi^{6}} d\xi = 32 \cdot I
\end{align*}
Usando Plancherel
\begin{equation*}
32 \cdot I = (2\pi)^{n}\int_{\RR}| f(x)|^{2} dx\overset{n = 1}{=} 2\pi \cdot \frac{16}{15} \ \ \implies \ \ I = \frac{\pi}{15}
\end{equation*}

\Soluzione

\begin{thm}
[Trasformata della derivata] Date $u, u'\in L^{1}(\RR)$ e $u\in C^{1}$
\begin{equation*}
\Fc \left\{\frac{d^{n}}{dx^{n}} u(x)\right\} = (i\xi)^{n} \cdot \Fc \{u(x)\}
\end{equation*}
\end{thm}
\begin{enumerate}
\item Nell'esercitazione scorsa abbiamo visto che:
\begin{equation*}
\hat{f}_{1}(\xi) = \pi e^{- | \xi |} \ \ \ \ \forall \xi \in \RR
\end{equation*}Osserviamo che
\begin{equation*}
f_{2}(x) = - \frac{1}{2} f_{1}'(x)
\end{equation*}Allora
\begin{equation*}
\hat{f}_{2}(\xi) = - \frac{1}{2}\Fc (f_{1}'(x), \xi) = - \frac{1}{2} \cdot i\xi \cdot \pi e^{- | \xi |} \ \ \ \ \forall \xi \in \RR
\end{equation*}
\item Notiamo che $f_{0}$ è limitata, allora $f_{0} \in L^{\infty}(\RR)$, mentre per $| x| \rightarrow \infty $ è asintotica a $1/x$, che non è integrabile, quindi $f_{0} \notin L^{1}(\RR)$. Tuttavia $f_{0} \in L^{2}(\RR)$.
\item $f_{0}$ è continua $\forall x\neq 0$, dove c'è una discontinuità di tipo salto di ampiezza $\pi $. Senza indugi e senza calcoli, deduciamo subito che
\begin{equation*}
f_{0}'(x)\overset{D(\RR)}{=} \pi \delta_{0}(x) - \frac{1}{1 + x^{2}}
\end{equation*}
\item Usando le proprietà della trasformata
\begin{align*}
\Fc (f_{0}'(x), \xi) & = \Fc (\pi \delta_{0} - f_{1}(x), \xi) = \pi - \pi e^{- | \xi |}\\
 & \\
i\xi \cdot\Fc (f_{0}(x), \xi) & = \Fc (f_{0}'(x), \xi) = \pi - \pi e^{- | \xi |}\\
 & \\
 & \implies \ \ \Fc (f_{0}(x), \xi) = \frac{\pi - \pi e^{- | \xi |}}{i\xi} = i\pi \frac{e^{- | \xi |} - 1}{\xi}
\end{align*}
\end{enumerate}

\Soluzione

Calcoliamo
\begin{equation*}
F(s) = \int^{+ \infty}_{0} e^{- st} f(t) dt = \int^{+ \infty}_{1} e^{- st} te^{2t} dt\ \overset{\text{ipp}}{=}\frac{1 - s}{(2 - s)^{2}} e^{2 - s}
\end{equation*}
cerco ascissa di convergenza, con $t \geq 1$
\begin{equation*}
e^{- \lambda t} te^{2t} = te^{t(2 - \lambda)} \in L^{1}(\RR) \ \ \iff \ \ \lambda > 2\ \ \implies \ \ \lambda_{f} = 2
\end{equation*}

\Soluzione

Usiamo le proprietà
\begin{equation*}
F(s) = \Lc\left\{H(t)\sum^{m}_{k = 0} a_{k} t^{k}\right\} = \sum^{m}_{k = 0} a_{k}\Lc\left\{H(t) t^{k}\right\} = \sum^{m}_{k = 0} a_{k}(- 1)^{k}\left(\frac{1}{s}\right)^{(k)} = (\star)
\end{equation*}
Provando vari $\displaystyle k$ mi accorgo che:
\begin{equation*}
(\star) = \sum^{m}_{k = 0} a_{k}\frac{k!}{s^{k + 1}}
\end{equation*}
È il risultato generale della \textbf{trasformata di un polinomio!}

\Soluzione

Abbiamo che
\begin{equation*}
f_{a}(x) = H(x) e^{ax^{2} + 2x}\sin^{2}(x) =
\begin{cases}
e^{ax^{2} + 2x}\sin^{2}(x), & x \geq 0\\
0, & x < 0
\end{cases}
\end{equation*}
Hanno il supporto in $\displaystyle [ 0, + \infty), \forall a\in \RR $. Moltiplichiamo per il solito esponenziale
\begin{equation*}
e^{- \lambda x} f_{a}(x) =
\begin{cases}
e^{ax^{2} + (2 - \lambda) x}\sin^{2}(x), & x \geq 0\\
0, & x < 0
\end{cases} \ \in L^{1} \ \ \iff \ \ a < 0
\end{equation*}
ottengo anche che l'ascissa di convergenza è $\displaystyle \lambda_{f} = - \infty $ poichè $\displaystyle \forall \lambda $ è integrabile.

Nel caso interessato
\begin{equation*}
a = 0\ \ \implies \ \ f_{0}(x) = H(x) e^{2x}\sin^{2}(x)
\end{equation*}
Calcoliamo
\begin{align*}
\Lc\{f_{0}(x), s\} & = \Lc\left\{H(x) e^{2x}\sin^{2}(x)\right\} \ = \Lc\left\{H(x) e^{2x}\frac{1 - \cos(2x)}{2}\right\}\\
 & = \frac{1}{2}\Lc\left\{H(x) e^{2x}\right\} - \frac{1}{2}\Lc\left\{H(x) e^{2x}\cos(2x)\right\}\\
 & = \frac{1}{2}\Lc\left\{H(x) e^{2x}\right\} - \frac{1}{2}\Lc\left\{H(x) e^{2x}\frac{e^{i2x} + e^{- i2x}}{2}\right\}\\
 & = \frac{1}{2}\Lc\left\{H(x) e^{2x}\right\} - \frac{1}{4}\Lc\left\{H(x) e^{2(1 + i) x}\right\} - \frac{1}{4}\Lc\left\{H(x) e^{2(1 - i) x}\right\}\\
 & = \frac{1}{2} \cdot \frac{1}{s - 2} - \frac{1}{4} \cdot \frac{1}{s - 2(1 + i)} - \frac{1}{4} \cdot \frac{1}{s - 2(1 - i)}\\
 & = \frac{1}{2}\frac{1}{s - 2} - \frac{1}{2}\frac{s - 2}{(s - 2)^{2} + 4}
\end{align*}

\Soluzione

Calcoliamo trasformata di $f$
\begin{align*}
\Lc\{f(t)\} & = \Lc\{H(t)\sin(\omega t)\} = \Lc\left\{H(t)\frac{e^{i\omega t} - e^{- i\omega t}}{2i}\right\}\\
 & = \frac{1}{2i}\Lc\left\{H(t) e^{i\omega t}\right\} - \frac{1}{2i}\Lc\left\{H(t) e^{- i\omega t}\right\}\\
 & = \frac{1}{2i} \cdot \frac{1}{s - i\omega} - \frac{1}{2i} \cdot \frac{1}{s + i\omega}\\
 & = \frac{1}{2i}\frac{s + i\omega - s + i\omega}{s^{2} + \omega^{2}}\\
 & = \frac{\omega}{s^{2} + \omega^{2}}
\end{align*}
Funzione $G(s)$.

Notiamo che
\begin{equation*}
\omega = 1\ \ \implies \ \ \Lc\{H(t)\sin(t)\} = \frac{1}{s^{2} + 1} = :\varphi (s) \ \ \implies \ \ G(s) = - \frac{1}{2} \varphi'(s)
\end{equation*}
Allora
\begin{equation*}
G(s) = - \frac{1}{2} \varphi'(s) = - \frac{1}{2}\frac{d}{ds}\Lc\{H(t)\sin(t)\} = \frac{1}{2}\Lc\{H(t) t\sin(t)\}
\end{equation*}
quindi la sua antitrasformata è
\begin{equation*}
g(t) = \frac{1}{2} H(t) t\sin(t)
\end{equation*}
Funzione $J(s)$.

Notiamo che
\begin{align*}
J(s) & = \frac{1}{s} G(s) = \frac{1}{s}\frac{1}{2}\Lc\{H(t) t\sin(t)\} = \frac{1}{2}\Lc\left\{\int^{t}_{0} H(\tau) \tau \sin(\tau) d\tau \right\}\\
 & \overset{\text{ipp}}{=}\frac{1}{2}\Lc\{(\sin(t) - t\cos(t)) H(t)\}
\end{align*}
quindi la sua antitrasformata è
\begin{equation*}
j(t) = \frac{1}{2}(\sin(t) - t\cos(t)) H(t)
\end{equation*}
Funzione $K(s)$.

Notiamo che
\begin{equation*}
K(s) = \frac{1}{s}\Lc\left\{\frac{1}{2}H(t)\sin(2t)\right\} = \frac{1}{2}\Lc\left\{\int^{t}_{0} H(\tau)\sin(2\tau) d\tau \right\} = \Lc\left\{\frac{1}{4}(1 - \cos(2t)) H(t)\right\}
\end{equation*}
quindi la sua antitrasformata è
\begin{equation*}
k(t) = \frac{1}{4}(1 - \cos(2t)) H(t)
\end{equation*}
\chapter{Esercitazione 10 - Boella}

\ParteEsercizi

\Esercizio{}

Determinare le soluzioni $u\in L^{1}$ di
\begin{equation*}
u + \frac{1}{2} u*e^{- | x|} = e^{- | x|}\cos x
\end{equation*}

\Esercizio{}

Determinare le soluzioni di
\begin{equation*}
\begin{cases}
u'' + 2u' + u = e^{- t} \chi_{(1, 2)}(t)\\
u(0) = u'(0) = 0
\end{cases}
\end{equation*}

\Esercizio{}

Determinare le soluzioni $u\in L^{1}$ di
\begin{equation*}
- u'' + u = \chi_{(- 1, 1)}(x)
\end{equation*}

\Esercizio{}

Determinare la soluzione $\Lc$ - trasformabile di
\begin{equation*}
\begin{cases}
u' + 2u = 2\int^{t}_{0} u(\tau)\sin(t - \tau) d\tau + 3\cos t\\
u(0) = 1
\end{cases}
\end{equation*}

\Esercizio{}

Determinare le soluzioni $u\in L^{1}$ di
\begin{equation*}
4u + 8u''*e^{- | x|} = x^{2} e^{- | x|}
\end{equation*}

\Esercizio{}

Determinare le soluzioni reali $u\in L^{1}$ di
\begin{equation*}
u'' + 2xu' + 4u = 0
\end{equation*}

\ParteSoluzioni

\Soluzione

\begin{gather*}
\Fc \left(e^{- | x|}\right) = \frac{2}{1 + \xi^{2}}\\
\begin{aligned}
\Fc \left(e^{- | x|}\cos x\right) & = \frac{1}{2}\Fc \left\{e^{ix} e^{- | x|} + e^{- ix} e^{- | x|}\right\} = \\
 & = \frac{1}{2}\left\{\frac{2}{1 + (\xi - 1)^{2}} + \frac{2}{1 + (\xi + 1)^{2}}\right\} = \\
 & = \frac{1}{\xi^{2} - 2\xi + 2} + \frac{1}{\xi^{2} - 2\xi + 2} = \frac{2\left(\xi^{2} + 2\right)}{\xi^{4} + 4}
\end{aligned}
\end{gather*}
Allora
\begin{equation*}
\hat{u} + \hat{u}\frac{1}{1 + \xi^{2}} = \frac{2\left(\xi^{2} + 2\right)}{\xi^{4} + 4} \ \ \implies \ \ \hat{u} \cdot \frac{2 + \xi^{2}}{1 + \xi^{2}} = \frac{2\left(\xi^{2} + 2\right)}{\xi^{4} + 4} \ \ \implies \ \ \hat{u} = \frac{2\left(1 + \xi^{2}\right)}{\xi^{4} + 4}
\end{equation*}
Antitrasformiamo
\begin{equation*}
u(x) = \frac{1}{2\pi}\int_{\RR} e^{ix\xi}\frac{2\left(1 + \xi^{2}\right)}{\xi^{4} + 4} d\xi
\end{equation*}
$\hat{u}$ reale pari, allora $u$ reale pari, quindi studiamo le $x > 0$ per poi specchiare.

Usando il Lemma di Jordan (semipiano superiore). Sia
\begin{equation*}
f(z) = \frac{2\left(1 + z^{2}\right) e^{ixz}}{z^{4} + 4} \ \ \implies \ \ u(x) = \frac{1}{2\pi} \cdot 2\pi i \cdot \sum_{\Im (z_{k}) > 0}\Res (f, z_{k})
\end{equation*}
Il denominatore si annulla in
\begin{equation*}
z^{4} + 4 = 0\ \ \iff \ \ z = \pm 1\pm i
\end{equation*}
Determiniamo i residui
\begin{align*}
\Res (f, 1 + i) & = \left. \frac{2\left(1 + z^{2}\right) e^{ixz}}{4z^{3}}\right|_{z = 1 + i} = \left. \frac{2z\left(1 + z^{2}\right) e^{ixz}}{4z^{4}}\right|_{z = 1 + i}\\
\Res (f, - 1 + i) & = \left. \frac{2\left(1 + z^{2}\right) e^{ixz}}{4z^{3}}\right|_{z = -1 + i} = \left. \frac{2z\left(1 + z^{2}\right) e^{ixz}}{4z^{4}}\right|_{z = -1 + i}
\end{align*}
calcoliamo
\begin{align*}
\sum_{\Im (z_{k}) > 0}\Res (f, z_{k}) & = - \frac{1}{8}\left\{(1 + i)(1 + 2i) e^{(- 1 + i) x} + (- 1 + i)(1 - 2i) e^{(- 1 - i) x}\right\}\\
 & = - \frac{e^{- x}}{8}\left\{(- 1 + 3i) e^{ix} + (1 + 3i) e^{- ix}\right\}
\end{align*}
Per le $x > 0$
\begin{align*}
u(x) & = i \cdot \sum_{\Im (z_{k}) > 0}\Res (f, z_{k})\\
 & = \frac{e^{- x}}{4}\left\{\frac{3 + i}{2} e^{ix} + \frac{3 - i}{2} e^{- ix}\right\}\\
 & = \frac{e^{- x}}{4}\left\{3\frac{e^{ix} + e^{- ix}}{2} - \frac{e^{ix} - e^{- ix}}{2i}\right\}\\
 & = \frac{e^{- x}}{4}(3\cos x - \sin x)
\end{align*}
Per le $x\in \RR $
\begin{equation*}
u(x) = \frac{e^{- | x|}}{4}(3\cos x - \sin| x|)
\end{equation*}

\Soluzione

\begin{gather*}
\begin{aligned}
\Lc\left(e^{- t} \chi_{(1, 2)}(t)\right) & = \int^{2}_{1} e^{- (s + 1) t} dt = \left[ - \frac{e^{- (s + 1) t}}{s + 1}\right]^{2}_{1}\\
 & = \frac{e^{- s - 1}}{s + 1} - \frac{e^{- 2s - 2}}{s + 1} = \frac{1}{e}\frac{e^{- s}}{s + 1} - \frac{1}{e^{2}}\frac{e^{- 2s}}{s + 1}
\end{aligned}\\
\\
\Lc(u) = U(s) \ \ \ \ \Lc(u') = sU(s) \ \ \ \ \Lc(u'') = s^{2} U(s)
\end{gather*}
Sostituiamo tutto
\begin{equation*}
\begin{aligned}
U(s)\left[ s^{2} + 2s + 1\right] & = \frac{1}{e}\frac{e^{- s}}{s + 1} - \frac{1}{e^{2}}\frac{e^{- 2s}}{s + 1}\\
U(s) & = \frac{1}{e}\frac{e^{- s}}{(s + 1)^{3}} - \frac{1}{e^{2}}\frac{e^{- 2s}}{(s + 1)^{3}}
\end{aligned}
\end{equation*}
Antitrasformiamo, ricordiamo che
\begin{equation*}
\Lc^{- 1}\left\{\frac{1}{s^{3}}\right\} = \frac{t^{2}}{2} H(t) \ \ \ \ \Lc^{- 1}\left\{\frac{1}{(s + 1)^{3}}\right\} = \frac{t^{2}}{2} e^{- t} H(t) \ \ \ \ \Lc\{f(t - \alpha)\} = e^{- \alpha s} F(s)
\end{equation*}
allora
\begin{align*}
u(t) & = \frac{1}{e}\frac{(t - 1)^{2}}{2} e^{- (t - 1)} H(t - 1) - \frac{1}{e^{2}}\frac{(t - 2)^{2}}{2} e^{- (t - 2)} H(t - 2)\\
 & = \frac{e^{- t}}{2}\left[(t - 1)^{2} H(t - 1) - (t - 2)^{2} H(t - 2)\right]\\
 & =
\begin{cases}
0, & t < 1\\
\frac{e^{- t}}{2}(t - 1)^{2}, & 1 \leq t < 2\\
\frac{e^{- t}}{2}(2t - 3), & t \geq 2
\end{cases}
\end{align*}

\Soluzione

Trasformiamo e usiamo il prodotto di convoluzione
\begin{equation*}
\left(\xi^{2} + 1\right)\hat{u} = \Fc \{\chi_{(- 1, 1)}(x)\}
\end{equation*}
allora
\begin{align*}
\implies \ \ \hat{u}(\xi) & = \frac{1}{1 + \xi^{2}} \cdot \Fc \{\chi_{(- 1, 1)}(x)\}\\
\implies \ \ u(x) & = \frac{1}{2} e^{- | x|} *\chi_{(- 1, 1)}(x)
\end{align*}
Ricordiamo che
\begin{equation*}
f*g = \int_{\RR} f(t) g(x - t) dt = \int_{\RR} f(x - t) g(t) dt
\end{equation*}
Quindi
\begin{equation*}
u(x) = \frac{1}{2}\int_{\RR} e^{- | t|} \chi_{(- 1, 1)}(x - t) dt
\end{equation*}
Dobbiamo dividere i vari casi
\begin{gather*}
\begin{aligned}
\chi_{(- 1, 1)}(x - t) & =
\begin{cases}
1, & - 1 < x - t < 1\\
0, & \text{altrove}
\end{cases}\\
 & =
\begin{cases}
1, & x - 1 < t < x + 1\\
0, & \text{altrove}
\end{cases}
\end{aligned}\\
\\
\implies \ \ u(x) = \frac{1}{2}\int^{x + 1}_{x - 1} e^{- | t|} dt =
\begin{cases}
A, & x < - 1\\
B, & - 1 \leq x \leq 1\\
C, & x > 1
\end{cases}
\end{gather*}
analizziamo i vari termini
\begin{align*}
A & = \frac{1}{2}\int^{x + 1}_{x - 1} e^{t} dt = \frac{1}{2}\left(e^{x + 1} - e^{x - 1}\right) = e^{x}\sinh 1\\
B & = \frac{1}{2}\left\{\int^{0}_{x - 1} e^{t} dt + \int^{x + 1}_{0} e^{- t} dt\right\} = \frac{1}{2}\left(1 - e^{x - 1} + 1 - e^{- x - 1}\right) = 1 - e^{- 1}\cosh x\\
C & = \frac{1}{2}\int^{x + 1}_{x - 1} e^{- t} dt = \frac{1}{2}\left(e^{- x - 1} - e^{- x + 1}\right) = e^{- x}\sinh 1
\end{align*}
mettendo insieme concludiamo
\begin{equation*}
u(x) =
\begin{cases}
1 - e^{- 1}\cosh x, & | x| < 1\\
e^{- | x|}\sinh 1, & | x| \geq 1
\end{cases}
\end{equation*}

\Soluzione

\begin{equation*}
\Lc(u) = U(s) \ \ \ \ \Lc(u') = sU(s) - 1
\end{equation*}
\begin{align*}
\implies \ \ sU(s) - 1 + 2U(s) & = 2U(s) \cdot \Lc\{\sin tH(t)\} + 3\Lc\{\cos t\}\\
sU - 1 + 2U & = U\frac{2}{s^{2} + 1} + 3\frac{s}{s^{2} + 1}\\
U\left(s + 2 - \frac{2}{s^{2} + 1}\right) & = 3\frac{s}{s^{2} + 1} + 1\\
U \cdot \frac{(s + 2)\left(s^{2} + 1\right) - 2}{\cancel{s^{2} + 1}} & = \frac{s^{2} + 3s + 1}{\cancel{s^{2} + 1}}\\
U(s) & = \frac{s^{2} + 3s + 1}{s^{3} + 2s^{2} + s} = \frac{s^{2} + 3s + 1}{s(s + 1)^{2}} = \\
 & = \frac{A}{s} + \frac{Bs + C}{(s + 1)^{2}} = \frac{A(s + 1)^{2} + (Bs + C) s}{s(s + 1)^{2}}\\
 & \\
 &
\begin{array}{l c l}
s = -1 & \rightarrow & B - C = -1\\
s = 1 & \rightarrow & 4A + B + C = 5\\
s = 0 & \rightarrow & A = 1\ \ \rightarrow \ \ B = 0\ \ C = 1
\end{array}\\
 & \ \ \ \ \\
U(s) & = \frac{1}{s} + \frac{1}{(s + 1)^{2}}\\
\implies \ \ u(t) & = H(t) + te^{- t} H(t)\\
 & = \left(1 + te^{- t}\right) H(t)
\end{align*}

\Soluzione

Trasformiamo
\begin{align*}
4\hat{u} - \frac{16\xi^{2}}{1 + \xi^{2}}\hat{u} & = - \frac{d^{2}}{d\xi^{2}}\frac{2}{1 + \xi^{2}}\\
\hat{u} \cdot \frac{4 - 12\xi^{2}}{1 + \xi^{2}} & = - \frac{d}{d\xi}\left[\frac{d}{d\xi}\frac{2}{1 + \xi^{2}}\right] = - \frac{d}{d\xi}\left[\frac{- 4\xi}{\left(1 + \xi^{2}\right)^{2}}\right]\\
 & = - \frac{- 4 \cdot \left(1 + \xi^{2}\right)^{2} + 4\xi \cdot 2\left(1 + \xi^{2}\right) 2\xi}{\left(1 + \xi^{2}\right)^{4}} = \frac{4 - 12\xi^{2}}{\left(1 + \xi^{2}\right)^{3}}
\end{align*}
quindi
\begin{equation*}
\hat{u} = \frac{1}{\left(1 + \xi^{2}\right)^{2}}
\end{equation*}
$\hat{u}$ reale pari
\begin{equation*}
u(x) = \frac{1}{2\pi}\int_{\RR} e^{ix\xi}\frac{1}{\left(1 + \xi^{2}\right)^{2}} d\xi
\end{equation*}
Per $x > 0$, usando il lemma di Jordan
\begin{align*}
u(x) & = \frac{1}{\cancel{2\pi}} \cdot \cancel{2\pi} i \cdot \Res \left\{\frac{e^{ixz}}{\left(1 + z^{2}\right)^{2}}, z = i\right\}\\
 & = i \cdot \frac{d}{dz}\left. \left[\cancel{(z - i)^{2}} \cdot \frac{e^{ixz}}{(z + i)^{2}\cancel{(z - i)^{2}}}\right]\right|_{z = i}\\
 & = i \cdot \left. \frac{ixe^{ixz}(z + i)^{2} - e^{ixz} 2(z + i)}{(z + i)^{4}}\right|_{z = i}\\
 & = i \cdot \frac{1}{16} e^{- x}(- 4ix - 4i)\\
 & = \frac{1}{4} \ e^{- x}(x + 1)
\end{align*}
Per $x\in \RR $
\begin{equation*}
u(x) = \frac{e^{- | x|}}{4}(| x| + 1)
\end{equation*}

\Soluzione

Trasformiamo i vari termini
\begin{equation*}
\Fc (u'') = - \xi^{2}\hat{u} \ \ \ \ \ \ \ \ \Fc (xu') = i\frac{d}{d\xi}(i\xi \hat{u}) = - \xi \hat{u}' - \hat{u}
\end{equation*}
sostituiamo
\begin{align*}
- \xi^{2}\hat{u} - 2\xi \hat{u}' - 2\hat{u} + 4\hat{u} & = 0\\
- 2\xi \hat{u}' + \left(2 - \xi^{2}\right)\hat{u} & = 0\\
\hat{u}' + \frac{2 - \xi^{2}}{- 2\xi}\hat{u} & = 0
\end{align*}
quindi la trasformata vale
\begin{equation*}
\hat{u}(\xi) = Ce^{- \int \frac{2 - \xi^{2}}{- 2\xi} d\xi} = C\xi e^{- \frac{\xi^{2}}{4}}
\end{equation*}
$\forall C$, $\hat{u}$ dispari allora $C = D \cdot i, \ D\in \RR $
\begin{equation*}
\ \hat{u}(\xi) = D \cdot i\xi e^{- \frac{\xi^{2}}{4}} \ \ \implies \ \ u(x) = D \cdot \frac{d}{dx}\left(e^{- x^{2}}\right) = K \cdot xe^{- x^{2}}
\end{equation*}
\chapter{Esercitazione 10 - Potrich}

\ParteEsercizi

\Esercizio{}

Determinare le soluzioni $\displaystyle \Fc $ - trasformabili della seguente equazione differenziale
\begin{equation*}
xu'' + 2u' - xu = xe^{- | x|}
\end{equation*}

\Esercizio{}

Determinare le soluzioni $\displaystyle u\in S'(\RR)$ dell'equazione differenziale
\begin{equation*}
- u'' + u = 1 + \delta_{0}''
\end{equation*}

\Esercizio{}

Trovare la soluzione $\Lc$ - trasformabile dell'equazione
\begin{equation*}
\begin{cases}
y'(t) - (t*y)(t) = 1\\
y(0) = - 1
\end{cases} \ \ \ \ \text{per} \ t > 0
\end{equation*}

\Esercizio{}

Determinare la soluzione $\Lc$ - trasformabile di
\begin{equation*}
\begin{cases}
2y(t) + y'(t) = 2\int^{t}_{0} y(\tau)\sin(t - \tau) d\tau + 3\cos(t)\\
y(0) = 1
\end{cases}
\end{equation*}

\Esercizio{}

Determinare la soluzione del seguente problema di Cauchy
\begin{equation*}
\begin{cases}
y'' + 2y' + y = e^{- t} \chi_{[ 1, 2]}(t)\\
y(0) = y'(0) = 0
\end{cases}
\end{equation*}

\Esercizio{}

Risolvere il problema integro - differenziale
\begin{equation*}
\begin{cases}
y'(t) = 1 + t + 8\int^{t}_{0}(t - \tau)^{2} y(\tau) d\tau \\
y(0) = 1
\end{cases}
\end{equation*}

\ParteSoluzioni

\Soluzione

Per le regole di derivazione
\begin{equation*}
\Fc \left\{u^{(n)}(x)\right\} = (i\xi)^{n}\Fc \{u(x)\} \ \ \implies \ \ \Fc \{u''(x)\} = - \xi^{2}\Fc \{u(x)\}
\end{equation*}
e inoltre che
\begin{equation*}
\frac{d^{n}}{d\xi^{n}}\Fc \{u(x)\} = \Fc \left\{(- ix)^{n} u(x)\right\} \ \ \implies \ \ \Fc \{xu(x)\} = i\frac{d}{d\xi}\Fc \{u(x)\}
\end{equation*}
Quindi
\begin{equation*}
\Fc \{xu''(x)\} = i\frac{d}{d\xi}\left(- \xi^{2}\Fc \{u(x)\}\right) = i\left[ - 2\xi \hat{u}(\xi) - \xi^{2}\frac{d}{d\xi}\hat{u}\right] = - 2i\xi \hat{u}(\xi) - i\xi^{2}\hat{u}'(\xi)
\end{equation*}
Trasformiamo il secondo termine
\begin{equation*}
\Fc \left\{e^{- | x|}, \xi \right\} = \frac{2}{1 + \xi^{2}}
\end{equation*}
Usando le proprietà
\begin{equation*}
\Fc \left\{xe^{- | x|}, \xi \right\} = i\frac{d}{d\xi}\frac{2}{1 + \xi^{2}}
\end{equation*}
Otteniamo
\begin{align*}
\cancel{- 2i\xi \hat{u}} - i\xi^{2}\hat{u}' + \cancel{2i\xi \hat{u}} - i\hat{u}' & = i\frac{d}{d\xi}\frac{2}{1 + \xi^{2}}\\
- \xi^{2}\hat{u}' - \hat{u}' & = \frac{d}{d\xi}\frac{2}{1 + \xi^{2}}\\
\hat{u}'\left(1 + \xi^{2}\right) & = - \frac{d}{d\xi}\frac{2}{1 + \xi^{2}} = - \frac{- 2 \cdot 2\xi}{\left(1 + \xi^{2}\right)^{2}} = \frac{4\xi}{\left(1 + \xi^{2}\right)^{2}}
\end{align*}
dato che $1 + \xi^{2} \neq 0$
\begin{equation*}
\hat{u}' = \frac{4\xi}{\left(1 + \xi^{2}\right)^{3}} \ \ \implies \ \ \hat{u} = - \frac{1}{\left(1 + \xi^{2}\right)^{2}} + c
\end{equation*}
Affinché $\hat{u}$ sia la trasformata di una funzione trasformabile, questa costante deve essere nulla, altrimenti non sarebbe infinitesima a infinito.
\begin{equation*}
\hat{u} = - \frac{1}{\left(1 + \xi^{2}\right)^{2}}
\end{equation*}
Andiamo ad antitrasformare.

$\hat{u}$ è reale pari, allora $u$ è reale pari. Per $x > 0$
\begin{equation*}
u(x) = \frac{1}{2\pi}\int_{\RR} e^{i\xi x}\hat{u}(\xi) d\xi = \frac{1}{2\pi}\int_{\RR} e^{i\xi x}\left[ - \frac{1}{\left(1 + \xi^{2}\right)^{2}}\right] d\xi =
\end{equation*}
applico il teorema dei residui e il Lemma di Jordan sulla semicirconferenza superiore
\begin{equation*}
= \frac{1}{2\pi}\int_{\RR} e^{i\xi x}\left[ - \frac{1}{\left(1 + \xi^{2}\right)^{2}}\right] d\xi = \frac{1}{2\pi} \cdot 2\pi i \cdot \Res \left(- \frac{e^{izx}}{\left(1 + z^{2}\right)^{2}}, z = i\right) =
\end{equation*}
è un polo del secondo ordine
\begin{equation*}
\begin{aligned}
= & - i\lim\limits_{z\rightarrow i}\frac{d}{dz}\cancel{(z - i)^{2}}\frac{e^{izx}}{\cancel{(z - i)^{2}}(z + i)^{2}} = - i\lim\limits_{z\rightarrow i}\frac{ixe^{izx}(z + i)^{2} - e^{izx} 2(z + i)}{(z + i)^{4}}\\
= & - i\frac{e^{- x}\left[ ix(2i)^{2} - 2(2i)\right]}{(2i)^{4}} = - i\frac{e^{- x}[ - 2x - 2]}{2^{3} i^{3}} = \frac{e^{- x}[ - x - 1]}{2^{2}} = - \frac{e^{- x}}{4}(x + 1)
\end{aligned}
\end{equation*}
Sfruttiamo la parità. Per $x\in \RR $
\begin{equation*}
u(x) = - \frac{e^{- | x|}}{4}(| x| + 1)
\end{equation*}

\Soluzione

Risolvo separatamente prima
\begin{equation*}
- u'' + u = \delta_{0}''
\end{equation*}
poi
\begin{equation*}
- u'' + u = 1
\end{equation*}
infine applico la sovrapposizione degli effetti.

Trasformiamo entrambi i membri
\begin{equation*}
\widehat{u''} = (i\xi)^{2}\hat{u} = - \xi^{2}\hat{u}
\end{equation*}
Possiamo calcolare la trasformata della derivata della Delta con la dualità
\begin{align*}
\langle \Fc \{\delta''(x), \xi \}, \varphi (x) \rangle & = \langle \delta''(x), \Fc \{\varphi (x), \xi \} \rangle \\
 & = - \langle \delta'(x), \frac{d}{d\xi}\Fc \{\varphi (x), \xi \} \rangle \\
 & = \langle \delta (x), \frac{d^{2}}{d\xi^{2}}\Fc \{\varphi (x), \xi \} \rangle \\
 & = \langle \delta (x), \Fc \left\{(- ix)^{2} \varphi (x), \xi \right\} \rangle \\
 & \overset{\xi = 0}{=}\int_{\RR}(- ix)^{2} \varphi (x) dx\\
 & = \langle (- ix)^{2}, \varphi (x) \rangle
\end{align*}
allora
\begin{equation*}
\Fc \{\delta''(x), \xi \} = (- i\xi)^{2} = - \xi^{2}
\end{equation*}
Sostituendo nell'equazione
\begin{equation*}
\xi^{2}\hat{u} + \hat{u} = - \xi^{2} \ \ \implies \ \ \left(1 + \xi^{2}\right)\hat{u} = - \xi^{2} \ \ \implies \ \ \hat{u}(\xi) = - \frac{\xi^{2}}{1 + \xi^{2}}
\end{equation*}
Notiamo che $\hat{u} \notin L^{1}$ quindi per forza dobbiamo ragionare nel senso delle distribuzioni.
\begin{equation*}
\hat{u}(\xi) = - 1 + \frac{1}{1 + \xi^{2}} \ \ \implies \ \ u(x) = - \delta + \frac{1}{2} e^{- | x|}
\end{equation*}
Per quanto riguarda
\begin{equation*}
- u'' + u = 1
\end{equation*}
l'unica soluzione in $\Sc'(\RR)$ è $u(x) = 1$, in conclusione
\begin{equation*}
u(x) = 1 - \delta + \frac{1}{2} e^{- | x|}
\end{equation*}

\Soluzione

Usiamo le proprietà della trasformata di Laplace
\begin{gather*}
\Lc\{y, s\} = Y(s) \ \ \ \ \Lc\left\{\frac{d}{dt} y(t), s\right\} = sY(s) - y\left(0^{+}\right)\\
\Lc\{f*g, s\} = F(s) G(s) \ \ \ \ \Lc\{tH(t)\} = \frac{1}{s^{2}}
\end{gather*}
Trasformando secondo Laplace i membri dell'equazione
\begin{align*}
sY(s) - \underbrace{y\left(0^{+}\right)}_{= -1} - \frac{1}{s^{2}} Y(s) & = \frac{1}{s}\\
\left(s - \frac{1}{s^{2}}\right) Y(s) & = \frac{1}{s} - 1\\
\frac{s^{3} - 1}{s^{\cancel{2}}} Y(s) & = \frac{1 - s}{\cancel{s}}\\
\frac{\cancel{(s - 1)}\left(s^{2} + s + 1\right)}{s} Y(s) & = - \cancel{(s - 1)}\\
Y(s) & = - \frac{s}{s^{2} + s + 1}
\end{align*}
Dobbiamo antitrasformare, ricordiamo che
\begin{equation*}
\Lc\{H(t)\cos(\omega t)\} = \frac{s}{s^{2} + \omega^{2}} \ \ \ \ \Lc\{H(t)\sin(\omega t)\} = \frac{\omega}{s^{2} + \omega^{2}}
\end{equation*}
Provo a ricondurmi a queste forme
\begin{equation*}
Y(s) = - \frac{s}{s^{2} + s + 1} = - \frac{s + \frac{1}{2} - \frac{1}{2}}{\left(s + \frac{1}{2}\right)^{2} + \frac{3}{4}} = - \frac{s + \frac{1}{2}}{\left(s + \frac{1}{2}\right)^{2} + \frac{3}{4}} + \frac{1}{2}\frac{1}{\left(s + \frac{1}{2}\right)^{2} + \frac{3}{4}}
\end{equation*}
Quindi
\begin{equation*}
y(t) = e^{- \frac{1}{2} t}\left[ - \cos\left(\frac{\sqrt{3}}{2} t\right) + \frac{1}{\sqrt{3}}\sin\left(\frac{\sqrt{3}}{2} t\right)\right]
\end{equation*}

\Soluzione

Trasformiamo i termini di
\begin{equation*}
2y(t) + y'(t) = 2\int^{t}_{0} y(\tau)\sin(t - \tau) d\tau + 3\cos(t)
\end{equation*}
\begin{align*}
\Lc\{y(t)\} & = Y(s)\\
\Lc\{y'(t)\} & = sY(s) - y\left(0^{+}\right)\\
\Lc\left\{\int^{t}_{0} y(\tau)\sin(t - \tau) d\tau \right\} & = \Lc\{(y*\sin)(t)\} = \Lc\{y(t)\} \cdot \Lc\{\sin(t)\}\\
\Lc\{H(t)\cos(\omega t)\} & = \frac{s}{s^{2} + \omega^{2}}\\
\Lc\{H(t)\sin(\omega t)\} & = \frac{\omega}{s^{2} + \omega^{2}}
\end{align*}
Sostituiamo
\begin{equation*}
2Y(s) + sY(s) - \underbrace{y\left(0^{+}\right)}_{= 1} = 2Y(s) \cdot \frac{1}{s^{2} + 1} + 3\frac{s}{s^{2} + 1}
\end{equation*}
procediamo coi calcoli per ricavare $Y(s)$
\begin{align*}
\left[ 2 + s - \frac{2}{s^{2} + 1}\right] Y(s) & = 1 + \frac{3s}{s^{2} + 1}\\
\left[\frac{2\left(s^{2} + 1\right) + s\left(s^{2} + 1\right) - 2}{\cancel{s^{2} + 1}}\right] Y(s) & = \frac{s^{2} + 1 + 3s}{\cancel{s^{2} + 1}}\\
\left[ 2s^{2} + \cancel{2} + s^{3} + s - \cancel{2}\right] Y(s) & = s^{2} + 1 + 3s\\
s\left(s^{2} + 2s + 1\right) Y(s) & = s^{2} + 1 + 3s\\
s(s + 1)^{2} Y(s) & = s^{2} + 1 + 3s\\
Y(s) & = \frac{s^{2} + 1 + 3s}{s(s + 1)^{2}}
\end{align*}
Antitrasformiamo, cerchiamo di ricondurci a qualcosa di noto coi fratti semplici
\begin{rem}
Per scomporre in fratti semplici bisogna seguire alcune regole. Se nella scomposizione del polinomio al denominatore compare:
\begin{equation*}
\begin{array}{l l}
\text{se a denom. c'è:} & \text{allora si associa}\\
x - a & \frac{A}{x - a}\\
(x - a)^{n} & \frac{A_{1}}{x - a} + \frac{A_{2}}{(x - a)^{2}} + \dotsc + \frac{A_{n}}{(x - a)^{n}}\\
\begin{array}{l}
x^{2} + bx + c\\
(\Delta < 0)
\end{array} & \frac{Ax + B}{x^{2} + bx + c}\\
\begin{array}{l}
\left(x^{2} + bx + c\right)^{n}\\
(\Delta < 0)
\end{array} & \frac{A_{1} x + B_{1}}{x^{2} + bx + c} + \frac{A_{2} x + B_{2}}{\left(x^{2} + bx + c\right)^{2}} + \dotsc + \frac{A_{n} x + B_{n}}{\left(x^{2} + bx + c\right)^{n}}
\end{array}
\end{equation*}
\end{rem}
\begin{equation*}
\frac{s^{2} + 1 + 3s}{s(s + 1)^{2}} = \frac{A}{s} + \frac{Bs + C}{(s + 1)^{2}} = \frac{A(s + 1)^{2} + s(Bs + C)}{s(s + 1)^{2}} = \dotsc
\end{equation*}
Otteniamo
\begin{equation*}
Y(s) = \frac{s^{2} + 1 + 3s}{s(s + 1)^{2}} = \frac{1}{s} + \frac{1}{(1 + s)^{2}}
\end{equation*}
Ricordando
\begin{equation*}
\Lc\{tH(t)\} = \frac{1}{s^{2}} \ \ \ \ \Lc\left\{e^{\alpha t} f(t)\right\} = F(s - \alpha)
\end{equation*}
Si ottiene
\begin{equation*}
y(t) = H(t) + tH(t) e^{- t} = H(t)\left(1 + te^{- t}\right)
\end{equation*}

\Soluzione

Per risolvere
\begin{equation*}
y'' + 2y' + y = e^{- t} \chi_{[ 1, 2]}(t)
\end{equation*}
trasformiamo e usiamo le proprietà
\begin{align*}
\Lc\{y(t)\} & = Y(s)\\
\Lc\{y'(t)\} & = sY(s) - \cancel{y\left(0^{+}\right)} = sY(s)\\
\Lc\{y''(t)\} & = s\left(sY(s) - \cancel{y\left(0^{+}\right)}\right) - \cancel{y'\left(0^{+}\right)} = s^{2} Y(s)\\
\Lc\left\{e^{- t} \chi_{[ 1, 2]}(t)\right\} & = \int^{\infty}_{0} e^{- st} e^{- t} \chi_{[ 1, 2]}(t) dt = \int^{2}_{1} e^{- (s + 1) t} dt\\
\end{align*}
Sostituiamo nell'equazione
\begin{align*}
s^{2} Y(s) + 2sY(s) + Y(s) & = \frac{e^{- s}}{e(s + 1)} - \frac{e^{- 2s}}{e^{2}(s + 1)}\\
(s + 1)^{2} Y(s) & = \frac{e^{- s}}{e(s + 1)} - \frac{e^{- 2s}}{e^{2}(s + 1)}\\
Y(s) & = \frac{e^{- s}}{e(s + 1)^{3}} - \frac{e^{- 2s}}{e^{2}(s + 1)^{3}}
\end{align*}
Per antitrasformare notiamo che
\begin{gather*}
\Lc\left\{e^{\alpha t} f(t)\right\} = F(s - \alpha) \ \ \ \ \Lc\left\{t^{2} H(t)\right\} = (- 1)^{2}\frac{d^{2}}{ds^{2}}\frac{1}{s} = \frac{d}{ds}\left(- \frac{1}{s^{2}}\right) = \frac{2}{s^{3}}\\
\Lc\{f(t - \alpha)\} = e^{- \alpha s} F(s)
\end{gather*}
Allora
\begin{gather*}
\Lc\left\{\frac{1}{2} t^{2} H(t)\right\} = \frac{1}{s^{3}} \ \ \implies \ \ \Lc\left\{\frac{1}{2} t^{2} H(t) e^{- t}\right\} = \frac{1}{(s + 1)^{3}}\\
\implies \ \ \Lc\left\{\frac{1}{2}(t - 1)^{2} H(t - 1) e^{- (t - 1)}\right\} = \frac{e^{- s}}{(s + 1)^{3}}
\end{gather*}
Quindi
\begin{equation*}
y(t) = \frac{1}{e} \cdot \frac{1}{2}(t - 1)^{2} H(t - 1) e^{- (t - 1)} - \frac{1}{e^{2}}\frac{1}{2}(t - 2)^{2} H(t - 2) e^{- (t - 2)}
\end{equation*}

\Soluzione

Per risolvere
\begin{equation*}
y'(t) = 1 + t + 8\int^{t}_{0}(t - \tau)^{2} y(\tau) d\tau
\end{equation*}
antitrasformiamo
\begin{align*}
\Lc\{y'(t)\} & = sY(s) - y\left(0^{+}\right) = sY(s) - 1\\
\Lc\{1H(t)\} & = \frac{1}{s}\\
\Lc\{tH(t)\} & = (- 1)^{1}\frac{d}{ds}\frac{1}{s} = \frac{1}{s^{2}}\\
\Lc\left\{\int^{t}_{0}(t - \tau)^{2} y(\tau) d\tau \right\} & = \Lc\left\{t^{2} *y(t)\right\} = \Lc\left\{t^{2} H(t)\right\} \cdot \Lc\{y(t)\}\\
 & = \frac{2}{s^{3}} \cdot Y(s)
\end{align*}
Sostituiamo nell'equazione
\begin{align*}
sY(s) - 1 & = \frac{1}{s} + \frac{1}{s^{2}} + 8 \cdot \frac{2}{s^{3}} \cdot Y(s)\\
\left(s - \frac{16}{s^{3}}\right) Y(s) & = \frac{1}{s} + \frac{1}{s^{2}} + 1\\
\left(\frac{s^{4} - 16}{s^{3}}\right) Y(s) & = \frac{s + 1 + s^{2}}{s^{2}}\\
\left(\frac{\left(s^{2} + 4\right)\left(s^{2} - 4\right)}{s}\right) Y(s) & = s + 1 + s^{2}\\
Y(s) & = \frac{s\left(s + 1 + s^{2}\right)}{\left(s^{2} + 4\right)(s + 2)(s - 2)}
\end{align*}
Scomponiamo in fratti semplici
\begin{align*}
\frac{s\left(s + 1 + s^{2}\right)}{\left(s^{2} + 4\right)(s + 2)(s - 2)} & = \frac{As + B}{s^{2} + 4} + \frac{C}{s + 2} + \frac{D}{s - 2}\\
\frac{s^{2} + s + s^{3}}{\left(s^{2} + 4\right)(s + 2)(s - 2)} & = \frac{(As + B)\left(s^{2} - 4\right) + C\left(s^{2} + 4\right)(s - 2) + D\left(s^{2} + 4\right)(s + 2)}{\left(s^{2} + 4\right)(s + 2)(s - 2)}\\
 & = \frac{As^{3} + Bs^{2} - 4As - 4B + C\left(s^{3} - 2s^{2} + 4s - 8\right)}{\left(s^{2} + 4\right)(s + 2)(s - 2)}\\
 & \ \ \ \ + \frac{D\left(s^{3} + 2s^{2} + 4s + 8\right)}{\left(s^{2} + 4\right)(s + 2)(s - 2)}\\
 & = \frac{s^{3}(A + C + D) + s^{2}(B - 2C + 2D) + s(- 4A + 4C + 4D)}{\left(s^{2} + 4\right)(s + 2)(s - 2)}\\
 & + \frac{- 4B - 8C + 8D}{\left(s^{2} + 4\right)(s + 2)(s - 2)}
\end{align*}
Uguagliamo i polinomi
\begin{equation*}
\begin{aligned}
1 & = A + C + D\\
1 & = B - 2C + 2D\\
1 & = - 4A + 4C + 4D\\
0 & = - 4B - 8C + 8D
\end{aligned} \ \ \iff \ \
\begin{bmatrix}
1 & 0 & 1 & 1\\
0 & 1 & - 2 & 2\\
- 4 & 0 & 4 & 4\\
0 & - 4 & - 8 & 8
\end{bmatrix}
\begin{bmatrix}
A\\
B\\
C\\
D
\end{bmatrix} =
\begin{bmatrix}
1\\
1\\
1\\
0
\end{bmatrix}
\end{equation*}
che ha come soluzione\footnote{Ovviamente trovata a mente e senza usare MATLAB.}
\begin{equation*}
\begin{bmatrix}
A\\
B\\
C\\
D
\end{bmatrix} =
\begin{bmatrix}
3/8\\
1/2\\
3/16\\
7/16
\end{bmatrix}
\end{equation*}
Quindi
\begin{equation*}
Y(s) = \frac{\frac{3}{8} s + \frac{1}{2}}{s^{2} + 4} + \frac{3}{16(s + 2)} + \frac{7}{16(s - 2)}
\end{equation*}
Notiamo che
\begin{equation*}
\frac{\frac{3}{8} s + \frac{1}{2}}{s^{2} + 4} = \frac{3}{8} \cdot \frac{s}{s^{2} + 4} + \frac{1}{2} \cdot \frac{1}{s^{2} + 4}
\end{equation*}
Antitrasformiamo
\begin{align*}
\Lc^{- 1}\left\{\frac{3}{8} \cdot \frac{s}{s^{2} + 4}\right\} & = \frac{3}{8}\Lc^{- 1}\left\{\frac{s}{s^{2} + 4}\right\} = \frac{3}{8}\cos(2t) H(t)\\
\Lc^{- 1}\left\{\frac{1}{2} \cdot \frac{1}{s^{2} + 4}\right\} & = \frac{1}{4}\Lc^{- 1}\left\{\frac{2}{s^{2} + 4}\right\} = \frac{1}{4}\sin(2t) H(t)
\end{align*}
In conclusione
\begin{equation*}
y(s) = \left[\frac{7}{16} e^{2t} + \frac{3}{16} e^{- 2t} + \frac{3}{8}\cos(2t) + \frac{1}{4}\sin(2t)\right] H(t)
\end{equation*}

 \cleardoublepage

\end{document}